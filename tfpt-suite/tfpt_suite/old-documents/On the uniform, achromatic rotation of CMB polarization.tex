\documentclass[final,5p,times,twocolumn]{elsarticle}

% --- Fonts & math
\usepackage[T1]{fontenc}
\usepackage[utf8]{inputenc}
\usepackage{amsmath,amssymb,mathtools}

% --- Figures & tables
\usepackage{graphicx}
\graphicspath{{./}{figures/}}
\usepackage{booktabs}
\usepackage[font=small,labelfont=bf]{caption}
\usepackage{float}     % per [H]
\usepackage{needspace} % per evitare buchi di colonna


% --- Typesetting
\usepackage{microtype}
\usepackage{etoolbox}
\usepackage{dblfloatfix}
\usepackage[section]{placeins}
\usepackage{balance}

% --- Units
\usepackage{siunitx}
\sisetup{detect-weight=true,detect-inline-weight=math}

% --- URLs & links
\usepackage{xcolor}
\usepackage{url}
\Urlmuskip=0mu plus 2mu
\def\UrlBreaks{\do\/\do-\do.\do_}
\usepackage{hyperref}
\hypersetup{colorlinks=true,linkcolor=blue,citecolor=blue,urlcolor=blue}

% --- Bibliography options (natbib is loaded by elsarticle)
\biboptions{numbers,sort&compress}

% --- Float & display spacing
\captionsetup{aboveskip=3pt, belowskip=2pt}
\setlength{\abovedisplayskip}{6pt}
\setlength{\belowdisplayskip}{6pt}
\setlength{\abovedisplayshortskip}{4pt}
\setlength{\belowdisplayshortskip}{4pt}
\setlength{\textfloatsep}{8pt plus 2pt minus 2pt}
\setlength{\floatsep}{6pt plus 2pt minus 2pt}
\setlength{\intextsep}{6pt plus 2pt minus 2pt}
\setlength{\dbltextfloatsep}{8pt plus 2pt minus 2pt}
\setlength{\dblfloatsep}{6pt plus 2pt minus 2pt}
\setcounter{topnumber}{3}
\setcounter{bottomnumber}{2}
\setcounter{totalnumber}{4}
\renewcommand{\topfraction}{0.9}
\renewcommand{\bottomfraction}{0.7}
\renewcommand{\textfraction}{0.07}
\renewcommand{\floatpagefraction}{0.8}
\renewcommand{\dbltopfraction}{0.9}
\renewcommand{\dblfloatpagefraction}{0.8}

% --- Compact bibliography
\apptocmd{\thebibliography}{%
  \scriptsize
  \setlength{\itemsep}{0pt}%
  \setlength{\parskip}{0pt}%
  \setlength{\leftmargin}{\parindent}%
  \setlength{\labelsep}{3pt}%
}{}{}

% --- Macros used
\newcommand{\betath}{\SI{0.243}{\degree}}
\newcommand{\sbetath}{\SI{0.003}{\degree}}
\newcommand{\phiZero}{\num{0.0533}\pm\num{0.0007}}

\journal{Physics Letters B}

\begin{document}

\begin{frontmatter}

\title{On the uniform, achromatic rotation of CMB polarization}


\author[Ahaus]{Stefan Hamann\fnref{orcidS}}\ead{sh@sh-future.de}
\author[Brescia]{Alessandro Rizzo\corref{cor}\fnref{orcidA}}\ead{a.rizzo@physiks.net}
\cortext[cor]{Corresponding author.}
\fntext[orcidS]{ORCID: \href{https://orcid.org/0000-0002-0983-2589}{0000-0002-0983-2589}.}
\fntext[orcidA]{ORCID: \href{https://orcid.org/0000-0002-8030-3540}{0000-0002-8030-3540}.}

\address[Ahaus]{Independent Researcher, Ahaus, Germany}
\address[Brescia]{Independent Researcher, Brescia, Italy}

\begin{abstract}
Polarization angles are sharp probes of spacetime geometry. On a metric-compatible Riemann--Cartan background, a single variational structure yields a uniform, frequency-independent rotation $\beta$ of CMB linear polarization without post hoc fitting. A Fujikawa computation fixes the abelian vertex $g_{a\gamma\gamma}=-1/(2\pi)$; geometric optics gives $d\psi/d\eta=\tfrac12\,g_{a\gamma\gamma}\,da/d\eta$; a conformal trace stationarity at fixed connection, normalized in the torsionless GR limit, fixes $|\Delta a|=\varphi_0=\phiZero$. Fit-free with respect to CMB data: after anomaly normalization and the GR renormalization condition, no parameters are tuned to observations; the resulting prediction is $\beta_{\rm th}=\varphi_0/(4\pi)=\betath\pm\sbetath$ (degrees). Planck PR4 and ACT DR6 are consistent with this level and show no $\nu^{-2}$ slope. Angles are quoted in our convention $\beta\equiv -\beta_{\rm exp}$. Index-level steps, robustness to non-axial torsion and minimal deformations, and calibration-aware likelihood details are provided in the Supplementary material (online).
\end{abstract}

\begin{keyword}
cosmic birefringence \sep CMB polarization \sep trace anomaly \sep axial anomaly \sep Riemann--Cartan torsion \sep geometric optics
\end{keyword}

\end{frontmatter}

\section{Setup and conventions}
Heaviside--Lorentz units with $c=\hbar=1$, signature $(+,-,-,-)$, and Levi--Civita symbol $\varepsilon_{0123}=+1$ are used; the dual is $\tilde F^{\mu\nu}=\tfrac12\,\varepsilon^{\mu\nu\rho\sigma}F_{\rho\sigma}$. Hats denote torsionless curvature, e.g.\ $\hat R_{\mu\nu\rho\sigma}$; $\nabla$ is the full Riemann--Cartan covariant derivative. Angles in equations are in radians. The polarization-angle convention matches $TB/EB$ analyses via $\beta\equiv -\beta_{\rm exp}$. The field $a$ is a \emph{dimensionless} axial phase (not the cosmological scale factor). To avoid dimensional ambiguities in the trace equation, the gravitational coupling is understood in the standard renormalized form $\bar\kappa:=\mu\,\kappa_{\rm phys}$; the bar is dropped below. Observables derived here are RG-stationary at one loop once the GR renormalization condition is imposed.

\paragraph*{Result (no fit).}
In the setting above, the chiral Jacobian in Fujikawa regularization fixes $g_{a\gamma\gamma}=-1/(2\pi)$ with the adopted dual and interaction conventions. The modified Maxwell equations imply $d\psi/d\eta=\tfrac12\,g_{a\gamma\gamma}\,da/d\eta$, so $\beta=\tfrac12\,g_{a\gamma\gamma}\,\Delta a$ along the light path. The trace channel of the single action at fixed connection yields a monotone condition $I(\kappa,L)=\kappa^{3}-2c_{3}^{2}\kappa^{2}-8b_{1}c_{3}^{6}L=0$ with $L:=\ln(1/\varphi_0)$, $\partial I/\partial L<0$. Imposing in the torsionless limit the renormalization condition $I(\kappa_{\rm P},L^\ast)=0$ at the GR point fixes $L^\ast$ uniquely and therefore $\varphi_0=e^{-L^\ast}$; no CMB data enter this fixation. With the physical orientation $\Delta a=-\varphi_0$, the prediction is $\beta_{\rm th}=\varphi_0/(4\pi)=\betath\pm\sbetath$ (degrees).

\section{Single action, field equations, and trace stationarity}
A metric-compatible Riemann--Cartan geometry with torsion $T^\rho{}_{\mu\nu}$ and contorsion $K^\rho{}_{\mu\nu}$ is considered; axial torsion is $S_\mu=\tfrac16\varepsilon_{\mu\nu\rho\sigma}T^{\nu\rho\sigma}$. The bulk action reads
\begin{align}
\label{eq:action}
S &= \int d^{4}x\,\sqrt{-g}\,\Big[
    \frac{1}{2\kappa^{2}}\big(\hat R - \nabla_{\lambda}K^{\lambda}{}_{\mu}{}^{\mu} + K_{\mu\nu\lambda}K^{\mu\nu\lambda}\big)
    - \tfrac{1}{4} F_{\mu\nu}F^{\mu\nu} \nonumber\\
&\hspace{0.9cm} - \tfrac{1}{4} g_{a\gamma\gamma}\, a \, F_{\mu\nu}\tilde F^{\mu\nu}
    + \tfrac{\lambda}{2}\,\partial_{\mu}a\,S^{\mu}
    + \tfrac{1}{3}\,U(\kappa,L)\Big] + S_{\partial\mathcal M},
\end{align}
with $L=\ln(1/\varphi_{0})$. Variation in $S^\mu$ gives
\begin{equation}
\label{eq:Sa_relation}
\partial_\mu a=\lambda\,S_\mu,
\end{equation}
and for purely axial torsion
\begin{equation}
\label{eq:Kexplicit}
K_{\mu\nu\rho}=\tfrac{1}{2}\,\varepsilon_{\mu\nu\rho\sigma}\,S^{\sigma}
=\frac{1}{2\lambda}\,\varepsilon_{\mu\nu\rho\sigma}\,\partial^{\sigma}a,
\end{equation}
which makes the leading eikonal transport achromatic. Metric variation at fixed connection yields
\begin{equation}
\label{eq:Einstein_like}
\mathcal Q_{\mu\nu}-\tfrac12 g_{\mu\nu}\,\mathcal Q
= \kappa^{2}\,T_{\mu\nu}^{(a)}+\kappa^{2}\,T_{\mu\nu}^{(\mathrm{EM})} + \tfrac{1}{3}\,g_{\mu\nu}\,U,
\end{equation}
with
\begin{equation}
\label{eq:Q_def}
\mathcal Q_{\mu\nu}
= \hat R_{\mu\nu}-\nabla_{\lambda}K^{\lambda}{}_{\mu\nu}+K_{\mu\alpha\beta}K_{\nu}{}^{\alpha\beta},\qquad
\mathcal Q=g^{\mu\nu}\mathcal Q_{\mu\nu}.
\end{equation}
A conformal variation $g_{\mu\nu}\to e^{2\sigma}g_{\mu\nu}$ at fixed connection isolates the trace channel. On the geometric-optics vacuum, matter enters only through the renormalization of $U$; normalizing
\begin{equation}
\label{eq:U_norm}
\frac{\partial U}{\partial L}=\frac{8}{3}\,b_{1}\,c_{3}^{6}.
\end{equation}
\noindent where $c_{3}=1/(8\pi)$ by the abelian anomaly (Sec.~\ref{sec:anom}).

\noindent Stationarity $\delta S/\delta\sigma=0$ gives
\begin{equation}
\label{eq:CFE}
I(\kappa,L):=\kappa^{3}-2\,c_{3}^{2}\kappa^{2}-8\,b_{1}\,c_{3}^{6}\,L=0.
\end{equation}
Since $\partial I/\partial L=-8\,b_{1}\,c_{3}^{6}<0$, for fixed $\kappa$ there is a unique root $L^\ast$, hence $\varphi_{0}=e^{-L^\ast}\in(0,1)$.
\begin{equation}
\label{eq:V_def}
\mathcal V_{\mu\nu}:=\mathcal Q^{\mathrm{TF}}_{\mu\nu}+g_{\mu\nu}\,I(\kappa,L).
\end{equation}
\noindent The full variation is equivalent to $\mathcal V_{\mu\nu}=0$.

\paragraph*{Fixed-connection conformal stationarity.}
“Metric variation at fixed connection’’ means that in the conformal variation $g_{\mu\nu}\!\to\!e^{2\sigma}g_{\mu\nu}$ the affine structure $\Gamma$ is held fixed (Palatini logic with metric-compatibility imposed), so the trace channel is isolated without altering the transport sector. This is a renormalization prescription, not a gauge choice; the Einstein--Hilbert boundary term ensures a well-posed variational principle for Dirichlet $(g,\Gamma)$ data.

\section{Anomaly normalization and achromatic transport}\label{sec:anom}
A local chiral rotation $\psi\!\to\!e^{\tfrac{i}{2}a(x)\gamma_{5}}\psi$ produces the Fujikawa Jacobian
\begin{equation}
\label{eq:Fujikawa_core}
\Delta\mathcal{L}_{\rm anom}=+\,\frac{a(x)}{8\pi}\,\partial_{\mu}K^{\mu},\qquad
K^{\mu}=\varepsilon^{\mu\nu\rho\sigma}A_{\nu}\partial_{\rho}A_{\sigma},
\end{equation}
and since $F_{\mu\nu}\tilde F^{\mu\nu}=2\,\partial_{\mu}K^{\mu}$, matching to $-\tfrac14 g_{a\gamma\gamma}aF_{\mu\nu}\tilde F^{\mu\nu}$ fixes
\begin{equation}
\label{eq:ga_fixed}
 g_{a\gamma\gamma}=-\frac{1}{2\pi}.
\end{equation}
Equivalently, matching $-\tfrac14\,g_{a\gamma\gamma}\,aF\tilde F=+\,c_{3}\,aF\tilde F$ gives $g_{a\gamma\gamma}=-4c_{3}$ and therefore $c_{3}=1/(8\pi)$ in our conventions. The modified Maxwell equations read
\begin{equation}
\nabla_{\mu}F^{\mu\nu}+\tfrac{1}{2}\,g_{a\gamma\gamma}\,\partial_{\mu}a\,\tilde F^{\mu\nu}=0,
\end{equation}
and in geometric optics
\begin{equation}
\frac{d\psi}{d\eta}=\frac{1}{2}\,g_{a\gamma\gamma}\,\frac{da}{d\eta}\quad\Rightarrow\quad
\beta=\frac{1}{2}\,g_{a\gamma\gamma}\,\Delta a.
\end{equation}

\paragraph*{Sign check.}
With $\varepsilon_{0123}=+1$ and the above dual, Eq.~\eqref{eq:Fujikawa_core} fixes the overall sign in \eqref{eq:ga_fixed}. Since $d\psi/d\eta=\tfrac12 g_{a\gamma\gamma} da/d\eta$ and $\Delta a=-\varphi_0<0$, one has $\beta_{\rm th}=\varphi_0/(4\pi)>0$, identical in sign to the angles reported by experiments after the $\beta\equiv-\beta_{\rm exp}$ conversion.

\section{Trace anomaly and fixation of $b_{1}$}
On metric-compatible Riemann--Cartan backgrounds with minimal couplings and purely axial torsion, the Maxwell operator is unchanged at one loop. The dilatation anomaly takes the form
\begin{equation}
\langle T^{\mu}{}_{\mu}\rangle=\frac{b_{1}}{16\pi^{2}}\,F_{\mu\nu}F^{\mu\nu}+\cdots,
\end{equation}
and $b_{1}$ equals the abelian trace coefficient of the Standard Model, $b_{1}=41/10$ in GUT normalization. Axial torsion redistributes purely geometric terms in the DeWitt--Seeley coefficient $a_{2}$, while the $F_{\mu\nu}F^{\mu\nu}$ coefficient retains its torsionless value at one loop under minimal couplings; non-axial torsion and non-minimal operators do not mix into $F^{2}$ at this order \cite{BOS1992,Shapiro2002}.

\paragraph*{EWSB mapping (explicit).}
With $g_{1}^{2}=\tfrac{5}{3}g^{\prime 2}$ and $A_{\mu}=\sin\theta_{W} B_{\mu}+\cos\theta_{W} W^{3}_{\mu}$, kinetic diagonalization removes $F_{\mu\nu}Z^{\mu\nu}$ at one loop under minimal couplings; the abelian trace piece projects onto the canonically normalized photon operator $F_{\mu\nu}F^{\mu\nu}$ with unit weight. Hence the coefficient in front of $F^{2}$ remains $b_{1}=41/10$ after EWSB at this order.

\section{Renormalization condition at the GR point and prediction}
The torsionless limit must reproduce the Einstein--Hilbert sector $(2\kappa_{\rm P}^{2})^{-1}R$ at the chosen renormalization scale; this is a renormalization condition, not a fit. Imposing
\begin{equation}
\label{eq:GRcond}
I(\kappa_{\rm P},L^\ast)=0
\end{equation}
fixes $L^\ast$ uniquely because $\partial I/\partial L=-8b_{1}c_{3}^{6}<0$, hence $\varphi_{0}=e^{-L^\ast}$ is determined without reference to CMB data. Using $c_{3}=1/(8\pi)$ from \eqref{eq:ga_fixed} and the Standard Model value $b_{1}=41/10$, one finds numerically
\[
\varphi_{0}=0.0533\pm 0.0007,\qquad
\beta_{\rm th}=\frac{\varphi_{0}}{4\pi}=0.243^{\circ}\pm 0.003^{\circ}.
\]
The quoted uncertainty is the direct propagation of $\delta\varphi_{0}$ through $\beta_{\rm th}=\varphi_{0}/(4\pi)$ (angles converted to degrees); scheme shifts in $U$ cancel against the GR renormalization condition, and higher-order geometric corrections are $\ll 10^{-2}\,{\rm deg}$ across CMB bands and are not included.

\paragraph*{RG stationarity.}
Differentiating $I(\kappa_{\rm P},L^\ast)=0$ with respect to $\ln\mu$ shows that the change $U\to U+\tfrac{8}{3}b_{1}c_{3}^{6}\delta\ell$ under $\mu\to\mu e^{\delta\ell}$ shifts $I$ by a constant that is removed by \eqref{eq:GRcond}, so $dL^{\ast}/d\ln\mu=0$ and $d\beta_{\rm th}/d\ln\mu=0$ at one loop.

\section{Consistency with current measurements}
All angles below are quoted in our convention $\beta\equiv-\beta_{\rm exp}$. Planck PR4 reports $\beta=\SI{0.300}{\degree}\pm\SI{0.110}{\degree}$ (68\% C.L.) and a dedicated ACT DR6 analysis with absolute-angle priors finds $\beta=\SI{0.215}{\degree}\pm\SI{0.074}{\degree}$ (68\% C.L.), both compatible with our prediction $\beta_{\rm th}=\varphi_0/(4\pi)$. Because Planck and ACT rely on independent absolute-angle calibrations with different priors and systematics, a common positive $\beta$ in our sign convention cannot be produced by miscalibration alone; the joint $\nu^{-2}$ slope fit consistent with zero further disfavors Faraday-like leakage. A joint multi-band fit of a Faraday-like slope $\beta(\nu)=\beta_{0}+m\,(\nu/\mathrm{GHz})^{-2}$ yields $m$ consistent with zero, as expected for achromatic transport.

\begin{figure}[H]
\centering
\includegraphics[width=0.88\columnwidth]{fig1_beta_comparison.pdf}
\caption{Uniform birefringence angle. Horizontal band: $\beta_{\rm th}\pm\sigma_{\rm th}$. Points: Planck PR4 and ACT DR6 (68\% C.L.), in our convention $\beta=-\beta_{\rm exp}$.}
\label{fig:beta_summary}
\end{figure}


\begin{table}[H]
\centering
\caption{Prediction vs.\ measurements (degrees; 68\% C.L.).}
\label{tab:theory_data_summary}
\begin{tabular}{lcc}
\toprule
Entry & Value & Note \\
\midrule
Prediction $\beta_{\rm th}$ & $0.243 \pm 0.003$ & from $\varphi_0/(4\pi)$ \\
Planck PR4 & $0.300 \pm 0.110$ & nearly full-sky \\
ACT DR6 & $0.215 \pm 0.074$ & absolute-angle priors \\
\bottomrule
\end{tabular}
\end{table}

\section{Robustness}
Non-axial torsion modifies $K^\rho{}_{\mu\nu}$ but leaves $g_{a\gamma\gamma}$ and the $F^{2}$ anomaly coefficient unchanged in the minimal theory \cite{BOS1992,Shapiro2002}. The uniform rotation depends on $\partial_\mu a$ via \eqref{eq:Sa_relation} and is achromatic at leading eikonal order \cite{HehlObukhov,ObukhovRubilar2002}. Small non-minimal operators organize as higher-derivative corrections to the constitutive tensor and would induce suppressed frequency dependence; the joint multi-band slope consistent with zero constrains such effects at the $\lesssim 10^{-2}\,\si{\degree}$ level across CMB bands. Being a total derivative, the Nieh--Yan density renormalizes $L$-independent geometric counterterms and does not renormalize the $aF\tilde F$ vertex or the $F^2$ coefficient in the minimal, metric-compatible setup.

\section{Falsifiability}
A single-experiment $5\sigma$ rejection of $\beta_{\rm th}$ against $\beta=0$ requires a total polarization-angle uncertainty $\sigma_{\rm tot}\lesssim \SI{0.049}{\degree}$; for $3\sigma$, $\sigma_{\rm tot}\lesssim \SI{0.081}{\degree}$. These thresholds set the target for absolute polarization-angle calibration; once reached, the prediction $\beta_{\rm th}=\varphi_0/(4\pi)$ is immediately testable.

\section{Summary}
A single variational structure on metric-compatible Riemann--Cartan backgrounds predicts a uniform, achromatic rotation of CMB polarization with no tunable parameters. The anomaly fixes $g_{a\gamma\gamma}$, geometric optics fixes the transport law, and the trace equation with the GR renormalization condition fixes $|\Delta a|=\varphi_0$, yielding $\beta_{\rm th}=\varphi_0/(4\pi)=\betath\pm\sbetath$. Present data are consistent with a constant rotation at this level and show no $\nu^{-2}$ slope. Algebraic steps, RG details, and calibration-aware likelihoods are provided in the Supplementary material (online).

% Optional supplement: compile even if the file is missing
\IfFileExists{supplementary.tex}{\input{supplementary}}{}

\balance
\begin{thebibliography}{99}

\bibitem{HehlObukhov}
F.~W.~Hehl, Y.~N.~Obukhov,
\textit{Foundations of Classical Electrodynamics},
Birkh\"auser (2003).

\bibitem{Ni1973}
W.-T.~Ni,
A nonmetric theory of gravity,
Phys.\ Rev.\ D \textbf{7}, 2880 (1973).

\bibitem{CFJ1990}
S.~M.~Carroll, G.~B.~Field, R.~Jackiw,
Limits on a Lorentz- and parity-violating modification of electrodynamics,
Phys.\ Rev.\ D \textbf{41}, 1231 (1990).

\bibitem{Lue1999}
A.~Lue, L.~Wang, M.~Kamionkowski,
Cosmological signature of new parity-violating interactions,
Phys.\ Rev.\ Lett.\ \textbf{83}, 1506 (1999).

\bibitem{MinamiKomatsu2020}
Y.~Minami, E.~Komatsu,
Simultaneous determination of cosmic birefringence and miscalibrated polarization angles,
Phys.\ Rev.\ Lett.\ \textbf{125}, 221301 (2020).

\bibitem{Shapiro2002}
I.~L.~Shapiro,
Physical aspects of the space-time torsion,
Phys.\ Rept.\ \textbf{357}, 113--213 (2002).

\bibitem{BOS1992}
I.~L.~Buchbinder, S.~D.~Odintsov, I.~L.~Shapiro,
\textit{Effective Action in Quantum Gravity},
IOP Publishing (1992), Ch.~10.

\bibitem{ObukhovRubilar2002}
Y.~N.~Obukhov, G.~F.~Rubilar,
Fresnel analysis of wave propagation in media with general linear constitutive law,
Phys.\ Rev.\ D \textbf{66}, 024042 (2002).

\bibitem{PlanckBirefringence2022PRL}
P.~Diego-Palazuelos \textit{et al.},
Cosmic Birefringence from Planck PR4,
Phys.\ Rev.\ Lett.\ \textbf{128}, 091302 (2022).

\bibitem{PlanckBirefringence2022arXiv}
P.~Diego-Palazuelos \textit{et al.},
Cosmic Birefringence from Planck Public Release 4,
arXiv:2203.04830 [astro-ph.CO] (2022).

\bibitem{ACTDR6}
P.~Diego-Palazuelos, E.~Komatsu,
Cosmic Birefringence from the Atacama Cosmology Telescope Data Release 6,
arXiv:2509.13654 [astro-ph.CO] (2025).

\bibitem{Aumont2020}
J.~Aumont \textit{et al.},
Absolute calibration of the polarisation angle for future CMB B-mode experiments,
Astron.\ Astrophys.\ \textbf{634}, A100 (2020).
doi:10.1051/0004-6361/201935785

\end{thebibliography}

\end{document}

