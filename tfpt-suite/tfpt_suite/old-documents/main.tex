
\documentclass[11pt,a4paper]{article}
\usepackage[utf8]{inputenc}
\usepackage[T1]{fontenc}
\usepackage[ngerman]{babel}
\usepackage{lmodern}
\usepackage{geometry}
\geometry{margin=2.5cm}
\usepackage{amsmath,amssymb,bm}
\usepackage{siunitx}
\usepackage{physics}
\usepackage{microtype}
\usepackage{graphicx}
\usepackage{booktabs}
\usepackage{hyperref}
\usepackage{cleveref}
\usepackage{tikz}
\usepackage{enumitem}
\graphicspath{{figures/}}

\title{Strukturelle Äquivalenz der Felder:\\
von der kubischen Fixpunktgleichung zur vereinheitlichten Feldgleichung}
\author{Stefan Hamann \and Alessandro Rizzo}
\date{Fassung vom 27.09.2025}

\begin{document}
\maketitle

\begin{abstract}
Wir formulieren das \emph{Prinzip der strukturellen Äquivalenz der Felder}: die bekannten Operatorgesetze
(Gauß, Faraday und Ampère) tragen dieselbe Form in vier Regimen ein und desselben Feldes: Elektromagnetismus,
Gravito Elektrodynamik (schwaches Feld), thermo gravitative Antwort (sehr schwaches Feld) und eine spinoriale
Erweiterung über ein $\beta$-Feld. Die dynamische Klammer bildet eine torsionsvolle Geometrie, deren kompakte Gleichung
\begin{equation*}
\big(R - \nabla\!\cdot\!K + K^2\big)_{AB}=\kappa^2\Big(T^{(a)}+T^{(\mathrm{EM})}\Big)_{AB}
\end{equation*}
mit modifizierten Maxwell-Gleichungen und einer festen Axion Photon-Kopplung folgt. Daraus ergibt sich eine
parameterfreie Vorhersage für die kosmische Birefringenz und eine einfache geometrische Selbstkonsistenz auf der
orientierbaren Doppelabdeckung, die die Genauigkeit der $\alpha$-Fixpunktlösung weiter steigert.
\end{abstract}

\section{Leitsatz: Strukturelle Äquivalenz}
Die vier Regime sind nicht verschiedene Kräfte, sondern Operator-Folgen desselben Feldes in
unterschiedlichen Skalenfenstern. \Cref{tab:operatoren} fasst die Struktur zusammen.

\begin{table}[h]
\centering
\caption{Vier Regime desselben Feldes unter denselben Operatoren.}
\label{tab:operatoren}
\small
\begin{tabular}{@{}llccc@{}}
\toprule
\textbf{Operator / Gesetz} & \textbf{Elektromagnetismus} & \textbf{Gravito EM (schwach)} & \textbf{Thermo Grav.~(sehr schwach)} & \textbf{Spinoriale Erweiterung ($\beta$)}\\
\midrule
Gauß ($\nabla\!\cdot$) & $\nabla\!\cdot \mathbf E=\rho_e/\varepsilon_0$
& $\nabla\!\cdot \mathbf g=-4\pi G\rho_m$
& Dichtefluktuationen $\to$ ``Temperatur''
& $\beta$ koppelt Dichte $\leftrightarrow$ Polarisation \\
Keine Monopole ($\nabla\!\cdot$) & $\nabla\!\cdot \mathbf B=0$
& $\nabla\!\cdot \mathbf B_g=0$
& thermische Isotropie des Vakuums
& lokale Paritätsverletzung (axionartig) \\
Faraday ($\nabla\!\times$) & $\nabla\!\times \mathbf E=-\partial_t\mathbf B$
& $\nabla\!\times \mathbf g=-\partial_t\mathbf B_g$
& Wärmefluss der Raumzeit
& Polarisationsrotation (kosm.\ Birefringenz) \\
Ampère ($\nabla\!\times$) & $\nabla\!\times \mathbf B=\mu_0\mathbf J+\mu_0\varepsilon_0\,\partial_t\mathbf E$
& $\nabla\!\times \mathbf B_g=-(4\pi G/c^2)\mathbf J_m+\partial_t\mathbf g$
& Entropieströme des Vakuums
& neuer spinorialer Strom $\propto \beta$ \\
Wellen ($\nabla^2$) & EM-Wellen & Gravitationswellen & thermische Wellen des Raums & spinoriale Wellen (``Übersetzung'') \\
\midrule
Konstanten & $\varepsilon_0,\mu_0,c$ & $G,c$ & $k_B$ & $\beta$ \\
\bottomrule
\end{tabular}
\end{table}

\section{Kompakte Geometriegleichung und Birefringenz}
Aus einer torsionsvollen Variation ergibt sich die vereinheitlichte Gleichung
\begin{equation}
\big(R-\nabla\!\cdot\!K+K^2\big)_{AB}=\kappa^2\Big(T^{(a)}+T^{(\mathrm{EM})}\Big)_{AB},
\qquad
\nabla_A F^{AB}+2c_3(\partial^B a)\tilde F_{AB}=0,
\label{eq:ufe}
\end{equation}
wobei der axion photon Vertex durch $g_{a\gamma\gamma}= -4c_3$ fixiert ist und $c_3=\tfrac{1}{8\pi}$.
Für ein homogenes $a(\eta)$ rotiert die Linearpolarisation um
\begin{equation}
\beta=\tfrac{1}{2}\,g_{a\gamma\gamma}\,\Delta a = 2c_3\,\Delta a=\frac{\varphi_0}{4\pi}\,,
\end{equation}
so dass mit $\varphi_0= \tfrac{1}{6\pi}+\tfrac{3}{256\pi^4}$ die geschlossene Vorhersage $\beta_{\text{th}}=0.2427^\circ$ folgt.

\begin{figure}[h]
\centering
\includegraphics[width=0.78\linewidth]{birefringence_pr4.png}
\caption{Kosmische Birefringenz: Planck PR4 Schätzungen mit Fehlerbalken im Vergleich zur \emph{parameterfreien}
Vorhersage $\beta_{\text{th}}=0.2427^\circ$.}
\label{fig:birefr}
\end{figure}

\section{Geometrische Selbstkonsistenz auf der Doppelabdeckung}
Die Feldenergie, die den Logarithmus $\kappa(\varphi_0)$ erzeugt, koppelt an genau jene Geometrie, die $\varphi_0$ setzt.
Auf der orientierbaren Doppelabdeckung addieren sich die beiden Randbeiträge; die minimal konsistente Antwort lautet
\begin{equation}
\varphi_0(\alpha)=\varphi_{\text{tree}}+\delta_{\text{top}}\,(1-2\alpha),\qquad
\varphi_{\text{tree}}=\frac{1}{6\pi},\quad \delta_{\text{top}}=\frac{3}{256\pi^4}.
\end{equation}
Eingesetzt in die kubische Normalform
\begin{equation}
\alpha^3- A\,\alpha^2- A\,c_3^2\,\kappa(\varphi_0)=0,\quad
A=2c_3^3,\quad \kappa=\frac{b_1}{2\pi}\ln\!\frac{1}{\varphi_0},\ \ b_1=\frac{41}{10},
\end{equation}
liefert dies eine implizite Lösung mit $\alpha^{-1}\approx 137.0359903901$, also eine Verschiebung um $\sim 3.6$ ppm
in Richtung der CODATA-Zahl.

\section{E8-Leiter als Strukturrahmen}
Die log-exakte Leiter
\begin{equation}
\varphi_n=\varphi_0\,e^{-\gamma(0)}\Big(\frac{D_n}{D_1}\Big)^{\lambda},\qquad
D_n=60-2n,\ \gamma(0)=0.834,\ \lambda=\frac{\gamma(0)}{\ln 248-\ln 60}
\end{equation}
ordnet Skalen ohne Fits und erzeugt einfache Verhältnisgesetze.
\Cref{fig:ladder} zeigt die Stufen $n=1\ldots 26$.

\begin{figure}[h]
\centering
\includegraphics[width=0.8\linewidth]{e8_ladder.png}
\caption{E8-Leiter: log-exakte Stufen $\varphi_n$ (siehe Text).}
\label{fig:ladder}
\end{figure}

\section{Visual: Doppelabdeckung und Randzählung}
Die Zahl $6\pi$ für den linearen Randkoeffizienten entsteht aus zwei echten Rändern plus der Naht $\Gamma$ der Möbius-Faser.
\begin{figure}[h]
\centering
\begin{tikzpicture}[scale=1.0]
% Cylindrical depiction with two boundaries and a seam
\draw[thick] (-3,2) to[out=0,in=180] (3,2);
\draw[thick] (-3,-2) to[out=0,in=180] (3,-2);
\draw[very thick,red] (0,-1.8) -- (0,1.8);
\node at (0.2,0) [right,red] {$\Gamma$ (Naht, effektiver Rand $2\pi$)};
\node at (-2.7,2.3) {oberer Rand $2\pi$};
\node at (-2.7,-2.3) {unterer Rand $2\pi$};
\end{tikzpicture}
\caption{Orientierbare Doppelabdeckung der Möbius-Faser: zwei Ränder plus Naht $\Gamma$ addieren zu $6\pi$.}
\label{fig:doublecover}
\end{figure}

\section{Schluss}
Ein und dieselbe Operatorstruktur spannt Elektromagnetismus, schwache Gravitation, thermo\-gravitative Antwort
und eine spinoriale Erweiterung auf. Die torsionsvolle Geometrie bündelt diese Struktur in einer kompakten Gleichung
und liefert über die Fixpunkte $c_3$ und $\varphi_0$ konkrete Zahlen mit Beobachtungsbezug. Die Doppelabdeckung
schließt die letzte Selbstkonsistenzlücke in der Normalform für $\alpha$.

\appendix
\section*{Notation und feste Zahlen}
\begin{itemize}[leftmargin=2em]
\item $c_3=\tfrac{1}{8\pi}$,\quad $\varphi_{\text{tree}}=\tfrac{1}{6\pi}$,\quad $\delta_{\text{top}}=\tfrac{3}{256\pi^4}$,\quad
$\varphi_0=\varphi_{\text{tree}}+\delta_{\text{top}}$.
\item $A=2c_3^3=\tfrac{1}{256\pi^3}$,\quad $\kappa=\tfrac{b_1}{2\pi}\ln(1/\varphi_0)$,\quad $b_1=\tfrac{41}{10}$ (GUT Norm).
\end{itemize}

\section*{Literaturhinweise}
\begin{enumerate}[label={[L\arabic*]}]
\item Vereinheitlichte Feldgleichung, axion Photon Kopplung und Birefringenz: interne Notiz \emph{``On the Unified Field Equation as a Consequence of a Quantum Fixed-Point Condition''} (2025).
\item Geometrische Selbstkonsistenz auf der Doppelabdeckung und ppm-Shift in $\alpha$: interne Notiz \emph{``Geometric Self Consistency on the Orientable Double Cover''} (2025).
\item TFPT V1.06: Topologie $c_3$, Geometrie $\varphi_0$ und E8-Leiter in geschlossener Form (2025).
\end{enumerate}

\end{document}
