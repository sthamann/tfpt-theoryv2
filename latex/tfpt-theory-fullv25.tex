%!TEX program = pdflatex
\documentclass[11pt,a4paper]{article}

% ===== PACKAGES =====
\usepackage[T1]{fontenc}
\usepackage[utf8]{inputenc}
\usepackage{lmodern}
\usepackage[margin=1in]{geometry}
\usepackage{amsmath,amssymb,amsthm}
\usepackage{mathtools}
\usepackage{tikz}
\usepackage{pgfplots}
\pgfplotsset{compat=1.18}
\usepgfplotslibrary{groupplots}
\usetikzlibrary{arrows.meta,positioning,decorations.markings,decorations.pathmorphing,calc,patterns,shapes.geometric}

% Define photon line style for Feynman diagrams
\tikzset{
    photon/.style={decorate, decoration={snake, amplitude=1.5pt, segment length=5pt}}
}
\usepackage{booktabs}
\usepackage{longtable}
\usepackage{array}
\usepackage{tabularx}
\usepackage{graphicx}
\usepackage{xcolor}
\usepackage{float}
\usepackage[hidelinks]{hyperref}
\usepackage{xurl}
\usepackage{microtype}
\usepackage{enumitem}
\usepackage{pifont}
\usepackage{siunitx}
\sisetup{
    detect-all,
    round-mode = figures,
    round-precision = 8,
    exponent-mode = input,
    exponent-product = \times,
    group-separator = {\,},
    group-minimum-digits = 4
}
\renewcommand{\_}{\string_\allowbreak}
\DeclareRobustCommand{\codepath}[1]{\begingroup\Urlmuskip=0mu plus 1mu\relax\path{#1}\endgroup}
\usepackage{tcolorbox}
\tcbuselibrary{theorems,skins,breakable}

% ===== SUITE-EXTRACTED SNIPPETS (v2.5) =====
% Centralized numerics/tables/plots extracted from tfpt-suite outputs.
% Inlined (standalone build): formerly % TFPT v2.5 verification snippets
% Auto-extracted from tfpt-suite report (mp.dps=80)
% Usage: % TFPT v2.5 verification snippets
% Auto-extracted from tfpt-suite report (mp.dps=80)
% Usage: % TFPT v2.5 verification snippets
% Auto-extracted from tfpt-suite report (mp.dps=80)
% Usage: \input{tfpt-v25-snippets.tex} somewhere after packages (siunitx, booktabs, pgfplots, tikz, float, tcolorbox).

% --- Numerics (high precision) ---
\newcommand{\TFPTcThreeNum}{0.039788735772973833942220940843128590508614911435114112186916836014724199408556634}
\newcommand{\TFPTvarphiTreeNum}{0.053051647697298445256294587790838120678153215246818816249222448019632265878075512}
\newcommand{\TFPTdeltaTopNum}{0.00012030447954708205299788417891154719697454455297232339122853166568719135947912407}
\newcommand{\TFPTvarphiZeroNum}{0.053171952176845527309292471969749667875127759799791139640450979685319457237554636}
\newcommand{\TFPTbetaRadNum}{0.0042312895113954151087468182088050163950728077266837257698944229030526732721556263}
\newcommand{\TFPTbetaDegNum}{0.24243503090092952849243152047951445246859512239848881592836053961070607871616298}
\newcommand{\TFPTgAggNum}{-0.15915494309189533576888376337251436203445964574045644874766734405981889679763422654}
\newcommand{\TFPTMOverMplNum}{0.000012564942083228448969980786638197975376707921748468781310516248914585154899192637}

% alpha: baseline self-consistent and refined two-defect
\newcommand{\TFPTalphaInvSelfNum}{137.03599410158383008148619622913336009409257904171246722475352841304700901323215}
\newcommand{\TFPTalphaInvTwoDefNum}{137.0359981932435869110224146843726714996000114935956757208545827539450219077872}
\newcommand{\TFPTalphaInvGOverFourNum}{137.03599921615843479026171751711397398605568500258186146747743646328176776397918}
\newcommand{\TFPTalphaInvCODATANum}{137.035999177}
\newcommand{\TFPTalphaInvCODATASigma}{0.000000021}
\newcommand{\TFPTalphaPpmTwoDefNum}{-0.0071788173837323352413040312917307695904786165477907034361613291839588627323760663}
\newcommand{\TFPTalphaPpmGOverFourNum}{0.00028575290453192123198929586322751817361006829116847864863933342766974073215940791}
\newcommand{\TFPTalphaZGOverFourNum}{1.8646873709648437}
\newcommand{\TFPTdeltaTwoGOverFourNum}{0.000000018091459748867855014525576506215873427709357460729151458474190949106463016708746}
\newcommand{\TFPTdeltaTwoOptimalNum}{1.79529470934477993569067565574491365815939289267429e-8}
\newcommand{\TFPTdeltaTwoOverDeltaTopNum}{1.49229248636677043432223527146417707405844271475241e-4}
\newcommand{\TFPTkEffNum}{1.97984799374161166800213443338796550879953534211621}

% alpha MSbar at MZ (primary comparison observable; alpha_precision_audit)
\newcommand{\TFPTalphaBarFiveInvMZPredNum}{127.94051574588975237575597444353510021974456512094373031810385305362533848241133}
\newcommand{\TFPTalphaBarFiveInvMZRefNum}{127.93}
\newcommand{\TFPTalphaBarFiveInvMZRefSigma}{0.008}
\newcommand{\TFPTalphaBarFiveInvMZZ}{1.314468236219047}

% two-loop RG fingerprints
\newcommand{\TFPTalphaThreePeVNum}{0.05243714153896733}
\newcommand{\TFPTalphaThreePeVRelDevPercent}{1.3819515887516742}

\newcommand{\TFPTmuCrossVarphiZeroGeV}{792005.555441517}
\newcommand{\TFPTmuCrosscThreeGeV}{216030451.17273504}

% R^2 / Starobinsky numbers (effective_action_r2)
\newcommand{\TFPTeffRtwoCoeffDimless}{1055668997.2030925063285921958931697786482178917384505743565947290283792722955}
\newcommand{\TFPTeffRtwoCoeffInAction}{527834498.60154625316429609794658488932410894586922528717829736451418963614867773}
\newcommand{\TFPTalphaR}{117864.3622405491}

% A_s for N=55..57
% NOTE: TeX control sequence names cannot contain digits, so we spell out N values.
\newcommand{\TFPTAsNfiftyfive}{2.0162081878496864e-9}
\newcommand{\TFPTAsNfiftysix}{2.09019136432946e-9}
\newcommand{\TFPTAsNfiftyseven}{2.165507571016076e-9}

% --- Omega_b conjecture ---
\newcommand{\TFPTOmegaBPred}{0.048940662665450112200545653760944651480054952073107413870556556782266783965399009}
\newcommand{\TFPTOmegaBRef}{0.049301692328524438476424754994611856173233055557122787616860658651215012327847394}
\newcommand{\TFPTOmegaBRefSigma}{0.0008568114274714093}
\newcommand{\TFPTOmegaBZ}{0.42136420161876677}
\newcommand{\TFPTCoeffFourPiMinusOne}{11.566370614359172953850573533118011536788677597500423283899778369231265625144836}

% --- CKM pipeline (rg-dressed, MZ) ---
\newcommand{\TFPTCKMA}{0.8164965809277260}
\newcommand{\TFPTCKMrho}{0.1666666666666667}
\newcommand{\TFPTCKMeta}{0.3333333333333333}
\newcommand{\TFPTCKMchiTwo}{44.05176935606062}
\newcommand{\TFPTCKMJarlskog}{2.767084e-5}

% --- PMNS baseline + eps scan ---
\newcommand{\TFPTsinSqThetaThirteen}{0.023108435158881104}
\newcommand{\TFPTthetaThirteenDeg}{8.743693}
\newcommand{\TFPTthetaTwelveDeg}{34.322526}
\newcommand{\TFPTepsRad}{0.008861992029474254}
\newcommand{\TFPTepsDeg}{0.5077547413674498}

% =============================================================================
% Tables and plots
% =============================================================================

% --- Core invariants table ---
\newcommand{\TFPTCoreInvariantTable}{%
\begin{table}[H]
\centering
\begin{tabular}{@{}lll@{}}
\toprule
Symbol & Definition & Value \\
\midrule
$c_3$ & $1/(8\pi)$ & \num{\TFPTcThreeNum} \\
$\varphi_{\mathrm{tree}}$ & $1/(6\pi)$ & \num{\TFPTvarphiTreeNum} \\
$\delta_{\mathrm{top}}$ & $48c_3^4 = 3/(256\pi^4)$ & \num{\TFPTdeltaTopNum} \\
$\varphi_0$ & $\varphi_{\mathrm{tree}}+\delta_{\mathrm{top}}$ & \num{\TFPTvarphiZeroNum} \\
$\beta_{\mathrm{rad}}$ & $\varphi_0/(4\pi)$ & \num{\TFPTbetaRadNum} \\
$\beta_{\mathrm{deg}}$ & $(180/\pi)\beta_{\mathrm{rad}}$ & \num{\TFPTbetaDegNum}\,$^\circ$ \\
$g_{a\gamma\gamma}$ & $-4c_3=-1/(2\pi)$ & \num{\TFPTgAggNum} \\
$M/\bar M_{\mathrm{Pl}}$ & $\sqrt{8\pi}\,c_3^4$ & \num{\TFPTMOverMplNum} \\
\bottomrule
\end{tabular}
\caption{TFPT invariants and derived scales, mp.dps=80.}
\end{table}%
}

% --- Alpha audit table (k-sensitivity + delta2) ---
\newcommand{\TFPTAlphaAuditTable}{%
\begin{table}[H]
\centering
\begin{tabular}{@{}llll@{}}
\toprule
Case & $\alpha^{-1}_{\mathrm{pred}}$ & ppm vs CODATA 2022 & Note \\
\midrule
baseline (self-consistent) & \num{\TFPTalphaInvSelfNum} & \num{-0.037000} & single defect \\
two-defect (suite default) & \num{\TFPTalphaInvTwoDefNum} & \num{\TFPTalphaPpmTwoDefNum} & $\delta_2=\delta_{\mathrm{top}}^2$ \\
two-defect (theoryv3, $g=5$) & \num{\TFPTalphaInvGOverFourNum} & \num{\TFPTalphaPpmGOverFourNum} & $\delta_2=\frac54\,\delta_{\mathrm{top}}^2$ \\
targeted match & \num{\TFPTalphaInvCODATANum} & 0 & requires $\delta_2=\num{\TFPTdeltaTwoOptimalNum}$ \\
\bottomrule
\end{tabular}
\caption{Fine-structure constant: baseline and two-defect refinements (suite default vs.\ theoryv3), plus the required second-order term for exact CODATA matching.}
\end{table}%
}

% --- Alpha backreaction exponent sensitivity plot (ppm vs k) ---
\newcommand{\TFPTAlphaSensitivityPlot}{%
\begin{figure}[H]
\centering
\begin{tikzpicture}
\begin{axis}[
    width=0.9\linewidth,
    xlabel={$k$ (backreaction exponent) in $\varphi(\alpha)=\varphi_{\mathrm{tree}}+\delta_{\mathrm{top}}e^{-k\alpha}$},
    ylabel={ppm$(\alpha^{-1}(0))$ vs CODATA 2022},
    grid=both,
    xmin=-0.1, xmax=3.1,
    ymin=-2.2, ymax=4.2,
    xtick={0,1,1.5,2,2.5,3},
    legend pos=south west,
]
\addplot+[mark=*] coordinates {
    (0.0, 3.665371791638453)
    (1.0, 1.8074246412353618)
    (1.5, 0.883514212976573)
    (2.0, -0.03703710120260405)
    (2.5, -0.9542414840493093)
    (3.0, -1.8681110743677323)
};
\addlegendentry{single-defect ($\delta_2=0$)}

\addplot+[only marks, mark=star, mark size=4pt] coordinates {
    (2.0, \TFPTalphaPpmTwoDefNum)
};
\addlegendentry{two-defect at $k=2$ ($\delta_2=\delta_{\mathrm{top}}^2$)}

\addplot[gray, dashed] coordinates {(2.0,-2.2) (2.0,4.2)};
\addlegendentry{$k=2$ (double cover)}

\addplot[gray, dotted] coordinates {(\TFPTkEffNum,-2.2) (\TFPTkEffNum,4.2)};
\addlegendentry{$k_{\mathrm{match}}\approx\num{\TFPTkEffNum}$ (diagnostic)}

\addplot[black!50, dashed] coordinates {(-0.1,0) (3.1,0)};
\end{axis}
\end{tikzpicture}
\caption{Backreaction exponent sensitivity for the CFE closure. The canonical value $k=2$ is fixed by the orientable double cover; $k_{\mathrm{match}}$ is the diagnostic value that would null the CODATA residual in the single-defect truncation.}
\end{figure}%
}

% --- Bounce transfer function data plot (from bounce_perturbations) ---
\newcommand{\TFPTTransferFunctionPlot}{%
\begin{figure}[H]
\centering
\begin{tikzpicture}
\begin{axis}[
    width=0.9\linewidth,
    xmode=log, ymode=log,
    xlabel={$\hat k$},
    ylabel={$T(\hat k)$},
    grid=both,
    legend pos=south east
]
\addplot+[mark=*] coordinates {
    (0.1,    0.0032255772038632126)
    (0.15199110829529336, 0.0035703003593638466)
    (0.23101297000831597, 0.00000000399020695466708)
    (0.35111917342151316, 0.000013156269894230775)
    (0.533669923120631,   0.03550967159062275)
    (0.8111308307896873,  1.6127872665381955)
    (1.2328467394420662,  1.0077896948137592)
    (1.873817422860383,   0.9810730551587763)
    (2.848035868435802,   1.004402539575707)
    (4.328761281083057,   1.0055829625872563)
    (6.5793322465756825,  1.006832617067307)
    (10.0,                1.007882155814805)
};
\addlegendentry{scalar}
\addplot+[mark=square*] coordinates {
    (0.1,    147270.05495615443)
    (0.15199110829529336, 53875.473521817745)
    (0.23101297000831597, 2.1302554245702106)
    (0.35111917342151316, 1.1806392634784857)
    (0.533669923120631,   1.1598025598647947)
    (0.8111308307896873,  1.0319915300605456)
    (1.2328467394420662,  1.035501225860808)
    (1.873817422860383,   1.008365931672867)
    (2.848035868435802,   0.9883585969391012)
    (4.328761281083057,   1.007608072068884)
    (6.5793322465756825,  1.0065676931381393)
    (10.0,                1.0018456357164773)
};
\addlegendentry{tensor}
\end{axis}
\end{tikzpicture}
\caption{Bounce transfer functions from \texttt{bounce\_perturbations}. Note: the two lowest $\hat k$ scalar points were flagged by Wronskian diagnostics in the test report.}
\end{figure}%
}

% --- CKM matrix (absolute values at MZ) ---
\newcommand{\TFPTCKMMatrixBlock}{%
\[
|V_{\mathrm{CKM}}|_{M_Z} \approx
\begin{pmatrix}
0.9744775404 & 0.2244586415 & 0.003441159690 \\
0.2243316946 & 0.9736442171 & 0.04113671446 \\
0.008295429526 & 0.04043830740 & 0.9991476013
\end{pmatrix}.
\]
}

% --- Yukawa textures (UV, used in ckm_full_pipeline) ---
\newcommand{\TFPTYukawaTextureBlock}{%
\begin{tcolorbox}[colback=gray!5, colframe=gray!60, title={\textbf{UV Yukawa texture used in the CKM pipeline}}, fonttitle=\bfseries\small]
The baseline verification pipeline uses a minimal hierarchical UV texture in powers of the TFPT Cabibbo parameter $\lambda$:
\[
\mathrm{diag}(Y_u^{UV})=(\lambda^8,\lambda^4,1),\qquad \mathrm{diag}(Y_d^{UV})=(\lambda^4,\lambda^2,\lambda^2)
\]
Numerically (using the report values $\lambda=\num{0.2244599705189737}$), this is
\[
Y_u^{UV}=\begin{pmatrix}
\num{6.4433424282582535e-06} & 0 & 0\\
0 & \num{2.5383739732864922e-03} & 0\\
0 & 0 & \num{1}
\end{pmatrix},\qquad
|Y_d^{UV}|=\begin{pmatrix}
\num{2.47358843e-03} & \num{1.13087378e-02} & \num{1.73373465e-04}\\
\num{5.69437735e-04} & \num{4.90544140e-02} & \num{2.07256140e-03}\\
\num{2.10569024e-05} & \num{2.03737406e-03} & \num{5.03393326e-02}
\end{pmatrix}.
\]
The down type texture is left rotated by a CKM proxy at the UV scale: $Y_d^{UV}=V_{UV}\,\mathrm{diag}(Y_d^{UV})$ (phase carried by $V_{UV}$).
\end{tcolorbox}%
}

% --- Yukawa texture heatmap (log10 magnitudes) ---
\newcommand{\TFPTYukawaHeatmapPlot}{%
\begin{figure}[H]
\centering
\begin{minipage}{0.48\linewidth}
\centering
\begin{tikzpicture}
\begin{axis}[
    title={$\log_{10}|Y_u^{UV}|$},
    width=\linewidth,
    height=0.9\linewidth,
    xmin=0.5, xmax=3.5,
    ymin=0.5, ymax=3.5,
    xtick={1,2,3}, ytick={1,2,3},
    y dir=reverse,
    axis on top,
    enlargelimits=false,
    colormap/hot,
    point meta min=-6, point meta max=0,
    tick label style={font=\scriptsize},
    title style={font=\small},
    colorbar=false,
]
\addplot[
    matrix plot*,
    point meta=explicit,
] coordinates {
    (1,1) [-5.1908888] (2,1) [-6]         (3,1) [-6]
    (1,2) [-6]         (2,2) [-2.5954444] (3,2) [-6]
    (1,3) [-6]         (2,3) [-6]         (3,3) [0]
};
\end{axis}
\end{tikzpicture}
\end{minipage}\hfill
\begin{minipage}{0.48\linewidth}
\centering
\begin{tikzpicture}
\begin{axis}[
    title={$\log_{10}|Y_d^{UV}|$},
    width=\linewidth,
    height=0.9\linewidth,
    xmin=0.5, xmax=3.5,
    ymin=0.5, ymax=3.5,
    xtick={1,2,3}, ytick={1,2,3},
    y dir=reverse,
    axis on top,
    enlargelimits=false,
    colormap/hot,
    point meta min=-6, point meta max=0,
    tick label style={font=\scriptsize},
    title style={font=\small},
    colorbar,
    colorbar style={ylabel={$\log_{10}|Y|$}, yticklabel style={font=\scriptsize}, ylabel style={font=\scriptsize}},
]
\addplot[
    matrix plot*,
    point meta=explicit,
] coordinates {
    (1,1) [-2.6066726] (2,1) [-1.9465859] (3,1) [-3.7610174]
    (1,2) [-3.2445538] (2,2) [-1.3093219] (3,2) [-2.6834926]
    (1,3) [-4.6766055] (2,3) [-2.6909292] (3,3) [-1.2980925]
};
\end{axis}
\end{tikzpicture}
\end{minipage}
\caption{Heatmap of $\log_{10}|Y_u^{UV}|$ and $\log_{10}|Y_d^{UV}|$ used by the \texttt{ckm\_full\_pipeline} benchmark.}
\end{figure}%
}

% --- Z3 permutation matrix P (PMNS module) ---
\newcommand{\TFPTZthreePermutationMatrix}{%
\[
P =
\begin{pmatrix}
0 & 1 & 0\\
0 & 0 & 1\\
1 & 0 & 0
\end{pmatrix}.
\]
}

% --- PMNS variant table ---
\newcommand{\TFPTPMNSVariantTable}{%
\begin{table}[H]
\centering
\begin{tabular}{@{}lcccc@{}}
\toprule
Variant & $\theta_{23}$ [deg] & $\Delta\theta_{23}$ [deg] & $\delta_{\mathrm{CP}}$ [deg] & $\Delta\delta$ [deg]\\
\midrule
baseline (TM1, Z3) & 45.0000 & 0.0000 & 90.0000 & 0.0000\\
L\_R23(+eps) & 45.5078 & +0.5078 & 90.0000 & 0.0000\\
L\_R23(-eps) & 44.4922 & -0.5078 & 90.0000 & 0.0000\\
R\_R23(+eps) & 45.4243 & +0.4243 & 88.1896 & -1.8104\\
R\_R23(-eps) & 44.5757 & -0.4243 & 91.8104 & +1.8104\\
L\_R23(+eps)*Z3phase & 45.5078 & +0.5078 & 90.0000 & 0.0000\\
L\_R23(-eps)*Z3phase & 44.4922 & -0.5078 & 90.0000 & 0.0000\\
\bottomrule
\end{tabular}
\caption{PMNS Z3-breaking scan with $\epsilon=\varphi_0/6=\num{\TFPTepsRad}$ rad.}
\end{table}%
}

% --- Omega_b table ---
\newcommand{\TFPTOmegaBTable}{%
\begin{table}[H]
\centering
\begin{tabular}{@{}lll@{}}
\toprule
Quantity & Expression & Value \\
\midrule
$\Omega_b$ (TFPT conjecture) & $(4\pi-1)\,\beta_{\mathrm{rad}}$ & \num{\TFPTOmegaBPred} \\
$\Omega_b$ (Planck-derived ref) & $\Omega_b h^2 / h^2$ & \num{\TFPTOmegaBRef} \,$\pm$\,\num{\TFPTOmegaBRefSigma} \\
z-score (ref) & $(\Omega_{b,\mathrm{pred}}-\Omega_{b,\mathrm{ref}})/\sigma$ & \num{\TFPTOmegaBZ} \\
\bottomrule
\end{tabular}
\caption{Baryon density conjecture: numeric prediction and reference comparison.}
\end{table}%
}

% --- k-calibration table (ell mapping) ---
\newcommand{\TFPTkCalibrationTable}{%
\begin{table}[H]
\centering
\begin{tabular}{@{}lccc@{}}
\toprule
Target $\ell$ & required $a_0/a_t$ (scalar) & required $a_0/a_t$ (tensor) & $N=\ln(a_0/a_t)$ (scalar) \\
\midrule
2   & $6.978\times 10^{58}$ & $2.590\times 10^{56}$ & 135.493 \\
30  & $4.652\times 10^{57}$ & $1.727\times 10^{55}$ & 132.785 \\
700 & $1.994\times 10^{56}$ & $7.400\times 10^{53}$ & 129.635 \\
\bottomrule
\end{tabular}
\caption{Mapping the bounce scale to CMB multipoles: required overall scaling $a_0/a_t$ to place features at given $\ell$ (from \texttt{k\_calibration}).}
\end{table}%
}

% --- Torsion bounds table ---
\newcommand{\TFPTTorsionBoundsTable}{%
\begin{table}[H]
\centering
\begin{tabular}{@{}lcc@{}}
\toprule
Component & bound on $|S_\mu|$ [GeV] & mapping note \\
\midrule
$S_T$ & $2.9\times 10^{-27}$ & direct $A_\mu$ bound \\
$S_X$ & $2.1\times 10^{-31}$ & direct $A_\mu$ bound \\
$S_Y$ & $2.5\times 10^{-31}$ & direct $A_\mu$ bound \\
$S_Z$ & $1.0\times 10^{-29}$ & direct $A_\mu$ bound \\
\bottomrule
\end{tabular}
\caption{Component-wise torsion bounds ingested by \texttt{torsion\_bounds\_mapping}; TFPT local vacuum prediction used in the suite is $S_\mu(\mathrm{today})=0$.}
\end{table}%
}
 somewhere after packages (siunitx, booktabs, pgfplots, tikz, float, tcolorbox).

% --- Numerics (high precision) ---
\newcommand{\TFPTcThreeNum}{0.039788735772973833942220940843128590508614911435114112186916836014724199408556634}
\newcommand{\TFPTvarphiTreeNum}{0.053051647697298445256294587790838120678153215246818816249222448019632265878075512}
\newcommand{\TFPTdeltaTopNum}{0.00012030447954708205299788417891154719697454455297232339122853166568719135947912407}
\newcommand{\TFPTvarphiZeroNum}{0.053171952176845527309292471969749667875127759799791139640450979685319457237554636}
\newcommand{\TFPTbetaRadNum}{0.0042312895113954151087468182088050163950728077266837257698944229030526732721556263}
\newcommand{\TFPTbetaDegNum}{0.24243503090092952849243152047951445246859512239848881592836053961070607871616298}
\newcommand{\TFPTgAggNum}{-0.15915494309189533576888376337251436203445964574045644874766734405981889679763422654}
\newcommand{\TFPTMOverMplNum}{0.000012564942083228448969980786638197975376707921748468781310516248914585154899192637}

% alpha: baseline self-consistent and refined two-defect
\newcommand{\TFPTalphaInvSelfNum}{137.03599410158383008148619622913336009409257904171246722475352841304700901323215}
\newcommand{\TFPTalphaInvTwoDefNum}{137.0359981932435869110224146843726714996000114935956757208545827539450219077872}
\newcommand{\TFPTalphaInvGOverFourNum}{137.03599921615843479026171751711397398605568500258186146747743646328176776397918}
\newcommand{\TFPTalphaInvCODATANum}{137.035999177}
\newcommand{\TFPTalphaInvCODATASigma}{0.000000021}
\newcommand{\TFPTalphaPpmTwoDefNum}{-0.0071788173837323352413040312917307695904786165477907034361613291839588627323760663}
\newcommand{\TFPTalphaPpmGOverFourNum}{0.00028575290453192123198929586322751817361006829116847864863933342766974073215940791}
\newcommand{\TFPTalphaZGOverFourNum}{1.8646873709648437}
\newcommand{\TFPTdeltaTwoGOverFourNum}{0.000000018091459748867855014525576506215873427709357460729151458474190949106463016708746}
\newcommand{\TFPTdeltaTwoOptimalNum}{1.79529470934477993569067565574491365815939289267429e-8}
\newcommand{\TFPTdeltaTwoOverDeltaTopNum}{1.49229248636677043432223527146417707405844271475241e-4}
\newcommand{\TFPTkEffNum}{1.97984799374161166800213443338796550879953534211621}

% alpha MSbar at MZ (primary comparison observable; alpha_precision_audit)
\newcommand{\TFPTalphaBarFiveInvMZPredNum}{127.94051574588975237575597444353510021974456512094373031810385305362533848241133}
\newcommand{\TFPTalphaBarFiveInvMZRefNum}{127.93}
\newcommand{\TFPTalphaBarFiveInvMZRefSigma}{0.008}
\newcommand{\TFPTalphaBarFiveInvMZZ}{1.314468236219047}

% two-loop RG fingerprints
\newcommand{\TFPTalphaThreePeVNum}{0.05243714153896733}
\newcommand{\TFPTalphaThreePeVRelDevPercent}{1.3819515887516742}

\newcommand{\TFPTmuCrossVarphiZeroGeV}{792005.555441517}
\newcommand{\TFPTmuCrosscThreeGeV}{216030451.17273504}

% R^2 / Starobinsky numbers (effective_action_r2)
\newcommand{\TFPTeffRtwoCoeffDimless}{1055668997.2030925063285921958931697786482178917384505743565947290283792722955}
\newcommand{\TFPTeffRtwoCoeffInAction}{527834498.60154625316429609794658488932410894586922528717829736451418963614867773}
\newcommand{\TFPTalphaR}{117864.3622405491}

% A_s for N=55..57
% NOTE: TeX control sequence names cannot contain digits, so we spell out N values.
\newcommand{\TFPTAsNfiftyfive}{2.0162081878496864e-9}
\newcommand{\TFPTAsNfiftysix}{2.09019136432946e-9}
\newcommand{\TFPTAsNfiftyseven}{2.165507571016076e-9}

% --- Omega_b conjecture ---
\newcommand{\TFPTOmegaBPred}{0.048940662665450112200545653760944651480054952073107413870556556782266783965399009}
\newcommand{\TFPTOmegaBRef}{0.049301692328524438476424754994611856173233055557122787616860658651215012327847394}
\newcommand{\TFPTOmegaBRefSigma}{0.0008568114274714093}
\newcommand{\TFPTOmegaBZ}{0.42136420161876677}
\newcommand{\TFPTCoeffFourPiMinusOne}{11.566370614359172953850573533118011536788677597500423283899778369231265625144836}

% --- CKM pipeline (rg-dressed, MZ) ---
\newcommand{\TFPTCKMA}{0.8164965809277260}
\newcommand{\TFPTCKMrho}{0.1666666666666667}
\newcommand{\TFPTCKMeta}{0.3333333333333333}
\newcommand{\TFPTCKMchiTwo}{44.05176935606062}
\newcommand{\TFPTCKMJarlskog}{2.767084e-5}

% --- PMNS baseline + eps scan ---
\newcommand{\TFPTsinSqThetaThirteen}{0.023108435158881104}
\newcommand{\TFPTthetaThirteenDeg}{8.743693}
\newcommand{\TFPTthetaTwelveDeg}{34.322526}
\newcommand{\TFPTepsRad}{0.008861992029474254}
\newcommand{\TFPTepsDeg}{0.5077547413674498}

% =============================================================================
% Tables and plots
% =============================================================================

% --- Core invariants table ---
\newcommand{\TFPTCoreInvariantTable}{%
\begin{table}[H]
\centering
\begin{tabular}{@{}lll@{}}
\toprule
Symbol & Definition & Value \\
\midrule
$c_3$ & $1/(8\pi)$ & \num{\TFPTcThreeNum} \\
$\varphi_{\mathrm{tree}}$ & $1/(6\pi)$ & \num{\TFPTvarphiTreeNum} \\
$\delta_{\mathrm{top}}$ & $48c_3^4 = 3/(256\pi^4)$ & \num{\TFPTdeltaTopNum} \\
$\varphi_0$ & $\varphi_{\mathrm{tree}}+\delta_{\mathrm{top}}$ & \num{\TFPTvarphiZeroNum} \\
$\beta_{\mathrm{rad}}$ & $\varphi_0/(4\pi)$ & \num{\TFPTbetaRadNum} \\
$\beta_{\mathrm{deg}}$ & $(180/\pi)\beta_{\mathrm{rad}}$ & \num{\TFPTbetaDegNum}\,$^\circ$ \\
$g_{a\gamma\gamma}$ & $-4c_3=-1/(2\pi)$ & \num{\TFPTgAggNum} \\
$M/\bar M_{\mathrm{Pl}}$ & $\sqrt{8\pi}\,c_3^4$ & \num{\TFPTMOverMplNum} \\
\bottomrule
\end{tabular}
\caption{TFPT invariants and derived scales, mp.dps=80.}
\end{table}%
}

% --- Alpha audit table (k-sensitivity + delta2) ---
\newcommand{\TFPTAlphaAuditTable}{%
\begin{table}[H]
\centering
\begin{tabular}{@{}llll@{}}
\toprule
Case & $\alpha^{-1}_{\mathrm{pred}}$ & ppm vs CODATA 2022 & Note \\
\midrule
baseline (self-consistent) & \num{\TFPTalphaInvSelfNum} & \num{-0.037000} & single defect \\
two-defect (suite default) & \num{\TFPTalphaInvTwoDefNum} & \num{\TFPTalphaPpmTwoDefNum} & $\delta_2=\delta_{\mathrm{top}}^2$ \\
two-defect (theoryv3, $g=5$) & \num{\TFPTalphaInvGOverFourNum} & \num{\TFPTalphaPpmGOverFourNum} & $\delta_2=\frac54\,\delta_{\mathrm{top}}^2$ \\
targeted match & \num{\TFPTalphaInvCODATANum} & 0 & requires $\delta_2=\num{\TFPTdeltaTwoOptimalNum}$ \\
\bottomrule
\end{tabular}
\caption{Fine-structure constant: baseline and two-defect refinements (suite default vs.\ theoryv3), plus the required second-order term for exact CODATA matching.}
\end{table}%
}

% --- Alpha backreaction exponent sensitivity plot (ppm vs k) ---
\newcommand{\TFPTAlphaSensitivityPlot}{%
\begin{figure}[H]
\centering
\begin{tikzpicture}
\begin{axis}[
    width=0.9\linewidth,
    xlabel={$k$ (backreaction exponent) in $\varphi(\alpha)=\varphi_{\mathrm{tree}}+\delta_{\mathrm{top}}e^{-k\alpha}$},
    ylabel={ppm$(\alpha^{-1}(0))$ vs CODATA 2022},
    grid=both,
    xmin=-0.1, xmax=3.1,
    ymin=-2.2, ymax=4.2,
    xtick={0,1,1.5,2,2.5,3},
    legend pos=south west,
]
\addplot+[mark=*] coordinates {
    (0.0, 3.665371791638453)
    (1.0, 1.8074246412353618)
    (1.5, 0.883514212976573)
    (2.0, -0.03703710120260405)
    (2.5, -0.9542414840493093)
    (3.0, -1.8681110743677323)
};
\addlegendentry{single-defect ($\delta_2=0$)}

\addplot+[only marks, mark=star, mark size=4pt] coordinates {
    (2.0, \TFPTalphaPpmTwoDefNum)
};
\addlegendentry{two-defect at $k=2$ ($\delta_2=\delta_{\mathrm{top}}^2$)}

\addplot[gray, dashed] coordinates {(2.0,-2.2) (2.0,4.2)};
\addlegendentry{$k=2$ (double cover)}

\addplot[gray, dotted] coordinates {(\TFPTkEffNum,-2.2) (\TFPTkEffNum,4.2)};
\addlegendentry{$k_{\mathrm{match}}\approx\num{\TFPTkEffNum}$ (diagnostic)}

\addplot[black!50, dashed] coordinates {(-0.1,0) (3.1,0)};
\end{axis}
\end{tikzpicture}
\caption{Backreaction exponent sensitivity for the CFE closure. The canonical value $k=2$ is fixed by the orientable double cover; $k_{\mathrm{match}}$ is the diagnostic value that would null the CODATA residual in the single-defect truncation.}
\end{figure}%
}

% --- Bounce transfer function data plot (from bounce_perturbations) ---
\newcommand{\TFPTTransferFunctionPlot}{%
\begin{figure}[H]
\centering
\begin{tikzpicture}
\begin{axis}[
    width=0.9\linewidth,
    xmode=log, ymode=log,
    xlabel={$\hat k$},
    ylabel={$T(\hat k)$},
    grid=both,
    legend pos=south east
]
\addplot+[mark=*] coordinates {
    (0.1,    0.0032255772038632126)
    (0.15199110829529336, 0.0035703003593638466)
    (0.23101297000831597, 0.00000000399020695466708)
    (0.35111917342151316, 0.000013156269894230775)
    (0.533669923120631,   0.03550967159062275)
    (0.8111308307896873,  1.6127872665381955)
    (1.2328467394420662,  1.0077896948137592)
    (1.873817422860383,   0.9810730551587763)
    (2.848035868435802,   1.004402539575707)
    (4.328761281083057,   1.0055829625872563)
    (6.5793322465756825,  1.006832617067307)
    (10.0,                1.007882155814805)
};
\addlegendentry{scalar}
\addplot+[mark=square*] coordinates {
    (0.1,    147270.05495615443)
    (0.15199110829529336, 53875.473521817745)
    (0.23101297000831597, 2.1302554245702106)
    (0.35111917342151316, 1.1806392634784857)
    (0.533669923120631,   1.1598025598647947)
    (0.8111308307896873,  1.0319915300605456)
    (1.2328467394420662,  1.035501225860808)
    (1.873817422860383,   1.008365931672867)
    (2.848035868435802,   0.9883585969391012)
    (4.328761281083057,   1.007608072068884)
    (6.5793322465756825,  1.0065676931381393)
    (10.0,                1.0018456357164773)
};
\addlegendentry{tensor}
\end{axis}
\end{tikzpicture}
\caption{Bounce transfer functions from \texttt{bounce\_perturbations}. Note: the two lowest $\hat k$ scalar points were flagged by Wronskian diagnostics in the test report.}
\end{figure}%
}

% --- CKM matrix (absolute values at MZ) ---
\newcommand{\TFPTCKMMatrixBlock}{%
\[
|V_{\mathrm{CKM}}|_{M_Z} \approx
\begin{pmatrix}
0.9744775404 & 0.2244586415 & 0.003441159690 \\
0.2243316946 & 0.9736442171 & 0.04113671446 \\
0.008295429526 & 0.04043830740 & 0.9991476013
\end{pmatrix}.
\]
}

% --- Yukawa textures (UV, used in ckm_full_pipeline) ---
\newcommand{\TFPTYukawaTextureBlock}{%
\begin{tcolorbox}[colback=gray!5, colframe=gray!60, title={\textbf{UV Yukawa texture used in the CKM pipeline}}, fonttitle=\bfseries\small]
The baseline verification pipeline uses a minimal hierarchical UV texture in powers of the TFPT Cabibbo parameter $\lambda$:
\[
\mathrm{diag}(Y_u^{UV})=(\lambda^8,\lambda^4,1),\qquad \mathrm{diag}(Y_d^{UV})=(\lambda^4,\lambda^2,\lambda^2)
\]
Numerically (using the report values $\lambda=\num{0.2244599705189737}$), this is
\[
Y_u^{UV}=\begin{pmatrix}
\num{6.4433424282582535e-06} & 0 & 0\\
0 & \num{2.5383739732864922e-03} & 0\\
0 & 0 & \num{1}
\end{pmatrix},\qquad
|Y_d^{UV}|=\begin{pmatrix}
\num{2.47358843e-03} & \num{1.13087378e-02} & \num{1.73373465e-04}\\
\num{5.69437735e-04} & \num{4.90544140e-02} & \num{2.07256140e-03}\\
\num{2.10569024e-05} & \num{2.03737406e-03} & \num{5.03393326e-02}
\end{pmatrix}.
\]
The down type texture is left rotated by a CKM proxy at the UV scale: $Y_d^{UV}=V_{UV}\,\mathrm{diag}(Y_d^{UV})$ (phase carried by $V_{UV}$).
\end{tcolorbox}%
}

% --- Yukawa texture heatmap (log10 magnitudes) ---
\newcommand{\TFPTYukawaHeatmapPlot}{%
\begin{figure}[H]
\centering
\begin{minipage}{0.48\linewidth}
\centering
\begin{tikzpicture}
\begin{axis}[
    title={$\log_{10}|Y_u^{UV}|$},
    width=\linewidth,
    height=0.9\linewidth,
    xmin=0.5, xmax=3.5,
    ymin=0.5, ymax=3.5,
    xtick={1,2,3}, ytick={1,2,3},
    y dir=reverse,
    axis on top,
    enlargelimits=false,
    colormap/hot,
    point meta min=-6, point meta max=0,
    tick label style={font=\scriptsize},
    title style={font=\small},
    colorbar=false,
]
\addplot[
    matrix plot*,
    point meta=explicit,
] coordinates {
    (1,1) [-5.1908888] (2,1) [-6]         (3,1) [-6]
    (1,2) [-6]         (2,2) [-2.5954444] (3,2) [-6]
    (1,3) [-6]         (2,3) [-6]         (3,3) [0]
};
\end{axis}
\end{tikzpicture}
\end{minipage}\hfill
\begin{minipage}{0.48\linewidth}
\centering
\begin{tikzpicture}
\begin{axis}[
    title={$\log_{10}|Y_d^{UV}|$},
    width=\linewidth,
    height=0.9\linewidth,
    xmin=0.5, xmax=3.5,
    ymin=0.5, ymax=3.5,
    xtick={1,2,3}, ytick={1,2,3},
    y dir=reverse,
    axis on top,
    enlargelimits=false,
    colormap/hot,
    point meta min=-6, point meta max=0,
    tick label style={font=\scriptsize},
    title style={font=\small},
    colorbar,
    colorbar style={ylabel={$\log_{10}|Y|$}, yticklabel style={font=\scriptsize}, ylabel style={font=\scriptsize}},
]
\addplot[
    matrix plot*,
    point meta=explicit,
] coordinates {
    (1,1) [-2.6066726] (2,1) [-1.9465859] (3,1) [-3.7610174]
    (1,2) [-3.2445538] (2,2) [-1.3093219] (3,2) [-2.6834926]
    (1,3) [-4.6766055] (2,3) [-2.6909292] (3,3) [-1.2980925]
};
\end{axis}
\end{tikzpicture}
\end{minipage}
\caption{Heatmap of $\log_{10}|Y_u^{UV}|$ and $\log_{10}|Y_d^{UV}|$ used by the \texttt{ckm\_full\_pipeline} benchmark.}
\end{figure}%
}

% --- Z3 permutation matrix P (PMNS module) ---
\newcommand{\TFPTZthreePermutationMatrix}{%
\[
P =
\begin{pmatrix}
0 & 1 & 0\\
0 & 0 & 1\\
1 & 0 & 0
\end{pmatrix}.
\]
}

% --- PMNS variant table ---
\newcommand{\TFPTPMNSVariantTable}{%
\begin{table}[H]
\centering
\begin{tabular}{@{}lcccc@{}}
\toprule
Variant & $\theta_{23}$ [deg] & $\Delta\theta_{23}$ [deg] & $\delta_{\mathrm{CP}}$ [deg] & $\Delta\delta$ [deg]\\
\midrule
baseline (TM1, Z3) & 45.0000 & 0.0000 & 90.0000 & 0.0000\\
L\_R23(+eps) & 45.5078 & +0.5078 & 90.0000 & 0.0000\\
L\_R23(-eps) & 44.4922 & -0.5078 & 90.0000 & 0.0000\\
R\_R23(+eps) & 45.4243 & +0.4243 & 88.1896 & -1.8104\\
R\_R23(-eps) & 44.5757 & -0.4243 & 91.8104 & +1.8104\\
L\_R23(+eps)*Z3phase & 45.5078 & +0.5078 & 90.0000 & 0.0000\\
L\_R23(-eps)*Z3phase & 44.4922 & -0.5078 & 90.0000 & 0.0000\\
\bottomrule
\end{tabular}
\caption{PMNS Z3-breaking scan with $\epsilon=\varphi_0/6=\num{\TFPTepsRad}$ rad.}
\end{table}%
}

% --- Omega_b table ---
\newcommand{\TFPTOmegaBTable}{%
\begin{table}[H]
\centering
\begin{tabular}{@{}lll@{}}
\toprule
Quantity & Expression & Value \\
\midrule
$\Omega_b$ (TFPT conjecture) & $(4\pi-1)\,\beta_{\mathrm{rad}}$ & \num{\TFPTOmegaBPred} \\
$\Omega_b$ (Planck-derived ref) & $\Omega_b h^2 / h^2$ & \num{\TFPTOmegaBRef} \,$\pm$\,\num{\TFPTOmegaBRefSigma} \\
z-score (ref) & $(\Omega_{b,\mathrm{pred}}-\Omega_{b,\mathrm{ref}})/\sigma$ & \num{\TFPTOmegaBZ} \\
\bottomrule
\end{tabular}
\caption{Baryon density conjecture: numeric prediction and reference comparison.}
\end{table}%
}

% --- k-calibration table (ell mapping) ---
\newcommand{\TFPTkCalibrationTable}{%
\begin{table}[H]
\centering
\begin{tabular}{@{}lccc@{}}
\toprule
Target $\ell$ & required $a_0/a_t$ (scalar) & required $a_0/a_t$ (tensor) & $N=\ln(a_0/a_t)$ (scalar) \\
\midrule
2   & $6.978\times 10^{58}$ & $2.590\times 10^{56}$ & 135.493 \\
30  & $4.652\times 10^{57}$ & $1.727\times 10^{55}$ & 132.785 \\
700 & $1.994\times 10^{56}$ & $7.400\times 10^{53}$ & 129.635 \\
\bottomrule
\end{tabular}
\caption{Mapping the bounce scale to CMB multipoles: required overall scaling $a_0/a_t$ to place features at given $\ell$ (from \texttt{k\_calibration}).}
\end{table}%
}

% --- Torsion bounds table ---
\newcommand{\TFPTTorsionBoundsTable}{%
\begin{table}[H]
\centering
\begin{tabular}{@{}lcc@{}}
\toprule
Component & bound on $|S_\mu|$ [GeV] & mapping note \\
\midrule
$S_T$ & $2.9\times 10^{-27}$ & direct $A_\mu$ bound \\
$S_X$ & $2.1\times 10^{-31}$ & direct $A_\mu$ bound \\
$S_Y$ & $2.5\times 10^{-31}$ & direct $A_\mu$ bound \\
$S_Z$ & $1.0\times 10^{-29}$ & direct $A_\mu$ bound \\
\bottomrule
\end{tabular}
\caption{Component-wise torsion bounds ingested by \texttt{torsion\_bounds\_mapping}; TFPT local vacuum prediction used in the suite is $S_\mu(\mathrm{today})=0$.}
\end{table}%
}
 somewhere after packages (siunitx, booktabs, pgfplots, tikz, float, tcolorbox).

% --- Numerics (high precision) ---
\newcommand{\TFPTcThreeNum}{0.039788735772973833942220940843128590508614911435114112186916836014724199408556634}
\newcommand{\TFPTvarphiTreeNum}{0.053051647697298445256294587790838120678153215246818816249222448019632265878075512}
\newcommand{\TFPTdeltaTopNum}{0.00012030447954708205299788417891154719697454455297232339122853166568719135947912407}
\newcommand{\TFPTvarphiZeroNum}{0.053171952176845527309292471969749667875127759799791139640450979685319457237554636}
\newcommand{\TFPTbetaRadNum}{0.0042312895113954151087468182088050163950728077266837257698944229030526732721556263}
\newcommand{\TFPTbetaDegNum}{0.24243503090092952849243152047951445246859512239848881592836053961070607871616298}
\newcommand{\TFPTgAggNum}{-0.15915494309189533576888376337251436203445964574045644874766734405981889679763422654}
\newcommand{\TFPTMOverMplNum}{0.000012564942083228448969980786638197975376707921748468781310516248914585154899192637}

% alpha: baseline self-consistent and refined two-defect
\newcommand{\TFPTalphaInvSelfNum}{137.03599410158383008148619622913336009409257904171246722475352841304700901323215}
\newcommand{\TFPTalphaInvTwoDefNum}{137.0359981932435869110224146843726714996000114935956757208545827539450219077872}
\newcommand{\TFPTalphaInvGOverFourNum}{137.03599921615843479026171751711397398605568500258186146747743646328176776397918}
\newcommand{\TFPTalphaInvCODATANum}{137.035999177}
\newcommand{\TFPTalphaInvCODATASigma}{0.000000021}
\newcommand{\TFPTalphaPpmTwoDefNum}{-0.0071788173837323352413040312917307695904786165477907034361613291839588627323760663}
\newcommand{\TFPTalphaPpmGOverFourNum}{0.00028575290453192123198929586322751817361006829116847864863933342766974073215940791}
\newcommand{\TFPTalphaZGOverFourNum}{1.8646873709648437}
\newcommand{\TFPTdeltaTwoGOverFourNum}{0.000000018091459748867855014525576506215873427709357460729151458474190949106463016708746}
\newcommand{\TFPTdeltaTwoOptimalNum}{1.79529470934477993569067565574491365815939289267429e-8}
\newcommand{\TFPTdeltaTwoOverDeltaTopNum}{1.49229248636677043432223527146417707405844271475241e-4}
\newcommand{\TFPTkEffNum}{1.97984799374161166800213443338796550879953534211621}

% alpha MSbar at MZ (primary comparison observable; alpha_precision_audit)
\newcommand{\TFPTalphaBarFiveInvMZPredNum}{127.94051574588975237575597444353510021974456512094373031810385305362533848241133}
\newcommand{\TFPTalphaBarFiveInvMZRefNum}{127.93}
\newcommand{\TFPTalphaBarFiveInvMZRefSigma}{0.008}
\newcommand{\TFPTalphaBarFiveInvMZZ}{1.314468236219047}

% two-loop RG fingerprints
\newcommand{\TFPTalphaThreePeVNum}{0.05243714153896733}
\newcommand{\TFPTalphaThreePeVRelDevPercent}{1.3819515887516742}

\newcommand{\TFPTmuCrossVarphiZeroGeV}{792005.555441517}
\newcommand{\TFPTmuCrosscThreeGeV}{216030451.17273504}

% R^2 / Starobinsky numbers (effective_action_r2)
\newcommand{\TFPTeffRtwoCoeffDimless}{1055668997.2030925063285921958931697786482178917384505743565947290283792722955}
\newcommand{\TFPTeffRtwoCoeffInAction}{527834498.60154625316429609794658488932410894586922528717829736451418963614867773}
\newcommand{\TFPTalphaR}{117864.3622405491}

% A_s for N=55..57
% NOTE: TeX control sequence names cannot contain digits, so we spell out N values.
\newcommand{\TFPTAsNfiftyfive}{2.0162081878496864e-9}
\newcommand{\TFPTAsNfiftysix}{2.09019136432946e-9}
\newcommand{\TFPTAsNfiftyseven}{2.165507571016076e-9}

% --- Omega_b conjecture ---
\newcommand{\TFPTOmegaBPred}{0.048940662665450112200545653760944651480054952073107413870556556782266783965399009}
\newcommand{\TFPTOmegaBRef}{0.049301692328524438476424754994611856173233055557122787616860658651215012327847394}
\newcommand{\TFPTOmegaBRefSigma}{0.0008568114274714093}
\newcommand{\TFPTOmegaBZ}{0.42136420161876677}
\newcommand{\TFPTCoeffFourPiMinusOne}{11.566370614359172953850573533118011536788677597500423283899778369231265625144836}

% --- CKM pipeline (rg-dressed, MZ) ---
\newcommand{\TFPTCKMA}{0.8164965809277260}
\newcommand{\TFPTCKMrho}{0.1666666666666667}
\newcommand{\TFPTCKMeta}{0.3333333333333333}
\newcommand{\TFPTCKMchiTwo}{44.05176935606062}
\newcommand{\TFPTCKMJarlskog}{2.767084e-5}

% --- PMNS baseline + eps scan ---
\newcommand{\TFPTsinSqThetaThirteen}{0.023108435158881104}
\newcommand{\TFPTthetaThirteenDeg}{8.743693}
\newcommand{\TFPTthetaTwelveDeg}{34.322526}
\newcommand{\TFPTepsRad}{0.008861992029474254}
\newcommand{\TFPTepsDeg}{0.5077547413674498}

% =============================================================================
% Tables and plots
% =============================================================================

% --- Core invariants table ---
\newcommand{\TFPTCoreInvariantTable}{%
\begin{table}[H]
\centering
\begin{tabular}{@{}lll@{}}
\toprule
Symbol & Definition & Value \\
\midrule
$c_3$ & $1/(8\pi)$ & \num{\TFPTcThreeNum} \\
$\varphi_{\mathrm{tree}}$ & $1/(6\pi)$ & \num{\TFPTvarphiTreeNum} \\
$\delta_{\mathrm{top}}$ & $48c_3^4 = 3/(256\pi^4)$ & \num{\TFPTdeltaTopNum} \\
$\varphi_0$ & $\varphi_{\mathrm{tree}}+\delta_{\mathrm{top}}$ & \num{\TFPTvarphiZeroNum} \\
$\beta_{\mathrm{rad}}$ & $\varphi_0/(4\pi)$ & \num{\TFPTbetaRadNum} \\
$\beta_{\mathrm{deg}}$ & $(180/\pi)\beta_{\mathrm{rad}}$ & \num{\TFPTbetaDegNum}\,$^\circ$ \\
$g_{a\gamma\gamma}$ & $-4c_3=-1/(2\pi)$ & \num{\TFPTgAggNum} \\
$M/\bar M_{\mathrm{Pl}}$ & $\sqrt{8\pi}\,c_3^4$ & \num{\TFPTMOverMplNum} \\
\bottomrule
\end{tabular}
\caption{TFPT invariants and derived scales, mp.dps=80.}
\end{table}%
}

% --- Alpha audit table (k-sensitivity + delta2) ---
\newcommand{\TFPTAlphaAuditTable}{%
\begin{table}[H]
\centering
\begin{tabular}{@{}llll@{}}
\toprule
Case & $\alpha^{-1}_{\mathrm{pred}}$ & ppm vs CODATA 2022 & Note \\
\midrule
baseline (self-consistent) & \num{\TFPTalphaInvSelfNum} & \num{-0.037000} & single defect \\
two-defect (suite default) & \num{\TFPTalphaInvTwoDefNum} & \num{\TFPTalphaPpmTwoDefNum} & $\delta_2=\delta_{\mathrm{top}}^2$ \\
two-defect (theoryv3, $g=5$) & \num{\TFPTalphaInvGOverFourNum} & \num{\TFPTalphaPpmGOverFourNum} & $\delta_2=\frac54\,\delta_{\mathrm{top}}^2$ \\
targeted match & \num{\TFPTalphaInvCODATANum} & 0 & requires $\delta_2=\num{\TFPTdeltaTwoOptimalNum}$ \\
\bottomrule
\end{tabular}
\caption{Fine-structure constant: baseline and two-defect refinements (suite default vs.\ theoryv3), plus the required second-order term for exact CODATA matching.}
\end{table}%
}

% --- Alpha backreaction exponent sensitivity plot (ppm vs k) ---
\newcommand{\TFPTAlphaSensitivityPlot}{%
\begin{figure}[H]
\centering
\begin{tikzpicture}
\begin{axis}[
    width=0.9\linewidth,
    xlabel={$k$ (backreaction exponent) in $\varphi(\alpha)=\varphi_{\mathrm{tree}}+\delta_{\mathrm{top}}e^{-k\alpha}$},
    ylabel={ppm$(\alpha^{-1}(0))$ vs CODATA 2022},
    grid=both,
    xmin=-0.1, xmax=3.1,
    ymin=-2.2, ymax=4.2,
    xtick={0,1,1.5,2,2.5,3},
    legend pos=south west,
]
\addplot+[mark=*] coordinates {
    (0.0, 3.665371791638453)
    (1.0, 1.8074246412353618)
    (1.5, 0.883514212976573)
    (2.0, -0.03703710120260405)
    (2.5, -0.9542414840493093)
    (3.0, -1.8681110743677323)
};
\addlegendentry{single-defect ($\delta_2=0$)}

\addplot+[only marks, mark=star, mark size=4pt] coordinates {
    (2.0, \TFPTalphaPpmTwoDefNum)
};
\addlegendentry{two-defect at $k=2$ ($\delta_2=\delta_{\mathrm{top}}^2$)}

\addplot[gray, dashed] coordinates {(2.0,-2.2) (2.0,4.2)};
\addlegendentry{$k=2$ (double cover)}

\addplot[gray, dotted] coordinates {(\TFPTkEffNum,-2.2) (\TFPTkEffNum,4.2)};
\addlegendentry{$k_{\mathrm{match}}\approx\num{\TFPTkEffNum}$ (diagnostic)}

\addplot[black!50, dashed] coordinates {(-0.1,0) (3.1,0)};
\end{axis}
\end{tikzpicture}
\caption{Backreaction exponent sensitivity for the CFE closure. The canonical value $k=2$ is fixed by the orientable double cover; $k_{\mathrm{match}}$ is the diagnostic value that would null the CODATA residual in the single-defect truncation.}
\end{figure}%
}

% --- Bounce transfer function data plot (from bounce_perturbations) ---
\newcommand{\TFPTTransferFunctionPlot}{%
\begin{figure}[H]
\centering
\begin{tikzpicture}
\begin{axis}[
    width=0.9\linewidth,
    xmode=log, ymode=log,
    xlabel={$\hat k$},
    ylabel={$T(\hat k)$},
    grid=both,
    legend pos=south east
]
\addplot+[mark=*] coordinates {
    (0.1,    0.0032255772038632126)
    (0.15199110829529336, 0.0035703003593638466)
    (0.23101297000831597, 0.00000000399020695466708)
    (0.35111917342151316, 0.000013156269894230775)
    (0.533669923120631,   0.03550967159062275)
    (0.8111308307896873,  1.6127872665381955)
    (1.2328467394420662,  1.0077896948137592)
    (1.873817422860383,   0.9810730551587763)
    (2.848035868435802,   1.004402539575707)
    (4.328761281083057,   1.0055829625872563)
    (6.5793322465756825,  1.006832617067307)
    (10.0,                1.007882155814805)
};
\addlegendentry{scalar}
\addplot+[mark=square*] coordinates {
    (0.1,    147270.05495615443)
    (0.15199110829529336, 53875.473521817745)
    (0.23101297000831597, 2.1302554245702106)
    (0.35111917342151316, 1.1806392634784857)
    (0.533669923120631,   1.1598025598647947)
    (0.8111308307896873,  1.0319915300605456)
    (1.2328467394420662,  1.035501225860808)
    (1.873817422860383,   1.008365931672867)
    (2.848035868435802,   0.9883585969391012)
    (4.328761281083057,   1.007608072068884)
    (6.5793322465756825,  1.0065676931381393)
    (10.0,                1.0018456357164773)
};
\addlegendentry{tensor}
\end{axis}
\end{tikzpicture}
\caption{Bounce transfer functions from \texttt{bounce\_perturbations}. Note: the two lowest $\hat k$ scalar points were flagged by Wronskian diagnostics in the test report.}
\end{figure}%
}

% --- CKM matrix (absolute values at MZ) ---
\newcommand{\TFPTCKMMatrixBlock}{%
\[
|V_{\mathrm{CKM}}|_{M_Z} \approx
\begin{pmatrix}
0.9744775404 & 0.2244586415 & 0.003441159690 \\
0.2243316946 & 0.9736442171 & 0.04113671446 \\
0.008295429526 & 0.04043830740 & 0.9991476013
\end{pmatrix}.
\]
}

% --- Yukawa textures (UV, used in ckm_full_pipeline) ---
\newcommand{\TFPTYukawaTextureBlock}{%
\begin{tcolorbox}[colback=gray!5, colframe=gray!60, title={\textbf{UV Yukawa texture used in the CKM pipeline}}, fonttitle=\bfseries\small]
The baseline verification pipeline uses a minimal hierarchical UV texture in powers of the TFPT Cabibbo parameter $\lambda$:
\[
\mathrm{diag}(Y_u^{UV})=(\lambda^8,\lambda^4,1),\qquad \mathrm{diag}(Y_d^{UV})=(\lambda^4,\lambda^2,\lambda^2)
\]
Numerically (using the report values $\lambda=\num{0.2244599705189737}$), this is
\[
Y_u^{UV}=\begin{pmatrix}
\num{6.4433424282582535e-06} & 0 & 0\\
0 & \num{2.5383739732864922e-03} & 0\\
0 & 0 & \num{1}
\end{pmatrix},\qquad
|Y_d^{UV}|=\begin{pmatrix}
\num{2.47358843e-03} & \num{1.13087378e-02} & \num{1.73373465e-04}\\
\num{5.69437735e-04} & \num{4.90544140e-02} & \num{2.07256140e-03}\\
\num{2.10569024e-05} & \num{2.03737406e-03} & \num{5.03393326e-02}
\end{pmatrix}.
\]
The down type texture is left rotated by a CKM proxy at the UV scale: $Y_d^{UV}=V_{UV}\,\mathrm{diag}(Y_d^{UV})$ (phase carried by $V_{UV}$).
\end{tcolorbox}%
}

% --- Yukawa texture heatmap (log10 magnitudes) ---
\newcommand{\TFPTYukawaHeatmapPlot}{%
\begin{figure}[H]
\centering
\begin{minipage}{0.48\linewidth}
\centering
\begin{tikzpicture}
\begin{axis}[
    title={$\log_{10}|Y_u^{UV}|$},
    width=\linewidth,
    height=0.9\linewidth,
    xmin=0.5, xmax=3.5,
    ymin=0.5, ymax=3.5,
    xtick={1,2,3}, ytick={1,2,3},
    y dir=reverse,
    axis on top,
    enlargelimits=false,
    colormap/hot,
    point meta min=-6, point meta max=0,
    tick label style={font=\scriptsize},
    title style={font=\small},
    colorbar=false,
]
\addplot[
    matrix plot*,
    point meta=explicit,
] coordinates {
    (1,1) [-5.1908888] (2,1) [-6]         (3,1) [-6]
    (1,2) [-6]         (2,2) [-2.5954444] (3,2) [-6]
    (1,3) [-6]         (2,3) [-6]         (3,3) [0]
};
\end{axis}
\end{tikzpicture}
\end{minipage}\hfill
\begin{minipage}{0.48\linewidth}
\centering
\begin{tikzpicture}
\begin{axis}[
    title={$\log_{10}|Y_d^{UV}|$},
    width=\linewidth,
    height=0.9\linewidth,
    xmin=0.5, xmax=3.5,
    ymin=0.5, ymax=3.5,
    xtick={1,2,3}, ytick={1,2,3},
    y dir=reverse,
    axis on top,
    enlargelimits=false,
    colormap/hot,
    point meta min=-6, point meta max=0,
    tick label style={font=\scriptsize},
    title style={font=\small},
    colorbar,
    colorbar style={ylabel={$\log_{10}|Y|$}, yticklabel style={font=\scriptsize}, ylabel style={font=\scriptsize}},
]
\addplot[
    matrix plot*,
    point meta=explicit,
] coordinates {
    (1,1) [-2.6066726] (2,1) [-1.9465859] (3,1) [-3.7610174]
    (1,2) [-3.2445538] (2,2) [-1.3093219] (3,2) [-2.6834926]
    (1,3) [-4.6766055] (2,3) [-2.6909292] (3,3) [-1.2980925]
};
\end{axis}
\end{tikzpicture}
\end{minipage}
\caption{Heatmap of $\log_{10}|Y_u^{UV}|$ and $\log_{10}|Y_d^{UV}|$ used by the \texttt{ckm\_full\_pipeline} benchmark.}
\end{figure}%
}

% --- Z3 permutation matrix P (PMNS module) ---
\newcommand{\TFPTZthreePermutationMatrix}{%
\[
P =
\begin{pmatrix}
0 & 1 & 0\\
0 & 0 & 1\\
1 & 0 & 0
\end{pmatrix}.
\]
}

% --- PMNS variant table ---
\newcommand{\TFPTPMNSVariantTable}{%
\begin{table}[H]
\centering
\begin{tabular}{@{}lcccc@{}}
\toprule
Variant & $\theta_{23}$ [deg] & $\Delta\theta_{23}$ [deg] & $\delta_{\mathrm{CP}}$ [deg] & $\Delta\delta$ [deg]\\
\midrule
baseline (TM1, Z3) & 45.0000 & 0.0000 & 90.0000 & 0.0000\\
L\_R23(+eps) & 45.5078 & +0.5078 & 90.0000 & 0.0000\\
L\_R23(-eps) & 44.4922 & -0.5078 & 90.0000 & 0.0000\\
R\_R23(+eps) & 45.4243 & +0.4243 & 88.1896 & -1.8104\\
R\_R23(-eps) & 44.5757 & -0.4243 & 91.8104 & +1.8104\\
L\_R23(+eps)*Z3phase & 45.5078 & +0.5078 & 90.0000 & 0.0000\\
L\_R23(-eps)*Z3phase & 44.4922 & -0.5078 & 90.0000 & 0.0000\\
\bottomrule
\end{tabular}
\caption{PMNS Z3-breaking scan with $\epsilon=\varphi_0/6=\num{\TFPTepsRad}$ rad.}
\end{table}%
}

% --- Omega_b table ---
\newcommand{\TFPTOmegaBTable}{%
\begin{table}[H]
\centering
\begin{tabular}{@{}lll@{}}
\toprule
Quantity & Expression & Value \\
\midrule
$\Omega_b$ (TFPT conjecture) & $(4\pi-1)\,\beta_{\mathrm{rad}}$ & \num{\TFPTOmegaBPred} \\
$\Omega_b$ (Planck-derived ref) & $\Omega_b h^2 / h^2$ & \num{\TFPTOmegaBRef} \,$\pm$\,\num{\TFPTOmegaBRefSigma} \\
z-score (ref) & $(\Omega_{b,\mathrm{pred}}-\Omega_{b,\mathrm{ref}})/\sigma$ & \num{\TFPTOmegaBZ} \\
\bottomrule
\end{tabular}
\caption{Baryon density conjecture: numeric prediction and reference comparison.}
\end{table}%
}

% --- k-calibration table (ell mapping) ---
\newcommand{\TFPTkCalibrationTable}{%
\begin{table}[H]
\centering
\begin{tabular}{@{}lccc@{}}
\toprule
Target $\ell$ & required $a_0/a_t$ (scalar) & required $a_0/a_t$ (tensor) & $N=\ln(a_0/a_t)$ (scalar) \\
\midrule
2   & $6.978\times 10^{58}$ & $2.590\times 10^{56}$ & 135.493 \\
30  & $4.652\times 10^{57}$ & $1.727\times 10^{55}$ & 132.785 \\
700 & $1.994\times 10^{56}$ & $7.400\times 10^{53}$ & 129.635 \\
\bottomrule
\end{tabular}
\caption{Mapping the bounce scale to CMB multipoles: required overall scaling $a_0/a_t$ to place features at given $\ell$ (from \texttt{k\_calibration}).}
\end{table}%
}

% --- Torsion bounds table ---
\newcommand{\TFPTTorsionBoundsTable}{%
\begin{table}[H]
\centering
\begin{tabular}{@{}lcc@{}}
\toprule
Component & bound on $|S_\mu|$ [GeV] & mapping note \\
\midrule
$S_T$ & $2.9\times 10^{-27}$ & direct $A_\mu$ bound \\
$S_X$ & $2.1\times 10^{-31}$ & direct $A_\mu$ bound \\
$S_Y$ & $2.5\times 10^{-31}$ & direct $A_\mu$ bound \\
$S_Z$ & $1.0\times 10^{-29}$ & direct $A_\mu$ bound \\
\bottomrule
\end{tabular}
\caption{Component-wise torsion bounds ingested by \texttt{torsion\_bounds\_mapping}; TFPT local vacuum prediction used in the suite is $S_\mu(\mathrm{today})=0$.}
\end{table}%
}

% -----------------------------------------------------------------------------
% TFPT v2.5 verification snippets
% Auto-extracted from tfpt-suite report (mp.dps=80)
% Usage: % TFPT v2.5 verification snippets
% Auto-extracted from tfpt-suite report (mp.dps=80)
% Usage: % TFPT v2.5 verification snippets
% Auto-extracted from tfpt-suite report (mp.dps=80)
% Usage: % TFPT v2.5 verification snippets
% Auto-extracted from tfpt-suite report (mp.dps=80)
% Usage: \input{tfpt-v25-snippets.tex} somewhere after packages (siunitx, booktabs, pgfplots, tikz, float, tcolorbox).

% --- Numerics (high precision) ---
\newcommand{\TFPTcThreeNum}{0.039788735772973833942220940843128590508614911435114112186916836014724199408556634}
\newcommand{\TFPTvarphiTreeNum}{0.053051647697298445256294587790838120678153215246818816249222448019632265878075512}
\newcommand{\TFPTdeltaTopNum}{0.00012030447954708205299788417891154719697454455297232339122853166568719135947912407}
\newcommand{\TFPTvarphiZeroNum}{0.053171952176845527309292471969749667875127759799791139640450979685319457237554636}
\newcommand{\TFPTbetaRadNum}{0.0042312895113954151087468182088050163950728077266837257698944229030526732721556263}
\newcommand{\TFPTbetaDegNum}{0.24243503090092952849243152047951445246859512239848881592836053961070607871616298}
\newcommand{\TFPTgAggNum}{-0.15915494309189533576888376337251436203445964574045644874766734405981889679763422654}
\newcommand{\TFPTMOverMplNum}{0.000012564942083228448969980786638197975376707921748468781310516248914585154899192637}

% alpha: baseline self-consistent and refined two-defect
\newcommand{\TFPTalphaInvSelfNum}{137.03599410158383008148619622913336009409257904171246722475352841304700901323215}
\newcommand{\TFPTalphaInvTwoDefNum}{137.0359981932435869110224146843726714996000114935956757208545827539450219077872}
\newcommand{\TFPTalphaInvGOverFourNum}{137.03599921615843479026171751711397398605568500258186146747743646328176776397918}
\newcommand{\TFPTalphaInvCODATANum}{137.035999177}
\newcommand{\TFPTalphaInvCODATASigma}{0.000000021}
\newcommand{\TFPTalphaPpmTwoDefNum}{-0.0071788173837323352413040312917307695904786165477907034361613291839588627323760663}
\newcommand{\TFPTalphaPpmGOverFourNum}{0.00028575290453192123198929586322751817361006829116847864863933342766974073215940791}
\newcommand{\TFPTalphaZGOverFourNum}{1.8646873709648437}
\newcommand{\TFPTdeltaTwoGOverFourNum}{0.000000018091459748867855014525576506215873427709357460729151458474190949106463016708746}
\newcommand{\TFPTdeltaTwoOptimalNum}{1.79529470934477993569067565574491365815939289267429e-8}
\newcommand{\TFPTdeltaTwoOverDeltaTopNum}{1.49229248636677043432223527146417707405844271475241e-4}
\newcommand{\TFPTkEffNum}{1.97984799374161166800213443338796550879953534211621}

% alpha MSbar at MZ (primary comparison observable; alpha_precision_audit)
\newcommand{\TFPTalphaBarFiveInvMZPredNum}{127.94051574588975237575597444353510021974456512094373031810385305362533848241133}
\newcommand{\TFPTalphaBarFiveInvMZRefNum}{127.93}
\newcommand{\TFPTalphaBarFiveInvMZRefSigma}{0.008}
\newcommand{\TFPTalphaBarFiveInvMZZ}{1.314468236219047}

% two-loop RG fingerprints
\newcommand{\TFPTalphaThreePeVNum}{0.05243714153896733}
\newcommand{\TFPTalphaThreePeVRelDevPercent}{1.3819515887516742}

\newcommand{\TFPTmuCrossVarphiZeroGeV}{792005.555441517}
\newcommand{\TFPTmuCrosscThreeGeV}{216030451.17273504}

% R^2 / Starobinsky numbers (effective_action_r2)
\newcommand{\TFPTeffRtwoCoeffDimless}{1055668997.2030925063285921958931697786482178917384505743565947290283792722955}
\newcommand{\TFPTeffRtwoCoeffInAction}{527834498.60154625316429609794658488932410894586922528717829736451418963614867773}
\newcommand{\TFPTalphaR}{117864.3622405491}

% A_s for N=55..57
% NOTE: TeX control sequence names cannot contain digits, so we spell out N values.
\newcommand{\TFPTAsNfiftyfive}{2.0162081878496864e-9}
\newcommand{\TFPTAsNfiftysix}{2.09019136432946e-9}
\newcommand{\TFPTAsNfiftyseven}{2.165507571016076e-9}

% --- Omega_b conjecture ---
\newcommand{\TFPTOmegaBPred}{0.048940662665450112200545653760944651480054952073107413870556556782266783965399009}
\newcommand{\TFPTOmegaBRef}{0.049301692328524438476424754994611856173233055557122787616860658651215012327847394}
\newcommand{\TFPTOmegaBRefSigma}{0.0008568114274714093}
\newcommand{\TFPTOmegaBZ}{0.42136420161876677}
\newcommand{\TFPTCoeffFourPiMinusOne}{11.566370614359172953850573533118011536788677597500423283899778369231265625144836}

% --- CKM pipeline (rg-dressed, MZ) ---
\newcommand{\TFPTCKMA}{0.8164965809277260}
\newcommand{\TFPTCKMrho}{0.1666666666666667}
\newcommand{\TFPTCKMeta}{0.3333333333333333}
\newcommand{\TFPTCKMchiTwo}{44.05176935606062}
\newcommand{\TFPTCKMJarlskog}{2.767084e-5}

% --- PMNS baseline + eps scan ---
\newcommand{\TFPTsinSqThetaThirteen}{0.023108435158881104}
\newcommand{\TFPTthetaThirteenDeg}{8.743693}
\newcommand{\TFPTthetaTwelveDeg}{34.322526}
\newcommand{\TFPTepsRad}{0.008861992029474254}
\newcommand{\TFPTepsDeg}{0.5077547413674498}

% =============================================================================
% Tables and plots
% =============================================================================

% --- Core invariants table ---
\newcommand{\TFPTCoreInvariantTable}{%
\begin{table}[H]
\centering
\begin{tabular}{@{}lll@{}}
\toprule
Symbol & Definition & Value \\
\midrule
$c_3$ & $1/(8\pi)$ & \num{\TFPTcThreeNum} \\
$\varphi_{\mathrm{tree}}$ & $1/(6\pi)$ & \num{\TFPTvarphiTreeNum} \\
$\delta_{\mathrm{top}}$ & $48c_3^4 = 3/(256\pi^4)$ & \num{\TFPTdeltaTopNum} \\
$\varphi_0$ & $\varphi_{\mathrm{tree}}+\delta_{\mathrm{top}}$ & \num{\TFPTvarphiZeroNum} \\
$\beta_{\mathrm{rad}}$ & $\varphi_0/(4\pi)$ & \num{\TFPTbetaRadNum} \\
$\beta_{\mathrm{deg}}$ & $(180/\pi)\beta_{\mathrm{rad}}$ & \num{\TFPTbetaDegNum}\,$^\circ$ \\
$g_{a\gamma\gamma}$ & $-4c_3=-1/(2\pi)$ & \num{\TFPTgAggNum} \\
$M/\bar M_{\mathrm{Pl}}$ & $\sqrt{8\pi}\,c_3^4$ & \num{\TFPTMOverMplNum} \\
\bottomrule
\end{tabular}
\caption{TFPT invariants and derived scales, mp.dps=80.}
\end{table}%
}

% --- Alpha audit table (k-sensitivity + delta2) ---
\newcommand{\TFPTAlphaAuditTable}{%
\begin{table}[H]
\centering
\begin{tabular}{@{}llll@{}}
\toprule
Case & $\alpha^{-1}_{\mathrm{pred}}$ & ppm vs CODATA 2022 & Note \\
\midrule
baseline (self-consistent) & \num{\TFPTalphaInvSelfNum} & \num{-0.037000} & single defect \\
two-defect (suite default) & \num{\TFPTalphaInvTwoDefNum} & \num{\TFPTalphaPpmTwoDefNum} & $\delta_2=\delta_{\mathrm{top}}^2$ \\
two-defect (theoryv3, $g=5$) & \num{\TFPTalphaInvGOverFourNum} & \num{\TFPTalphaPpmGOverFourNum} & $\delta_2=\frac54\,\delta_{\mathrm{top}}^2$ \\
targeted match & \num{\TFPTalphaInvCODATANum} & 0 & requires $\delta_2=\num{\TFPTdeltaTwoOptimalNum}$ \\
\bottomrule
\end{tabular}
\caption{Fine-structure constant: baseline and two-defect refinements (suite default vs.\ theoryv3), plus the required second-order term for exact CODATA matching.}
\end{table}%
}

% --- Alpha backreaction exponent sensitivity plot (ppm vs k) ---
\newcommand{\TFPTAlphaSensitivityPlot}{%
\begin{figure}[H]
\centering
\begin{tikzpicture}
\begin{axis}[
    width=0.9\linewidth,
    xlabel={$k$ (backreaction exponent) in $\varphi(\alpha)=\varphi_{\mathrm{tree}}+\delta_{\mathrm{top}}e^{-k\alpha}$},
    ylabel={ppm$(\alpha^{-1}(0))$ vs CODATA 2022},
    grid=both,
    xmin=-0.1, xmax=3.1,
    ymin=-2.2, ymax=4.2,
    xtick={0,1,1.5,2,2.5,3},
    legend pos=south west,
]
\addplot+[mark=*] coordinates {
    (0.0, 3.665371791638453)
    (1.0, 1.8074246412353618)
    (1.5, 0.883514212976573)
    (2.0, -0.03703710120260405)
    (2.5, -0.9542414840493093)
    (3.0, -1.8681110743677323)
};
\addlegendentry{single-defect ($\delta_2=0$)}

\addplot+[only marks, mark=star, mark size=4pt] coordinates {
    (2.0, \TFPTalphaPpmTwoDefNum)
};
\addlegendentry{two-defect at $k=2$ ($\delta_2=\delta_{\mathrm{top}}^2$)}

\addplot[gray, dashed] coordinates {(2.0,-2.2) (2.0,4.2)};
\addlegendentry{$k=2$ (double cover)}

\addplot[gray, dotted] coordinates {(\TFPTkEffNum,-2.2) (\TFPTkEffNum,4.2)};
\addlegendentry{$k_{\mathrm{match}}\approx\num{\TFPTkEffNum}$ (diagnostic)}

\addplot[black!50, dashed] coordinates {(-0.1,0) (3.1,0)};
\end{axis}
\end{tikzpicture}
\caption{Backreaction exponent sensitivity for the CFE closure. The canonical value $k=2$ is fixed by the orientable double cover; $k_{\mathrm{match}}$ is the diagnostic value that would null the CODATA residual in the single-defect truncation.}
\end{figure}%
}

% --- Bounce transfer function data plot (from bounce_perturbations) ---
\newcommand{\TFPTTransferFunctionPlot}{%
\begin{figure}[H]
\centering
\begin{tikzpicture}
\begin{axis}[
    width=0.9\linewidth,
    xmode=log, ymode=log,
    xlabel={$\hat k$},
    ylabel={$T(\hat k)$},
    grid=both,
    legend pos=south east
]
\addplot+[mark=*] coordinates {
    (0.1,    0.0032255772038632126)
    (0.15199110829529336, 0.0035703003593638466)
    (0.23101297000831597, 0.00000000399020695466708)
    (0.35111917342151316, 0.000013156269894230775)
    (0.533669923120631,   0.03550967159062275)
    (0.8111308307896873,  1.6127872665381955)
    (1.2328467394420662,  1.0077896948137592)
    (1.873817422860383,   0.9810730551587763)
    (2.848035868435802,   1.004402539575707)
    (4.328761281083057,   1.0055829625872563)
    (6.5793322465756825,  1.006832617067307)
    (10.0,                1.007882155814805)
};
\addlegendentry{scalar}
\addplot+[mark=square*] coordinates {
    (0.1,    147270.05495615443)
    (0.15199110829529336, 53875.473521817745)
    (0.23101297000831597, 2.1302554245702106)
    (0.35111917342151316, 1.1806392634784857)
    (0.533669923120631,   1.1598025598647947)
    (0.8111308307896873,  1.0319915300605456)
    (1.2328467394420662,  1.035501225860808)
    (1.873817422860383,   1.008365931672867)
    (2.848035868435802,   0.9883585969391012)
    (4.328761281083057,   1.007608072068884)
    (6.5793322465756825,  1.0065676931381393)
    (10.0,                1.0018456357164773)
};
\addlegendentry{tensor}
\end{axis}
\end{tikzpicture}
\caption{Bounce transfer functions from \texttt{bounce\_perturbations}. Note: the two lowest $\hat k$ scalar points were flagged by Wronskian diagnostics in the test report.}
\end{figure}%
}

% --- CKM matrix (absolute values at MZ) ---
\newcommand{\TFPTCKMMatrixBlock}{%
\[
|V_{\mathrm{CKM}}|_{M_Z} \approx
\begin{pmatrix}
0.9744775404 & 0.2244586415 & 0.003441159690 \\
0.2243316946 & 0.9736442171 & 0.04113671446 \\
0.008295429526 & 0.04043830740 & 0.9991476013
\end{pmatrix}.
\]
}

% --- Yukawa textures (UV, used in ckm_full_pipeline) ---
\newcommand{\TFPTYukawaTextureBlock}{%
\begin{tcolorbox}[colback=gray!5, colframe=gray!60, title={\textbf{UV Yukawa texture used in the CKM pipeline}}, fonttitle=\bfseries\small]
The baseline verification pipeline uses a minimal hierarchical UV texture in powers of the TFPT Cabibbo parameter $\lambda$:
\[
\mathrm{diag}(Y_u^{UV})=(\lambda^8,\lambda^4,1),\qquad \mathrm{diag}(Y_d^{UV})=(\lambda^4,\lambda^2,\lambda^2)
\]
Numerically (using the report values $\lambda=\num{0.2244599705189737}$), this is
\[
Y_u^{UV}=\begin{pmatrix}
\num{6.4433424282582535e-06} & 0 & 0\\
0 & \num{2.5383739732864922e-03} & 0\\
0 & 0 & \num{1}
\end{pmatrix},\qquad
|Y_d^{UV}|=\begin{pmatrix}
\num{2.47358843e-03} & \num{1.13087378e-02} & \num{1.73373465e-04}\\
\num{5.69437735e-04} & \num{4.90544140e-02} & \num{2.07256140e-03}\\
\num{2.10569024e-05} & \num{2.03737406e-03} & \num{5.03393326e-02}
\end{pmatrix}.
\]
The down type texture is left rotated by a CKM proxy at the UV scale: $Y_d^{UV}=V_{UV}\,\mathrm{diag}(Y_d^{UV})$ (phase carried by $V_{UV}$).
\end{tcolorbox}%
}

% --- Yukawa texture heatmap (log10 magnitudes) ---
\newcommand{\TFPTYukawaHeatmapPlot}{%
\begin{figure}[H]
\centering
\begin{minipage}{0.48\linewidth}
\centering
\begin{tikzpicture}
\begin{axis}[
    title={$\log_{10}|Y_u^{UV}|$},
    width=\linewidth,
    height=0.9\linewidth,
    xmin=0.5, xmax=3.5,
    ymin=0.5, ymax=3.5,
    xtick={1,2,3}, ytick={1,2,3},
    y dir=reverse,
    axis on top,
    enlargelimits=false,
    colormap/hot,
    point meta min=-6, point meta max=0,
    tick label style={font=\scriptsize},
    title style={font=\small},
    colorbar=false,
]
\addplot[
    matrix plot*,
    point meta=explicit,
] coordinates {
    (1,1) [-5.1908888] (2,1) [-6]         (3,1) [-6]
    (1,2) [-6]         (2,2) [-2.5954444] (3,2) [-6]
    (1,3) [-6]         (2,3) [-6]         (3,3) [0]
};
\end{axis}
\end{tikzpicture}
\end{minipage}\hfill
\begin{minipage}{0.48\linewidth}
\centering
\begin{tikzpicture}
\begin{axis}[
    title={$\log_{10}|Y_d^{UV}|$},
    width=\linewidth,
    height=0.9\linewidth,
    xmin=0.5, xmax=3.5,
    ymin=0.5, ymax=3.5,
    xtick={1,2,3}, ytick={1,2,3},
    y dir=reverse,
    axis on top,
    enlargelimits=false,
    colormap/hot,
    point meta min=-6, point meta max=0,
    tick label style={font=\scriptsize},
    title style={font=\small},
    colorbar,
    colorbar style={ylabel={$\log_{10}|Y|$}, yticklabel style={font=\scriptsize}, ylabel style={font=\scriptsize}},
]
\addplot[
    matrix plot*,
    point meta=explicit,
] coordinates {
    (1,1) [-2.6066726] (2,1) [-1.9465859] (3,1) [-3.7610174]
    (1,2) [-3.2445538] (2,2) [-1.3093219] (3,2) [-2.6834926]
    (1,3) [-4.6766055] (2,3) [-2.6909292] (3,3) [-1.2980925]
};
\end{axis}
\end{tikzpicture}
\end{minipage}
\caption{Heatmap of $\log_{10}|Y_u^{UV}|$ and $\log_{10}|Y_d^{UV}|$ used by the \texttt{ckm\_full\_pipeline} benchmark.}
\end{figure}%
}

% --- Z3 permutation matrix P (PMNS module) ---
\newcommand{\TFPTZthreePermutationMatrix}{%
\[
P =
\begin{pmatrix}
0 & 1 & 0\\
0 & 0 & 1\\
1 & 0 & 0
\end{pmatrix}.
\]
}

% --- PMNS variant table ---
\newcommand{\TFPTPMNSVariantTable}{%
\begin{table}[H]
\centering
\begin{tabular}{@{}lcccc@{}}
\toprule
Variant & $\theta_{23}$ [deg] & $\Delta\theta_{23}$ [deg] & $\delta_{\mathrm{CP}}$ [deg] & $\Delta\delta$ [deg]\\
\midrule
baseline (TM1, Z3) & 45.0000 & 0.0000 & 90.0000 & 0.0000\\
L\_R23(+eps) & 45.5078 & +0.5078 & 90.0000 & 0.0000\\
L\_R23(-eps) & 44.4922 & -0.5078 & 90.0000 & 0.0000\\
R\_R23(+eps) & 45.4243 & +0.4243 & 88.1896 & -1.8104\\
R\_R23(-eps) & 44.5757 & -0.4243 & 91.8104 & +1.8104\\
L\_R23(+eps)*Z3phase & 45.5078 & +0.5078 & 90.0000 & 0.0000\\
L\_R23(-eps)*Z3phase & 44.4922 & -0.5078 & 90.0000 & 0.0000\\
\bottomrule
\end{tabular}
\caption{PMNS Z3-breaking scan with $\epsilon=\varphi_0/6=\num{\TFPTepsRad}$ rad.}
\end{table}%
}

% --- Omega_b table ---
\newcommand{\TFPTOmegaBTable}{%
\begin{table}[H]
\centering
\begin{tabular}{@{}lll@{}}
\toprule
Quantity & Expression & Value \\
\midrule
$\Omega_b$ (TFPT conjecture) & $(4\pi-1)\,\beta_{\mathrm{rad}}$ & \num{\TFPTOmegaBPred} \\
$\Omega_b$ (Planck-derived ref) & $\Omega_b h^2 / h^2$ & \num{\TFPTOmegaBRef} \,$\pm$\,\num{\TFPTOmegaBRefSigma} \\
z-score (ref) & $(\Omega_{b,\mathrm{pred}}-\Omega_{b,\mathrm{ref}})/\sigma$ & \num{\TFPTOmegaBZ} \\
\bottomrule
\end{tabular}
\caption{Baryon density conjecture: numeric prediction and reference comparison.}
\end{table}%
}

% --- k-calibration table (ell mapping) ---
\newcommand{\TFPTkCalibrationTable}{%
\begin{table}[H]
\centering
\begin{tabular}{@{}lccc@{}}
\toprule
Target $\ell$ & required $a_0/a_t$ (scalar) & required $a_0/a_t$ (tensor) & $N=\ln(a_0/a_t)$ (scalar) \\
\midrule
2   & $6.978\times 10^{58}$ & $2.590\times 10^{56}$ & 135.493 \\
30  & $4.652\times 10^{57}$ & $1.727\times 10^{55}$ & 132.785 \\
700 & $1.994\times 10^{56}$ & $7.400\times 10^{53}$ & 129.635 \\
\bottomrule
\end{tabular}
\caption{Mapping the bounce scale to CMB multipoles: required overall scaling $a_0/a_t$ to place features at given $\ell$ (from \texttt{k\_calibration}).}
\end{table}%
}

% --- Torsion bounds table ---
\newcommand{\TFPTTorsionBoundsTable}{%
\begin{table}[H]
\centering
\begin{tabular}{@{}lcc@{}}
\toprule
Component & bound on $|S_\mu|$ [GeV] & mapping note \\
\midrule
$S_T$ & $2.9\times 10^{-27}$ & direct $A_\mu$ bound \\
$S_X$ & $2.1\times 10^{-31}$ & direct $A_\mu$ bound \\
$S_Y$ & $2.5\times 10^{-31}$ & direct $A_\mu$ bound \\
$S_Z$ & $1.0\times 10^{-29}$ & direct $A_\mu$ bound \\
\bottomrule
\end{tabular}
\caption{Component-wise torsion bounds ingested by \texttt{torsion\_bounds\_mapping}; TFPT local vacuum prediction used in the suite is $S_\mu(\mathrm{today})=0$.}
\end{table}%
}
 somewhere after packages (siunitx, booktabs, pgfplots, tikz, float, tcolorbox).

% --- Numerics (high precision) ---
\newcommand{\TFPTcThreeNum}{0.039788735772973833942220940843128590508614911435114112186916836014724199408556634}
\newcommand{\TFPTvarphiTreeNum}{0.053051647697298445256294587790838120678153215246818816249222448019632265878075512}
\newcommand{\TFPTdeltaTopNum}{0.00012030447954708205299788417891154719697454455297232339122853166568719135947912407}
\newcommand{\TFPTvarphiZeroNum}{0.053171952176845527309292471969749667875127759799791139640450979685319457237554636}
\newcommand{\TFPTbetaRadNum}{0.0042312895113954151087468182088050163950728077266837257698944229030526732721556263}
\newcommand{\TFPTbetaDegNum}{0.24243503090092952849243152047951445246859512239848881592836053961070607871616298}
\newcommand{\TFPTgAggNum}{-0.15915494309189533576888376337251436203445964574045644874766734405981889679763422654}
\newcommand{\TFPTMOverMplNum}{0.000012564942083228448969980786638197975376707921748468781310516248914585154899192637}

% alpha: baseline self-consistent and refined two-defect
\newcommand{\TFPTalphaInvSelfNum}{137.03599410158383008148619622913336009409257904171246722475352841304700901323215}
\newcommand{\TFPTalphaInvTwoDefNum}{137.0359981932435869110224146843726714996000114935956757208545827539450219077872}
\newcommand{\TFPTalphaInvGOverFourNum}{137.03599921615843479026171751711397398605568500258186146747743646328176776397918}
\newcommand{\TFPTalphaInvCODATANum}{137.035999177}
\newcommand{\TFPTalphaInvCODATASigma}{0.000000021}
\newcommand{\TFPTalphaPpmTwoDefNum}{-0.0071788173837323352413040312917307695904786165477907034361613291839588627323760663}
\newcommand{\TFPTalphaPpmGOverFourNum}{0.00028575290453192123198929586322751817361006829116847864863933342766974073215940791}
\newcommand{\TFPTalphaZGOverFourNum}{1.8646873709648437}
\newcommand{\TFPTdeltaTwoGOverFourNum}{0.000000018091459748867855014525576506215873427709357460729151458474190949106463016708746}
\newcommand{\TFPTdeltaTwoOptimalNum}{1.79529470934477993569067565574491365815939289267429e-8}
\newcommand{\TFPTdeltaTwoOverDeltaTopNum}{1.49229248636677043432223527146417707405844271475241e-4}
\newcommand{\TFPTkEffNum}{1.97984799374161166800213443338796550879953534211621}

% alpha MSbar at MZ (primary comparison observable; alpha_precision_audit)
\newcommand{\TFPTalphaBarFiveInvMZPredNum}{127.94051574588975237575597444353510021974456512094373031810385305362533848241133}
\newcommand{\TFPTalphaBarFiveInvMZRefNum}{127.93}
\newcommand{\TFPTalphaBarFiveInvMZRefSigma}{0.008}
\newcommand{\TFPTalphaBarFiveInvMZZ}{1.314468236219047}

% two-loop RG fingerprints
\newcommand{\TFPTalphaThreePeVNum}{0.05243714153896733}
\newcommand{\TFPTalphaThreePeVRelDevPercent}{1.3819515887516742}

\newcommand{\TFPTmuCrossVarphiZeroGeV}{792005.555441517}
\newcommand{\TFPTmuCrosscThreeGeV}{216030451.17273504}

% R^2 / Starobinsky numbers (effective_action_r2)
\newcommand{\TFPTeffRtwoCoeffDimless}{1055668997.2030925063285921958931697786482178917384505743565947290283792722955}
\newcommand{\TFPTeffRtwoCoeffInAction}{527834498.60154625316429609794658488932410894586922528717829736451418963614867773}
\newcommand{\TFPTalphaR}{117864.3622405491}

% A_s for N=55..57
% NOTE: TeX control sequence names cannot contain digits, so we spell out N values.
\newcommand{\TFPTAsNfiftyfive}{2.0162081878496864e-9}
\newcommand{\TFPTAsNfiftysix}{2.09019136432946e-9}
\newcommand{\TFPTAsNfiftyseven}{2.165507571016076e-9}

% --- Omega_b conjecture ---
\newcommand{\TFPTOmegaBPred}{0.048940662665450112200545653760944651480054952073107413870556556782266783965399009}
\newcommand{\TFPTOmegaBRef}{0.049301692328524438476424754994611856173233055557122787616860658651215012327847394}
\newcommand{\TFPTOmegaBRefSigma}{0.0008568114274714093}
\newcommand{\TFPTOmegaBZ}{0.42136420161876677}
\newcommand{\TFPTCoeffFourPiMinusOne}{11.566370614359172953850573533118011536788677597500423283899778369231265625144836}

% --- CKM pipeline (rg-dressed, MZ) ---
\newcommand{\TFPTCKMA}{0.8164965809277260}
\newcommand{\TFPTCKMrho}{0.1666666666666667}
\newcommand{\TFPTCKMeta}{0.3333333333333333}
\newcommand{\TFPTCKMchiTwo}{44.05176935606062}
\newcommand{\TFPTCKMJarlskog}{2.767084e-5}

% --- PMNS baseline + eps scan ---
\newcommand{\TFPTsinSqThetaThirteen}{0.023108435158881104}
\newcommand{\TFPTthetaThirteenDeg}{8.743693}
\newcommand{\TFPTthetaTwelveDeg}{34.322526}
\newcommand{\TFPTepsRad}{0.008861992029474254}
\newcommand{\TFPTepsDeg}{0.5077547413674498}

% =============================================================================
% Tables and plots
% =============================================================================

% --- Core invariants table ---
\newcommand{\TFPTCoreInvariantTable}{%
\begin{table}[H]
\centering
\begin{tabular}{@{}lll@{}}
\toprule
Symbol & Definition & Value \\
\midrule
$c_3$ & $1/(8\pi)$ & \num{\TFPTcThreeNum} \\
$\varphi_{\mathrm{tree}}$ & $1/(6\pi)$ & \num{\TFPTvarphiTreeNum} \\
$\delta_{\mathrm{top}}$ & $48c_3^4 = 3/(256\pi^4)$ & \num{\TFPTdeltaTopNum} \\
$\varphi_0$ & $\varphi_{\mathrm{tree}}+\delta_{\mathrm{top}}$ & \num{\TFPTvarphiZeroNum} \\
$\beta_{\mathrm{rad}}$ & $\varphi_0/(4\pi)$ & \num{\TFPTbetaRadNum} \\
$\beta_{\mathrm{deg}}$ & $(180/\pi)\beta_{\mathrm{rad}}$ & \num{\TFPTbetaDegNum}\,$^\circ$ \\
$g_{a\gamma\gamma}$ & $-4c_3=-1/(2\pi)$ & \num{\TFPTgAggNum} \\
$M/\bar M_{\mathrm{Pl}}$ & $\sqrt{8\pi}\,c_3^4$ & \num{\TFPTMOverMplNum} \\
\bottomrule
\end{tabular}
\caption{TFPT invariants and derived scales, mp.dps=80.}
\end{table}%
}

% --- Alpha audit table (k-sensitivity + delta2) ---
\newcommand{\TFPTAlphaAuditTable}{%
\begin{table}[H]
\centering
\begin{tabular}{@{}llll@{}}
\toprule
Case & $\alpha^{-1}_{\mathrm{pred}}$ & ppm vs CODATA 2022 & Note \\
\midrule
baseline (self-consistent) & \num{\TFPTalphaInvSelfNum} & \num{-0.037000} & single defect \\
two-defect (suite default) & \num{\TFPTalphaInvTwoDefNum} & \num{\TFPTalphaPpmTwoDefNum} & $\delta_2=\delta_{\mathrm{top}}^2$ \\
two-defect (theoryv3, $g=5$) & \num{\TFPTalphaInvGOverFourNum} & \num{\TFPTalphaPpmGOverFourNum} & $\delta_2=\frac54\,\delta_{\mathrm{top}}^2$ \\
targeted match & \num{\TFPTalphaInvCODATANum} & 0 & requires $\delta_2=\num{\TFPTdeltaTwoOptimalNum}$ \\
\bottomrule
\end{tabular}
\caption{Fine-structure constant: baseline and two-defect refinements (suite default vs.\ theoryv3), plus the required second-order term for exact CODATA matching.}
\end{table}%
}

% --- Alpha backreaction exponent sensitivity plot (ppm vs k) ---
\newcommand{\TFPTAlphaSensitivityPlot}{%
\begin{figure}[H]
\centering
\begin{tikzpicture}
\begin{axis}[
    width=0.9\linewidth,
    xlabel={$k$ (backreaction exponent) in $\varphi(\alpha)=\varphi_{\mathrm{tree}}+\delta_{\mathrm{top}}e^{-k\alpha}$},
    ylabel={ppm$(\alpha^{-1}(0))$ vs CODATA 2022},
    grid=both,
    xmin=-0.1, xmax=3.1,
    ymin=-2.2, ymax=4.2,
    xtick={0,1,1.5,2,2.5,3},
    legend pos=south west,
]
\addplot+[mark=*] coordinates {
    (0.0, 3.665371791638453)
    (1.0, 1.8074246412353618)
    (1.5, 0.883514212976573)
    (2.0, -0.03703710120260405)
    (2.5, -0.9542414840493093)
    (3.0, -1.8681110743677323)
};
\addlegendentry{single-defect ($\delta_2=0$)}

\addplot+[only marks, mark=star, mark size=4pt] coordinates {
    (2.0, \TFPTalphaPpmTwoDefNum)
};
\addlegendentry{two-defect at $k=2$ ($\delta_2=\delta_{\mathrm{top}}^2$)}

\addplot[gray, dashed] coordinates {(2.0,-2.2) (2.0,4.2)};
\addlegendentry{$k=2$ (double cover)}

\addplot[gray, dotted] coordinates {(\TFPTkEffNum,-2.2) (\TFPTkEffNum,4.2)};
\addlegendentry{$k_{\mathrm{match}}\approx\num{\TFPTkEffNum}$ (diagnostic)}

\addplot[black!50, dashed] coordinates {(-0.1,0) (3.1,0)};
\end{axis}
\end{tikzpicture}
\caption{Backreaction exponent sensitivity for the CFE closure. The canonical value $k=2$ is fixed by the orientable double cover; $k_{\mathrm{match}}$ is the diagnostic value that would null the CODATA residual in the single-defect truncation.}
\end{figure}%
}

% --- Bounce transfer function data plot (from bounce_perturbations) ---
\newcommand{\TFPTTransferFunctionPlot}{%
\begin{figure}[H]
\centering
\begin{tikzpicture}
\begin{axis}[
    width=0.9\linewidth,
    xmode=log, ymode=log,
    xlabel={$\hat k$},
    ylabel={$T(\hat k)$},
    grid=both,
    legend pos=south east
]
\addplot+[mark=*] coordinates {
    (0.1,    0.0032255772038632126)
    (0.15199110829529336, 0.0035703003593638466)
    (0.23101297000831597, 0.00000000399020695466708)
    (0.35111917342151316, 0.000013156269894230775)
    (0.533669923120631,   0.03550967159062275)
    (0.8111308307896873,  1.6127872665381955)
    (1.2328467394420662,  1.0077896948137592)
    (1.873817422860383,   0.9810730551587763)
    (2.848035868435802,   1.004402539575707)
    (4.328761281083057,   1.0055829625872563)
    (6.5793322465756825,  1.006832617067307)
    (10.0,                1.007882155814805)
};
\addlegendentry{scalar}
\addplot+[mark=square*] coordinates {
    (0.1,    147270.05495615443)
    (0.15199110829529336, 53875.473521817745)
    (0.23101297000831597, 2.1302554245702106)
    (0.35111917342151316, 1.1806392634784857)
    (0.533669923120631,   1.1598025598647947)
    (0.8111308307896873,  1.0319915300605456)
    (1.2328467394420662,  1.035501225860808)
    (1.873817422860383,   1.008365931672867)
    (2.848035868435802,   0.9883585969391012)
    (4.328761281083057,   1.007608072068884)
    (6.5793322465756825,  1.0065676931381393)
    (10.0,                1.0018456357164773)
};
\addlegendentry{tensor}
\end{axis}
\end{tikzpicture}
\caption{Bounce transfer functions from \texttt{bounce\_perturbations}. Note: the two lowest $\hat k$ scalar points were flagged by Wronskian diagnostics in the test report.}
\end{figure}%
}

% --- CKM matrix (absolute values at MZ) ---
\newcommand{\TFPTCKMMatrixBlock}{%
\[
|V_{\mathrm{CKM}}|_{M_Z} \approx
\begin{pmatrix}
0.9744775404 & 0.2244586415 & 0.003441159690 \\
0.2243316946 & 0.9736442171 & 0.04113671446 \\
0.008295429526 & 0.04043830740 & 0.9991476013
\end{pmatrix}.
\]
}

% --- Yukawa textures (UV, used in ckm_full_pipeline) ---
\newcommand{\TFPTYukawaTextureBlock}{%
\begin{tcolorbox}[colback=gray!5, colframe=gray!60, title={\textbf{UV Yukawa texture used in the CKM pipeline}}, fonttitle=\bfseries\small]
The baseline verification pipeline uses a minimal hierarchical UV texture in powers of the TFPT Cabibbo parameter $\lambda$:
\[
\mathrm{diag}(Y_u^{UV})=(\lambda^8,\lambda^4,1),\qquad \mathrm{diag}(Y_d^{UV})=(\lambda^4,\lambda^2,\lambda^2)
\]
Numerically (using the report values $\lambda=\num{0.2244599705189737}$), this is
\[
Y_u^{UV}=\begin{pmatrix}
\num{6.4433424282582535e-06} & 0 & 0\\
0 & \num{2.5383739732864922e-03} & 0\\
0 & 0 & \num{1}
\end{pmatrix},\qquad
|Y_d^{UV}|=\begin{pmatrix}
\num{2.47358843e-03} & \num{1.13087378e-02} & \num{1.73373465e-04}\\
\num{5.69437735e-04} & \num{4.90544140e-02} & \num{2.07256140e-03}\\
\num{2.10569024e-05} & \num{2.03737406e-03} & \num{5.03393326e-02}
\end{pmatrix}.
\]
The down type texture is left rotated by a CKM proxy at the UV scale: $Y_d^{UV}=V_{UV}\,\mathrm{diag}(Y_d^{UV})$ (phase carried by $V_{UV}$).
\end{tcolorbox}%
}

% --- Yukawa texture heatmap (log10 magnitudes) ---
\newcommand{\TFPTYukawaHeatmapPlot}{%
\begin{figure}[H]
\centering
\begin{minipage}{0.48\linewidth}
\centering
\begin{tikzpicture}
\begin{axis}[
    title={$\log_{10}|Y_u^{UV}|$},
    width=\linewidth,
    height=0.9\linewidth,
    xmin=0.5, xmax=3.5,
    ymin=0.5, ymax=3.5,
    xtick={1,2,3}, ytick={1,2,3},
    y dir=reverse,
    axis on top,
    enlargelimits=false,
    colormap/hot,
    point meta min=-6, point meta max=0,
    tick label style={font=\scriptsize},
    title style={font=\small},
    colorbar=false,
]
\addplot[
    matrix plot*,
    point meta=explicit,
] coordinates {
    (1,1) [-5.1908888] (2,1) [-6]         (3,1) [-6]
    (1,2) [-6]         (2,2) [-2.5954444] (3,2) [-6]
    (1,3) [-6]         (2,3) [-6]         (3,3) [0]
};
\end{axis}
\end{tikzpicture}
\end{minipage}\hfill
\begin{minipage}{0.48\linewidth}
\centering
\begin{tikzpicture}
\begin{axis}[
    title={$\log_{10}|Y_d^{UV}|$},
    width=\linewidth,
    height=0.9\linewidth,
    xmin=0.5, xmax=3.5,
    ymin=0.5, ymax=3.5,
    xtick={1,2,3}, ytick={1,2,3},
    y dir=reverse,
    axis on top,
    enlargelimits=false,
    colormap/hot,
    point meta min=-6, point meta max=0,
    tick label style={font=\scriptsize},
    title style={font=\small},
    colorbar,
    colorbar style={ylabel={$\log_{10}|Y|$}, yticklabel style={font=\scriptsize}, ylabel style={font=\scriptsize}},
]
\addplot[
    matrix plot*,
    point meta=explicit,
] coordinates {
    (1,1) [-2.6066726] (2,1) [-1.9465859] (3,1) [-3.7610174]
    (1,2) [-3.2445538] (2,2) [-1.3093219] (3,2) [-2.6834926]
    (1,3) [-4.6766055] (2,3) [-2.6909292] (3,3) [-1.2980925]
};
\end{axis}
\end{tikzpicture}
\end{minipage}
\caption{Heatmap of $\log_{10}|Y_u^{UV}|$ and $\log_{10}|Y_d^{UV}|$ used by the \texttt{ckm\_full\_pipeline} benchmark.}
\end{figure}%
}

% --- Z3 permutation matrix P (PMNS module) ---
\newcommand{\TFPTZthreePermutationMatrix}{%
\[
P =
\begin{pmatrix}
0 & 1 & 0\\
0 & 0 & 1\\
1 & 0 & 0
\end{pmatrix}.
\]
}

% --- PMNS variant table ---
\newcommand{\TFPTPMNSVariantTable}{%
\begin{table}[H]
\centering
\begin{tabular}{@{}lcccc@{}}
\toprule
Variant & $\theta_{23}$ [deg] & $\Delta\theta_{23}$ [deg] & $\delta_{\mathrm{CP}}$ [deg] & $\Delta\delta$ [deg]\\
\midrule
baseline (TM1, Z3) & 45.0000 & 0.0000 & 90.0000 & 0.0000\\
L\_R23(+eps) & 45.5078 & +0.5078 & 90.0000 & 0.0000\\
L\_R23(-eps) & 44.4922 & -0.5078 & 90.0000 & 0.0000\\
R\_R23(+eps) & 45.4243 & +0.4243 & 88.1896 & -1.8104\\
R\_R23(-eps) & 44.5757 & -0.4243 & 91.8104 & +1.8104\\
L\_R23(+eps)*Z3phase & 45.5078 & +0.5078 & 90.0000 & 0.0000\\
L\_R23(-eps)*Z3phase & 44.4922 & -0.5078 & 90.0000 & 0.0000\\
\bottomrule
\end{tabular}
\caption{PMNS Z3-breaking scan with $\epsilon=\varphi_0/6=\num{\TFPTepsRad}$ rad.}
\end{table}%
}

% --- Omega_b table ---
\newcommand{\TFPTOmegaBTable}{%
\begin{table}[H]
\centering
\begin{tabular}{@{}lll@{}}
\toprule
Quantity & Expression & Value \\
\midrule
$\Omega_b$ (TFPT conjecture) & $(4\pi-1)\,\beta_{\mathrm{rad}}$ & \num{\TFPTOmegaBPred} \\
$\Omega_b$ (Planck-derived ref) & $\Omega_b h^2 / h^2$ & \num{\TFPTOmegaBRef} \,$\pm$\,\num{\TFPTOmegaBRefSigma} \\
z-score (ref) & $(\Omega_{b,\mathrm{pred}}-\Omega_{b,\mathrm{ref}})/\sigma$ & \num{\TFPTOmegaBZ} \\
\bottomrule
\end{tabular}
\caption{Baryon density conjecture: numeric prediction and reference comparison.}
\end{table}%
}

% --- k-calibration table (ell mapping) ---
\newcommand{\TFPTkCalibrationTable}{%
\begin{table}[H]
\centering
\begin{tabular}{@{}lccc@{}}
\toprule
Target $\ell$ & required $a_0/a_t$ (scalar) & required $a_0/a_t$ (tensor) & $N=\ln(a_0/a_t)$ (scalar) \\
\midrule
2   & $6.978\times 10^{58}$ & $2.590\times 10^{56}$ & 135.493 \\
30  & $4.652\times 10^{57}$ & $1.727\times 10^{55}$ & 132.785 \\
700 & $1.994\times 10^{56}$ & $7.400\times 10^{53}$ & 129.635 \\
\bottomrule
\end{tabular}
\caption{Mapping the bounce scale to CMB multipoles: required overall scaling $a_0/a_t$ to place features at given $\ell$ (from \texttt{k\_calibration}).}
\end{table}%
}

% --- Torsion bounds table ---
\newcommand{\TFPTTorsionBoundsTable}{%
\begin{table}[H]
\centering
\begin{tabular}{@{}lcc@{}}
\toprule
Component & bound on $|S_\mu|$ [GeV] & mapping note \\
\midrule
$S_T$ & $2.9\times 10^{-27}$ & direct $A_\mu$ bound \\
$S_X$ & $2.1\times 10^{-31}$ & direct $A_\mu$ bound \\
$S_Y$ & $2.5\times 10^{-31}$ & direct $A_\mu$ bound \\
$S_Z$ & $1.0\times 10^{-29}$ & direct $A_\mu$ bound \\
\bottomrule
\end{tabular}
\caption{Component-wise torsion bounds ingested by \texttt{torsion\_bounds\_mapping}; TFPT local vacuum prediction used in the suite is $S_\mu(\mathrm{today})=0$.}
\end{table}%
}
 somewhere after packages (siunitx, booktabs, pgfplots, tikz, float, tcolorbox).

% --- Numerics (high precision) ---
\newcommand{\TFPTcThreeNum}{0.039788735772973833942220940843128590508614911435114112186916836014724199408556634}
\newcommand{\TFPTvarphiTreeNum}{0.053051647697298445256294587790838120678153215246818816249222448019632265878075512}
\newcommand{\TFPTdeltaTopNum}{0.00012030447954708205299788417891154719697454455297232339122853166568719135947912407}
\newcommand{\TFPTvarphiZeroNum}{0.053171952176845527309292471969749667875127759799791139640450979685319457237554636}
\newcommand{\TFPTbetaRadNum}{0.0042312895113954151087468182088050163950728077266837257698944229030526732721556263}
\newcommand{\TFPTbetaDegNum}{0.24243503090092952849243152047951445246859512239848881592836053961070607871616298}
\newcommand{\TFPTgAggNum}{-0.15915494309189533576888376337251436203445964574045644874766734405981889679763422654}
\newcommand{\TFPTMOverMplNum}{0.000012564942083228448969980786638197975376707921748468781310516248914585154899192637}

% alpha: baseline self-consistent and refined two-defect
\newcommand{\TFPTalphaInvSelfNum}{137.03599410158383008148619622913336009409257904171246722475352841304700901323215}
\newcommand{\TFPTalphaInvTwoDefNum}{137.0359981932435869110224146843726714996000114935956757208545827539450219077872}
\newcommand{\TFPTalphaInvGOverFourNum}{137.03599921615843479026171751711397398605568500258186146747743646328176776397918}
\newcommand{\TFPTalphaInvCODATANum}{137.035999177}
\newcommand{\TFPTalphaInvCODATASigma}{0.000000021}
\newcommand{\TFPTalphaPpmTwoDefNum}{-0.0071788173837323352413040312917307695904786165477907034361613291839588627323760663}
\newcommand{\TFPTalphaPpmGOverFourNum}{0.00028575290453192123198929586322751817361006829116847864863933342766974073215940791}
\newcommand{\TFPTalphaZGOverFourNum}{1.8646873709648437}
\newcommand{\TFPTdeltaTwoGOverFourNum}{0.000000018091459748867855014525576506215873427709357460729151458474190949106463016708746}
\newcommand{\TFPTdeltaTwoOptimalNum}{1.79529470934477993569067565574491365815939289267429e-8}
\newcommand{\TFPTdeltaTwoOverDeltaTopNum}{1.49229248636677043432223527146417707405844271475241e-4}
\newcommand{\TFPTkEffNum}{1.97984799374161166800213443338796550879953534211621}

% alpha MSbar at MZ (primary comparison observable; alpha_precision_audit)
\newcommand{\TFPTalphaBarFiveInvMZPredNum}{127.94051574588975237575597444353510021974456512094373031810385305362533848241133}
\newcommand{\TFPTalphaBarFiveInvMZRefNum}{127.93}
\newcommand{\TFPTalphaBarFiveInvMZRefSigma}{0.008}
\newcommand{\TFPTalphaBarFiveInvMZZ}{1.314468236219047}

% two-loop RG fingerprints
\newcommand{\TFPTalphaThreePeVNum}{0.05243714153896733}
\newcommand{\TFPTalphaThreePeVRelDevPercent}{1.3819515887516742}

\newcommand{\TFPTmuCrossVarphiZeroGeV}{792005.555441517}
\newcommand{\TFPTmuCrosscThreeGeV}{216030451.17273504}

% R^2 / Starobinsky numbers (effective_action_r2)
\newcommand{\TFPTeffRtwoCoeffDimless}{1055668997.2030925063285921958931697786482178917384505743565947290283792722955}
\newcommand{\TFPTeffRtwoCoeffInAction}{527834498.60154625316429609794658488932410894586922528717829736451418963614867773}
\newcommand{\TFPTalphaR}{117864.3622405491}

% A_s for N=55..57
% NOTE: TeX control sequence names cannot contain digits, so we spell out N values.
\newcommand{\TFPTAsNfiftyfive}{2.0162081878496864e-9}
\newcommand{\TFPTAsNfiftysix}{2.09019136432946e-9}
\newcommand{\TFPTAsNfiftyseven}{2.165507571016076e-9}

% --- Omega_b conjecture ---
\newcommand{\TFPTOmegaBPred}{0.048940662665450112200545653760944651480054952073107413870556556782266783965399009}
\newcommand{\TFPTOmegaBRef}{0.049301692328524438476424754994611856173233055557122787616860658651215012327847394}
\newcommand{\TFPTOmegaBRefSigma}{0.0008568114274714093}
\newcommand{\TFPTOmegaBZ}{0.42136420161876677}
\newcommand{\TFPTCoeffFourPiMinusOne}{11.566370614359172953850573533118011536788677597500423283899778369231265625144836}

% --- CKM pipeline (rg-dressed, MZ) ---
\newcommand{\TFPTCKMA}{0.8164965809277260}
\newcommand{\TFPTCKMrho}{0.1666666666666667}
\newcommand{\TFPTCKMeta}{0.3333333333333333}
\newcommand{\TFPTCKMchiTwo}{44.05176935606062}
\newcommand{\TFPTCKMJarlskog}{2.767084e-5}

% --- PMNS baseline + eps scan ---
\newcommand{\TFPTsinSqThetaThirteen}{0.023108435158881104}
\newcommand{\TFPTthetaThirteenDeg}{8.743693}
\newcommand{\TFPTthetaTwelveDeg}{34.322526}
\newcommand{\TFPTepsRad}{0.008861992029474254}
\newcommand{\TFPTepsDeg}{0.5077547413674498}

% =============================================================================
% Tables and plots
% =============================================================================

% --- Core invariants table ---
\newcommand{\TFPTCoreInvariantTable}{%
\begin{table}[H]
\centering
\begin{tabular}{@{}lll@{}}
\toprule
Symbol & Definition & Value \\
\midrule
$c_3$ & $1/(8\pi)$ & \num{\TFPTcThreeNum} \\
$\varphi_{\mathrm{tree}}$ & $1/(6\pi)$ & \num{\TFPTvarphiTreeNum} \\
$\delta_{\mathrm{top}}$ & $48c_3^4 = 3/(256\pi^4)$ & \num{\TFPTdeltaTopNum} \\
$\varphi_0$ & $\varphi_{\mathrm{tree}}+\delta_{\mathrm{top}}$ & \num{\TFPTvarphiZeroNum} \\
$\beta_{\mathrm{rad}}$ & $\varphi_0/(4\pi)$ & \num{\TFPTbetaRadNum} \\
$\beta_{\mathrm{deg}}$ & $(180/\pi)\beta_{\mathrm{rad}}$ & \num{\TFPTbetaDegNum}\,$^\circ$ \\
$g_{a\gamma\gamma}$ & $-4c_3=-1/(2\pi)$ & \num{\TFPTgAggNum} \\
$M/\bar M_{\mathrm{Pl}}$ & $\sqrt{8\pi}\,c_3^4$ & \num{\TFPTMOverMplNum} \\
\bottomrule
\end{tabular}
\caption{TFPT invariants and derived scales, mp.dps=80.}
\end{table}%
}

% --- Alpha audit table (k-sensitivity + delta2) ---
\newcommand{\TFPTAlphaAuditTable}{%
\begin{table}[H]
\centering
\begin{tabular}{@{}llll@{}}
\toprule
Case & $\alpha^{-1}_{\mathrm{pred}}$ & ppm vs CODATA 2022 & Note \\
\midrule
baseline (self-consistent) & \num{\TFPTalphaInvSelfNum} & \num{-0.037000} & single defect \\
two-defect (suite default) & \num{\TFPTalphaInvTwoDefNum} & \num{\TFPTalphaPpmTwoDefNum} & $\delta_2=\delta_{\mathrm{top}}^2$ \\
two-defect (theoryv3, $g=5$) & \num{\TFPTalphaInvGOverFourNum} & \num{\TFPTalphaPpmGOverFourNum} & $\delta_2=\frac54\,\delta_{\mathrm{top}}^2$ \\
targeted match & \num{\TFPTalphaInvCODATANum} & 0 & requires $\delta_2=\num{\TFPTdeltaTwoOptimalNum}$ \\
\bottomrule
\end{tabular}
\caption{Fine-structure constant: baseline and two-defect refinements (suite default vs.\ theoryv3), plus the required second-order term for exact CODATA matching.}
\end{table}%
}

% --- Alpha backreaction exponent sensitivity plot (ppm vs k) ---
\newcommand{\TFPTAlphaSensitivityPlot}{%
\begin{figure}[H]
\centering
\begin{tikzpicture}
\begin{axis}[
    width=0.9\linewidth,
    xlabel={$k$ (backreaction exponent) in $\varphi(\alpha)=\varphi_{\mathrm{tree}}+\delta_{\mathrm{top}}e^{-k\alpha}$},
    ylabel={ppm$(\alpha^{-1}(0))$ vs CODATA 2022},
    grid=both,
    xmin=-0.1, xmax=3.1,
    ymin=-2.2, ymax=4.2,
    xtick={0,1,1.5,2,2.5,3},
    legend pos=south west,
]
\addplot+[mark=*] coordinates {
    (0.0, 3.665371791638453)
    (1.0, 1.8074246412353618)
    (1.5, 0.883514212976573)
    (2.0, -0.03703710120260405)
    (2.5, -0.9542414840493093)
    (3.0, -1.8681110743677323)
};
\addlegendentry{single-defect ($\delta_2=0$)}

\addplot+[only marks, mark=star, mark size=4pt] coordinates {
    (2.0, \TFPTalphaPpmTwoDefNum)
};
\addlegendentry{two-defect at $k=2$ ($\delta_2=\delta_{\mathrm{top}}^2$)}

\addplot[gray, dashed] coordinates {(2.0,-2.2) (2.0,4.2)};
\addlegendentry{$k=2$ (double cover)}

\addplot[gray, dotted] coordinates {(\TFPTkEffNum,-2.2) (\TFPTkEffNum,4.2)};
\addlegendentry{$k_{\mathrm{match}}\approx\num{\TFPTkEffNum}$ (diagnostic)}

\addplot[black!50, dashed] coordinates {(-0.1,0) (3.1,0)};
\end{axis}
\end{tikzpicture}
\caption{Backreaction exponent sensitivity for the CFE closure. The canonical value $k=2$ is fixed by the orientable double cover; $k_{\mathrm{match}}$ is the diagnostic value that would null the CODATA residual in the single-defect truncation.}
\end{figure}%
}

% --- Bounce transfer function data plot (from bounce_perturbations) ---
\newcommand{\TFPTTransferFunctionPlot}{%
\begin{figure}[H]
\centering
\begin{tikzpicture}
\begin{axis}[
    width=0.9\linewidth,
    xmode=log, ymode=log,
    xlabel={$\hat k$},
    ylabel={$T(\hat k)$},
    grid=both,
    legend pos=south east
]
\addplot+[mark=*] coordinates {
    (0.1,    0.0032255772038632126)
    (0.15199110829529336, 0.0035703003593638466)
    (0.23101297000831597, 0.00000000399020695466708)
    (0.35111917342151316, 0.000013156269894230775)
    (0.533669923120631,   0.03550967159062275)
    (0.8111308307896873,  1.6127872665381955)
    (1.2328467394420662,  1.0077896948137592)
    (1.873817422860383,   0.9810730551587763)
    (2.848035868435802,   1.004402539575707)
    (4.328761281083057,   1.0055829625872563)
    (6.5793322465756825,  1.006832617067307)
    (10.0,                1.007882155814805)
};
\addlegendentry{scalar}
\addplot+[mark=square*] coordinates {
    (0.1,    147270.05495615443)
    (0.15199110829529336, 53875.473521817745)
    (0.23101297000831597, 2.1302554245702106)
    (0.35111917342151316, 1.1806392634784857)
    (0.533669923120631,   1.1598025598647947)
    (0.8111308307896873,  1.0319915300605456)
    (1.2328467394420662,  1.035501225860808)
    (1.873817422860383,   1.008365931672867)
    (2.848035868435802,   0.9883585969391012)
    (4.328761281083057,   1.007608072068884)
    (6.5793322465756825,  1.0065676931381393)
    (10.0,                1.0018456357164773)
};
\addlegendentry{tensor}
\end{axis}
\end{tikzpicture}
\caption{Bounce transfer functions from \texttt{bounce\_perturbations}. Note: the two lowest $\hat k$ scalar points were flagged by Wronskian diagnostics in the test report.}
\end{figure}%
}

% --- CKM matrix (absolute values at MZ) ---
\newcommand{\TFPTCKMMatrixBlock}{%
\[
|V_{\mathrm{CKM}}|_{M_Z} \approx
\begin{pmatrix}
0.9744775404 & 0.2244586415 & 0.003441159690 \\
0.2243316946 & 0.9736442171 & 0.04113671446 \\
0.008295429526 & 0.04043830740 & 0.9991476013
\end{pmatrix}.
\]
}

% --- Yukawa textures (UV, used in ckm_full_pipeline) ---
\newcommand{\TFPTYukawaTextureBlock}{%
\begin{tcolorbox}[colback=gray!5, colframe=gray!60, title={\textbf{UV Yukawa texture used in the CKM pipeline}}, fonttitle=\bfseries\small]
The baseline verification pipeline uses a minimal hierarchical UV texture in powers of the TFPT Cabibbo parameter $\lambda$:
\[
\mathrm{diag}(Y_u^{UV})=(\lambda^8,\lambda^4,1),\qquad \mathrm{diag}(Y_d^{UV})=(\lambda^4,\lambda^2,\lambda^2)
\]
Numerically (using the report values $\lambda=\num{0.2244599705189737}$), this is
\[
Y_u^{UV}=\begin{pmatrix}
\num{6.4433424282582535e-06} & 0 & 0\\
0 & \num{2.5383739732864922e-03} & 0\\
0 & 0 & \num{1}
\end{pmatrix},\qquad
|Y_d^{UV}|=\begin{pmatrix}
\num{2.47358843e-03} & \num{1.13087378e-02} & \num{1.73373465e-04}\\
\num{5.69437735e-04} & \num{4.90544140e-02} & \num{2.07256140e-03}\\
\num{2.10569024e-05} & \num{2.03737406e-03} & \num{5.03393326e-02}
\end{pmatrix}.
\]
The down type texture is left rotated by a CKM proxy at the UV scale: $Y_d^{UV}=V_{UV}\,\mathrm{diag}(Y_d^{UV})$ (phase carried by $V_{UV}$).
\end{tcolorbox}%
}

% --- Yukawa texture heatmap (log10 magnitudes) ---
\newcommand{\TFPTYukawaHeatmapPlot}{%
\begin{figure}[H]
\centering
\begin{minipage}{0.48\linewidth}
\centering
\begin{tikzpicture}
\begin{axis}[
    title={$\log_{10}|Y_u^{UV}|$},
    width=\linewidth,
    height=0.9\linewidth,
    xmin=0.5, xmax=3.5,
    ymin=0.5, ymax=3.5,
    xtick={1,2,3}, ytick={1,2,3},
    y dir=reverse,
    axis on top,
    enlargelimits=false,
    colormap/hot,
    point meta min=-6, point meta max=0,
    tick label style={font=\scriptsize},
    title style={font=\small},
    colorbar=false,
]
\addplot[
    matrix plot*,
    point meta=explicit,
] coordinates {
    (1,1) [-5.1908888] (2,1) [-6]         (3,1) [-6]
    (1,2) [-6]         (2,2) [-2.5954444] (3,2) [-6]
    (1,3) [-6]         (2,3) [-6]         (3,3) [0]
};
\end{axis}
\end{tikzpicture}
\end{minipage}\hfill
\begin{minipage}{0.48\linewidth}
\centering
\begin{tikzpicture}
\begin{axis}[
    title={$\log_{10}|Y_d^{UV}|$},
    width=\linewidth,
    height=0.9\linewidth,
    xmin=0.5, xmax=3.5,
    ymin=0.5, ymax=3.5,
    xtick={1,2,3}, ytick={1,2,3},
    y dir=reverse,
    axis on top,
    enlargelimits=false,
    colormap/hot,
    point meta min=-6, point meta max=0,
    tick label style={font=\scriptsize},
    title style={font=\small},
    colorbar,
    colorbar style={ylabel={$\log_{10}|Y|$}, yticklabel style={font=\scriptsize}, ylabel style={font=\scriptsize}},
]
\addplot[
    matrix plot*,
    point meta=explicit,
] coordinates {
    (1,1) [-2.6066726] (2,1) [-1.9465859] (3,1) [-3.7610174]
    (1,2) [-3.2445538] (2,2) [-1.3093219] (3,2) [-2.6834926]
    (1,3) [-4.6766055] (2,3) [-2.6909292] (3,3) [-1.2980925]
};
\end{axis}
\end{tikzpicture}
\end{minipage}
\caption{Heatmap of $\log_{10}|Y_u^{UV}|$ and $\log_{10}|Y_d^{UV}|$ used by the \texttt{ckm\_full\_pipeline} benchmark.}
\end{figure}%
}

% --- Z3 permutation matrix P (PMNS module) ---
\newcommand{\TFPTZthreePermutationMatrix}{%
\[
P =
\begin{pmatrix}
0 & 1 & 0\\
0 & 0 & 1\\
1 & 0 & 0
\end{pmatrix}.
\]
}

% --- PMNS variant table ---
\newcommand{\TFPTPMNSVariantTable}{%
\begin{table}[H]
\centering
\begin{tabular}{@{}lcccc@{}}
\toprule
Variant & $\theta_{23}$ [deg] & $\Delta\theta_{23}$ [deg] & $\delta_{\mathrm{CP}}$ [deg] & $\Delta\delta$ [deg]\\
\midrule
baseline (TM1, Z3) & 45.0000 & 0.0000 & 90.0000 & 0.0000\\
L\_R23(+eps) & 45.5078 & +0.5078 & 90.0000 & 0.0000\\
L\_R23(-eps) & 44.4922 & -0.5078 & 90.0000 & 0.0000\\
R\_R23(+eps) & 45.4243 & +0.4243 & 88.1896 & -1.8104\\
R\_R23(-eps) & 44.5757 & -0.4243 & 91.8104 & +1.8104\\
L\_R23(+eps)*Z3phase & 45.5078 & +0.5078 & 90.0000 & 0.0000\\
L\_R23(-eps)*Z3phase & 44.4922 & -0.5078 & 90.0000 & 0.0000\\
\bottomrule
\end{tabular}
\caption{PMNS Z3-breaking scan with $\epsilon=\varphi_0/6=\num{\TFPTepsRad}$ rad.}
\end{table}%
}

% --- Omega_b table ---
\newcommand{\TFPTOmegaBTable}{%
\begin{table}[H]
\centering
\begin{tabular}{@{}lll@{}}
\toprule
Quantity & Expression & Value \\
\midrule
$\Omega_b$ (TFPT conjecture) & $(4\pi-1)\,\beta_{\mathrm{rad}}$ & \num{\TFPTOmegaBPred} \\
$\Omega_b$ (Planck-derived ref) & $\Omega_b h^2 / h^2$ & \num{\TFPTOmegaBRef} \,$\pm$\,\num{\TFPTOmegaBRefSigma} \\
z-score (ref) & $(\Omega_{b,\mathrm{pred}}-\Omega_{b,\mathrm{ref}})/\sigma$ & \num{\TFPTOmegaBZ} \\
\bottomrule
\end{tabular}
\caption{Baryon density conjecture: numeric prediction and reference comparison.}
\end{table}%
}

% --- k-calibration table (ell mapping) ---
\newcommand{\TFPTkCalibrationTable}{%
\begin{table}[H]
\centering
\begin{tabular}{@{}lccc@{}}
\toprule
Target $\ell$ & required $a_0/a_t$ (scalar) & required $a_0/a_t$ (tensor) & $N=\ln(a_0/a_t)$ (scalar) \\
\midrule
2   & $6.978\times 10^{58}$ & $2.590\times 10^{56}$ & 135.493 \\
30  & $4.652\times 10^{57}$ & $1.727\times 10^{55}$ & 132.785 \\
700 & $1.994\times 10^{56}$ & $7.400\times 10^{53}$ & 129.635 \\
\bottomrule
\end{tabular}
\caption{Mapping the bounce scale to CMB multipoles: required overall scaling $a_0/a_t$ to place features at given $\ell$ (from \texttt{k\_calibration}).}
\end{table}%
}

% --- Torsion bounds table ---
\newcommand{\TFPTTorsionBoundsTable}{%
\begin{table}[H]
\centering
\begin{tabular}{@{}lcc@{}}
\toprule
Component & bound on $|S_\mu|$ [GeV] & mapping note \\
\midrule
$S_T$ & $2.9\times 10^{-27}$ & direct $A_\mu$ bound \\
$S_X$ & $2.1\times 10^{-31}$ & direct $A_\mu$ bound \\
$S_Y$ & $2.5\times 10^{-31}$ & direct $A_\mu$ bound \\
$S_Z$ & $1.0\times 10^{-29}$ & direct $A_\mu$ bound \\
\bottomrule
\end{tabular}
\caption{Component-wise torsion bounds ingested by \texttt{torsion\_bounds\_mapping}; TFPT local vacuum prediction used in the suite is $S_\mu(\mathrm{today})=0$.}
\end{table}%
}
 somewhere after packages (siunitx, booktabs, pgfplots, tikz, float, tcolorbox).

% --- Numerics (high precision) ---
\newcommand{\TFPTcThreeNum}{0.039788735772973833942220940843128590508614911435114112186916836014724199408556634}
\newcommand{\TFPTvarphiTreeNum}{0.053051647697298445256294587790838120678153215246818816249222448019632265878075512}
\newcommand{\TFPTdeltaTopNum}{0.00012030447954708205299788417891154719697454455297232339122853166568719135947912407}
\newcommand{\TFPTvarphiZeroNum}{0.053171952176845527309292471969749667875127759799791139640450979685319457237554636}
\newcommand{\TFPTbetaRadNum}{0.0042312895113954151087468182088050163950728077266837257698944229030526732721556263}
\newcommand{\TFPTbetaDegNum}{0.24243503090092952849243152047951445246859512239848881592836053961070607871616298}
\newcommand{\TFPTgAggNum}{-0.15915494309189533576888376337251436203445964574045644874766734405981889679763422654}
\newcommand{\TFPTMOverMplNum}{0.000012564942083228448969980786638197975376707921748468781310516248914585154899192637}

% alpha: baseline self-consistent and refined two-defect
\newcommand{\TFPTalphaInvSelfNum}{137.03599410158383008148619622913336009409257904171246722475352841304700901323215}
\newcommand{\TFPTalphaInvTwoDefNum}{137.0359981932435869110224146843726714996000114935956757208545827539450219077872}
\newcommand{\TFPTalphaInvGOverFourNum}{137.03599921615843479026171751711397398605568500258186146747743646328176776397918}
\newcommand{\TFPTalphaInvCODATANum}{137.035999177}
\newcommand{\TFPTalphaInvCODATASigma}{0.000000021}
\newcommand{\TFPTalphaPpmTwoDefNum}{-0.0071788173837323352413040312917307695904786165477907034361613291839588627323760663}
\newcommand{\TFPTalphaPpmGOverFourNum}{0.00028575290453192123198929586322751817361006829116847864863933342766974073215940791}
\newcommand{\TFPTalphaZGOverFourNum}{1.8646873709648437}
\newcommand{\TFPTdeltaTwoGOverFourNum}{0.000000018091459748867855014525576506215873427709357460729151458474190949106463016708746}
\newcommand{\TFPTdeltaTwoOptimalNum}{1.79529470934477993569067565574491365815939289267429e-8}
\newcommand{\TFPTdeltaTwoOverDeltaTopNum}{1.49229248636677043432223527146417707405844271475241e-4}
\newcommand{\TFPTkEffNum}{1.97984799374161166800213443338796550879953534211621}

% alpha MSbar at MZ (primary comparison observable; alpha_precision_audit)
\newcommand{\TFPTalphaBarFiveInvMZPredNum}{127.94051574588975237575597444353510021974456512094373031810385305362533848241133}
\newcommand{\TFPTalphaBarFiveInvMZRefNum}{127.93}
\newcommand{\TFPTalphaBarFiveInvMZRefSigma}{0.008}
\newcommand{\TFPTalphaBarFiveInvMZZ}{1.314468236219047}

% two-loop RG fingerprints
\newcommand{\TFPTalphaThreePeVNum}{0.05243714153896733}
\newcommand{\TFPTalphaThreePeVRelDevPercent}{1.3819515887516742}

\newcommand{\TFPTmuCrossVarphiZeroGeV}{792005.555441517}
\newcommand{\TFPTmuCrosscThreeGeV}{216030451.17273504}

% R^2 / Starobinsky numbers (effective_action_r2)
\newcommand{\TFPTeffRtwoCoeffDimless}{1055668997.2030925063285921958931697786482178917384505743565947290283792722955}
\newcommand{\TFPTeffRtwoCoeffInAction}{527834498.60154625316429609794658488932410894586922528717829736451418963614867773}
\newcommand{\TFPTalphaR}{117864.3622405491}

% A_s for N=55..57
% NOTE: TeX control sequence names cannot contain digits, so we spell out N values.
\newcommand{\TFPTAsNfiftyfive}{2.0162081878496864e-9}
\newcommand{\TFPTAsNfiftysix}{2.09019136432946e-9}
\newcommand{\TFPTAsNfiftyseven}{2.165507571016076e-9}

% --- Omega_b conjecture ---
\newcommand{\TFPTOmegaBPred}{0.048940662665450112200545653760944651480054952073107413870556556782266783965399009}
\newcommand{\TFPTOmegaBRef}{0.049301692328524438476424754994611856173233055557122787616860658651215012327847394}
\newcommand{\TFPTOmegaBRefSigma}{0.0008568114274714093}
\newcommand{\TFPTOmegaBZ}{0.42136420161876677}
\newcommand{\TFPTCoeffFourPiMinusOne}{11.566370614359172953850573533118011536788677597500423283899778369231265625144836}

% --- CKM pipeline (rg-dressed, MZ) ---
\newcommand{\TFPTCKMA}{0.8164965809277260}
\newcommand{\TFPTCKMrho}{0.1666666666666667}
\newcommand{\TFPTCKMeta}{0.3333333333333333}
\newcommand{\TFPTCKMchiTwo}{44.05176935606062}
\newcommand{\TFPTCKMJarlskog}{2.767084e-5}

% --- PMNS baseline + eps scan ---
\newcommand{\TFPTsinSqThetaThirteen}{0.023108435158881104}
\newcommand{\TFPTthetaThirteenDeg}{8.743693}
\newcommand{\TFPTthetaTwelveDeg}{34.322526}
\newcommand{\TFPTepsRad}{0.008861992029474254}
\newcommand{\TFPTepsDeg}{0.5077547413674498}

% =============================================================================
% Tables and plots
% =============================================================================

% --- Core invariants table ---
\newcommand{\TFPTCoreInvariantTable}{%
\begin{table}[H]
\centering
\begin{tabular}{@{}lll@{}}
\toprule
Symbol & Definition & Value \\
\midrule
$c_3$ & $1/(8\pi)$ & \num{\TFPTcThreeNum} \\
$\varphi_{\mathrm{tree}}$ & $1/(6\pi)$ & \num{\TFPTvarphiTreeNum} \\
$\delta_{\mathrm{top}}$ & $48c_3^4 = 3/(256\pi^4)$ & \num{\TFPTdeltaTopNum} \\
$\varphi_0$ & $\varphi_{\mathrm{tree}}+\delta_{\mathrm{top}}$ & \num{\TFPTvarphiZeroNum} \\
$\beta_{\mathrm{rad}}$ & $\varphi_0/(4\pi)$ & \num{\TFPTbetaRadNum} \\
$\beta_{\mathrm{deg}}$ & $(180/\pi)\beta_{\mathrm{rad}}$ & \num{\TFPTbetaDegNum}\,$^\circ$ \\
$g_{a\gamma\gamma}$ & $-4c_3=-1/(2\pi)$ & \num{\TFPTgAggNum} \\
$M/\bar M_{\mathrm{Pl}}$ & $\sqrt{8\pi}\,c_3^4$ & \num{\TFPTMOverMplNum} \\
\bottomrule
\end{tabular}
\caption{TFPT invariants and derived scales, mp.dps=80.}
\end{table}%
}

% --- Alpha audit table (k-sensitivity + delta2) ---
\newcommand{\TFPTAlphaAuditTable}{%
\begin{table}[H]
\centering
\small
\setlength{\tabcolsep}{4pt}
\begin{tabularx}{\linewidth}{@{}lllX@{}}
\toprule
Case & $\alpha^{-1}_{\mathrm{pred}}$ & ppm vs CODATA 2022 & Note \\
\midrule
one-defect truncation (baseline) & \num[round-mode=places,round-precision=12]{\TFPTalphaInvSelfNum} & \num[round-mode=places,round-precision=6]{-0.037000} & $\delta_2=0$ \\
two-defect template ($g=4$) & \num[round-mode=places,round-precision=12]{\TFPTalphaInvTwoDefNum} & \num[round-mode=places,round-precision=6]{\TFPTalphaPpmTwoDefNum} & $\delta_2=\delta_{\mathrm{top}}^2$ \\
two-defect derived ($g=5$) & \num[round-mode=places,round-precision=12]{\TFPTalphaInvGOverFourNum} & \num[round-mode=places,round-precision=6]{\TFPTalphaPpmGOverFourNum} & $\delta_2=\frac54\,\delta_{\mathrm{top}}^2$ \\
targeted match & \num[round-mode=places,round-precision=9]{\TFPTalphaInvCODATANum} & 0 & requires $\delta_2=\num[round-mode=figures,round-precision=6]{\TFPTdeltaTwoOptimalNum}$ \\
\bottomrule
\end{tabularx}
\caption{Fine-structure constant: baseline and two-defect refinements (suite default vs.\ theoryv3), plus the required second-order term for exact CODATA matching.}
\end{table}%
}

% --- Alpha backreaction exponent sensitivity plot (ppm vs k) ---
\newcommand{\TFPTAlphaSensitivityPlot}{%
\begin{figure}[H]
\centering
\begin{tikzpicture}
\begin{axis}[
    width=0.9\linewidth,
    xlabel={$k$ (backreaction exponent) in $\varphi(\alpha)=\varphi_{\mathrm{tree}}+\delta_{\mathrm{top}}e^{-k\alpha}$},
    ylabel={ppm$(\alpha^{-1}(0))$ vs CODATA 2022},
    grid=both,
    xmin=-0.1, xmax=3.1,
    ymin=-2.2, ymax=4.2,
    xtick={0,1,1.5,2,2.5,3},
    legend pos=south west,
]
\addplot+[mark=*] coordinates {
    (0.0, 3.665371791638453)
    (1.0, 1.8074246412353618)
    (1.5, 0.883514212976573)
    (2.0, -0.03703710120260405)
    (2.5, -0.9542414840493093)
    (3.0, -1.8681110743677323)
};
\addlegendentry{single-defect ($\delta_2=0$)}

\addplot+[only marks, mark=star, mark size=4pt] coordinates {
    (2.0, \TFPTalphaPpmTwoDefNum)
};
\addlegendentry{two-defect template ($g=4$: $\delta_2=\delta_{\mathrm{top}}^2$)}

\addplot+[only marks, mark=diamond*, mark size=3.5pt] coordinates {
    (2.0, \TFPTalphaPpmGOverFourNum)
};
\addlegendentry{derived partition ($g=5$: $\delta_2=\tfrac54\,\delta_{\mathrm{top}}^2$)}

\addplot[gray, dashed] coordinates {(2.0,-2.2) (2.0,4.2)};
\addlegendentry{$k=2$ (double cover)}

\addplot[gray, dotted] coordinates {(\TFPTkEffNum,-2.2) (\TFPTkEffNum,4.2)};
\addlegendentry{$k_{\mathrm{match}}\approx\num{\TFPTkEffNum}$ (diagnostic)}

\addplot[black!50, dashed] coordinates {(-0.1,0) (3.1,0)};
\end{axis}
\end{tikzpicture}
\caption{Backreaction exponent sensitivity for the CFE closure. The canonical value $k=2$ is fixed by the orientable double cover; $k_{\mathrm{match}}$ is the diagnostic value that would null the CODATA residual in the single-defect truncation.}
\end{figure}%
}

% --- Bounce transfer function data plot (from bounce_perturbations) ---
\newcommand{\TFPTTransferFunctionPlot}{%
\begin{figure}[H]
\centering
\begin{tikzpicture}
\begin{axis}[
    width=0.9\linewidth,
    xmode=log, ymode=log,
    xlabel={$\hat k$},
    ylabel={$T(\hat k)$},
    grid=both,
    legend pos=south east
]
\addplot+[only marks, mark=x, mark size=3.0pt, red] coordinates {
    (0.1,    0.0032255772038632126)
    (0.15199110829529336, 0.0035703003593638466)
};
\addlegendentry{scalar (flagged by Wronskian diagnostic)}
\addplot+[mark=*] coordinates {
    (0.23101297000831597, 0.00000000399020695466708)
    (0.35111917342151316, 0.000013156269894230775)
    (0.533669923120631,   0.03550967159062275)
    (0.8111308307896873,  1.6127872665381955)
    (1.2328467394420662,  1.0077896948137592)
    (1.873817422860383,   0.9810730551587763)
    (2.848035868435802,   1.004402539575707)
    (4.328761281083057,   1.0055829625872563)
    (6.5793322465756825,  1.006832617067307)
    (10.0,                1.007882155814805)
};
\addlegendentry{scalar (kept)}
\addplot+[mark=square*] coordinates {
    (0.1,    147270.05495615443)
    (0.15199110829529336, 53875.473521817745)
    (0.23101297000831597, 2.1302554245702106)
    (0.35111917342151316, 1.1806392634784857)
    (0.533669923120631,   1.1598025598647947)
    (0.8111308307896873,  1.0319915300605456)
    (1.2328467394420662,  1.035501225860808)
    (1.873817422860383,   1.008365931672867)
    (2.848035868435802,   0.9883585969391012)
    (4.328761281083057,   1.007608072068884)
    (6.5793322465756825,  1.0065676931381393)
    (10.0,                1.0018456357164773)
};
\addlegendentry{tensor}
\end{axis}
\end{tikzpicture}
\caption{Bounce transfer functions from \texttt{bounce\_perturbations}. The two lowest-$\hat k$ scalar points are shown but marked as flagged by the Wronskian diagnostic (numerical consistency check).}
\end{figure}%
}

% --- CKM matrix (absolute values at MZ) ---
\newcommand{\TFPTCKMMatrixBlock}{%
\[
|V_{\mathrm{CKM}}|_{M_Z} \approx
\begin{pmatrix}
0.9744775404 & 0.2244586415 & 0.003441159690 \\
0.2243316946 & 0.9736442171 & 0.04113671446 \\
0.008295429526 & 0.04043830740 & 0.9991476013
\end{pmatrix}.
\]
}

% --- Yukawa textures (UV, used in ckm_full_pipeline) ---
\newcommand{\TFPTYukawaTextureBlock}{%
\begin{tcolorbox}[colback=gray!5, colframe=gray!60, breakable,
title={\textbf{UV Yukawa texture used in the CKM pipeline}}, fonttitle=\bfseries\small, fontupper=\small]
The baseline verification pipeline uses a minimal hierarchical UV texture in powers of the TFPT Cabibbo parameter $\lambda$:
\[
\mathrm{diag}(Y_u^{UV})=(\lambda^8,\lambda^4,1),\qquad \mathrm{diag}(Y_d^{UV})=(\lambda^4,\lambda^2,\lambda^2)
\]
Numerically (using the report values $\lambda=\num{0.2244599705189737}$), this is
\begingroup
\sisetup{round-mode=figures, round-precision=6}
\setlength{\arraycolsep}{3pt}
\[
Y_u^{UV}=\begin{pmatrix}
\num{6.4433424282582535e-06} & 0 & 0\\
0 & \num{2.5383739732864922e-03} & 0\\
0 & 0 & \num{1}
\end{pmatrix},\qquad
|Y_d^{UV}|=\begin{pmatrix}
\num{2.47358843e-03} & \num{1.13087378e-02} & \num{1.73373465e-04}\\
\num{5.69437735e-04} & \num{4.90544140e-02} & \num{2.07256140e-03}\\
\num{2.10569024e-05} & \num{2.03737406e-03} & \num{5.03393326e-02}
\end{pmatrix}.
\]
\endgroup
The down type texture is left rotated by a CKM proxy at the UV scale: $Y_d^{UV}=V_{UV}\,\mathrm{diag}(Y_d^{UV})$ (phase carried by $V_{UV}$).
\end{tcolorbox}%
}

% --- Yukawa texture heatmap (log10 magnitudes) ---
\newcommand{\TFPTYukawaHeatmapPlot}{%
\begin{figure}[H]
\centering
\begin{minipage}{0.48\linewidth}
\centering
\begin{tikzpicture}
\begin{axis}[
    title={$\log_{10}|Y_u^{UV}|$},
    width=\linewidth,
    height=0.9\linewidth,
    xmin=0.5, xmax=3.5,
    ymin=0.5, ymax=3.5,
    xtick={1,2,3}, ytick={1,2,3},
    y dir=reverse,
    axis on top,
    enlargelimits=false,
    colormap/hot,
    point meta min=-6, point meta max=0,
    tick label style={font=\scriptsize},
    title style={font=\small},
    colorbar=false,
]
\addplot[
    matrix plot*,
    mesh/cols=3,
    point meta=explicit,
] coordinates {
    (1,1) [-5.1908888] (2,1) [-6]         (3,1) [-6]
    (1,2) [-6]         (2,2) [-2.5954444] (3,2) [-6]
    (1,3) [-6]         (2,3) [-6]         (3,3) [0]
};
\end{axis}
\end{tikzpicture}
\end{minipage}\hfill
\begin{minipage}{0.48\linewidth}
\centering
\begin{tikzpicture}
\begin{axis}[
    title={$\log_{10}|Y_d^{UV}|$},
    width=\linewidth,
    height=0.9\linewidth,
    xmin=0.5, xmax=3.5,
    ymin=0.5, ymax=3.5,
    xtick={1,2,3}, ytick={1,2,3},
    y dir=reverse,
    axis on top,
    enlargelimits=false,
    colormap/hot,
    point meta min=-6, point meta max=0,
    tick label style={font=\scriptsize},
    title style={font=\small},
    colorbar,
    colorbar style={ylabel={$\log_{10}|Y|$}, yticklabel style={font=\scriptsize}, ylabel style={font=\scriptsize}},
]
\addplot[
    matrix plot*,
    mesh/cols=3,
    point meta=explicit,
] coordinates {
    (1,1) [-2.6066726] (2,1) [-1.9465859] (3,1) [-3.7610174]
    (1,2) [-3.2445538] (2,2) [-1.3093219] (3,2) [-2.6834926]
    (1,3) [-4.6766055] (2,3) [-2.6909292] (3,3) [-1.2980925]
};
\end{axis}
\end{tikzpicture}
\end{minipage}
\caption{Heatmap of $\log_{10}|Y_u^{UV}|$ and $\log_{10}|Y_d^{UV}|$ used by the \texttt{ckm\_full\_pipeline} benchmark.}
\end{figure}%
}

% --- Z3 permutation matrix P (PMNS module) ---
\newcommand{\TFPTZthreePermutationMatrix}{%
\[
P =
\begin{pmatrix}
0 & 1 & 0\\
0 & 0 & 1\\
1 & 0 & 0
\end{pmatrix}.
\]
}

% --- PMNS variant table ---
\newcommand{\TFPTPMNSVariantTable}{%
\begin{table}[H]
\centering
\begin{tabular}{@{}lcccc@{}}
\toprule
Variant & $\theta_{23}$ [deg] & $\Delta\theta_{23}$ [deg] & $\delta_{\mathrm{CP}}$ [deg] & $\Delta\delta$ [deg]\\
\midrule
baseline (TM1, Z3) & 45.0000 & 0.0000 & 90.0000 & 0.0000\\
L\_R23(+eps) & 45.5078 & +0.5078 & 90.0000 & 0.0000\\
L\_R23(-eps) & 44.4922 & -0.5078 & 90.0000 & 0.0000\\
R\_R23(+eps) & 45.4243 & +0.4243 & 88.1896 & -1.8104\\
R\_R23(-eps) & 44.5757 & -0.4243 & 91.8104 & +1.8104\\
L\_R23(+eps)*Z3phase & 45.5078 & +0.5078 & 90.0000 & 0.0000\\
L\_R23(-eps)*Z3phase & 44.4922 & -0.5078 & 90.0000 & 0.0000\\
\bottomrule
\end{tabular}
\caption{PMNS Z3-breaking scan with $\epsilon=\varphi_0/6=\num{\TFPTepsRad}$ rad.}
\end{table}%
}

% --- Omega_b table ---
\newcommand{\TFPTOmegaBTable}{%
\begin{table}[H]
\centering
\begin{tabular}{@{}lll@{}}
\toprule
Quantity & Expression & Value \\
\midrule
$\Omega_b$ (TFPT conjecture) & $(4\pi-1)\,\beta_{\mathrm{rad}}$ & \num{\TFPTOmegaBPred} \\
$\Omega_b$ (Planck-derived ref) & $\Omega_b h^2 / h^2$ & \num{\TFPTOmegaBRef} \,$\pm$\,\num{\TFPTOmegaBRefSigma} \\
z-score (ref) & $(\Omega_{b,\mathrm{pred}}-\Omega_{b,\mathrm{ref}})/\sigma$ & \num[round-mode=places,round-precision=3]{\TFPTOmegaBZ} \\
\bottomrule
\end{tabular}
\caption{Baryon density conjecture: numeric prediction and reference comparison.}
\end{table}%
}

% --- k-calibration table (ell mapping) ---
\newcommand{\TFPTkCalibrationTable}{%
\begin{table}[H]
\centering
\begin{tabular}{@{}lccc@{}}
\toprule
Target $\ell$ & required $a_0/a_t$ (scalar) & required $a_0/a_t$ (tensor) & $N=\ln(a_0/a_t)$ (scalar) \\
\midrule
2   & $6.978\times 10^{58}$ & $2.590\times 10^{56}$ & 135.493 \\
30  & $4.652\times 10^{57}$ & $1.727\times 10^{55}$ & 132.785 \\
700 & $1.994\times 10^{56}$ & $7.400\times 10^{53}$ & 129.635 \\
\bottomrule
\end{tabular}
\caption{Mapping the bounce scale to CMB multipoles: required overall scaling $a_0/a_t$ to place features at given $\ell$ (from \texttt{k\_calibration}).}
\end{table}%
}

% --- Torsion bounds table ---
\newcommand{\TFPTTorsionBoundsTable}{%
\begin{table}[H]
\centering
\begin{tabular}{@{}lcc@{}}
\toprule
Component & bound on $|S_\mu|$ [GeV] & mapping note \\
\midrule
$S_T$ & $2.9\times 10^{-27}$ & direct $A_\mu$ bound \\
$S_X$ & $2.1\times 10^{-31}$ & direct $A_\mu$ bound \\
$S_Y$ & $2.5\times 10^{-31}$ & direct $A_\mu$ bound \\
$S_Z$ & $1.0\times 10^{-29}$ & direct $A_\mu$ bound \\
\bottomrule
\end{tabular}
\caption{Component-wise torsion bounds ingested by \texttt{torsion\_bounds\_mapping}; TFPT local vacuum prediction used in the suite is $S_\mu(\mathrm{today})=0$.}
\end{table}%
}

% ===== SUITE MANIFEST (AUTO-GENERATED; reviewer-proof counts) =====
% Produced by: tfpt-suite/tfpt_suite/suite_manifest.py (scans out/ + out_physics/).
% Provides: \TFPTSuite* macros and \TFPTSuiteProblemModules.
% Inlined (standalone build): formerly % AUTO-GENERATED by tfpt_suite.suite_manifest.py — do not edit by hand.
% generated_at_utc: 2026-01-28T08:03:20.850002+00:00

\newcommand{\TFPTSuiteModulesTotal}{132}
\newcommand{\TFPTSuiteUniqueModulesTotal}{66}
\newcommand{\TFPTSuiteUniqueModulesEngineering}{66}
\newcommand{\TFPTSuiteUniqueModulesPhysics}{66}
\newcommand{\TFPTSuiteChecksTotal}{688}
\newcommand{\TFPTSuiteModulesWithWarn}{6}
\newcommand{\TFPTSuiteModulesWithFail}{1}
\newcommand{\TFPTSuitePlotsExpectedTotal}{46}
\newcommand{\TFPTSuitePlotsPresentTotal}{46}
\newcommand{\TFPTSuitePlotsMissingTotal}{0}
\newcommand{\TFPTSuitePlotsInvalidTotal}{0}

% Problem modules (WARN/FAIL) and plot inventory issues:
\newcommand{\TFPTSuiteProblemModules}{%
\begin{itemize}
\item \textbf{FAIL modules:}
\begin{itemize}
\item \texttt{torsion\_condensate} (physics): checks=8, WARN=2, FAIL=1 (out: \texttt{out\_physics/torsion\_condensate})
\end{itemize}
\item \textbf{WARN modules:}
\begin{itemize}
\item \texttt{alpha\_on\_shell\_bridge} (engineering): checks=7, WARN=1, FAIL=0 (out: \texttt{out/alpha\_on\_shell\_bridge})
\item \texttt{matching\_finite\_pieces} (engineering): checks=5, WARN=1, FAIL=0 (out: \texttt{out/matching\_finite\_pieces})
\item \texttt{torsion\_condensate} (engineering): checks=8, WARN=3, FAIL=0 (out: \texttt{out/torsion\_condensate})
\item \texttt{alpha\_on\_shell\_bridge} (physics): checks=7, WARN=1, FAIL=0 (out: \texttt{out\_physics/alpha\_on\_shell\_bridge})
\item \texttt{matching\_finite\_pieces} (physics): checks=5, WARN=1, FAIL=0 (out: \texttt{out\_physics/matching\_finite\_pieces})
\item \texttt{torsion\_condensate} (physics): checks=8, WARN=2, FAIL=1 (out: \texttt{out\_physics/torsion\_condensate})
\end{itemize}
\end{itemize}
}%


% -----------------------------------------------------------------------------
% AUTO-GENERATED by tfpt_suite.suite_manifest.py — do not edit by hand.
% generated_at_utc: 2026-01-28T08:03:20.850002+00:00

\newcommand{\TFPTSuiteModulesTotal}{132}
\newcommand{\TFPTSuiteUniqueModulesTotal}{66}
\newcommand{\TFPTSuiteUniqueModulesEngineering}{66}
\newcommand{\TFPTSuiteUniqueModulesPhysics}{66}
\newcommand{\TFPTSuiteChecksTotal}{688}
\newcommand{\TFPTSuiteModulesWithWarn}{6}
\newcommand{\TFPTSuiteModulesWithFail}{1}
\newcommand{\TFPTSuitePlotsExpectedTotal}{46}
\newcommand{\TFPTSuitePlotsPresentTotal}{46}
\newcommand{\TFPTSuitePlotsMissingTotal}{0}
\newcommand{\TFPTSuitePlotsInvalidTotal}{0}

% Problem modules (WARN/FAIL) and plot inventory issues:
\newcommand{\TFPTSuiteProblemModules}{%
\begin{itemize}
\item \textbf{FAIL modules:}
\begin{itemize}
\item \texttt{torsion\_condensate} (physics): checks=8, WARN=2, FAIL=1 (out: \texttt{out\_physics/torsion\_condensate})
\end{itemize}
\item \textbf{WARN modules:}
\begin{itemize}
\item \texttt{alpha\_on\_shell\_bridge} (engineering): checks=7, WARN=1, FAIL=0 (out: \texttt{out/alpha\_on\_shell\_bridge})
\item \texttt{matching\_finite\_pieces} (engineering): checks=5, WARN=1, FAIL=0 (out: \texttt{out/matching\_finite\_pieces})
\item \texttt{torsion\_condensate} (engineering): checks=8, WARN=3, FAIL=0 (out: \texttt{out/torsion\_condensate})
\item \texttt{alpha\_on\_shell\_bridge} (physics): checks=7, WARN=1, FAIL=0 (out: \texttt{out\_physics/alpha\_on\_shell\_bridge})
\item \texttt{matching\_finite\_pieces} (physics): checks=5, WARN=1, FAIL=0 (out: \texttt{out\_physics/matching\_finite\_pieces})
\item \texttt{torsion\_condensate} (physics): checks=8, WARN=2, FAIL=1 (out: \texttt{out\_physics/torsion\_condensate})
\end{itemize}
\end{itemize}
}%

% ===== THEORYV3 MANIFEST (AUTO-GENERATED; compressed evidence layer) =====
% Produced by: tfpt-suite/theoryv3/theoryv3_manifest.py (scans theoryv3/out).
% Provides: \TFPTTheoryVThree* macros and a compact evidence table.
% Inlined (standalone build): formerly % AUTO-GENERATED by tfpt-suite/theoryv3/theoryv3_manifest.py — do not edit by hand.
% generated_at_utc: 2026-01-28T08:00:46.431406+00:00

\newcommand{\TFPTTheoryVThreeModulesTotal}{14}
\newcommand{\TFPTTheoryVThreeChecksTotal}{50}
\newcommand{\TFPTTheoryVThreeChecksPassTotal}{40}
\newcommand{\TFPTTheoryVThreeChecksWarnTotal}{10}
\newcommand{\TFPTTheoryVThreeChecksFailTotal}{0}
\newcommand{\TFPTTheoryVThreeModulesWithWarn}{4}
\newcommand{\TFPTTheoryVThreeModulesWithFail}{0}
\newcommand{\TFPTTheoryVThreePlotsExpectedTotal}{16}
\newcommand{\TFPTTheoryVThreePlotsPresentTotal}{16}
\newcommand{\TFPTTheoryVThreePlotsMissingTotal}{0}
\newcommand{\TFPTTheoryVThreePlotsInvalidTotal}{0}


% -----------------------------------------------------------------------------
% AUTO-GENERATED by tfpt-suite/theoryv3/theoryv3_manifest.py — do not edit by hand.
% generated_at_utc: 2026-01-28T08:00:46.431406+00:00

\newcommand{\TFPTTheoryVThreeModulesTotal}{14}
\newcommand{\TFPTTheoryVThreeChecksTotal}{50}
\newcommand{\TFPTTheoryVThreeChecksPassTotal}{40}
\newcommand{\TFPTTheoryVThreeChecksWarnTotal}{10}
\newcommand{\TFPTTheoryVThreeChecksFailTotal}{0}
\newcommand{\TFPTTheoryVThreeModulesWithWarn}{4}
\newcommand{\TFPTTheoryVThreeModulesWithFail}{0}
\newcommand{\TFPTTheoryVThreePlotsExpectedTotal}{16}
\newcommand{\TFPTTheoryVThreePlotsPresentTotal}{16}
\newcommand{\TFPTTheoryVThreePlotsMissingTotal}{0}
\newcommand{\TFPTTheoryVThreePlotsInvalidTotal}{0}

% ===== COLORS =====
\definecolor{axiomcolor}{RGB}{0,80,160}
\definecolor{theoremcolor}{RGB}{0,120,60}
\definecolor{corollarycolor}{RGB}{120,60,0}
\definecolor{definitioncolor}{RGB}{100,0,100}
\definecolor{topologyblue}{RGB}{50,100,180}
\definecolor{geometrygreen}{RGB}{50,150,80}
\definecolor{quantumorange}{RGB}{220,120,40}
\definecolor{observablepurple}{RGB}{140,60,160}

% ===== THEOREM ENVIRONMENTS =====
\newtcbtheorem[number within=section]{axiom}{Axiom}{
    colback=axiomcolor!5,
    colframe=axiomcolor,
    fonttitle=\bfseries,
    separator sign={.},
    description delimiters parenthesis
}{ax}

\newtcbtheorem[number within=section]{theorem}{Theorem}{
    colback=theoremcolor!5,
    colframe=theoremcolor,
    fonttitle=\bfseries,
    separator sign={.},
    description delimiters parenthesis
}{thm}

\newtcbtheorem[number within=section]{corollary}{Corollary}{
    colback=corollarycolor!5,
    colframe=corollarycolor,
    fonttitle=\bfseries,
    separator sign={.},
    description delimiters parenthesis
}{cor}

\newtcbtheorem[number within=section]{lemma}{Lemma}{
    colback=theoremcolor!5,
    colframe=theoremcolor,
    fonttitle=\bfseries,
    separator sign={.},
    description delimiters parenthesis
}{lem}

\newtcbtheorem[number within=section]{definition}{Definition}{
    colback=definitioncolor!5,
    colframe=definitioncolor,
    fonttitle=\bfseries,
    separator sign={.},
    description delimiters parenthesis
}{def}

\newenvironment{remark}{\par\smallskip\noindent\textbf{Remark.}}{\par\smallskip}

% ===== COMMANDS =====
\newcommand{\Mpl}{\bar{M}_{\mathrm{Pl}}}
\newcommand{\cthree}{c_3}
\newcommand{\phiz}{\varphi_0}
\newcommand{\phitree}{\varphi_{\mathrm{tree}}}
\newcommand{\deltatop}{\delta_{\mathrm{top}}}
\newcommand{\bone}{b_1}
\newcommand{\gagg}{g_{a\gamma\gamma}}

% ===== TITLE =====
\title{%
    \textbf{Topological Fixed Point Theory (TFPT)}\\[0.5em]
    \Large A Parameter-Free Derivation of the Fine-Structure Constant\\
    from Topology, Geometry, and Quantum Consistency
}
\author{%
    Stefan Hamann \and Alessandro Rizzo
}
\date{Version 2.5 --- 27 January 2026}

\begin{document}
\maketitle

\begin{abstract}
We present the complete axiomatic foundation of Topological Fixed Point Theory (TFPT), 
which derives the fine-structure constant $\alpha$ and other fundamental quantities 
from eight minimal assumptions rooted in topology, geometry, and quantum field theory. 
The theory rests on two fundamental invariants: the topological coupling 
$\cthree = 1/(8\pi)$ from 11D Chern--Simons quantization and the geometric scale 
$\phiz = 1/(6\pi) + 3/(256\pi^4)$ from M\"obius fiber geometry. 
These invariants, combined with the Standard Model abelian trace $\bone = 41/10$, 
yield a cubic fixed-point equation (CFE) whose unique positive root is fixed by geometric self-consistency on the orientable double cover ($k=2$). 
A two-defect correction derived from SU(5) holonomy degeneracy ($g=3+2=5$) yields $\delta_2=\frac54\,\delta_{\mathrm{top}}^2$ and 
$\alpha^{-1}(0)=\num[round-mode=places,round-precision=12]{\TFPTalphaInvGOverFourNum}$ (CODATA $z=\num[round-mode=places,round-precision=3]{\TFPTalphaZGOverFourNum}$). 
The one-defect truncation baseline yields $\alpha^{-1}(0)=\num[round-mode=places,round-precision=10]{\TFPTalphaInvSelfNum}$ (a $-0.037$ ppm deviation, but excluded at CODATA precision) and is reported only for series control and sensitivity diagnostics.
Under the declared $\overline{\mathrm{MS}}$-at-$M_Z$ comparison policy, this implies 
$\overline{\alpha}^{(5)}(M_Z)^{-1}=\num[round-mode=places,round-precision=6]{\TFPTalphaBarFiveInvMZPredNum}$ (PDG~2024 $z=\num[round-mode=places,round-precision=3]{\TFPTalphaBarFiveInvMZZ}$). 
The same kernel fixes the axion--photon coupling $g_{a\gamma\gamma}=-4c_3$ and predicts cosmic birefringence 
$\beta=\phiz/(4\pi)=\num[round-mode=places,round-precision=4]{\TFPTbetaDegNum}^\circ$, consistent with Planck PR4 observations at the 0.5--1.7$\sigma$ level.
\end{abstract}

%=============================================================================
% KERNEL AND DISCRETE GRAMMAR (Compiler view)
%=============================================================================
\section*{Kernel and Discrete Grammar}
\addcontentsline{toc}{section}{Kernel and Discrete Grammar}

\begin{tcolorbox}[colback=gray!4, colframe=gray!60, title={\textbf{Kernel specification (inputs $\to$ outputs)}}, fonttitle=\bfseries\small]
\textbf{Primitive constant:} $\pi$.

\textbf{Structural input (fixed, not fitted):} the Standard Model abelian trace $\bone=41/10$ (determined by SM particle content and GUT normalization; not a TFPT knob).

\textbf{Notation:} we denote the geometric scale by $\varphi_0$ throughout (suite key: \texttt{varphi0}); no separate $\phi_0$ object is introduced.

\textbf{Kernel invariants (computed):}
\[
c_3=\frac{1}{8\pi},\qquad 
\varphi_{\mathrm{tree}}=\frac{1}{6\pi},\qquad
\delta_{\mathrm{top}}=48c_3^4=\frac{3}{256\pi^4},\qquad
\varphi_0=\varphi_{\mathrm{tree}}+\delta_{\mathrm{top}}.
\]

\textbf{Discrete grammar (no continuous fits):}
\begin{itemize}[leftmargin=1.5em, topsep=2pt, itemsep=1pt]
    \item \textbf{Double cover class:} $k=2$ (exactly two sheets) fixes the backreaction exponent in $\varphi_0(\alpha)$.
    \item \textbf{Holonomy degeneracy:} SU(5) hypercharge holonomy splits into eigenspaces of dimensions $\{3,2\}$, hence $g:=3+2=5$ (a discrete degeneracy label, \emph{not} a gauge coupling).
\end{itemize}

\textbf{Kernel outputs (examples):}
\begin{itemize}[leftmargin=1.5em, topsep=2pt, itemsep=1pt]
    \item $g_{a\gamma\gamma}=-4c_3=-1/(2\pi)$,
    \item $\beta_{\mathrm{rad}}=\varphi_0/(4\pi)$ and $\beta_{\mathrm{deg}}=(180/\pi)\beta_{\mathrm{rad}}$,
    \item two-defect weight $\delta_2=\tfrac{g}{4}\delta_{\mathrm{top}}^2$ (so $\delta_2/\delta_{\mathrm{top}}^2=1.25$),
    \item E$_8$ damping anchor $\gamma(0)=\tfrac{g}{g+1}=\tfrac56$.
\end{itemize}

\textbf{Evidence principle:} Every headline number in this paper is either proved from A1--A8 or reproduced by a named verification module in \codepath{tfpt-suite/} (and \codepath{tfpt-suite/theoryv3/} for discrete audits).
\end{tcolorbox}

\begin{figure}[H]
\centering
\begin{tikzpicture}[
    box/.style={draw, rounded corners, minimum width=3.2cm, minimum height=1.05cm, align=center, font=\small},
    arrow/.style={-{Stealth[length=3mm]}, thick}
]
    \node[box, fill=topologyblue!15] (kernel) {Kernel\\$\pi\to(c_3,\varphi_0)$\\SM content: $b_1$\\grammar: $k=2$, $g=5$};
    \node[box, right=1.2cm of kernel, fill=quantumorange!15] (couplings) {Couplings\\CFE $\Rightarrow \alpha$\\policy: $\overline{\alpha}^{(5)}(M_Z)$};
    \node[box, right=1.2cm of couplings, fill=geometrygreen!15] (scales) {Scales\\$R^2$, $f_a$, $\rho_\Lambda$ candidates};
    \node[box, right=1.2cm of scales, fill=observablepurple!15] (flavor) {Flavor anchors\\$\lambda$, CKM/PMNS};
    \node[box, right=1.2cm of flavor, fill=observablepurple!15] (cosmo) {Cosmology anchors\\$\Omega_b$, $n_s,r,A_s$, $k\to\ell$};

    \draw[arrow] (kernel) -- (couplings);
    \draw[arrow] (couplings) -- (scales);
    \draw[arrow] (scales) -- (flavor);
    \draw[arrow] (flavor) -- (cosmo);
\end{tikzpicture}
\caption{Compiler view: TFPT separates a minimal kernel (discrete invariants) from policy layers (scheme/matching, thresholds, $k\to\ell$ mapping) and produces finite candidate sets rather than continuous fits.}
\label{fig:compiler}
\end{figure}

\begin{table}[H]
\centering
\begin{tabular}{@{}lll@{}}
\toprule
Output & Value & Evidence (module) \\
\midrule
$\alpha^{-1}(0)$ (two-defect, $g=5$) & \num[round-mode=places,round-precision=12]{\TFPTalphaInvGOverFourNum} & \texttt{theoryv3/out/defect\_partition\_g5\_audit} \\
$\overline{\alpha}^{(5)}(M_Z)^{-1}$ ($\overline{\mathrm{MS}}$ policy) & \num[round-mode=places,round-precision=6]{\TFPTalphaBarFiveInvMZPredNum} & \texttt{out/alpha\_precision\_audit} \\
$\beta_{\mathrm{deg}}$ (birefringence) & \num[round-mode=places,round-precision=4]{\TFPTbetaDegNum}$^\circ$ & \texttt{out/birefringence\_tomography} \\
$\Omega_b$ (APS seam candidate) & \num[round-mode=places,round-precision=6]{\TFPTOmegaBPred} & \texttt{out/omega\_b\_conjecture\_scan} \\
$\sin^2\theta_{13}$ (TM1/Z$_3$) & \num[round-mode=places,round-precision=6]{\TFPTsinSqThetaThirteen} & \texttt{out/pmns\_z3\_breaking} \\
\bottomrule
\end{tabular}
\caption{Derived constants and anchors from the kernel. Each entry is either proved in-text or reproduced by a named suite module (paths in \codepath{tfpt-suite/out/...} and \codepath{tfpt-suite/out_physics/...}, or \codepath{tfpt-suite/theoryv3/out/...}).}
\label{tab:constant-factory}
\end{table}

\begin{figure}[H]
\centering
\begin{tikzpicture}[
    node/.style={draw, rounded corners, align=center, font=\small, minimum width=3.1cm, minimum height=0.9cm},
    arrow/.style={-{Stealth[length=3mm]}, thick}
]
    \node[node, fill=geometrygreen!15] (hol) {SU(5) holonomy\\degeneracy $\{3,2\}$};
    \node[node, right=1.2cm of hol, fill=topologyblue!15] (g) {$g:=3+2=5$\\(discrete node)};
    \node[node, below left=0.9cm and 0.6cm of g, fill=quantumorange!15] (d2) {$\delta_2=\frac{g}{4}\delta_{\mathrm{top}}^2$\\(two-defect weight)};
    \node[node, below=0.9cm of g, fill=quantumorange!15] (gam) {$\gamma(0)=\frac{g}{g+1}$\\(E$_8$ damping)};
    \node[node, below right=0.9cm and 0.6cm of g, fill=quantumorange!15] (uni) {$\Delta b_3$ candidates include $g/2$\\(unification patch)};

    \draw[arrow] (hol) -- (g);
    \draw[arrow] (g) -- (d2);
    \draw[arrow] (g) -- (gam);
    \draw[arrow] (g) -- (uni);
\end{tikzpicture}
\caption{Crosslink map: the same discrete holonomy degeneracy $g=5$ anchors independent sectors (defect expansion, E$_8$ damping, and a discrete unification patch candidate).}
\label{fig:crosslink}
\end{figure}

%=============================================================================
% CLAIM MAP - Reader's Contract
%=============================================================================
\vspace{1em}
\begin{tcolorbox}[colback=gray!5, colframe=gray!60, title={\textbf{Claim Map: What This Paper Establishes}}, fonttitle=\bfseries]
\textbf{Theorems (given axioms A1--A8):}
\begin{itemize}[leftmargin=1.5em, topsep=2pt, itemsep=1pt]
    \item Fine-structure constant (CFE closure): one-defect truncation gives $\alpha^{-1}(0)=\num[round-mode=places,round-precision=10]{\TFPTalphaInvSelfNum}$ ($-0.037$ ppm vs CODATA~2022). A two-defect refinement anchored by holonomy degeneracy $g=5$ gives $\alpha^{-1}(0)=\num[round-mode=places,round-precision=12]{\TFPTalphaInvGOverFourNum}$ (CODATA $z=\num[round-mode=places,round-precision=3]{\TFPTalphaZGOverFourNum}$).
    \item Cosmic birefringence: $\beta = \num[round-mode=places,round-precision=4]{\TFPTbetaDegNum}^\circ$ (UFE + minimal topological excursion, $n=1$)
    \item Cabibbo angle: $\lambda = 0.2245$ at $-0.15\%$ deviation
    \item Two-loop RG fingerprints: $\alpha_3(1\,\mathrm{PeV})=\num[round-mode=places,round-precision=6]{\TFPTalphaThreePeVNum}$ (rel.\ dev.\ vs.\ $\varphi_0$: $\num[round-mode=places,round-precision=3]{\TFPTalphaThreePeVRelDevPercent}\%$) and exact crossing scales $\alpha_3(\mu)=\varphi_0$, $\alpha_3(\mu)=c_3$ (checked by \texttt{two\_loop\_rg\_fingerprints} in \texttt{tfpt-suite/})
\end{itemize}

\textbf{Structural extensions (status: testable):}
\begin{itemize}[leftmargin=1.5em, topsep=2pt, itemsep=1pt]
    \item E$_8$ cascade with $\gamma(0)=g/(g+1)=5/6$ (anchored by holonomy degeneracy $g=5$)
    \item Block constants $\zeta_B$ with discrete indices $k_B, r_B$
    \item Z$_3$ flavor architecture with single phase $\delta$
    \item PMNS: $\sin^2\theta_{13} \approx 0.0231$ via TM1 pattern (few-\% dev.)
    \item Inflation: $A_s \sim 2\times10^{-9}$, $n_s \approx 0.964$, $r \approx 0.0038$ via $R^2$ completion (derived under assumptions K1--K3; reproduced by the verification module \texttt{effective\_action\_r2} in \texttt{tfpt-suite/})
    \item Bounce transfer function $T(k)$: large-scale features from torsion-driven bounce (conditional; evaluated by the verification module \texttt{bounce\_perturbations} in \texttt{tfpt-suite/})
    \item Axion DM: $m_a \approx 64\,\mu$eV, $\nu = 15.6$ GHz (background + DM modes)
\end{itemize}

\textbf{Direct falsification paths:}
\begin{itemize}[leftmargin=1.5em, topsep=2pt, itemsep=1pt]
    \item Any future $\alpha$ measurement deviating $>1$ ppm from TFPT
    \item Birefringence $\beta$ outside $[0.1^\circ, 0.4^\circ]$ with $<0.05^\circ$ error
    \item RG fingerprint $\alpha_3(1\,\mathrm{PeV})$ deviating $>2\%$ from $\phiz$ (see \texttt{two\_loop\_rg\_fingerprints})
    \item Inflation $r$ outside $[0.001, 0.005]$ with precision $\sigma_r < 0.001$
    \item PMNS $\sin^2\theta_{13}$ deviating $>10\%$ from $0.0231$
    \item Axion haloscope non-detection at $15.6 \pm 0.5$ GHz (given local DM density)
\end{itemize}
\end{tcolorbox}

%=============================================================================
% ASSUMPTION LEDGER - Transparency about what is assumed
%=============================================================================
\vspace{0.5em}
\begin{tcolorbox}[colback=yellow!5, colframe=yellow!60!black, 
    title={\textbf{Assumption Ledger: Established vs.\ New}}, fonttitle=\bfseries]

\textbf{Established physics (citable):}
\begin{itemize}[leftmargin=1.5em, topsep=2pt, itemsep=1pt]
    \item Riemann--Cartan geometry with torsion (Hehl et al.~1976)
    \item Fujikawa anomaly and chiral measure (Fujikawa 1979)
    \item Background-field gauge and Ward identities (standard QFT)
    \item Maxwell convention: U(1) field strength defined as $F=dA$ (gauge invariance exact in a Riemann--Cartan background)
    \item SM abelian trace $\bone = 41/10$ (particle content, GUT normalization)
    \item GR limit requirement ($K \to 0 \Rightarrow$ Einstein--Hilbert)
\end{itemize}

\textbf{Structural inputs (to be justified or tested):}
\begin{center}
\begin{tabular}{@{}p{4cm}p{4cm}p{4.5cm}@{}}
\toprule
\textbf{Input} & \textbf{Justification} & \textbf{Falsification Test}\\
\midrule
Orientable double cover (A2) & Spinor structure requires orientability & CPT tests, spin-torsion bounds\\
$\phitree = 1/(6\pi)$ from $\mathbb{Z}_2$ gluing term (A3) & $\eta$-gluing on cut double cover & Change gluing class/topology $\Rightarrow$ shift $\alpha$\\
$\deltatop = 48\cthree^4$ & Spin-lifted $\mathbb{Z}_2$ deficit factor & See Appendix~\ref{app:48deriv}\\
Backreaction exponent $=2$ & Defect-sector reweighting on the orientable double cover (Lemma~\ref{lem:defectreweight}) & Sensitivity analysis (Sec.~\ref{sec:sensitivity})\\
Holonomy degeneracy $g=5$ & SU(5) hypercharge holonomy eigenspaces $\{3,2\}$ (kernel crosslink) & Crosslink failures in $\{\delta_2,\gamma(0),$ unification patch$\}$\\
$\delta_2=\frac54\,\deltatop^2$ (two-defect) & Defect partition enumeration anchored by $g=5$ (theoryv3 audit) & $\alpha^{-1}(0)$ shifts outside CODATA window\\
$\Delta a_{\text{top}} = n\,\phiz$ (birefringence) & Topological excursion quantization (Definition~\ref{def:topoexcursion}); Sec.~\ref{sec:deltaa} gives one minimal dynamical realization & $\beta(z)$ tomography inconsistent\\
$\gamma(0)=g/(g+1)=5/6$ (E$_8$) & Anchored by holonomy degeneracy $g=5$ (crosslink) & Block-scale deviations $>5\%$\\
$R^2$ completion with $M/\Mpl=\sqrt{8\pi}\cthree^4$ & Derived under assumptions K1--K3 (Appendix~\ref{app:r2}); reproduced by \texttt{effective\_action\_r2} in \texttt{tfpt-suite/} & $(A_s,n_s,r)$ inconsistent\\
\bottomrule
\end{tabular}
\end{center}

\textbf{Parameter count:} The core CFE uses 0 free parameters. With $\gamma(0)=g/(g+1)=5/6$, the E$_8$ cascade 
introduces no continuous fit parameters (only discrete E$_8$ data and the TFPT invariants). 
The inputs above are \emph{structural choices}, not continuous fits.
\end{tcolorbox}

\tableofcontents
%=============================================================================
\section{Introduction: The Problem of Parameters}
%=============================================================================

The Standard Model of particle physics contains approximately 19 free parameters 
that must be determined experimentally. Among these, the fine-structure constant 
$\alpha \approx 1/137$ stands out as a dimensionless number whose value has no 
explanation within conventional frameworks.

TFPT addresses this problem by demonstrating that $\alpha$ is not a free parameter 
but a \emph{fixed point} of an effective potential, uniquely determined by:
\begin{enumerate}[label=(\roman*)]
    \item Topological quantization from 11D Chern--Simons theory
    \item Geometric reduction on a M\"obius fiber
    \item Quantum consistency via background-field gauge Ward identities
\end{enumerate}

\begin{figure}[H]
\centering
\begin{tikzpicture}[
    scale=1.0,
    box/.style={draw, rounded corners, minimum width=2.8cm, minimum height=1cm, align=center, font=\small},
    arrow/.style={-{Stealth[length=3mm]}, thick}
]
    % Boxes
    \node[box, fill=topologyblue!20] (topo) at (0,3) {Topology\\$\cthree = \frac{1}{8\pi}$};
    \node[box, fill=geometrygreen!20] (geom) at (4,3) {Geometry\\$\phiz = \frac{1}{6\pi}+\frac{3}{256\pi^4}$};
    \node[box, fill=quantumorange!20] (qft) at (2,1.5) {Quantum FPE\\CFE: $\alpha^3 - 2\cthree^3\alpha^2 - \ldots = 0$};
    \node[box, fill=observablepurple!20] (obs1) at (-0.5,0) {$\alpha^{-1}(0)$\\$\num[round-mode=places,round-precision=4]{\TFPTalphaInvGOverFourNum}$};
    \node[box, fill=observablepurple!20] (obs2) at (2,0) {$\gagg$\\$-4\cthree$};
    \node[box, fill=observablepurple!20] (obs3) at (4.5,0) {$\beta$\\$\num[round-mode=places,round-precision=4]{\TFPTbetaDegNum}^\circ$};
    
    % Arrows
    \draw[arrow, topologyblue] (topo) -- (qft);
    \draw[arrow, geometrygreen] (geom) -- (qft);
    \draw[arrow, observablepurple] (qft) -- (obs1);
    \draw[arrow, observablepurple] (qft) -- (obs2);
    \draw[arrow, observablepurple] (qft) -- (obs3);
    
    % Labels
    \node[font=\footnotesize, topologyblue] at (-0.8,2.3) {fixes $A$};
    \node[font=\footnotesize, geometrygreen] at (4.8,2.3) {fixes $\mathcal{K}$};
\end{tikzpicture}
\caption{Logical flow of TFPT: topological and geometric invariants determine quantum 
fixed-point parameters, which yield observables without free inputs.}
\label{fig:flow}
\end{figure}

%=============================================================================
\section{Axiomatic Foundation}
%=============================================================================

TFPT is built on eight axioms, organized into three categories: geometric structure, 
quantum field theory, and closure conditions. Each axiom is either derived from 
established physics or explicitly marked as a new postulate.

%-----------------------------------------------------------------------------
\subsection{Geometric Axioms (A1--A4)}
%-----------------------------------------------------------------------------

\begin{axiom}{Riemann--Cartan Geometry}{RC}
The fundamental spacetime connection possesses torsion; the Levi-Civita 
connection is a limiting case when torsion vanishes.

\tcblower
\textbf{Status:} Established (Einstein--Cartan theory, Hehl et al.~1976).

\textbf{Consequence:} Axial torsion couples to spinor fields and is dynamically active.
\end{axiom}

\begin{axiom}{Orientable Double Cover}{ODC}
The physically relevant manifold is the orientable double cover $\widetilde{M}$ 
of a non-orientable base with M\"obius structure.

\tcblower
\textbf{Status:} New assumption; mathematically rigorous, physically motivated 
by spinor structure, chirality, and anomaly cancellation.

\textbf{Consequence:} The geometry carries two identical boundary contributions 
plus a seam $\Gamma$, yielding a factor of 2 in backreaction.
\end{axiom}

\begin{tcolorbox}[colback=blue!3, colframe=blue!40, title={\textbf{Lemma (Spin Structure Requires Orientability)}}, fonttitle=\bfseries\itshape]
If a theory defines fundamental spinors globally, the effective manifold must be 
orientable. When the base topology is non-orientable (M\"obius structure), the 
physical carrier space is necessarily the orientable double cover.

\textbf{Physical necessity:}
\begin{enumerate}[leftmargin=1.5em, topsep=2pt, itemsep=1pt]
    \item Chiral fermions require a globally consistent spin structure; this fails 
        on non-orientable manifolds.
    \item Anomaly cancellation (ABJ) demands well-defined $\gamma_5$ eigenvalues, 
        which exist only on the orientable cover.
    \item The double cover provides exactly two sheets, which fix the exponent in
        $\phiz(\alpha) = \phitree + \deltatop e^{-2\alpha}$.
    \item The ``seam'' $\Gamma$ contributes a discrete spectral gluing term in the
        $\eta$-invariant, supplying the third $2\pi$ contribution in the normalization.
\end{enumerate}

\textbf{Causal chain:} Spinors $\to$ orientability required $\to$ double cover 
$\to$ (Axiom A3) gluing spectral term $\to$ 3 curvature contributions $\to$ $\phitree = 1/(6\pi)$.
\end{tcolorbox}

\begin{axiom}{Gauss--Bonnet Normalization with APS Gluing Term}{GB}
The geometric normalization on the orientable double cover $\widetilde{M}$ is fixed by the
spectral gluing term associated with the $\mathbb{Z}_2$ identification along the seam $\Gamma$.
Cutting along $\Gamma$ yields a manifold-with-boundary $\widetilde{M}_\Gamma$ with components
$(C_1,C_2,\Gamma_+,\Gamma_-)$ and a Dirac operator equipped with APS boundary conditions on
$C_1,C_2$ and a unitary matching operator $U_\Gamma$ on $\Gamma_\pm$. The $\eta$-invariant gluing
formula produces a discrete seam contribution $\Delta_\Gamma$, and for the minimal nontrivial
$\mathbb{Z}_2$ gluing class one has $\Delta_\Gamma = 2\pi$.

\tcblower
\textbf{Status:} Standard APS/$\eta$-gluing framework; the discrete seam term follows from the
minimal $\mathbb{Z}_2$ matching class (Appendix~\ref{app:seam}).

\textbf{Consequence:} Two physical boundary cycles ($2\pi+2\pi$) plus $\Delta_\Gamma=2\pi$ give
total curvature $6\pi$, fixing
\begin{equation}
    \phitree = \frac{1}{6\pi}.
\end{equation}
\end{axiom}

\noindent\textbf{Why this is well-defined.} The seam is encoded as a gluing condition, not as an
ad hoc boundary. In APS index theory, the gluing formula for $\eta$-invariants adds a discrete
spectral-flow term determined by the matching operator $U_\Gamma$. This makes the seam
contribution quantized and computable, rather than a geometric guess.

\begin{tcolorbox}[colback=green!3, colframe=green!50!black, 
    title={\textbf{Proof Sketch for $\phitree = 1/(6\pi)$ (via $\eta$-gluing)}}, fonttitle=\bfseries\small]
\textbf{Setup:} Cut the orientable double cover along $\Gamma$ to obtain
$\widetilde{M}_\Gamma$ with boundary components $(C_1,C_2,\Gamma_+,\Gamma_-)$.
Impose APS boundary conditions on $C_{1,2}$ and gluing by a unitary operator
$U_\Gamma$ between $\Gamma_+$ and $\Gamma_-$.

\textbf{Step 1 (Boundary operator):} On the seam, take the tangential Dirac operator
\begin{equation}
    D_\Gamma = i\,\frac{d}{d\theta}, \qquad \theta\in[0,2\pi).
\end{equation}

\textbf{Step 2 (Minimal gluing class):} The minimal M\"obius class is modeled by the
unitary phase
\begin{equation}
    U_\Gamma(\theta)=e^{i\theta}.
\end{equation}

\textbf{Step 3 (Spectral flow):} Interpolate the boundary condition
$\psi(\theta+2\pi)=e^{i\varphi}\psi(\theta)$ for $\varphi\in[0,2\pi]$. The eigenvalues are
\begin{equation}
    \lambda_n(\varphi)=n+\frac{\varphi}{2\pi}, \qquad n\in\mathbb{Z},
\end{equation}
so exactly one eigenvalue crosses zero as $\varphi$ runs from $0$ to $2\pi$:
$\mathrm{SF}(U_\Gamma)=1$.

\textbf{Step 4 (Gluing term):} The APS gluing theorem yields
$\eta(\widetilde{M}) = \eta(\widetilde{M}_\Gamma) + 2\,\mathrm{SF}(U_\Gamma)$,
so the seam contribution is $\Delta_\Gamma=2\pi$.

\textbf{Step 5 (Normalization):} The two physical boundary cycles contribute $2\pi+2\pi$,
and the seam adds $2\pi$, so $R_{\text{total}}=6\pi$. The stationarity condition
$\varphi\cdot R_{\text{total}}=1$ then gives $\phitree=1/(6\pi)$.
\end{tcolorbox}

\begin{axiom}{Discrete Topological Corrections}{TC}
Additional contributions to $\phiz$ arise only from topological invariants 
that are discrete and universal.

\tcblower
\textbf{Status:} Consequence of gauge-invariance constraints.

\textbf{Consequence:} The topological surcharge is fixed by the spin lift of the
$\mathbb{Z}_2$ identification, which yields an order-4 element in the spin bundle
and hence a universal deficit factor $(1-1/4)$ per minimal sector. Cutting along the
seam yields four local corner patches $(C_1,\Gamma_+)$, $(C_1,\Gamma_-)$, $(C_2,\Gamma_+)$,
$(C_2,\Gamma_-)$ in the fundamental domain, so $F=4$ and the deficit sum is:
\begin{equation}
    \deltatop 
    = \frac{1}{256\pi^4}\sum_{p=1}^{4}\left(1-\frac{1}{4}\right)
    = \frac{3}{256\pi^4}
    = 48\cthree^4
\end{equation}
See Appendix~\ref{app:48deriv} for the explicit sector count.
\end{axiom}

\begin{figure}[H]
\centering
\begin{tikzpicture}[scale=1.2]
    % M\"obius strip representation
    \fill[blue!10] (0,0) ellipse (2.5 and 1);
    
    % Top boundary
    \draw[thick, blue] (0,1) ellipse (2.2 and 0.3);
    \node[blue, font=\footnotesize] at (3.1,1) {top boundary ($2\pi$)};
    
    % Bottom boundary
    \draw[thick, blue] (0,-1) ellipse (2.2 and 0.3);
    \node[blue, font=\footnotesize] at (3.1,-1) {bottom boundary ($2\pi$)};
    
    % Seam
    \draw[thick, green!60!black, dashed] (-2.5,0) -- (2.5,0);
    \node[green!60!black, font=\footnotesize] at (2.9,0) {seam $\Gamma$ ($2\pi$)};
    
    % Arrow showing M\"obius identification
    \draw[-{Stealth}, thick, red] (2.8,0.7) arc[start angle=40, end angle=-40, radius=0.8];
    \node[red, font=\footnotesize] at (3.35,0.55) {M\"obius};
    \node[red, font=\footnotesize] at (3.35,0.25) {twist};
    
    % Total
    \node[font=\small] at (0,-1.8) {Total integrated curvature: $3 \times 2\pi = 6\pi$};
    \node[font=\small] at (0,-2.2) {$\Rightarrow \phitree = 1/(6\pi)$};
\end{tikzpicture}
\caption{Orientable double cover $\widetilde{M}$: two boundary cycles plus the discrete
$\eta$-gluing seam term $\Gamma$ contribute $6\pi$ total curvature, fixing $\phitree$.}
\label{fig:doublecover}
\end{figure}

%-----------------------------------------------------------------------------
\subsection{Quantum Field Theory Axioms (A5--A6)}
%-----------------------------------------------------------------------------

\begin{tcolorbox}[colback=yellow!5, colframe=yellow!60!black, 
    title={\textbf{Conventions: Units, Masses, and Normalizations}}, fonttitle=\bfseries]

\textbf{Units and constants:}
\begin{itemize}[leftmargin=1.5em, topsep=2pt, itemsep=1pt]
    \item Natural units: $\hbar = c = 1$; energies in GeV, lengths in GeV$^{-1}$.
    \item Reference data: CODATA 2022, PDG 2024.
    \item Metric signature: $(+,-,-,-)$.
    \item Levi-Civita tensor: $\varepsilon^{0123} = +1$, $\tilde{F}^{\mu\nu} = \frac{1}{2}\varepsilon^{\mu\nu\rho\sigma}F_{\rho\sigma}$.
\end{itemize}

\textbf{Planck mass definitions:}
\begin{center}
\begin{tabular}{@{}lll@{}}
\toprule
\textbf{Symbol} & \textbf{Definition} & \textbf{Value}\\
\midrule
$M_{\mathrm{Pl}}$ (unreduced) & $G^{-1/2}$ & $1.221 \times 10^{19}$ GeV\\
$\Mpl$ (reduced) & $(8\pi G)^{-1/2}$ & $2.435 \times 10^{18}$ GeV\\
\bottomrule
\end{tabular}
\end{center}
\textbf{Convention:} Block formulas (Sec.~\ref{sec:e8}) use $M_{\mathrm{Pl}}$ (unreduced). 
Relation: $M_{\mathrm{Pl}} = \sqrt{8\pi}\,\Mpl$.

\textbf{Axion sector:}
\begin{itemize}[leftmargin=1.5em, topsep=2pt, itemsep=1pt]
    \item Axion field $a$: Dimensionless, $2\pi$-periodic. Physical field: $a_{\text{phys}} = f_a \cdot a$.
    \item Chern--Simons term: $\mathcal{L} \supset \cthree\, a\, F\tilde{F}$ with $\cthree = 1/(8\pi)$.
    \item Coupling: $\gagg = -4\cthree$ is dimensionless; $g_{a\gamma\gamma}^{\text{(phys)}} = \gagg / f_a$ has dimension $[\text{energy}]^{-1}$.
    \item \textbf{Maxwell convention:} We define the field strength by the exterior derivative $F := dA$, i.e.\ $F_{AB}=2\partial_{[A}A_{B]}$, independent of the torsionful affine connection (we do not use $F_{AB}=2\hat{\nabla}_{[A}A_{B]}$). Therefore U(1) gauge invariance is exact in the electromagnetic sector.
\end{itemize}
\end{tcolorbox}

\begin{axiom}{Axion--Photon Anomaly Coupling}{APC}
The unique additional coupling in the low-energy effective theory is a 
pseudoscalar axion--photon interaction of the form $\cthree\, a F\tilde{F}$, 
fixed by the abelian anomaly (Fujikawa method).

\tcblower
\textbf{Status:} Standard in anomaly physics; follows from chiral measure.

\textbf{Consequence:} The axion--photon coupling is
\begin{equation}
    \cthree = \frac{1}{8\pi}
\end{equation}
with no remaining normalization freedom.
\end{axiom}

\begin{axiom}{Modulus Closure in Background-Field Gauge}{BFG}
We introduce a dimensionless coupling modulus $\sigma$ with electromagnetic kinetic function
\begin{equation}
    \mathcal{L} \supset -\frac{1}{4} f(\sigma)\,F_{\mu\nu}F^{\mu\nu}
    + \frac{1}{2}(\partial\sigma)^2,
    \qquad 
    \alpha(\sigma)=\frac{1}{4\pi f(\sigma)}.
\end{equation}
The physical vacuum extremizes the effective action with respect to $\sigma$:
\begin{equation}
    \frac{\delta\Gamma_{\mathrm{eff}}}{\delta\sigma}=0.
\end{equation}
In background-field gauge, the effective potential can be written as
$V_{\mathrm{eff}}(\sigma)=U(\alpha(\sigma)) + V_{\mathrm{geo}}(\sigma)$, where
$V_{\mathrm{geo}}'(\sigma_\star)=0$ by geometric closure.
If $\alpha(\sigma)$ is monotonic near the vacuum, this is equivalent to
\begin{equation}
    \frac{\partial U}{\partial \alpha}=0,
    \qquad U(\alpha)=U(\sigma(\alpha)).
\end{equation}

\tcblower
\textbf{Status:} Standard in modulus stabilization; background-field gauge provides a clean definition of $\Gamma_{\mathrm{eff}}$.

\textbf{Consequence:} $\alpha$ is fixed by a modulus extremum, not treated as a variational constant.
\end{axiom}

%-----------------------------------------------------------------------------
\subsection{Closure Axioms (A7--A8)}
%-----------------------------------------------------------------------------

\begin{axiom}{Standard Model Hypercharge Trace}{SMT}
The hypercharge trace invariant, summed over all SM fermions with GUT normalization 
$(5/3)$, is:
\begin{equation}
    \bone = \frac{5}{3} \sum_{\text{SM fermions}} Y_f^2 = \frac{41}{10}
\end{equation}
In TFPT this quantity appears as the \emph{spectral quadratic index} of the
hypercharge-coupled boundary operators in the regulated determinant
(Appendix~\ref{app:boundary}), not as an RG flow coefficient.

\tcblower
\textbf{Status:} Direct consequence of SM particle content; independent of RG scale.

\textbf{Important distinction:} $\bone$ is a \emph{hypercharge trace invariant}, 
not the one-loop beta function coefficient $b_1^{(\text{RG})}$. While numerically 
identical, their physical roles differ:
\begin{itemize}[leftmargin=1.5em, topsep=2pt, itemsep=1pt]
    \item $b_1^{(\text{RG})}$: Controls energy dependence $d\alpha_1/d\ln\mu$
    \item $\bone$ (TFPT): Discrete index counting charge content; scale-independent
\end{itemize}

\textbf{Consequence:} No exotic particles or modified charge structure required.
\end{axiom}

\begin{tcolorbox}[colback=yellow!3, colframe=yellow!50!black, 
    title={\textbf{Scale Consistency: Why $\bone$ (UV) Determines $\alpha$ (IR)}}, fonttitle=\bfseries\small]
\textbf{Potential objection:} The abelian trace $\bone = 41/10$ is computed from the full SM spectrum 
(quarks, leptons, gauge bosons) at high energies. How can it determine $\alpha$ in the Thomson 
limit ($Q^2 \to 0$)?

\textbf{Resolution:} The CFE is a stationarity (closure) condition for the effective action; it is \emph{not} an RG differential equation.

\begin{enumerate}[leftmargin=1.5em, topsep=2pt, itemsep=1pt]
    \item \textbf{Spectral origin:} In the APS framework with flat $U(1)_Y$ holonomies on the boundary cycles,
        the regulated determinant produces a universal trace coefficient proportional to $\sum_f Y_f^2$.
        Thresholds affect non-universal finite terms, not this discrete spectral index.
    \item \textbf{Field-content fixed coefficient:} $\bone$ is the GUT-normalized hypercharge trace 
        $\frac{5}{3}\sum_f Y_f^2$ of the full SM spectrum. Once the field content and charge assignments are fixed, 
        $\bone$ is a discrete input and does not ``run'' as a function of $\mu$.
    \item \textbf{UV log coefficient, not IR running:} In the one-loop effective action, $\bone$ multiplies the UV logarithmic term 
        coming from the abelian charge trace. TFPT uses this UV coefficient inside the stationarity condition; it does not attempt to compute 
        the infrared running $\alpha(\mu)$.
    \item \textbf{Threshold independence (anomaly matching):} Below mass thresholds, particles decouple from IR loops, but the UV charge-trace 
        coefficient is fixed by the UV-complete spectrum. The sum $41/10$ is the asymptotic value and remains the relevant discrete input.
    \item \textbf{CFE vs.\ RG:} Standard RG asks ``how does $\alpha(\mu)$ evolve?'' The CFE asks ``what value of $\alpha$ makes the effective action stationary 
        given $(\cthree,\phiz,\bone)$?'' It is a closure condition, not a flow equation.
\end{enumerate}

\textbf{Reviewer-safe statement:} We use $\bone$ as the discrete spectral trace coefficient of the
hypercharge-coupled boundary operators in the UV logarithmic term of the regulated determinant.
TFPT explicitly does \emph{not} claim to compute the IR running $\alpha(\mu)$ or threshold matching.

\textbf{Prediction:} If additional charged particles exist beyond the SM (e.g., SUSY partners), 
$\bone$ would change and the CFE would predict a \emph{different} $\alpha$. This is testable.
\end{tcolorbox}

\begin{axiom}{GR Renormalization Condition}{GRN}
In the torsion-free limit ($K \to 0$), the theory must exactly reproduce 
Einstein--Hilbert gravity.

\tcblower
\textbf{Status:} Physical consistency requirement.

\textbf{Consequence:} Fixes the gravitational coupling structure:
$\kappa^2 = 8\pi G$, with the dimensionless ratio $\xi = \cthree/\phiz$ constrained by topology.
The dimensional scale $M_{\mathrm{Pl}}$ remains an external input.
\end{axiom}

%=============================================================================
\section{Derivation of Fundamental Constants}
%=============================================================================

%-----------------------------------------------------------------------------
\subsection{\texorpdfstring{The Geometric Scale $\phiz$}{The Geometric Scale phi0}}
%-----------------------------------------------------------------------------

\begin{theorem}{Existence and Uniqueness of $\phiz$}{phi0}
From Axioms A3 and A4, the geometric scale is uniquely determined:
\begin{equation}
    \phiz = \phitree + \deltatop = \frac{1}{6\pi} + \frac{3}{256\pi^4} = \num[round-mode=places,round-precision=17]{\TFPTvarphiZeroNum}
\end{equation}
\end{theorem}

\begin{proof}
From A3 ($\eta$-gluing): the M\"obius double cover contributes two physical boundary
cycles ($2\pi+2\pi$) plus a discrete seam gluing term ($2\pi$), giving
$R_{\text{total}}=6\pi$. The stationarity condition $\partial_\varphi V_{\text{eff}} = 0$
under the normalization $\chi = \varphi R = 1$ yields $\phitree = 1/(6\pi)$.

From A4: the unique gauge-invariant correction at leading order is 
$\deltatop = 48\cthree^4 = 3/(256\pi^4)$.
\end{proof}

\begin{tcolorbox}[colback=gray!5, colframe=gray!60, breakable,
    title={\textbf{Dictionary: $\beta_{\text{rad}}$ and Canonical Consequences}}, fonttitle=\bfseries\small]
\textbf{Definition (notation):}
\[
\beta_{\text{rad}} := \frac{\phiz}{4\pi}.
\]
\textit{Note:} dimensionless; equals the birefringence angle in radians for the minimal topological excursion $\Delta a_{\text{top}}=\phiz$.

\textbf{Immediate algebraic consequences:}
\begin{equation}
    \phiz = 4\pi\beta_{\text{rad}}, 
    \qquad 
    \beta_{\text{deg}} := \frac{180}{\pi}\,\beta_{\text{rad}}.
\end{equation}
\begin{equation}
    \gagg = -4\cthree = -\frac{1}{2\pi},
    \qquad
    \deltatop = 48\cthree^4 = \frac{3}{256\pi^4}.
\end{equation}

\textbf{Remark:} Many later ``$\beta_{\text{rad}}$-centric'' relations are compact rewrites of earlier definitions. 
We catalogue their logical status (definition vs.\ consequence vs.\ lemma vs.\ prediction) in Appendix~\ref{app:identities}.
\end{tcolorbox}

% Suite-extracted invariant table (high precision)
\TFPTCoreInvariantTable

%-----------------------------------------------------------------------------
\subsection{The Cubic Fixed-Point Equation (CFE)}
%-----------------------------------------------------------------------------

\begin{definition}{Effective Potential Parameters}{params}
From the 11D Chern--Simons reduction (A5) and modulus-closure structure (A6):
\begin{align}
    A &= 2\cthree^3 = \frac{1}{256\pi^3} \approx 1.2598 \times 10^{-4}\\[0.5em]
    \mathcal{K} &= \frac{\bone}{2\pi}\ln\frac{1}{\phiz} \approx 1.9147
\end{align}
\textbf{Notation:} We use $\mathcal{K}$ (calligraphic) for this CFE parameter to 
distinguish it from $\kappa^2 = 8\pi G$ (gravitational coupling).
\end{definition}

\begin{tcolorbox}[colback=gray!5, colframe=gray!60,
    title={\textbf{Lemma (Geometric Fixing of the Log)}}, fonttitle=\bfseries\small]
In TFPT the one-loop logarithm is the ratio of a UV scale to a geometrically fixed IR scale:
\begin{equation}
    \ln\frac{1}{\phiz} = \ln\frac{\Lambda}{\mu_{\text{geo}}}.
\end{equation}
Here $\mu_{\text{geo}} \equiv \ell_{\text{geo}}^{-1}$ is fixed by the M\"obius curvature
normalization ($R_{\text{total}}=6\pi$), and $\phiz = \ell_{\text{UV}}/\ell_{\text{geo}}$.
Thus the renormalization point is not a free choice but is geometrically determined.
\end{tcolorbox}

\begin{theorem}{Cubic Fixed-Point Equation}{CFE}
From Axioms A5--A7 and the modulus-closure condition (A6), the vacuum extremum
$\delta\Gamma_{\mathrm{eff}}/\delta\sigma=0$ is equivalent to $\partial U/\partial\alpha = 0$
and yields the cubic fixed-point equation:
\begin{equation}
    \boxed{\alpha^3 - 2\cthree^3\,\alpha^2 - 8\bone\cthree^6\ln\frac{1}{\phiz(\alpha)} = 0}
    \label{eq:CFE}
\end{equation}
This equation has exactly one positive real root.
\end{theorem}

\begin{proof}
In background-field gauge, the one-loop effective potential takes the form
\begin{equation}
    U(\alpha) = \frac{A}{4}\alpha^4 - \frac{2}{3}A\cthree^3\alpha^3 - 8A\bone\cthree^6 L\,\alpha + \ldots
\end{equation}
where $L = \ln(1/\phiz)$. The modulus-closure condition 
$\delta\Gamma_{\mathrm{eff}}/\delta\sigma = 0$ reduces to $\partial U/\partial\alpha = 0$
(monotonic $\alpha(\sigma)$), giving
\begin{equation}
    A\bigl[\alpha^3 - 2\cthree^3\alpha^2 - 8\bone\cthree^6 L\bigr] = 0
\end{equation}
(The factor 2 arises from differentiating the $2/3$ coefficient of the cubic term.)
For $A \neq 0$, this is the CFE. Descartes' rule of signs guarantees exactly 
one positive real root.
\end{proof}

\begin{figure}[H]
\centering
\begin{tikzpicture}
\begin{axis}[
    width=12cm, height=7cm,
    xlabel={$\alpha$},
    ylabel={$f(\alpha) = \alpha^3 - A\alpha^2 - A\cthree^2\mathcal{K}$},
    xmin=0.0068, xmax=0.0078,
    ymin=-3e-8, ymax=3e-8,
    grid=major,
    axis lines=middle,
    xtick={0.0069, 0.0071, 0.0073, 0.0075, 0.0077},
    ytick={-2e-8, -1e-8, 0, 1e-8, 2e-8},
    scaled y ticks=false,
    yticklabel style={/pgf/number format/sci},
    legend pos=north west
]
    % CFE cubic
    \addplot[domain=0.0068:0.0078, samples=200, thick, blue] 
        {x^3 - 2*0.039788735773^3*x^2 - 8*4.1*0.039788735773^6*2.9439};
    \addlegendentry{CFE cubic $f(\alpha)$}
    
    % Root marker
    \addplot[only marks, mark=*, mark size=3pt, red] 
        coordinates {(0.00729733, 0)};
    \node[red, above right] at (axis cs:0.00729733, 0.3e-8) 
        {$\alpha^{-1} = 137.0365$};
    
    % Zero line
    \addplot[domain=0.0068:0.0078, dashed, gray] {0};
\end{axis}
\end{tikzpicture}
\caption{The CFE has a unique positive root at $\alpha \approx 0.007297$, 
corresponding to $\alpha^{-1} \approx 137.0365$ (baseline with fixed $\phiz$).}
\label{fig:CFE}
\end{figure}

%-----------------------------------------------------------------------------
\subsection{Geometric Self-Consistency (Backreaction)}
%-----------------------------------------------------------------------------

\subsubsection{Backreaction as Structural Response}

The response $\phiz(\alpha) = \phitree + \deltatop e^{-2\alpha}$ is the closed form
consequence of defect sector reweighting on the orientable double cover. The linear form
is the $\mathcal{O}(\alpha)$ expansion, not a separate assumption.

\textbf{Step 1: The CFE uses $L = \ln(1/\phiz)$ as input.}
The cubic fixed-point equation \eqref{eq:CFE} depends on $\phiz$ only through 
the logarithm $L$. Any shift in $\phiz$ propagates via:
\begin{equation}
    \delta L = -\frac{\delta\phiz}{\phiz}
\end{equation}

\textbf{Step 2: Defect sector reweighting in the Euclidean partition function.}
We treat the discrete correction as a defect sector weight. An additive deformation of the
defect action by $\alpha$ implies exact exponential reweighting:
\begin{lemma}{Defect Sector Reweighting}{defectreweight}
Let the Euclidean partition function split into sectors
\begin{equation}
    Z(\alpha)=Z_{\mathrm{triv}}(\alpha)+Z_{\mathrm{def}}(\alpha),
\end{equation}
with defect weight $\delta_{\mathrm{top}}(\alpha)=Z_{\mathrm{def}}(\alpha)/Z_{\mathrm{triv}}(\alpha)$.
If the defect action is deformed additively,
\begin{equation}
    S_{\mathrm{def}}(\alpha)=S_{\mathrm{def}}(0)+\alpha\,\mathcal{J},
\end{equation}
then
\begin{equation}
    \delta_{\mathrm{top}}(\alpha)=\delta_{\mathrm{top}}\,e^{-\alpha\mathcal{J}}.
\end{equation}
For a degree-$d$ cover with identical sheet contributions, $\mathcal{J}=d\,\mathcal{J}_1$.
With the minimal deck-invariant normalization $\mathcal{J}_1=1$, the double cover gives
$\delta_{\mathrm{top}}(\alpha)=\delta_{\mathrm{top}}\,e^{-2\alpha}$. \hfill QED
\end{lemma}

\begin{tcolorbox}[colback=blue!3, colframe=blue!40, 
    title={\textbf{Lemma (Cover-Degree Scaling of Deck-Invariant Response)}}, fonttitle=\bfseries\small]
Let $p:\widetilde{M}\to M$ be a degree-$d$ covering and let $\mathcal{J}$ be a deck-invariant
source functional. Then
\begin{equation}
    \int_{\widetilde{M}}\mathcal{J} = d \int_{M}\mathcal{J}.
\end{equation}
If the response of $\phiz$ is built from such an integrated source, the effective
weight acquires a factor $d$. For the orientable double cover, $d=2$.
\end{tcolorbox}

\textbf{Step 3: Double cover fixes the exponent.}
The defect-sector lemma and cover-degree scaling fix the exponent to $2$ for the
orientable double cover. Adding the baseline yields the exact response:
\begin{equation}
    \phiz(\alpha) = \phitree + \deltatop\,e^{-2\alpha}.
\end{equation}

\begin{remark}
\textbf{Series control (no new parameters):} The exact form implies
\begin{equation}
    \phiz(\alpha)=\phitree+\deltatop\left(1-2\alpha+2\alpha^2-\frac{4}{3}\alpha^3+\cdots\right),
\end{equation}
so all higher coefficients are fixed (e.g.\ $c_2=2$). Numerically, even the
$\alpha^2$ term shifts $\phiz$ by $\lesssim 10^{-8}$ since
$\alpha\approx 0.0073$ and $\deltatop\approx 1.2\times10^{-4}$.
\end{remark}

\begin{theorem}{Self-Consistent $\alpha$ with Derived Two-Defect Partition}{selfconsistent}
Including the minimal backreaction response on the orientable double cover \emph{and} the derived two-defect occupancy weight anchored by holonomy degeneracy $g=5$:
\begin{align}
    \phiz(\alpha) &= \phitree + \deltatop e^{-2\alpha} + \delta_2 e^{-4\alpha} + \cdots,
    \label{eq:backreaction}\\
    \delta_2 &= \frac{g}{4}\,\delta_{\mathrm{top}}^2 = \frac54\,\delta_{\mathrm{top}}^2,
    \qquad (g=5),
\end{align}
the self-consistent solution of the CFE yields the v2.5 headline value:
\begin{equation}
    \boxed{\alpha^{-1}_{\text{TFPT}}(0) = \num[round-mode=places,round-precision=12]{\TFPTalphaInvGOverFourNum}}
\end{equation}
with CODATA 2022 diagnostic $z=\num[round-mode=places,round-precision=3]{\TFPTalphaZGOverFourNum}$ (see \codepath{tfpt-suite/theoryv3/out/defect_partition_g5_audit}).
\end{theorem}

\begin{lemma}{One-defect truncation (diagnostic baseline; excluded at CODATA precision)}{oneDefectTruncation}
Setting $\delta_2=0$ in \eqref{eq:backreaction} yields the one-defect truncation
\[
\alpha^{-1}(0)=\num[round-mode=places,round-precision=10]{\TFPTalphaInvSelfNum},
\]
which is useful as an expansion baseline and for sensitivity plots, but is not the v2.5 headline prediction.
\end{lemma}

\begin{remark}
\textbf{Suite diagnostic template (for comparison):} The legacy ordered-occupancy template $\delta_2=\delta_{\mathrm{top}}^2$ corresponds to $g=4$ in $\delta_2=(g/4)\delta_{\mathrm{top}}^2$ and yields
$\alpha^{-1}(0)=\num[round-mode=places,round-precision=10]{\TFPTalphaInvTwoDefNum}$ (ppm$=\num[round-mode=places,round-precision=4]{\TFPTalphaPpmTwoDefNum}$), tracked by \codepath{tfpt-suite/out/alpha_precision_audit}.
\end{remark}

\begin{proof}
On the orientable double cover $\widetilde{M}$, defect-sector reweighting with a
deck-invariant source fixes the exponent in \eqref{eq:backreaction}
(Lemma~\ref{lem:defectreweight}). The self-consistent solution is obtained by iteration:
\begin{enumerate}
    \item Start with $\alpha_0$ from the CFE with fixed $\phiz$.
    \item Update $\phiz(\alpha_n)$ via \eqref{eq:backreaction}.
    \item Solve CFE for $\alpha_{n+1}$.
    \item Iterate until convergence.
\end{enumerate}
Convergence is rapid (typically 3--4 iterations).
\end{proof}

\begin{table}[H]
\centering
\small
\setlength{\tabcolsep}{4pt}
\begin{tabularx}{\linewidth}{@{}lccX@{}}
\toprule
\textbf{Scenario} & $\alpha^{-1}$ & \textbf{Deviation from CODATA} & \textbf{Status}\\
\midrule
Baseline (fixed $\phiz$) & 137.03650146 & $+3.67$ ppm & structural\\
Self-consistent $\phiz(\alpha)$ & \num[round-mode=places,round-precision=10]{\TFPTalphaInvSelfNum} & $-0.037$ ppm & \textbf{closed-form defect reweighting (one-defect)}\\
Two-defect refinement ($+\tfrac54\deltatop^2 e^{-4\alpha}$) & \num[round-mode=places,round-precision=12]{\TFPTalphaInvGOverFourNum} & $+\num[round-mode=places,round-precision=6]{\TFPTalphaPpmGOverFourNum}$ ppm & \textbf{derived (holonomy $g=5$)}\\
Two-defect (suite diagnostic) ($+\deltatop^2 e^{-4\alpha}$) & \num[round-mode=places,round-precision=10]{\TFPTalphaInvTwoDefNum} & $\num[round-mode=places,round-precision=4]{\TFPTalphaPpmTwoDefNum}$ ppm & parameter-free template (diagnostic)\\
CODATA 2022 & \num[round-mode=places,round-precision=9]{\TFPTalphaInvCODATANum} & --- & reference\\
\bottomrule
\end{tabularx}
\caption{Impact of geometric self-consistency on $\alpha$ prediction.}
\label{tab:alpha}
\end{table}

% Suite/theoryv3 audit table for α
\TFPTAlphaAuditTable

\subsubsection{Sensitivity Analysis}
\label{sec:sensitivity}

The sub-ppm precision of the $\alpha$ prediction depends critically on the backreaction 
coefficient. We analyze the sensitivity to demonstrate this is not fine-tuning but 
a genuine geometric constraint.

\textbf{Exponent variation (diagnostic):}
\begin{center}
\begin{tabular}{@{}ccc@{}}
\toprule
\textbf{Exponent $k$} & $\alpha^{-1}$ & \textbf{Deviation (ppm)}\\
\midrule
$k = 0$ (no backreaction) & 137.03650146 & $+3.67$\\
$k = 1$ & 137.03624686 & $+1.81$\\
$k = 1.5$ & 137.03612025 & $+0.88$\\
$\mathbf{k = 2}$ (double cover) & $\mathbf{\num[round-mode=places,round-precision=10]{\TFPTalphaInvSelfNum}}$ & $\mathbf{-0.037}$\\
$k = 2.5$ & 137.03586841 & $-0.95$\\
$k = 3$ & 137.03574318 & $-1.87$\\
\bottomrule
\end{tabular}
\end{center}

\textbf{Key observations:}
\begin{enumerate}[leftmargin=1.5em, topsep=2pt, itemsep=1pt]
    \item The zero crossing occurs near $k \approx 1.97$ in this diagnostic.
    \item The exponent $k=2$ is not fitted; it follows from the double cover having two sheets.
    \item Sensitivity: $\Delta(\alpha^{-1})/\Delta k \approx 2.5 \times 10^{-4}$ per unit of $k$, 
        or about 1.8 ppm per unit change.
\end{enumerate}

\textbf{Why the exponent must be exactly 2:}

The orientable double cover has exactly two sheets. Field energy on each sheet couples 
identically to the boundary data. Any coefficient $\neq 2$ would require:
\begin{itemize}[leftmargin=1.5em, topsep=2pt, itemsep=1pt]
    \item Asymmetric coupling (violating the cover symmetry), or
    \item Additional sheets (violating the minimal orientable completion), or
    \item A non-geometric origin (abandoning the axiom structure).
\end{itemize}

Although $-0.037$ ppm appears small, CODATA 2022 has $\sigma(\alpha^{-1})\simeq 2.1\times 10^{-8}$, i.e.\ a ppm-scale uncertainty of only $\sim 1.5\times 10^{-4}$ ppm. The one-defect truncation is therefore \emph{excluded} at CODATA precision and is shown here only as a diagnostic baseline; the derived two-defect partition (Sec.~\ref{sec:sensitivity}; Theorem~\ref{thm:selfconsistent}) supplies the required second-order correction.

\TFPTAlphaSensitivityPlot

%-----------------------------------------------------------------------------
\subsection{Comparison and Scheme Policy (No Hidden Knobs)}
\label{sec:comparison-policy}
%-----------------------------------------------------------------------------

\begin{tcolorbox}[colback=yellow!3, colframe=yellow!60!black,
    title={\textbf{Primary comparison observable: $\overline{\alpha}^{(5)}(M_Z)$ ($\overline{\mathrm{MS}}$)}}, fonttitle=\bfseries\small]
\textbf{Why this matters:} Precision electroweak fits and PDG summaries use the $\overline{\mathrm{MS}}$ coupling at $M_Z$, $\overline{\alpha}^{(5)}(M_Z)$, as the comparison observable. CODATA $\alpha(0)$ is an on-shell/IR reference and is therefore treated as \emph{secondary diagnostic} once scheme and threshold bookkeeping is made explicit.

\textbf{Declared policy (suite):}
\begin{itemize}[leftmargin=1.5em, topsep=2pt, itemsep=1pt]
    \item \textbf{Primary reference:} $\overline{\alpha}^{(5)}(M_Z)^{-1} = \num{\TFPTalphaBarFiveInvMZRefNum} \pm \num{\TFPTalphaBarFiveInvMZRefSigma}$ (PDG 2024).
    \item \textbf{Secondary diagnostic:} $\alpha^{-1}(0)=\num[round-mode=places,round-precision=9]{\TFPTalphaInvCODATANum}$ (CODATA 2022).
\end{itemize}

\textbf{Result (from TFPT + declared running inputs):}
\[
\overline{\alpha}^{(5)}(M_Z)^{-1}=\num[round-mode=places,round-precision=6]{\TFPTalphaBarFiveInvMZPredNum},
\qquad
z=\num[round-mode=places,round-precision=3]{\TFPTalphaBarFiveInvMZZ}.
\]

\textbf{Reviewer-safe statement:} TFPT fixes $\alpha(0)$ via the CFE closure. The conversion to $\overline{\alpha}^{(5)}(M_Z)$ uses standard SM/QED running inputs (leptonic 1-loop, PDG hadronic vacuum polarization, and declared finite pieces), tracked by \codepath{tfpt-suite/out/alpha_precision_audit}. The inverse bridge back to $\alpha(0)$ is \emph{diagnostic only} because the CODATA uncertainty budget is not commensurate with scheme/matching systematics.
\end{tcolorbox}

%=============================================================================
\section{The Unified Field Equation (UFE)}
%=============================================================================

%-----------------------------------------------------------------------------
\subsection{Torsionful Geometric Action}
%-----------------------------------------------------------------------------

\begin{definition}{UFE Action}{UFEaction}
On a manifold $M^{4+n}$ with metric $\hat{g}_{AB}$ and torsionful connection 
$\hat{\Gamma}^A_{BC}$, the action is:
\begin{equation}
    S = \int d^{4+n}x\sqrt{-\hat{g}}\left[
        \frac{1}{2\hat{\kappa}^2}\hat{R}(\hat{\Gamma}) 
        - \frac{1}{4}F_{AB}F^{AB}
        + \frac{1}{2}\hat{g}^{AB}\partial_A a\,\partial_B a 
        - V(a) 
        + \cthree\, a\, F_{AB}\tilde{F}^{AB}
    \right]
\end{equation}
Here $\hat{R}(\hat{\Gamma})$ denotes the Ricci scalar constructed from the full torsionful connection $\hat{\Gamma}$.
\end{definition}

\noindent\textbf{Minimal torsion sector.} In the parameter-free baseline we set $\mathcal{L}_{\mathrm{tors}}\equiv 0$.
All torsion dependence enters through $\hat{R}(\hat{\Gamma})$ with $\hat{\Gamma}=\Gamma(\hat{g})+K$.
The quadratic torsion treatment used for the one-loop $R^2$ completion is stated explicitly in Definition~\ref{def:R2torsion} and Appendix~\ref{app:r2}.

\noindent\textbf{Maxwell convention.} We treat the U(1) gauge potential $A$ as a one-form and define the field strength by the exterior derivative
$F := dA$, i.e.\ $F_{AB}=2\partial_{[A}A_{B]}$. This definition is independent of the torsionful affine connection $\hat{\Gamma}$; in particular, we do not use $F_{AB}=2\hat{\nabla}_{[A}A_{B]}$. Therefore U(1) gauge invariance is exact in the electromagnetic sector.

\noindent\textbf{Torsion and contorsion.} The torsion of the full connection is $T^A_{\;BC}:=2\hat{\Gamma}^A_{[BC]}$. We decompose
$\hat{\Gamma}^A_{BC}=\Gamma^A_{BC}(\hat{g})+K^A_{BC}$, where $\Gamma(\hat{g})$ is the Levi-Civita connection of $\hat{g}_{AB}$ and $K^A_{BC}$ is the contorsion tensor.

\begin{theorem}{Unified Field Equations}{UFE}
Variation of the action yields the compact tensor form of the UFE. Since $\delta\hat{g}^{AB}$ is symmetric, only the symmetric part of the torsionful Ricci tensor contributes. Define
$\hat{G}_{(AB)}(\hat{\Gamma})$ by
\begin{equation*}
    \hat{G}_{(AB)}(\hat{\Gamma}) := \hat{R}_{(AB)}(\hat{\Gamma}) - \frac{1}{2}\hat{g}_{AB}\hat{R}(\hat{\Gamma}).
\end{equation*}
Then:
\begin{align}
    \hat{G}_{(AB)}(\hat{\Gamma})
        &= \hat{\kappa}^2\bigl(T^{(a)}_{AB} + T^{(\text{EM})}_{AB}\bigr)
        \label{eq:UFE}\\[0.5em]
    \nabla^{\mathrm{LC}}_A F^{AB} + 2\cthree(\partial^B a)\tilde{F}_{AB} &= 0
        \label{eq:maxwell}\\[0.5em]
    \nabla^{\mathrm{LC}}_A\tilde{F}^{AB} &= 0
        \label{eq:bianchi}
\end{align}
In Eqs.~\eqref{eq:maxwell}--\eqref{eq:bianchi} the covariant derivative is the Levi-Civita derivative of $\hat{g}_{AB}$ (equivalently $\nabla^{\mathrm{LC}}_A X^A=(1/\sqrt{-\hat{g}})\,\partial_A(\sqrt{-\hat{g}}\,X^A)$).
When expanding Eq.~\eqref{eq:UFE} in Levi-Civita plus contorsion variables (with $\hat{\Gamma}=\Gamma(\hat{g})+K$), all covariant derivatives are Levi-Civita.
In the torsion vacuum $(K \to 0)$, this reduces to Einstein--Hilbert gravity 
by Axiom A8.
\end{theorem}

\begin{tcolorbox}[colback=gray!5, colframe=gray!60,
title={\textbf{Why the UFE uses the symmetric torsionful Einstein tensor}}, fonttitle=\bfseries\small]
Start from the torsionful Einstein--Hilbert term
\[
S_{\mathrm{EH}}=\frac{1}{2\hat{\kappa}^2}\int d^{4+n}x\,\sqrt{-\hat{g}}\;\hat{R},
\qquad
\hat{R}=\hat{g}^{AB}\hat{R}_{AB}(\hat{\Gamma}).
\]
Vary $\hat{g}^{AB}$ while holding $\hat{\Gamma}$ fixed:
\[
\delta\hat{R}=\hat{R}_{AB}\,\delta\hat{g}^{AB},
\qquad
\delta\sqrt{-\hat{g}}=-\frac12\sqrt{-\hat{g}}\;\hat{g}_{AB}\,\delta\hat{g}^{AB}.
\]
Hence $\delta S_{\mathrm{EH}}\propto\bigl(\hat{R}_{AB}-\tfrac12\hat{g}_{AB}\hat{R}\bigr)\delta\hat{g}^{AB}$.
Since $\delta\hat{g}^{AB}$ is symmetric, only the $(AB)$ part contributes, so the unambiguous vacuum equation is
\[
\hat{G}_{(AB)}(\hat{\Gamma})\equiv \hat{R}_{(AB)}(\hat{\Gamma})-\frac12\hat{g}_{AB}\hat{R}(\hat{\Gamma})=0.
\]
\end{tcolorbox}

\begin{tcolorbox}[colback=gray!5, colframe=gray!60,
title={\textbf{Notation: Ricci tensor under }\(\hat{\Gamma}=\Gamma(\hat{g})+K\)}, fonttitle=\bfseries\small]
We use $\nabla$ for the Levi-Civita derivative of $\hat{g}_{AB}$ in this decomposition.
Define the contracted rank-two tensors
\[
(\nabla K)_{BD}:=\nabla_A K^A{}_{DB}-\nabla_D K^A{}_{AB},
\qquad
(K^2)_{BD}:=K^A{}_{AE}K^E{}_{DB}-K^A{}_{DE}K^E{}_{AB}.
\]
With these conventions the identity is literal:
\[
\hat{R}_{BD}=R_{BD}+(\nabla K)_{BD}+(K^2)_{BD}.
\]
\end{tcolorbox}

\begin{tcolorbox}[colback=blue!3, colframe=blue!40,
    title={\textbf{Lemma (Torsion-Induced Bounce Term)}}, fonttitle=\bfseries\small]
In Einstein--Cartan-type setups where torsion is algebraic, integrating out the
axial torsion produces an effective spin--spin interaction
$(\bar{\psi}\gamma_5\gamma_\mu\psi)^2$. In a homogeneous cosmological background this term
scales as $a^{-6}$ and contributes a repulsive component at small scale factor,
allowing a non-singular bounce. This provides the minimal bridge from the UFE to
bounce scenarios used in Sec.~\ref{sec:bounce}.
\end{tcolorbox}

%-----------------------------------------------------------------------------
\subsection{Present-Day Torsion Regimes and Local Tests}
\label{sec:torsion-regimes}
%-----------------------------------------------------------------------------

TFPT distinguishes between \emph{cosmological torsion} (which may be active in the early universe and drive a bounce) and \emph{local vacuum torsion} (which is constrained by laboratory and astrophysical bounds on Lorentz violation).
To avoid hidden assumptions, we treat ``torsion today'' as an explicit regime choice rather than a vague qualitative statement.

\begin{tcolorbox}[colback=yellow!3, colframe=yellow!60!black,
    title={\textbf{Regime declaration (baseline vs.\ falsifiable alternatives)}}, fonttitle=\bfseries\small]
\textbf{Baseline (suite):} local vacuum torsion today vanishes,
\[
S_\mu(\text{today})=0,
\]
so local bounds are automatically satisfied and only cosmological torsion dynamics remain relevant for early-universe completions (bounce, condensates).

\textbf{Falsifiable alternatives:} if a TFPT completion predicts a nonzero local $S_\mu$, it must lie below the component-wise bounds and should be stated as a concrete regime proposal (the suite provides an exportable JSON ``regime designer'' to make such proposals explicit).
\end{tcolorbox}

\TFPTTorsionBoundsTable

\begin{tcolorbox}[colback=gray!5, colframe=gray!60,
    title={\textbf{Two concrete torsion test benchmarks (suite; regime designer)}}, fonttitle=\bfseries\small]
The suite includes an explicit ``regime designer'' that translates spin-polarized media into a toy torsion estimate and compares it to vetted component-wise bounds (module \texttt{ux\_torsion\_regime\_designer}). 
The intent is not to claim a derived microscopic coupling here, but to make \emph{testability regimes} concrete.
\end{tcolorbox}

\begin{table}[H]
\centering
\begin{tabular}{@{}llll@{}}
\toprule
\textbf{Benchmark} & \textbf{Coupling scale} & \textbf{Density} & \textbf{$|S_\mu|$ for $c_{\rm spin}=1$ [GeV]} \\
\midrule
Electron gas (lab) & $M_{\rm Pl}^{-1}$ & $n\sim 10^{23}\,\mathrm{cm^{-3}}$ & $\sim \num{3.24e-56}$ \\
Nuclear matter core & TFPT $M^{-1}$ & $n\sim 0.16\,\mathrm{fm^{-3}}$ & $\sim \num{3.28e-31}$ \\
\bottomrule
\end{tabular}
\caption{Order-of-magnitude torsion benchmarks from \codepath{tfpt-suite/out/unconventional/ux_torsion_regime_designer}. In the lab benchmark, the signal is hopelessly small under Planck coupling; in dense matter under TFPT-$M$ coupling, the estimate approaches current bounds (depending on the effective $c_{\rm spin}$).}
\end{table}

\begin{figure}[H]
\centering
\begin{tikzpicture}
\begin{axis}[
    width=0.9\linewidth,
    height=0.42\linewidth,
    ymode=log,
    ymin=1e-60, ymax=1e-26,
    xmin=0.5, xmax=3.5,
    xtick={1,2,3},
    xticklabels={Electron gas (lab), Nuclear matter core, Extreme dense spin fluid},
    x tick label style={rotate=20, anchor=east, font=\scriptsize},
    ylabel={$|S_\mu|$ [GeV] (toy estimate; $c_{\rm spin}=1$)},
    grid=both,
    legend style={font=\scriptsize},
    legend pos=north west,
]
\addplot+[only marks, mark=*, mark size=2.5pt] coordinates {
    (1,3.2396753580225e-56)
    (2,3.2832238430084956e-31)
    (3,4.104029803760619e-30)
};
\addlegendentry{scenario anchors}

\addplot[red, dashed, thick] coordinates {(0.5,2.1e-31) (3.5,2.1e-31)};
\addlegendentry{tightest bound ($2.1\times10^{-31}$ GeV)}
\end{axis}
\end{tikzpicture}
\caption{Suite regime scan (\texttt{ux\_torsion\_regime\_designer}): torsion estimates and bound-saturation targets across illustrative spin-medium scenarios.}
\end{figure}

%-----------------------------------------------------------------------------
\subsection{Gravitational Coupling from Invariants}
%-----------------------------------------------------------------------------

\begin{corollary}{Gravitational Coupling Consistency}{kappa}
The gravitational coupling $\kappa^2 = 8\pi G$ (with dimension $[\text{energy}]^{-2}$) 
is constrained by consistency with the dimensionless ratio:
\begin{equation}
    \xi \equiv \frac{\cthree}{\phiz}
\end{equation}

\textbf{Dimensional analysis:} Both $\cthree$ and $\phiz$ are dimensionless. The 
physical gravitational coupling $G$ enters when restoring dimensions:
\begin{equation}
    \kappa^2 = 8\pi G = \frac{1}{\Mpl^2}
\end{equation}
The TFPT claim is that the \emph{ratio} $\cthree/\phiz$ is fixed by topology, 
not that $G$ is derived from dimensionless constants alone.

\textbf{Tree-level value:} With $\phitree = 1/(6\pi)$ and $\cthree = 1/(8\pi)$:
\begin{equation}
    \xi_{\text{tree}} = \frac{1/(8\pi)}{1/(6\pi)} = \frac{6\pi}{8\pi} = \frac{3}{4}
\end{equation}

\textbf{With backreaction:} Using $\phiz \approx 0.05317$:
\begin{equation}
    \xi = \frac{0.03979}{0.05317} \approx 0.7483
\end{equation}
This is a shift of $-0.22\%$ from the tree-level value.

\textbf{Physical interpretation:} The ratio $\xi$ determines the relative strength 
of axion-mediated and gravitational interactions. It is \emph{not} a derivation of 
Newton's constant $G$, which remains an input (via $M_{\text{Pl}}$).
\end{corollary}

%=============================================================================
\section{Observable Predictions}
%=============================================================================

%-----------------------------------------------------------------------------
\subsection{Axion--Photon Coupling}
%-----------------------------------------------------------------------------

\begin{corollary}{Axion--Photon Coupling}{gagg}
The CFE fixes the axion--photon vertex uniquely:
\begin{equation}
    \boxed{\gagg = -4\cthree = -\frac{1}{2\pi} \approx -0.1592}
\end{equation}
This determines cosmic birefringence without free parameters.
\end{corollary}

%-----------------------------------------------------------------------------
\subsection{Cosmic Birefringence}
%-----------------------------------------------------------------------------

\begin{definition}{Topological Excursion Quantization}{topoexcursion}
The topological axion excursion is quantized in discrete sector changes:
\begin{equation}
    \Delta a_{\text{top}} = n\,\phiz, \qquad n\in\mathbb{Z},
\end{equation}
with $n=1$ the minimal nontrivial class. The corresponding birefringence is
$\beta = \Delta a_{\text{top}}/(4\pi) = n\,\phiz/(4\pi)$.
\end{definition}

\begin{theorem}{Cosmic Birefringence Prediction}{birefringence}
A homogeneous topological axion field $a_{\text{top}}(\eta)$ rotates the polarization plane of light by:
\begin{equation}
    \frac{d\beta}{d\eta} = 2\cthree\frac{da_{\text{top}}}{d\eta} 
    \quad\Rightarrow\quad 
    \beta = \frac{\Delta a_{\text{top}}}{4\pi}
\end{equation}
Using the coupling convention $\gagg=-4\cthree$ (Corollary~\ref{cor:gagg}) and the sign convention for $\beta$ stated below, the transport law is equivalently
$\frac{d\beta}{d\eta} = -\frac{1}{2}\gagg\,\frac{da_{\text{top}}}{d\eta} = 2\cthree\,\frac{da_{\text{top}}}{d\eta}$.
With the minimal nontrivial sector $n=1$ (Definition~\ref{def:topoexcursion}):
\begin{equation}
    \boxed{\beta_{\text{th}} = \frac{\phiz}{4\pi} = 0.0042312895\ldots~\text{rad} = 0.2424^\circ}
\end{equation}
\end{theorem}

\subsubsection{Dynamical Completion of the Excursion}
\label{sec:deltaa}

\begin{tcolorbox}[colback=red!3, colframe=red!50!black, 
    title={\textbf{One Minimal Dynamical Realization: Late-Time Shifted Axion Attractor}}, fonttitle=\bfseries\small]
\textbf{Goal:} Provide one minimal dynamical realization compatible with the topological quantization (Definition~\ref{def:topoexcursion}).

\textbf{Shifted periodic potential:} The smallest periodic completion that enforces the geometric shift is
\begin{equation}
    V(a_{\text{top}},t)=\Lambda^4\Bigl[1-\cos\bigl(a_{\text{top}}-\phiz\,s(t)\bigr)\Bigr],
    \label{eq:shifted-axion-potential}
\end{equation}
where $s(t)$ transitions from $0$ to $1$ at late times (e.g.\ when a TFPT block becomes dynamically relevant).
Then the instantaneous minimum shifts from $a=0$ (early) to $a=\phiz$ (late).

\textbf{Attractor consequence (conditional):} If a late-time transition occurs after recombination and the background mode relaxes toward the new minimum,
\begin{equation}
    \Delta a_{\text{top}} = a_{\text{top}}(t_0)-a_{\text{top}}(t_{\mathrm{rec}}) \simeq \phiz \quad (n=1)
    \quad\Rightarrow\quad
    \beta \simeq \frac{\phiz}{4\pi},
\end{equation}
recovering the TFPT birefringence prediction as an attractor rather than a bare assumption.

\textbf{Timing without continuous tuning (one realization):} The effective mass is $m_{\mathrm{eff}}\sim \Lambda^2/f_a$.
Requiring $m_{\mathrm{eff}}\sim H_0$ delays the roll to very late times (robustly after recombination). With $f_a\sim 10^{11}$~GeV this corresponds to
$\Lambda\sim \sqrt{H_0 f_a}\sim 10^{-6}$~eV (order-of-magnitude).

\textbf{Discrete realization via a TFPT block (optional):} One may treat $\Lambda$ itself as the scale of an additional ultra-light TFPT block,
generated by the same universal suppression $\zeta_B \sim (\pi\cthree)\exp(-\beta_B\pi\cthree)\exp(-k_B/\cthree)$ with $k_B=\frac{3}{2}I_1(B)$.
A minimal rational charge set $\{1,2/3,2/3\}$ gives $I_1=1+2\cdot(4/9)=17/9$ and $k=17/6$, yielding $\Lambda$ naturally in the $\mu$eV range for late cascade levels,
and hence $m_{\mathrm{eff}}\sim 10^{-32}$~eV (cosmologically late roll) without introducing a continuous tuning knob.

\textbf{New test:} If the field is still evolving today, birefringence becomes redshift-dependent:
\begin{equation}
    \beta(z)=\frac{a(t_0)-a(t(z))}{4\pi}.
\end{equation}
This can be constrained by birefringence tomography using CMB and polarized sources across redshift.
\end{tcolorbox}

\begin{tcolorbox}[colback=orange!5, colframe=orange!60!black, 
    title={\textbf{Sign Convention for Birefringence}}, fonttitle=\bfseries\small]
\textbf{Definition:} $\beta > 0$ corresponds to a counter-clockwise rotation of the 
polarization plane when looking along the photon propagation direction (IAU convention).

\textbf{Relation to data pipelines:}
\begin{itemize}[leftmargin=1.5em, topsep=2pt, itemsep=1pt]
    \item Planck PR4: Reports $\beta$ in this convention. Our $\beta_{\text{th}} = +0.2424^\circ$ 
        is directly comparable.
    \item ACT DR6: Uses same sign convention; recent result $\beta = 0.215^\circ \pm 0.074^\circ$ 
        is consistent with TFPT at $0.4\sigma$.
    \item Some analyses use $\beta \equiv -\beta_{\text{obs}}$; we do \emph{not} adopt this.
\end{itemize}

\textbf{Dominant systematic:} Absolute polarization angle calibration (via Crab nebula or 
EB-nulling on Galactic dust). Current calibration uncertainty $\sim 0.1^\circ$ is comparable 
to statistical errors.
\end{tcolorbox}

\begin{table}[H]
\centering
\begin{tabular}{@{}lccccc@{}}
\toprule
\textbf{Planck PR4 Case} & $\beta_{\text{obs}}$ [$^\circ$] & $1\sigma$ [$^\circ$] & 
    $\beta_{\text{th}}$ [$^\circ$] & $|\Delta|/\sigma$\\
\midrule
Full-sky (raw EB) & 0.30 & 0.11 & 0.2424 & 0.52\\
Foreground-corrected & 0.36 & 0.11 & 0.2424 & 1.07\\
COMMANDER template & 0.16 & 0.05 & 0.2424 & 1.65\\
\bottomrule
\end{tabular}
\caption{Cosmic birefringence: TFPT prediction vs.\ Planck PR4 observations.}
\label{tab:birefringence}
\end{table}

\begin{figure}[H]
\centering
\begin{tikzpicture}
\begin{axis}[
    width=11cm, height=6cm,
    ylabel={$\beta$ [degrees]},
    symbolic x coords={Full-sky, Foreground-corr., COMMANDER},
    xtick=data,
    ymin=0, ymax=0.5,
    ytick={0, 0.1, 0.2, 0.3, 0.4, 0.5},
    grid=major,
    legend pos=north east,
]
    % Observations with error bars
    \addplot[
        only marks,
        mark=square*,
        mark size=4pt,
        blue,
        error bars/.cd,
        y dir=both, y explicit
    ] coordinates {(Full-sky,0.30) +- (0,0.11) (Foreground-corr.,0.36) +- (0,0.11) (COMMANDER,0.16) +- (0,0.05)};
    \addlegendentry{Planck PR4}
    
    % Theory line
    \addplot[thick, red, dashed, domain=0:3] coordinates {(Full-sky,0.2424) (Foreground-corr.,0.2424) (COMMANDER,0.2424)};
    \addlegendentry{$\beta_{\text{th}} = 0.2424^\circ$}
\end{axis}
\end{tikzpicture}
\caption{Cosmic birefringence: TFPT prediction compared with Planck PR4 
measurements. Theory is within 0.5--1.7$\sigma$ of all observations.}
\label{fig:birefringence}
\end{figure}

%-----------------------------------------------------------------------------
\subsection{Two-Loop RG Fingerprints}
%-----------------------------------------------------------------------------

\begin{corollary}{RG Fingerprints}{RGfingerprints}
The TFPT invariants leave signatures in the RG flow of gauge couplings:
\begin{align}
    \alpha_3(1\,\text{PeV}) &\approx \phiz \quad 
        \text{(two-loop extrapolated: } \num[round-mode=places,round-precision=6]{\TFPTalphaThreePeVNum}, \text{ dev: } -\num[round-mode=places,round-precision=3]{\TFPTalphaThreePeVRelDevPercent}\%\text{)}\\
    \alpha_3(\mu_{\cthree}) &\approx \cthree \quad 
        \text{(crossing at } \mu_{\cthree}\approx \num[round-mode=places,round-precision=3]{\TFPTmuCrosscThreeGeV}\ \text{GeV, dev: } 0.00\%\text{)}
\end{align}
These values are obtained by two-loop RG running from PDG boundary conditions at $M_Z$ using the declared PyR@TE model (checked by \texttt{two\_loop\_rg\_fingerprints} in \texttt{tfpt-suite/}; not direct measurements at the quoted scales). The crossing scale for $\alpha_3(\mu)=\phiz$ is $\mu\approx \num[round-mode=places,round-precision=0]{\TFPTmuCrossVarphiZeroGeV}$~GeV.
\end{corollary}

\begin{figure}[H]
\centering
\begin{tikzpicture}
\begin{axis}[
    width=10cm, height=6cm,
    ybar,
    bar width=25pt,
    ylabel={Deviation [\%]},
    symbolic x coords={$\alpha_3(1\,\text{PeV})$, $\alpha_3(\mu_{c_3})$},
    xtick=data,
    ymin=-1.6, ymax=0.2,
    nodes near coords,
    nodes near coords align={vertical},
    every node near coord/.append style={font=\small},
    grid=major,
]
    \addplot[fill=green!50] coordinates 
        {($\alpha_3(1\,\text{PeV})$, -\TFPTalphaThreePeVRelDevPercent) ($\alpha_3(\mu_{c_3})$, 0.000)};
\end{axis}
\end{tikzpicture}
\caption{Two-loop RG fingerprints: percent-level deviation from TFPT predictions at 1~PeV, and a constructed crossing for $\alpha_3(\mu_{\cthree})=\cthree$.}
\label{fig:twoloop}
\end{figure}

%=============================================================================
% PART II: STRUCTURAL EXTENSIONS
%=============================================================================
\vspace{1em}
\begin{center}
\rule{0.8\textwidth}{0.5pt}\\[0.5em]
{\Large\bfseries Part II: Structural Extensions}\\[0.3em]
{\small The following sections extend the core CFE framework. They are \\
testable structural postulates that do not affect the $\alpha$, $\beta$, \\
or RG fingerprint predictions established in Part I.}\\[0.5em]
\rule{0.8\textwidth}{0.5pt}
\end{center}
\vspace{1em}

%=============================================================================
\section{\texorpdfstring{The E$_8$ Cascade and Block Constants}{The E8 Cascade and Block Constants}}
\label{sec:e8}
%=============================================================================

%-----------------------------------------------------------------------------
\subsection{Nilpotent Orbit Structure}
%-----------------------------------------------------------------------------

\begin{definition}{E$_8$ Orbit Graph}{orbitgraph}
Let $\mathcal{N}(\mathfrak{e}_8)$ denote the set of nilpotent orbits of the 
exceptional Lie algebra $\mathfrak{e}_8$. For each orbit $\mathcal{O}$, define:
\begin{itemize}[leftmargin=1.5em, topsep=2pt, itemsep=1pt]
    \item $\dim(\mathcal{O})$: the dimension of the orbit as a variety
    \item $D(\mathcal{O}) = 248 - \dim(\mathcal{O})$: the ``centralizer dimension''
    \item $h(\mathcal{O})$: the height (sum of coefficients in weighted Dynkin diagram)
\end{itemize}
The \textbf{orbit graph} $G = (V, E)$ has vertices $V = \mathcal{N}(\mathfrak{e}_8)$ 
and directed edges $\mathcal{O} \to \mathcal{O}'$ iff:
\begin{enumerate}[leftmargin=1.5em, topsep=2pt, itemsep=1pt]
    \item $\mathcal{O}' \subset \overline{\mathcal{O}}$ (closure containment)
    \item $D(\mathcal{O}) - D(\mathcal{O}') = 2$ (minimal step)
\end{enumerate}
\end{definition}

\begin{theorem}{E$_8$ Chain Uniqueness}{chainunique}
Consider the closure ordering on $\mathcal{N}(\mathfrak{e}_8)$. There exists a 
\textbf{unique saturated chain} from the subregular orbit ($D = 58$) to the 
principal orbit ($D = 8$) satisfying:
\begin{enumerate}[leftmargin=1.5em, topsep=2pt, itemsep=1pt]
    \item All edges are cover relations with $\Delta D = 2$
    \item At each layer, the orbit of minimal height is selected
    \item The Levi structure is monotonically extended
\end{enumerate}
This chain has exactly 26 steps, yielding the sequence $D_n = 60 - 2n$ for $n = 1, \ldots, 26$.
\end{theorem}

\begin{proof}[Proof sketch]
The proof proceeds by explicit construction on the Hasse diagram of 
$\mathcal{N}(\mathfrak{e}_8)$:
\begin{enumerate}[leftmargin=1.5em, topsep=2pt, itemsep=1pt]
    \item \textbf{Layer structure:} Orbits are partitioned by $D$-value into 
        27 layers ($D \in \{60, 58, \ldots, 8\}$).
    \item \textbf{Cover relations:} For E$_8$, the cover relations are classified 
        by Bala--Carter labels. At each $D$-value, there may be multiple orbits, 
        but the height function provides a total order.
    \item \textbf{Uniqueness:} At each step $D \to D-2$, we select the unique 
        orbit that (a) is reachable from the current orbit via a cover relation, 
        and (b) has minimal height among all reachable candidates.
    \item \textbf{Verification:} The complete chain is given in Appendix~\ref{app:E8cert} 
        with explicit Bala--Carter labels.
\end{enumerate}
\end{proof}

%-----------------------------------------------------------------------------
\subsection{Log-Exact Ladder Formula}
%-----------------------------------------------------------------------------

\begin{lemma}{Holonomy degeneracy anchor for $\gamma(0)$}{holonomydeg}
TFPT uses the SU(5) hypercharge holonomy data already fixed elsewhere in the construction.
On the fundamental splitting, the holonomy spectrum separates into two eigenspaces of dimensions $3$ and $2$, hence we define the discrete degeneracy count
\[
g:=3+2=5.
\]
The minimal damping ratio compatible with a single additional trivial sector is then
\[
\gamma(0):=\frac{g}{g+1}=\frac56.
\]
\end{lemma}

\begin{definition}{E$_8$ Cascade Ladder}{E8cascade}
For $n \geq 1$, the ladder values are:
\begin{equation}
    \varphi_n = \phiz \cdot e^{-\gamma(0)} \cdot 
        \left(\frac{D_n}{D_1}\right)^\lambda, \quad D_n = 60 - 2n
    \label{eq:ladder}
\end{equation}
with:
\begin{align}
    \gamma(0) &= \frac{g}{g+1}=\frac{5}{6} \quad \text{(holonomy anchor; Lemma~\ref{lem:holonomydeg})}\\
    \lambda &= \frac{\gamma(0)}{\ln(248) - \ln(60)} = \frac{\gamma(0)}{\ln(248/60)} 
        \approx 0.587233
\end{align}
\end{definition}

\textbf{Parameter derivation (not fitted):}
\begin{itemize}[leftmargin=1.5em, topsep=2pt, itemsep=2pt]
    \item \textbf{$D_n = 60 - 2n$:} Forced by Theorem~\ref{thm:chainunique}. The 
        arithmetic sequence is not assumed but \emph{derived} from the unique chain.
    \item \textbf{$\gamma(0)$:} Fixed by SU(5) holonomy degeneracy $g=3+2=5$, giving $\gamma(0)=g/(g+1)=5/6$ (Lemma~\ref{lem:holonomydeg}). No continuous fit is introduced.
    \item \textbf{$\lambda$:} The exponent 248 is $\dim(\mathfrak{e}_8)$; the exponent 60 
        is $D_{\max}$ (adjoint orbit). The ratio $\gamma(0)/\ln(248/60)$ ensures scale 
        consistency across the full ladder.
\end{itemize}

\begin{remark}
\textbf{Numerical stability:} $\gamma(0)=5/6$ differs from the previously used rounded value $0.834$ at the $0.08\%$ level, leaving the cascade scales and block table stable within rounding. 
This supplies a discrete holonomy anchor (via $g=5$) while preserving the percent-level structure tests.
\end{remark}

\begin{tcolorbox}[colback=blue!3, colframe=blue!40, 
    title={\textbf{Lemma (Log-Exact vs.\ Quadratic)}}, fonttitle=\bfseries\small]
The damping function $\gamma(n)$ is \emph{log-exact}, not quadratic:
\begin{equation}
    \gamma(n) = \lambda \ln\frac{D_n}{D_{n+1}} = \lambda \ln\frac{60-2n}{58-2n} 
    \approx \frac{\lambda}{29-n} \quad (n \ll 29)
\end{equation}
A quadratic fit $\gamma(n) = a + bn + cn^2$ is diagnostic only and deviates 
systematically at large $n$. The log-exact form is the \emph{theory}; quadratic 
is an approximation valid near $n \approx 0$.
\end{tcolorbox}

\begin{remark}
The E$_8$ cascade is a \emph{structural postulate} with testable consequences. 
It does not affect the core CFE result for $\alpha$, which depends only on 
$(\cthree, \phiz, \bone)$. Even if the cascade is modified or rejected, the 
$\alpha$ prediction stands independently.
\end{remark}

\begin{figure}[H]
\centering
\begin{tikzpicture}
\begin{axis}[
    width=12cm, height=7cm,
    xlabel={Cascade step $n$},
    ylabel={$\varphi_n$},
    xmin=0, xmax=26,
    ymin=0, ymax=0.06,
    grid=major,
    legend pos=north east,
]
    % Ladder values (approximate)
    \addplot[thick, blue, mark=o, mark size=2pt] coordinates {
        (1, 0.053) (2, 0.050) (3, 0.048) (4, 0.046) (5, 0.044)
        (6, 0.042) (7, 0.040) (8, 0.038) (9, 0.037) (10, 0.035)
        (11, 0.033) (12, 0.031) (13, 0.030) (14, 0.028) (15, 0.027)
        (16, 0.026) (17, 0.025) (18, 0.024) (19, 0.023) (20, 0.022)
        (21, 0.021) (22, 0.020) (23, 0.019) (24, 0.018) (25, 0.017)
    };
    \addlegendentry{$\varphi_n$ ladder}
    
    % Key scale markers
    \addplot[only marks, mark=square*, mark size=4pt, red] 
        coordinates {(12, 0.031)};
    \node[red, right, font=\footnotesize] at (axis cs:12.5, 0.031) {EW ($n=12$)};
    
    \addplot[only marks, mark=square*, mark size=4pt, orange] 
        coordinates {(15, 0.028)};
    \node[orange, right, font=\footnotesize] at (axis cs:15.5, 0.028) {Hadronic ($n=15$)};
    
    \addplot[only marks, mark=square*, mark size=4pt, green!60!black] 
        coordinates {(10, 0.033)};
    \node[green!60!black, left, font=\footnotesize] at (axis cs:9.5, 0.033) {PQ ($n=10$)};
\end{axis}
\end{tikzpicture}
\caption{E$_8$ log-exact ladder $\varphi_n$. Key Standard Model scales 
(EW, Hadronic, PQ) are marked.}
\label{fig:ladder}
\end{figure}

%-----------------------------------------------------------------------------
\subsection{Block Constant Formula and Derivation}
%-----------------------------------------------------------------------------

\begin{tcolorbox}[colback=green!3, colframe=green!50!black, 
    title={\textbf{Lemma (Boundary Trace Index)}}, fonttitle=\bfseries]
For a 6D Weyl fermion zero mode on a boundary cycle $C_i$ with flat $U(1)_Y$ 
connection (holonomy $\oint_{C_i} A$), the variation of $\ln\det(\not{\!D})$ 
under the connection is proportional to the quadratic charge index:
\begin{equation}
    I_1(B) = \sum_{\Phi \in B} \sum_{i \in U(1)_Y} q_i(\Phi)^2
\end{equation}
The proportionality constant is fixed by the topological normalization $\cthree$.
\end{tcolorbox}

\begin{theorem}{Block Index Formula}{blockindex}
On the orientable double cover with three boundary cycles $(C_1, C_2, C_T)$, 
the block coefficient is:
\begin{equation}
    k_B = \frac{3}{2}\, I_1(B)
\end{equation}
where the factor $3/2$ arises from: (i) factor 3 from three boundary cycles, 
(ii) factor $1/2$ from the double cover counting.
\end{theorem}

\begin{proof}[Proof sketch]
The APS index theorem on manifolds with boundary gives a $\eta$-invariant 
contribution proportional to boundary data. For flat connections on $C_i$, 
this reduces to a sum over charge squares. The three cycles contribute 
additively. The double cover halves the effective weight (one physical 
representation spans two sheets). See Appendix~\ref{app:boundary} for details.
\end{proof}

\begin{corollary}{Abelian Trace Consistency}{tracecons}
The number 41 appears twice in TFPT:
\begin{enumerate}[leftmargin=1.5em, topsep=2pt, itemsep=1pt]
    \item As $\bone = 41/10$ in the 4D RG beta function (CFE)
    \item As $k_{\text{EW}} = 41/32$ in the electroweak block ($I_1^{\text{EW}} = 41/48$)
\end{enumerate}
Both originate from the \textbf{same} quadratic hypercharge trace in the SM spectrum:
\begin{equation}
    \sum_{\text{SM}} Y^2 = \frac{41}{6} \quad \text{(GUT normalization)}
\end{equation}
This is not a coincidence but a structural echo of the abelian anomaly.
\end{corollary}

\begin{definition}{Block Constants}{blocks}
Physical scales $X_B$ are derived via the \textbf{three-step algorithm}:

\textbf{Step 1:} Evaluate ladder $\varphi_{n_B}$ from \eqref{eq:ladder}.

\textbf{Step 2:} Compute block factor:
\begin{equation}
    \zeta_B = (\pi\cthree) \cdot e^{-\beta_B\pi\cthree} 
        \cdot e^{-k_B/\cthree}
\end{equation}
where $\beta_B = (8 - r_B)/8$, $r_B$ is the sector rank, and $k_B = \frac{3}{2}I_1(B)$.
We adopt the minimal choice $G_B=1$, i.e.\ no additional per-block group prefactor beyond the universal $E_8$ ladder $\varphi_n$.

\textbf{Step 3:} Compute scale (using \textbf{unreduced} Planck mass $M_{\mathrm{Pl}}$):
\begin{equation}
    X_B = \zeta_B \cdot M_{\mathrm{Pl}} \cdot \varphi_{n_B}
\end{equation}
\end{definition}

\begin{table}[H]
\centering
\begin{tabular}{@{}lccccccc@{}}
\toprule
\textbf{Block} & $n$ & $r_B$ & $k_B$ & $I_1$ & $X_{\text{th}}$ [GeV] & 
    $X_{\text{ref}}$ [GeV] & \textbf{Ratio}\\
\midrule
Electroweak & 12 & 2 & 41/32 & 41/48 & 251.1 & 246.2 & 1.020\\
Hadronic & 15 & 5 & 3/2 & 1 & 0.968 & 0.938 & 1.032\\
Pion & 16 & 5 & 51/32 & 17/16 & 0.088 & 0.092 & 0.957\\
Neutrino (seesaw) & 5 & 4 & 1/8 & 1/12 & $1.31\times10^{15}$ & (paper ref.) & 1.000\\
Axion (PQ) & 10 & 1 & 1/2 & 1/3 & $8.86\times10^{10}$ & (derived) & 1.000\\
\bottomrule
\end{tabular}
\caption{Block constants derived from TFPT. Note: $X_{\text{ref}}$ for Neutrino 
and Axion are internal paper benchmarks (not direct measurements and not used as inputs).}
\label{tab:blocks}
\end{table}

\begin{figure}[H]
\centering
\begin{tikzpicture}
\begin{axis}[
    width=10cm, height=5cm,
    xbar,
    bar width=12pt,
    xlabel={Ratio (theory/experimental)},
    symbolic y coords={Pion, Hadronic, Electroweak, Axion, Neutrino},
    ytick=data,
    xmin=0.9, xmax=1.1,
    xtick={0.9, 0.95, 1.0, 1.05, 1.1},
    grid=major,
    nodes near coords,
    nodes near coords align={horizontal},
    every node near coord/.append style={font=\footnotesize},
]
    \addplot[fill=blue!50] coordinates {(0.957,Pion) (1.032,Hadronic) (1.020,Electroweak) (1.000,Axion) (1.000,Neutrino)};
    
    % Perfect agreement line
    \draw[thick, red, dashed] (axis cs:1.0,Pion) -- (axis cs:1.0,Neutrino);
\end{axis}
\end{tikzpicture}
\caption{Block constant ratios. All scales agree within a few percent.}
\label{fig:blocks}
\end{figure}

%-----------------------------------------------------------------------------
\subsection{Neutrino Seesaw Predictions}
%-----------------------------------------------------------------------------

The neutrino block at $n = 5$ provides concrete predictions for the seesaw mechanism.

\begin{corollary}{Seesaw Scale and Neutrino Masses}{seesaw}
From the block formula:
\begin{equation}
    M_R = \zeta_{\text{NR}} \cdot M_{\mathrm{Pl}} \cdot \varphi_5 
    \approx 1.31 \times 10^{15}~\text{GeV}
\end{equation}
With the Higgs VEV $v_H \approx 246$ GeV and $\mathcal{O}(1)$ Yukawa coupling $y_{\nu 3} \sim 1$:
\begin{equation}
    m_{\nu 3} \simeq \frac{(y_{\nu 3} v_H)^2}{M_R} 
    \approx \frac{(246~\text{GeV})^2}{1.31 \times 10^{15}~\text{GeV}} 
    \approx 0.046~\text{eV}
\end{equation}
This yields the atmospheric mass-squared splitting:
\begin{equation}
    \Delta m^2_{31} \approx m_{\nu 3}^2 \approx 2.1 \times 10^{-3}~\text{eV}^2
\end{equation}
\textbf{Comparison:} Experimental value $\Delta m^2_{31} = (2.453 \pm 0.033) \times 10^{-3}$ eV$^2$ 
(normal ordering). Agreement: $\sim 15\%$.
\end{corollary}

\begin{remark}
The 15\% deviation is expected: (i) $y_{\nu 3} = 1$ is an approximation; the 
actual Yukawa may differ by $\mathcal{O}(1)$ factors. (ii) The seesaw formula 
neglects sub-leading contributions. The key prediction is the \emph{scale} 
$M_R \sim 10^{15}$ GeV, which is robust.
\end{remark}

%=============================================================================
\section{\texorpdfstring{Flavor Structure: The Z$_3$ Architecture}{Flavor Structure: The Z3 Architecture}}
%=============================================================================

%-----------------------------------------------------------------------------
\subsection{\texorpdfstring{Universal Phase $\delta$}{Universal Phase delta}}
%-----------------------------------------------------------------------------

\begin{tcolorbox}[colback=purple!3, colframe=purple!50!black, 
    title={\textbf{Lemma (Cross-Ratio from Boundary Monodromy)}}, fonttitle=\bfseries\small]
On a conformally flat boundary geometry, the unique projectively invariant quantity 
from four marked points is the cross-ratio. Any boundary monodromy acts as a 
PSL$(2,\mathbb{R})$ transformation. Therefore, relations depending only on boundary 
monodromies must be rational functions of a M\"obius map.

\textbf{Proof idea:} Flat Wilson lines on the boundary act as holomorphic automorphisms. 
The chirality projectors allow only projective data. The cross-ratio is the unique 
projective invariant.
\end{tcolorbox}

\begin{definition}{Flavor Phase}{delta}
A single universal phase $\delta$ governs all mass hierarchies:
\begin{equation}
    \delta_\star = \frac{3}{5} + \frac{\phiz}{6} \approx 0.6089
\end{equation}

\textbf{Derivation of $\delta_\star$ (first-order closure):}
\begin{enumerate}[leftmargin=1.5em, topsep=2pt, itemsep=1pt]
    \item The leading term $3/5$ arises from the abelian trace in GUT normalization 
        (same trace appearing in $\bone$ and $k_B$).
    \item The correction $\phiz/6$ comes from the M\"obius length scale: $\phiz$ 
        enters linearly via conformal rescaling of the boundary, and the factor 
        $1/6$ is the boundary normalization from the double cover ($6\pi$ total curvature).
    \item Higher orders are $\mathcal{O}(\phiz^2) \sim 10^{-3}$, negligible.
\end{enumerate}

\textbf{Empirical extraction:}
\begin{equation}
    \delta_M = \frac{\sqrt{m_\tau/m_\mu} - 1}{\sqrt{m_\tau/m_\mu} + 1} \approx 0.6079
\end{equation}
Agreement: $|\delta_\star - \delta_M| / \delta_M \approx -0.16\%$.
\end{definition}

\begin{tcolorbox}[colback=gray!5, colframe=gray!50, 
    title={\textbf{Why These Cusps?}}, fonttitle=\bfseries\small]
The M\"obius map is evaluated at cusps $y \in \{1, 1/3, 2/3\}$. These are the 
\textbf{unique} rational values satisfying:
\begin{enumerate}[leftmargin=1.5em, topsep=2pt, itemsep=1pt]
    \item Boundary holonomies are quantized $\Rightarrow$ cusps must be rational.
    \item GUT normalization compatibility: $1/3$ and $2/3$ match the hypercharge 
        assignments in SU(5) embedding.
    \item Sector separation: $y = 1$ (leptons), $y = 1/3$ (down-type), $y = 2/3$ (up-type).
\end{enumerate}
This is a classification result: given these constraints, the cusp set is unique.
\end{tcolorbox}

%-----------------------------------------------------------------------------
\subsection{M\"obius Mass Relations}
%-----------------------------------------------------------------------------

\begin{corollary}{Mass Ratios from M\"obius Map}{massratios}
The M\"obius map $M_y(\delta) = (y + \delta)/(y - \delta)$ evaluated at 
cusps $y \in \{1, 1/3, 2/3\}$ yields mass ratios:

\begin{center}
\begin{tabular}{@{}lcccc@{}}
\toprule
\textbf{Ratio} & \textbf{M\"obius Form} & \textbf{Prediction} & \textbf{Empirical} & \textbf{Status}\\
\midrule
$m_\tau/m_\mu$ & $(M_1(\delta))^2$ & 16.82 & 16.82 & anchor\\
$m_\mu/m_e$ & $(M_1(\delta)|M_{1/3}(\delta)|)^2$ & 197.6 & $\approx 207$ & pre-RG\\
$m_c/m_u$ & $(M_{2/3}(\delta))^2$ & 470.5 & $\approx 588$ & scheme-dep.\\
$m_t/m_c$ & $(2/3/(2/3-\delta))^2$ & 128.7 & $\approx 136$ & scheme-dep.\\
\bottomrule
\end{tabular}
\end{center}

\textbf{Status key:}
\begin{itemize}[leftmargin=1.5em, topsep=2pt, itemsep=1pt]
    \item \textit{anchor}: Input used to fix $\delta$; consistency check only.
    \item \textit{pre-RG}: Prediction before RG running; final comparison requires 
        evolving to common scale.
    \item \textit{scheme-dep.}: Quark masses are renormalization-scheme dependent 
        ($\overline{\text{MS}}$ vs.\ pole mass). Raw ratios are indicative; 
        precise comparison requires full RG treatment.
\end{itemize}
\end{corollary}

%-----------------------------------------------------------------------------
\subsection{Cabibbo Angle}
%-----------------------------------------------------------------------------

\begin{corollary}{Cabibbo Angle}{Cabibbo}
The Cabibbo angle is fixed parameter-free by $\phiz$:
\begin{equation}
    \begin{aligned}
        \lambda = \sin\theta_C 
        &= \sqrt{\phiz}\left(1 - \tfrac{1}{2}\phiz\right) \\
        &= \sqrt{4\pi\beta_{\text{rad}}}\left(1 - 2\pi\beta_{\text{rad}}\right) \\
        &= 0.22445997\ldots
    \end{aligned}
\end{equation}
PDG reference: $\lambda = 0.2248$. Deviation: $-0.15\%$.
\end{corollary}

\begin{tcolorbox}[colback=gray!5, colframe=gray!60, 
    title={\textbf{CKM Conventions and Scale (Clarification)}}, fonttitle=\bfseries\small]
\textbf{Convention:} We use the PDG/Wolfenstein identification $\lambda = |V_{us}|$.

\textbf{Scope:} TFPT uses the Cabibbo relation above as a low-energy prediction anchored in $\phiz$; it does \emph{not} compute the full CKM fit, RG evolution of CKM parameters, or threshold matching.
Any extension to a full CKM prediction must specify scheme and reference scale (e.g.\ $\overline{\mathrm{MS}}$ at $M_Z$) and propagate experimental/hadronic uncertainties.
\textbf{Disposition:} A publication-grade CKM matrix reconstruction is treated as a separate, RG-complete program; the verification suite ships a deterministic RG-dressed texture pipeline (PyR@TE-generated two-loop betas with explicit threshold bookkeeping and a declared reference-scale comparison) to make the conventions and extraction steps explicit.
\end{tcolorbox}

\begin{tcolorbox}[colback=gray!5, colframe=gray!60,
    title={\textbf{Convention Lock (Holdout Protocol; avoids ``convention shopping'')}}, fonttitle=\bfseries\small]
\textbf{Problem:} Even if all candidates are discrete, a ``pick the smallest $\chi^2$'' rule can accidentally select a convention that fits a subset of entries but fails badly on the remaining ones.

\textbf{Suite protocol (unconventional audit):} We therefore run a deterministic holdout scan (\texttt{ux\_flavor\_holdout\_search}) with
\[
\text{fit keys}=\{V_{ud},V_{us},V_{cd},V_{cs},V_{cb},V_{ts},V_{tb}\},\qquad
\text{holdout}=\{V_{ub},V_{td}\}.
\]
\textbf{Outcome:} The best-fit candidate (\texttt{koide\_pi\_over\_12}) achieves $\chi^2_{\rm fit}\simeq 2.48$ but catastrophically fails the holdout ($\chi^2_{\rm holdout}\simeq 186$). The stable convention is \texttt{pi\_times\_1\_minus\_delta} with $\chi^2_{\rm total}\simeq 5.11$ and $\chi^2_{\rm holdout}\simeq 2.38$, and is treated as the preferred default (\texttt{v107sm\_default}) rather than as a post-hoc pick.
\end{tcolorbox}

\begin{figure}[H]
\centering
\begin{tikzpicture}
\begin{groupplot}[
    group style={group size=2 by 1, horizontal sep=1.2cm},
    ybar,
    bar width=10pt,
    width=0.48\textwidth,
    height=0.42\textwidth,
    xmin=0.5, xmax=4.5,
    xtick={1,2,3,4},
    xticklabels={
        \texttt{pi\_times\_1\_minus\_delta},
        \texttt{koide\_pi\_over\_12},
        \texttt{pi\_times\_delta},
        \texttt{2pi\_times\_delta}
    },
    x tick label style={rotate=25, anchor=east, font=\scriptsize},
    grid=both,
    legend style={font=\scriptsize},
]
\nextgroupplot[
    title={fit $\chi^2$},
    ylabel={$\chi^2_{\rm fit}$},
    ymin=0, ymax=6.2,
]
\addplot coordinates {
    (1,2.7273332483922754)
    (2,2.479112421420773)
    (3,4.05899978942247)
    (4,5.162857066748017)
};

\nextgroupplot[
    title={holdout $\chi^2$},
    ylabel={$\chi^2_{\rm holdout}$},
    ymode=log,
    ymin=1, ymax=1000,
]
\addplot coordinates {
    (1,2.378204155291164)
    (2,168.10522472179576)
    (3,173.9429926914698)
    (4,347.6044990986585)
};
\end{groupplot}
\end{tikzpicture}
\caption{Unconventional holdout scan (\texttt{ux\_flavor\_holdout\_search}): discrete CKM convention candidates ranked by fit-$\chi^2$ with explicit holdout penalties.}
\end{figure}

\subsubsection{Suite benchmark: RG-dressed CKM pipeline (explicit conventions)}
The shipped suite includes a deterministic CKM reconstruction pipeline with declared UV textures, two-loop RG dressing, and a reference-scale comparison at $M_Z$ (module \texttt{ckm\_full\_pipeline}). The baseline benchmark gives a CKM matrix close to PDG at the level quantified by the reported $\chi^2=\num{\TFPTCKMchiTwo}$.

\begin{tcolorbox}[colback=yellow!3, colframe=yellow!60!black, breakable,
title={\textbf{Benchmark hygiene: explicit external phase override (no hidden fit)}}, fonttitle=\bfseries\small]
\textbf{What is compared:} the pipeline benchmarks the \emph{magnitudes} $|V_{ij}|$ at a declared reference scale (native $m_t$, comparison at $M_Z$) under a fully explicit RG/matching policy.

\textbf{External anchor (declared):} the CKM CP phase is currently anchored by a declared convention,
\[
\delta_{\mathrm{CKM}} := \delta_{\mathrm{CP}}(\mathrm{PDG}) = \pi/3,
\]
i.e.\ the suite runs in \texttt{delta\_mode=external\_delta\_cp\_override} for this benchmark (see \codepath{tfpt-suite/out/ckm_full_pipeline/report.txt}).

\textbf{Status:} this section is therefore a \emph{benchmark pipeline} (deterministic, audited, no continuous fitting), not a closed TFPT prediction of $\delta_{\mathrm{CKM}}$. Replacing the external phase override by a discrete topological phase selection is an explicit roadmap item.
\end{tcolorbox}

\TFPTCKMMatrixBlock

\TFPTYukawaTextureBlock

\TFPTYukawaHeatmapPlot

%-----------------------------------------------------------------------------
\subsection{PMNS Neutrino Mixing: TM1 Pattern}
%-----------------------------------------------------------------------------

\begin{tcolorbox}[colback=blue!3, colframe=blue!50!black, 
    title={\textbf{PMNS: closed sub-block vs.\ open interface}}, fonttitle=\bfseries]
TFPT provides a \emph{closed} prediction for $(\theta_{13},\theta_{12})$ via a single kernel identity plus the TM1 pattern. The remaining PMNS degrees of freedom $(\theta_{23},\delta_{\mathrm{CP}})$ are treated as an \emph{open interface} requiring explicit Z$_3$-breaking and (eventually) correlated Yukawa/RG treatment; they must not be read as closed predictions at this stage.

\textbf{Key identity} (from $\beta_{\text{rad}} = \phiz/(4\pi)$):
\begin{equation}
    \sin^2\theta_{13} = 4\pi\,\beta_{\text{rad}}\, e^{-\gamma(0)} 
    = \frac{\phiz}{8\pi\cthree}\, e^{-\gamma(0)} = \num[round-mode=places,round-precision=6]{\TFPTsinSqThetaThirteen}
\end{equation}
where $\gamma(0)=g/(g+1)=5/6$ is fixed by holonomy degeneracy (Definition~\ref{def:E8cascade}; not fitted). 
This predicts $\theta_{13} \approx 8.74^\circ$.

\textbf{TM1 sum rule} (parameter-free $\theta_{12}$):
\begin{equation}
    \sin^2\theta_{12} = \frac{1}{3}\bigl(1 - 2\sin^2\theta_{13}\bigr) = 0.318
\end{equation}

\textbf{Leading-order (interface) values:} Z$_3$ symmetry gives $\sin^2\theta_{23}=0.5$ and $\delta_{\mathrm{CP}}=\pm 90^\circ$ at leading order, but these are not treated as closed outputs without the breaking corrections below.

\medskip
\begin{center}
\begin{tabular}{@{}lll@{}}
\toprule
\textbf{Category} & \textbf{Quantity} & \textbf{Status}\\
\midrule
Closed (kernel + TM1) & $(\sin^2\theta_{13},\sin^2\theta_{12})$ & derived / testable\\
Open interface (requires Z$_3$ breaking + RG) & $(\theta_{23},\delta_{\mathrm{CP}})$ & candidates / scan (not closed)\\
\bottomrule
\end{tabular}
\end{center}

\begin{center}
\begin{tabular}{@{}lcccc@{}}
\toprule
\textbf{Parameter} & \textbf{TFPT} & \textbf{Experiment} & \textbf{Deviation} & \textbf{Status}\\
\midrule
$\sin^2\theta_{13}$ & 0.02311 & 0.0220 & $+5.04\%$ & \textbf{closed (testable)}\\
$\sin^2\theta_{12}$ & 0.318 & 0.307 & $+3.56\%$ & \textbf{closed (derived)}\\
$\sin^2\theta_{23}$ & 0.500 (LO) & 0.546 & $-8.4\%$ & \textbf{open interface}\\
$\delta_{\text{CP}}$ & $\pm 90^\circ$ (LO) & $\sim 195^\circ$ & --- & \textbf{open interface}\\
\bottomrule
\end{tabular}
\end{center}

\textbf{Refinement path}: The $\theta_{23}$ deviation and $\delta_{\mathrm{CP}}$ phase require 
Z$_3$ breaking corrections of order $\phiz/6 \approx 0.009$, which shift $\theta_{23}$ toward 
the experimental value. Full treatment requires the CKM--PMNS texture correlation via 
Yukawa RG running (see Appendix~\ref{app:flavor}).
\end{tcolorbox}

\TFPTZthreePermutationMatrix

\TFPTPMNSVariantTable

%=============================================================================
\section{\texorpdfstring{Inflation: $R^2$ Completion (Starobinsky Limit)}{Inflation: R2 Completion (Starobinsky Limit)}}
%=============================================================================

The TFPT topological correction scale is set by $\cthree^4$ (cf.\ $\deltatop = 48\cthree^4$). 
In effective gravity, the leading higher-curvature completion compatible with general covariance is an $R^2$ term.

\begin{definition}{Starobinsky $R^2$ Completion}{R2scale}
We consider the Starobinsky-type completion
\begin{equation}
    S_{\text{grav}} \supset \int d^4x\sqrt{-g}\;\frac{\Mpl^2}{2}\left(R + \frac{1}{6M^2}R^2\right).
    \label{eq:R2action}
\end{equation}
The scale $M$ is derived from the torsionful UFE effective action under
assumptions K1--K3 in Section~\ref{sec:r2derive}, yielding the TFPT prediction in
\eqref{eq:R2scale} (Appendix~\ref{app:r2}).
\end{definition}

\begin{theorem}{Starobinsky Predictions (Amplitude, Tilt, Tensors)}{inflation}
At large e-fold number $N$, the $R^2$ completion yields:
\begin{align}
    n_s &= 1 - \frac{2}{N}, \label{eq:ns}\\
    r &= \frac{12}{N^2}, \label{eq:r}\\
    A_s &\simeq \frac{N^2}{24\pi^2}\left(\frac{M}{\Mpl}\right)^2. \label{eq:As}
\end{align}
\end{theorem}

\begin{table}[H]
\centering
\begin{tabular}{@{}ccccc@{}}
\toprule
$N$ & $n_s$ & $r$ & $A_s$ & \textbf{Status}\\
\midrule
55 & 0.9636 & 0.0040 & $2.02\times10^{-9}$ & viable\\
\textbf{56} & \textbf{0.9643} & \textbf{0.0038} & $\mathbf{2.09\times10^{-9}}$ & \textbf{benchmark}\\
57 & 0.9649 & 0.0037 & $2.17\times10^{-9}$ & viable\\
\bottomrule
\end{tabular}
\caption{Starobinsky $R^2$ inflation with $M/\Mpl=\sqrt{8\pi}\cthree^4$. The scalar amplitude $A_s$ is close to the observed $2.1\times10^{-9}$ for $N\simeq 56$.}
\label{tab:inflation}
\end{table}

\subsection{\texorpdfstring{$R^2$ Term from the Torsionful Effective Action}{R2 Term from the Torsionful Effective Action}}
\label{sec:r2derive}

\begin{definition}{Relevant Torsion Degrees}{R2torsion}
Decompose torsion into irreducible parts $(S_\mu,\,T_\mu,\,q_{\mu\nu\rho})$ (axial, trace, tensor).
In the minimal UFE setting the dominant coupling is through axial torsion $S_\mu$, and the
torsion sector is treated at quadratic order so it can be integrated out in the one-loop
effective action.
\end{definition}

\begin{theorem}{Induced $R^2$ Term from UFE}{R2derive}
Under assumptions K1--K3, integrating out the torsion sector in a Riemann--Cartan background
generates a local effective action of the form
\begin{equation}
    \Gamma_{\mathrm{eff}} \supset \int d^4x\sqrt{-g}\;\frac{\Mpl^2}{2}
    \left(R + \frac{1}{6M^2}R^2 + \ldots\right),
\end{equation}
with $M$ fixed by the TFPT invariants once the GR renormalization condition is imposed.
\end{theorem}

\textbf{Proof sketch.} Use background-field gauge on a Riemann--Cartan background, cast the
quadratic torsion operator into Laplace type, and extract the $a_2$ heat-kernel coefficient.
The $R^2$ term follows from the local part of $\Gamma_{\mathrm{eff}}$. Assumptions K1--K3 and
the matching step are listed in Appendix~\ref{app:r2}.

\subsection{Coefficient Identification and GR Renormalization}

\begin{theorem}{TFPT $R^2$ Scale}{R2coeff}
Using the GR renormalization condition (A8) to fix the scheme, the coefficient of the
$R^2$ term yields
\begin{equation}
    \frac{M}{\Mpl}=\sqrt{8\pi}\;\cthree^4
    = 1.2565\times 10^{-5}.
    \label{eq:R2scale}
\end{equation}
Here the square root applies only to $8\pi$, consistent with the numerical value above.
\end{theorem}

\begin{tcolorbox}[colback=green!3, colframe=green!50!black, 
    title={\textbf{Derivation Provided (Conditional)}}, fonttitle=\bfseries\small]
Appendix~\ref{app:r2} lists assumptions K1--K4 and sketches the one-loop effective-action
calculation in a Riemann--Cartan background. The full operator/heat-kernel evaluation is
implemented in the \texttt{effective\_action\_r2} verification module (see \texttt{tfpt-suite/}).
\end{tcolorbox}

\subsection{\texorpdfstring{Bounce + $R^2$ Perturbations and Transfer Function}{Bounce + R2 Perturbations and Transfer Function}}
\label{sec:bounce}

\begin{definition}{Bounce-to-Inflation Background}{bouncebg}
Assume a non-singular bounce driven by the torsion-induced $a^{-6}$ term
(Lemma above), followed by a transition into the $R^2$ inflationary phase.
We take the CMB pivot scales to exit the horizon during the $R^2$ phase.
\end{definition}

\begin{definition}{Mukhanov--Sasaki Variables in $f(R)$}{msvars}
For $f(R)$ gravity with $F \equiv df/dR$, scalar and tensor perturbations are
described by the Mukhanov--Sasaki variables with stability conditions
$F>0$ and $f''(R)>0$ throughout the evolution.
\end{definition}

\begin{theorem}{Transfer-Function Dressing of Starobinsky Spectra}{bounce}
If the pivot modes exit during the $R^2$ phase, the standard Starobinsky
predictions
$n_s = 1 - 2/N$ and $r = 12/N^2$ hold up to a transfer function $T(k)$:
\begin{equation}
    \mathcal{P}_\zeta(k) = T^2(k)\,\mathcal{P}_\zeta^{(R^2)}(k),
    \qquad T(k)\to 1 \text{ for } k \gg k_{\text{bounce}}.
\end{equation}
Deviations from $T(k)=1$ are confined to the largest scales set by the bounce.
\end{theorem}

\begin{corollary}{Large-Scale Bounce Signatures}{bouncefeatures}
The bounce can imprint a large-scale suppression or oscillatory feature in
$\mathcal{P}_\zeta(k)$ near $k_{\text{bounce}}$, providing a direct falsification
target. The computation of $T(k)$ is given in Appendix~\ref{app:bounce}.
\end{corollary}

%=============================================================================
\section{Axion Cosmology and Dark Matter}
%=============================================================================

%-----------------------------------------------------------------------------
\subsection{Two-Field Structure: Topological vs.\ QCD Axion}
%-----------------------------------------------------------------------------

The TFPT axion sector is split into two distinct fields to avoid an internal
contradiction (a single $\mu$eV axion cannot remain quasi-static over Hubble times):
\begin{equation}
    \boxed{a_{\text{top}} \;\neq\; a_{\text{QCD}}},
    \qquad
    \mathcal{L} \supset \cthree\, a_{\text{top}} F\tilde{F}
    + \frac{a_{\text{QCD}}}{f_a}\,G\tilde{G}.
\end{equation}

\begin{tcolorbox}[colback=orange!5, colframe=orange!60!black, 
    title={\textbf{Two-Field Axion Structure}}, fonttitle=\bfseries]
\begin{itemize}[leftmargin=1.5em, topsep=2pt, itemsep=2pt]
    \item \textbf{$a_{\text{top}}$ (topological field):} Ultralight, quasi-static background
        responsible for birefringence.
        \begin{itemize}[leftmargin=1em, topsep=1pt]
            \item Excursion: $\Delta a_{\text{top}} = n\,\phiz$ (minimal $n=1$)
            \item Time scale: cosmological (Hubble time; no rapid oscillations)
            \item Observable: CMB polarization rotation $\beta = n\,\phiz/(4\pi) = 0.2424^\circ$ for $n=1$
        \end{itemize}
    \item \textbf{$a_{\text{QCD}}$ (QCD axion):} Oscillating field forming dark matter with
        $f_a$ fixed by the PQ block.
        \begin{itemize}[leftmargin=1em, topsep=1pt]
            \item Initial misalignment: $\theta_i \in [0,\pi]$ from $S^1/\mathbb{Z}_2$ field space (see below)
            \item Time scale: $\tau \sim m_a^{-1} \sim 10^{-11}$ s (for $m_a \sim 64\,\mu$eV)
            \item Observable: axion haloscope signal at $\nu \approx 15.6$ GHz
        \end{itemize}
\end{itemize}

\textbf{Minimal mixing:} We adopt the discrete/vanishing mixing limit as the minimal model,
so no new continuous parameters are introduced.
\end{tcolorbox}
\textbf{Why two fields?} A single $\mu$eV axion oscillates at $\sim 10^{10}$ Hz. Over CMB 
propagation time ($\sim 10^{17}$ s), it completes $\sim 10^{27}$ oscillations, averaging 
the birefringence to zero. A distinct ultralight $a_{\text{top}}$ is therefore required.

%-----------------------------------------------------------------------------
\subsection{Dark Matter Relic Density}
%-----------------------------------------------------------------------------

\begin{theorem}{Axion Dark Matter Parameters}{axionDM}
For the QCD axion field $a_{\text{QCD}}$ from the E$_8$ cascade (PQ block at $n = 10$):
\begin{align}
    f_a &= 8.86 \times 10^{10}~\text{GeV} \quad \text{(decay constant)}\\
    m_a &= \frac{m_\pi f_\pi}{f_a}\sqrt{\frac{m_u m_d}{(m_u + m_d)^2}} \approx 64\,\mu\text{eV}\\
    \nu_{\text{haloscope}} &= \frac{m_a}{2\pi\hbar} \approx 15.56~\text{GHz}
\end{align}
\end{theorem}

\textbf{Relic density accounting:}

Including anharmonicity and string-wall production, a standard estimate is
\begin{equation}
    \Omega_a h^2 \simeq 0.12\left(\frac{f_a}{10^{12}~\text{GeV}}\right)^{1.17}\theta_i^2\,F(\theta_i)\,C_{\mathrm{str}},
\end{equation}
where $C_{\mathrm{str}}$ encodes additional production from string-wall networks (order-unity QCD cosmology input, not a TFPT fit parameter).
For $f_a = 8.86 \times 10^{10}$~GeV this gives:
\begin{itemize}[leftmargin=1.5em, topsep=2pt, itemsep=1pt]
    \item For $C_{\mathrm{str}}=1$ (misalignment only), solving yields $\theta_i \approx 2.43$ rad.
    \item \textbf{Suite default (post-inflation PQ):} random $\theta_i$ with $\theta_{\mathrm{rms}}=\pi/\sqrt{3}$ and a \emph{discrete} string/domain-wall enhancement $C_{\mathrm{str}}=7/3$ (minimal cusp/charge set $\{1,2/3,2/3\}$) gives $\Omega_a h^2 \approx 0.123$ (matches $\Omega_{\mathrm{DM}}h^2\simeq 0.12$ at the $\sim 2\%$ level; physics-mode suite gate PASS).
\end{itemize}

\begin{tcolorbox}[colback=yellow!5, colframe=yellow!60!black, 
    title={\textbf{Resolution: Topologically Preferred $\theta_i$}}, fonttitle=\bfseries\small]
\textbf{Option A (topological sector):} On the non-orientable M\"obius fiber, the pseudoscalar
identification enforces $a(x)\mapsto -a(\tau x)$. For constant initial data this implies
$a \equiv -a \;(\mathrm{mod}\;2\pi)$ and hence $a \in \{0,\pi\}$. Equivalently, the field space
is $S^1/\mathbb{Z}_2$, reducing the fundamental domain to $[0,\pi]$ and making $\theta_i$
naturally of order one.

\textbf{Anharmonic enhancement:} Near $\theta_i \to \pi$,
\begin{equation}
    F(\theta_i) \sim \left[\ln\frac{e}{1 - \theta_i/\pi}\right]^{7/6}.
\end{equation}

\textbf{Scenario branches:}
\begin{itemize}[leftmargin=1.5em, topsep=2pt, itemsep=1pt]
    \item \textbf{Pre-inflation PQ (no strings/walls):} $C_{\mathrm{str}}=1$ and a homogeneous $\theta_i$.
    \item \textbf{Post-inflation PQ (strings/walls):} $N_{\mathrm{DW}}=1$ for stability; string-wall networks give $C_{\mathrm{str}}\sim 3$ to $4$.
\end{itemize}

\textbf{Option B (alternative):} Multi-component dark matter:
\begin{itemize}[leftmargin=1.5em, topsep=2pt, itemsep=1pt]
    \item \textbf{Not required in the shipped suite policy:} with the post-inflation RMS branch and discrete $C_{\mathrm{str}}=7/3$, the QCD axion can already saturate the observed DM abundance (torsion-DM remains an optional extension if a different cosmological branch under-produces axions).
\end{itemize}
\end{tcolorbox}

%-----------------------------------------------------------------------------
\subsection{Dark Energy as a Defect-Suppressed Torsion Condensate (candidate)}
\label{sec:dark-energy}
%-----------------------------------------------------------------------------

\begin{tcolorbox}[colback=gray!5, colframe=gray!60,
title={\textbf{Closure-level candidate: discrete $\rho_\Lambda$ with no continuous knobs}}, fonttitle=\bfseries\small]
We test the discrete defect-suppression candidate
\[
\phi_\ast := n\,e^{-\alpha^{-1}(0)/2},
\qquad
\rho_\Lambda \approx (\Mpl \phi_\ast)^4,
\qquad
\Lambda=\rho_\Lambda/\Mpl^2,
\qquad
K_{\mathrm{rms}}=2\sqrt{\Lambda}.
\]
The two-defect sector uses $\delta_2=\tfrac54\delta_{\mathrm{top}}^2$ and the corresponding $\alpha^{-1}(0)$ (Sec.~\ref{sec:sensitivity}).

\textbf{Finite discrete normalization scan:} we scan
\[
n\in\left\{1,\;1/\sqrt{2},\;1/\sqrt{3},\;1/2,\;1/3^{1/4}\right\}
\]
against the $\Lambda$CDM target ledger (from \texttt{k\_calibration.json} in the suite).
The best branch is $n=1/2$, yielding a log$_{10}$ mismatch in $\rho_\Lambda$ of
\[
\Delta\log_{10}\rho_\Lambda \approx 0.086~\text{dex},
\]
within the declared order-of-magnitude falsifiability window (see \codepath{tfpt-suite/theoryv3/out/dark_energy_exponential_audit}).

\textbf{Why $n=1/2$ is preferred:} the double cover suggests a two-sector split normalization $n=1/2$; this is checked explicitly by \codepath{tfpt-suite/theoryv3/out/dark_energy_norm_half_origin_audit}.

\textbf{Dynamical completion status:} a spectral-flow-quantized gap-equation version (torsion condensate) exists as a physics-suite module (\texttt{torsion\_condensate}). At the current policy settings it is \emph{close but not closed}: no discrete $n$ hits within $z\le 2$, with best $z\simeq 2.14$ (log$_{10}$ mismatch $\simeq 0.215$ dex). The paper therefore separates (i) the robust normalization audit (PASS candidate) from (ii) the dynamical condensate closure (open interface; FAIL gate in physics mode), to avoid optimistic labeling.
\end{tcolorbox}

%=============================================================================
\section{\texorpdfstring{$\beta_{\text{rad}}$-Centric Identity Summary}{beta_rad-Centric Identity Summary}}
%=============================================================================

The parameter $\beta_{\text{rad}} := \phiz/(4\pi) \approx 0.00423$ provides a compact dictionary for expressing multiple quantities. 
Most equalities in this section are algebraic rewrites of earlier definitions rather than independent new inputs (see Appendix~\ref{app:identities}). 
Inflation observables additionally depend on a dynamical completion (here: the conditional $R^2$ sector in Theorem~\ref{thm:inflation}, assumptions K1--K3).

\begin{table}[H]
\centering
\begin{tabular}{@{}lll@{}}
\toprule
\textbf{Observable} & \textbf{$\beta_{\text{rad}}$ Form} & \textbf{Value}\\
\midrule
Cosmic birefringence & $\beta_{\text{deg}} = \frac{180}{\pi}\,\beta_{\text{rad}}$ & \num[round-mode=places,round-precision=4]{\TFPTbetaDegNum}$^\circ$\\
Neutrino $\theta_{13}$ & $\sin^2\theta_{13} = 4\pi\,\beta_{\text{rad}}\,e^{-\gamma(0)}$ & \num[round-mode=places,round-precision=6]{\TFPTsinSqThetaThirteen}\\
Baryon density (APS seam candidate) & $\Omega_b = (4\pi - 1)\,\beta_{\text{rad}}$ & \num[round-mode=places,round-precision=6]{\TFPTOmegaBPred}\\
\bottomrule
\end{tabular}
\caption{Observables expressed through the single scale $\beta_{\text{rad}} = \phiz/(4\pi)$.}
\label{tab:beta}
\end{table}

\begin{remark}
The $\Omega_b$ identity is no longer a free conjectural coefficient: the APS seam term satisfies $\Delta_\Gamma=2\pi$ (spectral flow equals winding in the minimal nontrivial class).
A natural coefficient candidate is then $K:=2\Delta_\Gamma-1=4\pi-1$, giving $\Omega_b=K\,\beta_{\mathrm{rad}}$.
What remains open is the embedding of the seam operator into the full 4D/6D Dirac operator and a full $\eta$-gluing evaluation on the complete TFPT geometry.
\end{remark}

\TFPTOmegaBTable

%=============================================================================
\section{Summary of Claims and Status}
%=============================================================================

\begingroup
\small
\renewcommand{\arraystretch}{1.15}
\begin{longtable}{@{}p{3cm}p{4cm}p{3cm}p{5.2cm}@{}}
\caption{TFPT status dashboard. ``Exact'': $<0.1\%$; ``Solid'': $<1\%$; ``Minor'': $<3\%$; ``Moderate'': $<5\%$.}\label{tab:status}\\
\toprule
\textbf{Domain} & \textbf{Quantity} & \textbf{Deviation} & \textbf{Status}\\
\midrule
\endfirsthead
\toprule
\textbf{Domain} & \textbf{Quantity} & \textbf{Deviation} & \textbf{Status}\\
\midrule
\endhead
\midrule
\multicolumn{4}{r}{\small Continued on next page.}\\
\midrule
\endfoot
\bottomrule
\endlastfoot
\multicolumn{4}{l}{\textit{Structural (parameter-free)}}\\
\quad CFE & $\alpha^{-1}(0)=\num[round-mode=places,round-precision=12]{\TFPTalphaInvGOverFourNum}$ & $z=\num[round-mode=places,round-precision=3]{\TFPTalphaZGOverFourNum}$ & \textcolor{green!60!black}{\textbf{Derived (double cover + $g=5$ defect partition; no fits)}}\\
\quad UFE & $\beta = \num[round-mode=places,round-precision=4]{\TFPTbetaDegNum}^\circ$ & 0.5--1.7$\sigma$ & \textcolor{green!60!black}{\textbf{Solid}}\\
\quad Cabibbo & $\lambda = \num[round-mode=places,round-precision=4]{0.2244599705}$ & $-0.15\%$ & \textcolor{green!60!black}{\textbf{Solid}}\\
\midrule
\multicolumn{4}{l}{\textit{RG-dressed}}\\
\quad Two-loop & $\alpha_3(1\,\text{PeV})$ & $-1.38\%$ & \textcolor{blue}{\textbf{Good}}\\
\quad Two-loop & $\alpha_3(\mu_{\cthree})$ & $0.00\%$ & \textcolor{green!60!black}{\textbf{Exact}}\\
\quad Unification gate & $\alpha_1,\alpha_2,\alpha_3$ & mismatch $\sim 0.38\%$ & \textcolor{green!60!black}{\textbf{Solid (explicit gate; discrete policy scan)}}\\
\midrule
\multicolumn{4}{l}{\textit{Structural Extensions (Part II)}}\\
\quad PMNS ($\theta_{13}$) & $\sin^2\theta_{13} = \num[round-mode=places,round-precision=5]{0.0231084352}$ & $+5.04\%$ & \textcolor{blue}{\textbf{Testable}}\\
\quad PMNS ($\theta_{12}$) & $\sin^2\theta_{12} = \num[round-mode=places,round-precision=3]{0.318}$ & $+3.56\%$ & \textcolor{blue}{\textbf{Derived}}\\
\quad Inflation ($n_s$) & $n_s \approx 0.9643$ ($N\approx 56$) & within errors & \textcolor{blue}{\textbf{Derived (K1--K3)}}\\
\quad Inflation ($r$) & $r \approx 0.0038$ & future & \textcolor{blue}{\textbf{Derived (K1--K3)}}\\
\quad Bounce transfer function & $T(k)$ large-scale features & future & \textcolor{orange}{\textbf{Conditional (bounce module)}}\\
\quad Axion DM & $m_a \approx 64\,\mu\text{eV}$ & $\theta_i$ model & \textcolor{orange}{\textbf{Structured}}\\
\midrule
\multicolumn{4}{l}{\textit{Suite-closure (physics-mode; assumption-explicit proxy modules)}}\\
\quad Global consistency (suite, physics) & $\chi^2$ scorecard & $p\simeq 0.435$ ($\chi^2\simeq 15.23$, 15 dof) & \textcolor{green!60!black}{\textbf{PASS (dashboard; no full likelihood)}}\\
\quad Flavor (suite, physics) & CKM/PMNS $\chi^2$ & $p_{\mathrm{CKM}}\simeq 0.242$, $p_{\mathrm{PMNS}}\simeq 0.525$ & \textcolor{green!60!black}{\textbf{PASS (imported $\chi^2$ terms)}}\\
\quad Axion DM (relic) & $\Omega_a h^2 \approx 0.123$ & $+2.2\%$ & \textcolor{blue}{\textbf{Derived (post-inflation; $C_{\mathrm{str}}=7/3$)}}\\
\quad Dark energy (norm) & $\rho_\Lambda$ via $e^{-\alpha^{-1}/2}$ & 0.086 dex & \textcolor{green!60!black}{\textbf{Candidate PASS (theoryv3 audit; $n=1/2$)}}\\
\quad Dark energy (condensate) & $\rho_\Lambda$ torsion gap equation & $z\simeq 2.14$ (0.215 dex) & \textcolor{orange}{\textbf{Close, but not closed (FAIL in physics suite)}}\\
\quad BBN & $Y_p, D/H, N_\mathrm{eff}$ & within $2\sigma$ & \textcolor{blue}{\textbf{Consistent (suite)}}\\
\quad Baryogenesis & $\eta_b \approx 5.8\times10^{-10}$ & within 0.5 dex & \textcolor{blue}{\textbf{Proxy (suite)}}\\
\quad Arrow of time & $\log_{10}(S_H)\approx 14.1$ & $>0$ & \textcolor{blue}{\textbf{Proxy (suite)}}\\
\quad GW background & $\Omega_{\rm gw}\approx 3\times10^{-17}$ & $\ll$ bounds & \textcolor{blue}{\textbf{Consistent (suite)}}\\
\quad Precision QED audit & TFPT-scale NP & $\ll$ anomaly scales & \textcolor{blue}{\textbf{Consistent (suite)}}\\
\midrule
\multicolumn{4}{l}{\textit{Requires Further Development}}\\
\quad CKM full matrix & Yukawa textures & RG-dressed baseline + $\chi^2$ diagnostic & \textcolor{blue}{\textbf{Implemented (suite; p-value gate PASS)}}\\
\quad EW scale (constant factory) & $v$ (EW vev) & $320~\mathrm{GeV}$ vs $246~\mathrm{GeV}$ & \textcolor{orange}{\textbf{Pending (candidate; theoryv3 constant factory)}}\\
\quad Lepton ladder (constant factory) & $m_e, m_\mu, m_\tau$ & depends on $v$ candidate & \textcolor{orange}{\textbf{Pending (candidate chain)}}\\
\quad Hadronic scales & $m_p$ placeholder; $m_\pi$ pending & --- & \textcolor{orange}{\textbf{Open (not derived)}}\\
\quad Derived $H_0$ (diagnostic) & from $\Omega_b$ and $\eta_b$ proxy & $z_{\mathrm{Planck}}\simeq -2.99$ & \textcolor{blue}{\textbf{Diagnostic (theoryv3 WARN)}}\\
\quad $\Omega_b$ identity & $(4\pi-1)\beta_{\text{rad}}$ & $z=\num[round-mode=places,round-precision=3]{\TFPTOmegaBZ}$ & \textcolor{blue}{\textbf{Proxy (APS seam candidate; embedding open)}}\\
\end{longtable}
\endgroup

%=============================================================================
\section{What TFPT Does \emph{Not} Assume}
%=============================================================================

The following are \emph{not} required:
\begin{itemize}[leftmargin=2em]
    \item[\ding{55}] Slow-roll inflation (specific model)
    \item[\ding{55}] String landscape or anthropic selection
    \item[\ding{55}] Extra dimensions as free parameters
    \item[\ding{55}] Dark matter entirely from axions
    \item[\ding{55}] New particles beyond minimal axion sector
    \item[\ding{55}] Any fitted or adjusted parameters
\end{itemize}

%=============================================================================
\section{Conclusions}
%=============================================================================

TFPT demonstrates that the fine-structure constant $\alpha$ can be derived 
from first principles using eight minimal axioms rooted in topology, geometry, 
and quantum consistency. The key results are:

\begin{enumerate}
    \item The topological coupling $\cthree = 1/(8\pi)$ from 11D Chern--Simons 
        quantization and the geometric scale $\phiz = 1/(6\pi) + 3/(256\pi^4)$ 
        from M\"obius fiber reduction are the only fundamental inputs.
    
    \item The cubic fixed-point equation (CFE) has a unique positive root. With the derived two-defect term $\delta_2=\frac54\delta_{\mathrm{top}}^2$ (holonomy $g=5$), this yields
        $\alpha^{-1}(0)=\num[round-mode=places,round-precision=12]{\TFPTalphaInvGOverFourNum}$ (one-defect truncation: $\num[round-mode=places,round-precision=10]{\TFPTalphaInvSelfNum}$).
    
    \item The same invariants determine the axion--photon coupling 
        $\gagg = -4\cthree$ and predict cosmic birefringence 
        $\beta = \num[round-mode=places,round-precision=4]{\TFPTbetaDegNum}^\circ$, consistent with observations.
    
    \item An E$_8$ cascade organizes all physical scales from Planck to 
        electroweak, with deviations typically below 5\%.
\end{enumerate}

The theory is falsifiable: any significant revision of $\alpha$, $\beta$, 
or the two-loop RG fingerprints would invalidate the framework.

%=============================================================================
\appendix
\section{Numerical Constants (High Precision)}
%=============================================================================

\begin{table}[H]
\centering
\small
\begin{tabular}{@{}ll@{}}
\toprule
\textbf{Quantity} & \textbf{Value}\\
\midrule
$\cthree = 1/(8\pi)$ & \num[round-mode=places,round-precision=12]{\TFPTcThreeNum}\\
$\phitree = 1/(6\pi)$ & \num[round-mode=places,round-precision=12]{\TFPTvarphiTreeNum}\\
$\deltatop = 3/(256\pi^4)$ & \num[round-mode=places,round-precision=12]{\TFPTdeltaTopNum}\\
$\phiz = \phitree + \deltatop$ & \num[round-mode=places,round-precision=12]{\TFPTvarphiZeroNum}\\
$\phiz(\alpha)$ (self-consistent) & \num[round-mode=places,round-precision=12]{0.053170209119023745}\\
$\beta_{\text{rad}} = \phiz/(4\pi)$ & \num[round-mode=places,round-precision=12]{\TFPTbetaRadNum}\\
$\beta_{\text{deg}} = (180/\pi)\,\beta_{\text{rad}}$ & \num[round-mode=places,round-precision=6]{\TFPTbetaDegNum}$^\circ$\\
$g_{a\gamma\gamma}$ & \num[round-mode=places,round-precision=12]{\TFPTgAggNum}\\
$\gamma(0)=g/(g+1)$ (E$_8$) & $\tfrac56$\\
$\lambda$ (E$_8$ exponent) & \num[round-mode=places,round-precision=12]{0.587233190787837868}\\
$M/\Mpl$ ($R^2$ scale) & \num[round-mode=places,round-precision=12]{\TFPTMOverMplNum}\\
$A = 2\cthree^3$ & \num[round-mode=figures,round-precision=8]{1.2598255637968552e-4}\\
$\mathcal{K} = (\bone/2\pi)\ln(1/\phiz)$ & \num[round-mode=places,round-precision=10]{1.91468479457699546}\\
$\alpha^{-1}_{\text{baseline}}$ & \num[round-mode=places,round-precision=9]{137.036501464885822}\\
$\alpha^{-1}_{\text{TFPT}}$ (one-defect) & \num[round-mode=places,round-precision=12]{\TFPTalphaInvSelfNum}\\
$\alpha^{-1}_{\text{TFPT}}$ (two-defect, $g=5$) & \num[round-mode=places,round-precision=12]{\TFPTalphaInvGOverFourNum}\\
$\alpha^{-1}_{\text{TFPT}}$ (two-defect, suite diagnostic) & \num[round-mode=places,round-precision=12]{\TFPTalphaInvTwoDefNum}\\
$\alpha^{-1}_{\text{CODATA 2022}}$ & \num[round-mode=places,round-precision=9]{\TFPTalphaInvCODATANum}\\
Accuracy (ppm; one-defect) & \num[round-mode=places,round-precision=6]{-0.03704}\\
Accuracy (ppm; two-defect, $g=5$) & \num[round-mode=places,round-precision=6]{\TFPTalphaPpmGOverFourNum}\\
Accuracy (ppm; two-defect, suite diagnostic) & \num[round-mode=places,round-precision=6]{\TFPTalphaPpmTwoDefNum}\\
$\xi = \cthree/\phiz$ & \num[round-mode=places,round-precision=9]{0.748303083562547799}\\
\bottomrule
\end{tabular}
\caption{High-precision numerical values of TFPT constants.}
\label{tab:constants}
\end{table}

%=============================================================================
\section{Derivation of the CFE from the Effective Potential}
%=============================================================================

In background-field gauge, the one-loop effective potential for the electromagnetic 
coupling modulus takes the form. Here $U(\alpha)$ is the induced potential for $\sigma$
after rewriting $\alpha=\alpha(\sigma)$:
\begin{equation}
    U(\alpha) = \frac{1}{4}(F_{\mu\nu}F^{\mu\nu})_{\text{eff}} 
        + \text{quantum corrections}
\end{equation}

The full structure, including the Chern--Simons contribution from 11D reduction, is:
\begin{equation}
    U(\alpha) = \frac{A}{4}\alpha^4 - \frac{2}{3}A\cthree^3\alpha^3 
        - 8A\bone\cthree^6 L\,\alpha + \mathcal{O}(\alpha^5)
    \label{eq:potential}
\end{equation}
where $L = \ln(1/\phiz)$ encodes the geometric scale.

The modulus-closure condition $\delta\Gamma_{\mathrm{eff}}/\delta\sigma=0$ reduces to
$\partial U/\partial\alpha = 0$ (monotonic $\alpha(\sigma)$), giving:
\begin{equation}
    \frac{\partial U}{\partial\alpha} = A\alpha^3 - 2A\cthree^3\alpha^2 
        - 8A\bone\cthree^6 L = 0
    \label{eq:stationarity}
\end{equation}

Dividing by $A$ (which is nonzero):
\begin{equation}
    \alpha^3 - 2\cthree^3\alpha^2 - 8\bone\cthree^6 L = 0
    \label{eq:CFE-appendix}
\end{equation}

This is the CFE. The combinatorial factors arise from specific one-loop diagrams:

\textbf{(i) Factor 2 in $2\cthree^3\alpha^2$:}

The cubic term in the effective potential comes from the vacuum polarization 
(``bubble'') diagram with axion insertion:
\begin{center}
\begin{tikzpicture}[scale=0.8]
    \draw[thick, photon] (0,0) -- (1.5,0);
    \draw[thick] (1.5,0) arc (180:0:0.75);
    \draw[thick] (1.5,0) arc (-180:0:0.75);
    \draw[thick, photon] (3,0) -- (4.5,0);
    \draw[thick, dashed] (2.25,0.75) -- (2.25,1.5);
    \node at (2.25,1.8) {\small $a$};
    \node at (0,-0.3) {\small $\gamma$};
    \node at (4.5,-0.3) {\small $\gamma$};
\end{tikzpicture}
\end{center}
The $\varepsilon^{\mu\nu\rho\sigma}\varepsilon_{\alpha\beta\gamma\delta}$ contraction from two 
$F\tilde{F}$ vertices produces a factor 2 from the antisymmetric tensor algebra:
\begin{equation}
    \varepsilon^{\mu\nu\rho\sigma}\varepsilon_{\mu\nu\alpha\beta} = -2(\delta^\rho_\alpha\delta^\sigma_\beta - \delta^\rho_\beta\delta^\sigma_\alpha)
\end{equation}

\textbf{(ii) Factor 8 and $\cthree^6$ in $8\bone\cthree^6 L$:}

The log-dependent term has two components:

\textbf{Origin of $\cthree^6$:} The factor $\cthree^6$ arises from the product structure 
of the effective action. With $A = 2\cthree^3$ as the overall amplitude:
\begin{equation}
    8\bone\cthree^6 L = 8\bone \cdot \cthree^3 \cdot \cthree^3 \cdot L
    = 4 \cdot A \cdot \cthree^3 \cdot \bone L
\end{equation}
The factorization proceeds as follows:
\begin{itemize}[leftmargin=1.5em, topsep=2pt, itemsep=1pt]
    \item One factor of $\cthree^3$ comes from three $\cthree\, aF\tilde{F}$ vertices 
        in the anomaly chain (triangle diagram).
    \item The second factor of $\cthree^3$ comes from the amplitude normalization $A/2 = \cthree^3$.
    \item The coefficient 8 decomposes as $8 = 4 \times 2$, where 4 is from $|D_2|^2$ 
        (two $\mathbb{Z}_2$ reflections) and 2 is from the double cover sheets.
\end{itemize}
\textbf{Consistency check:} $8\cthree^6 = 8 \times (8\pi)^{-6} = 8/(8\pi)^6$, which matches 
the numerical coefficient in the CFE.

\textbf{Origin of factor 8:} The dihedral symmetry $D_4$ of the 4-point amplitude 
contributes $|D_4| = 8$. This arises from the 4 rotations and 4 reflections that 
leave the fermion loop topology invariant:
\begin{center}
\begin{tikzpicture}[scale=0.8]
    \draw[thick] (0,0) rectangle (2,2);
    \draw[thick, photon] (-0.8,0) -- (0,0);
    \draw[thick, photon] (-0.8,2) -- (0,2);
    \draw[thick, photon] (2,0) -- (2.8,0);
    \draw[thick, photon] (2,2) -- (2.8,2);
    \node at (1,-0.5) {\small fermion loop};
    \node at (1,1) {\small $D_4$};
\end{tikzpicture}
\end{center}

\textbf{The $\bone$ factor:} $\bone = 41/10$ counts the hypercharge-weighted fermion 
contributions summed over the SM spectrum. It enters as $\sum_f Y_f^2$, where the 
sum runs over all SM fermions with their multiplicities.

\textbf{(iii) Factor 2/3 in potential:}

The coefficient $2/3$ in the cubic term of \eqref{eq:potential} ensures that 
differentiation produces the factor 2 in \eqref{eq:stationarity}:
\begin{equation}
    \frac{d}{d\alpha}\left(-\frac{2}{3}A\cthree^3\alpha^3\right) = -2A\cthree^3\alpha^2
\end{equation}
This is not a choice but a consequence of the integral representation of the 
effective potential: the cubic term arises from a triangle subgraph.

\textbf{Why no $\alpha^2$ term?}

The absence of an $\alpha^2$ term in the CFE is not accidental. In background-field gauge, 
quadratic terms are absorbed into the field-strength renormalization. The Ward identity 
ensures that physical observables depend only on odd powers of $\alpha$ (gauge invariance 
requires $F^2 \sim \alpha^2$ to enter as a full factor, not as a correction to $\alpha$).

\begin{tcolorbox}[colback=purple!3, colframe=purple!50!black, 
    title={\textbf{Why Integer Coefficients? (No $\zeta(3)$, $\ln 2$, etc.)}}, fonttitle=\bfseries\small]
\textbf{Potential objection:} In typical QFT calculations (e.g., anomalous magnetic moment $g-2$), 
coefficients involve transcendental numbers like $\zeta(3)$, $\pi^2/12$, $\ln 2$. Why are the 
CFE coefficients (2, 8, 48) simple integers?

\textbf{Resolution:} The CFE is not a perturbative loop expansion but a \emph{topological constraint}.

\begin{enumerate}[leftmargin=1.5em, topsep=2pt, itemsep=1pt]
    \item \textbf{Origin of transcendentals:} In $g-2$, transcendentals arise from 
        \emph{integrating over continuous momenta} in Feynman integrals. The integrals 
        produce polylogarithms, which evaluate to $\zeta$-values.
    
    \item \textbf{Topological discreteness:} In TFPT, the relevant integrals are over 
        \emph{discrete topological data}: Euler characteristics, winding numbers, and 
        symmetry group orders. These are integers by definition.
    
    \item \textbf{Combinatorial origin:}
        \begin{itemize}[leftmargin=1em, topsep=1pt, itemsep=1pt]
            \item Factor 2: From $\varepsilon^{\mu\nu\rho\sigma}$ contraction (antisymmetry)
            \item Factor 8: From $|D_4| = 8$ (dihedral group of the box)
            \item Factor 48: \emph{Mnemonic} $2 \times 24$ (double cover $\times$ permutations); the derivation is the spin-lifted deficit with four cut patches (Appendix~\ref{app:48deriv})
        \end{itemize}
    
    \item \textbf{Where $\pi$ appears:} The factors of $\pi$ in $\cthree = 1/(8\pi)$ and 
        $\phitree = 1/(6\pi)$ come from the \emph{normalization} of circles ($2\pi$ per cycle), 
        not from loop integrals. They are geometric, not perturbative.
\end{enumerate}

\textbf{Diagnostic:} If transcendentals appeared in the CFE, it would indicate non-topological 
dynamics. Their absence is evidence that the fixed-point structure is genuinely topological.
\end{tcolorbox}

\begin{remark}
The consistency between \eqref{eq:potential} and \eqref{eq:CFE-appendix} can be 
verified directly: $\frac{d}{d\alpha}\bigl(-\frac{2}{3}A\cthree^3\alpha^3\bigr) 
= -2A\cthree^3\alpha^2$, matching the required coefficient.
\end{remark}

%=============================================================================
\section{Verification Suite Summary}
%=============================================================================

The TFPT verification suite (\texttt{tfpt-suite/}) turns paper claims into deterministic, assumption-explicit computations. All counts in this section are auto-generated from the suite artifacts (no hardcoded ``39 modules''):
\[
\text{unique modules (registry)}=\TFPTSuiteUniqueModulesTotal,\qquad
\text{module artifacts (engineering+physics)}=\TFPTSuiteModulesTotal,\qquad
\text{checks total}=\TFPTSuiteChecksTotal.
\]
It ships:
\begin{itemize}[leftmargin=1.5em, topsep=2pt, itemsep=1pt]
    \item a \textbf{conventional suite} (registry-driven; core invariants, $\alpha$ closure + policy bridge, RG/matching, flavor pipelines, cosmology/bounce wiring, torsion/birefringence, dashboards),
    \item an \textbf{unconventional suite} of search/audit/design tools (for closing explicit ToE gaps without hiding assumptions),
    \item a \textbf{theoryv3} branch of discrete audits that formalizes the ``kernel + crosslinks'' thesis (holonomy $g=5$, defect partition, constant factory).
\end{itemize}
Each module emits \texttt{results.json} (machine-readable), \texttt{report.txt} (human-readable), and optional plots; PDF reports summarize runs.

\subsection{Known Gaps and Status (Auto-Generated)}
\label{sec:known-gaps}

\begin{tcolorbox}[colback=yellow!3, colframe=yellow!60!black,
title={\textbf{Current WARN/FAIL inventory (from suite manifest)}}, fonttitle=\bfseries\small]
\TFPTSuiteProblemModules
\end{tcolorbox}

% Compact table of problem modules (WARN/FAIL/plot issues)
% Inlined (standalone build): formerly % AUTO-GENERATED by tfpt_suite.suite_manifest.py — do not edit by hand.
\begin{table}[H]
\centering
\small
\begin{tabular}{@{}llrrrrrr@{}}
\toprule
Module & Mode & Checks & PASS & WARN & FAIL & Plots(exp) & Plots(ok)\\
\midrule
\texttt{alpha\_on\_shell\_bridge} & engineering & 7 & 6 & 1 & 0 & 0 & 0\\
\texttt{matching\_finite\_pieces} & engineering & 5 & 4 & 1 & 0 & 0 & 0\\
\texttt{torsion\_condensate} & engineering & 8 & 5 & 3 & 0 & 0 & 0\\
\texttt{alpha\_on\_shell\_bridge} & physics & 7 & 6 & 1 & 0 & 0 & 0\\
\texttt{matching\_finite\_pieces} & physics & 5 & 4 & 1 & 0 & 0 & 0\\
\texttt{torsion\_condensate} & physics & 8 & 5 & 2 & 1 & 0 & 0\\
\bottomrule
\end{tabular}
\caption{Suite manifest: compact table of WARN/FAIL modules and plot issues (auto-generated).}
\end{table}

% -----------------------------------------------------------------------------
% AUTO-GENERATED by tfpt_suite.suite_manifest.py — do not edit by hand.
\begin{table}[H]
\centering
\small
\begin{tabular}{@{}llrrrrrr@{}}
\toprule
Module & Mode & Checks & PASS & WARN & FAIL & Plots(exp) & Plots(ok)\\
\midrule
\texttt{alpha\_on\_shell\_bridge} & engineering & 7 & 6 & 1 & 0 & 0 & 0\\
\texttt{matching\_finite\_pieces} & engineering & 5 & 4 & 1 & 0 & 0 & 0\\
\texttt{torsion\_condensate} & engineering & 8 & 5 & 3 & 0 & 0 & 0\\
\texttt{alpha\_on\_shell\_bridge} & physics & 7 & 6 & 1 & 0 & 0 & 0\\
\texttt{matching\_finite\_pieces} & physics & 5 & 4 & 1 & 0 & 0 & 0\\
\texttt{torsion\_condensate} & physics & 8 & 5 & 2 & 1 & 0 & 0\\
\bottomrule
\end{tabular}
\caption{Suite manifest: compact table of WARN/FAIL modules and plot issues (auto-generated).}
\end{table}

\subsection{theoryv3 Evidence Layer (Compressed, Auto-Generated)}
\label{sec:theoryv3-evidence}

The \texttt{theoryv3} analysis branch is a compact ``evidence layer'' for the discrete kernel and crosslink logic. Current inventory (auto-generated from \codepath{tfpt-suite/theoryv3/out/}):
\[
\text{modules}=\TFPTTheoryVThreeModulesTotal,\qquad
\text{checks}=\TFPTTheoryVThreeChecksTotal\ (\mathrm{PASS/WARN/FAIL}=\TFPTTheoryVThreeChecksPassTotal/\TFPTTheoryVThreeChecksWarnTotal/\TFPTTheoryVThreeChecksFailTotal),\qquad
\text{plots expected/present}=\TFPTTheoryVThreePlotsExpectedTotal/\TFPTTheoryVThreePlotsPresentTotal.
\]

% Inlined (standalone build): formerly % AUTO-GENERATED by tfpt-suite/theoryv3/theoryv3_manifest.py — do not edit by hand.
\begin{table}[H]
\centering
\small
\begin{tabular}{@{}lllcc@{}}
\toprule
Module & Signal & Result & Checks & Plots OK\\
\midrule
\texttt{alpha\_backreaction\_sensitivity\_audit} & α backreaction sensitivity (k-grid) & PASS & 2/2 & yes\\
\texttt{axion\_dm\_audit} & Axion DM audit & PASS & 2/2 & yes\\
\texttt{baryon\_consistency\_audit} & Baryon consistency / H0 audit & WARN & 3/4 & yes\\
\texttt{constant\_factory\_audit} & Constant factory audit (derived outputs) & WARN & 2/3 & yes\\
\texttt{dark\_energy\_exponential\_audit} & Dark energy exponential candidate & PASS & 2/2 & yes\\
\texttt{dark\_energy\_norm\_half\_origin\_audit} & Dark energy normalization n=1/2 scan & PASS & 2/2 & yes\\
\texttt{defect\_partition\_g5\_audit} & Two-defect partition (δ2=(5/4)δ\_top\textasciicircum{}2) & PASS & 4/4 & yes\\
\texttt{flavor\_pattern\_audit} & Flavor pattern audit (texture invariants) & PASS & 3/3 & yes\\
\texttt{g5\_crosslink\_audit} & Crosslink consistency (g=5 anchors) & PASS & 3/3 & yes\\
\texttt{g5\_origin\_audit} & Holonomy degeneracy g=5 & PASS & 2/2 & yes\\
\texttt{pmns\_tm1\_audit} & PMNS TM1 audit & PASS & 2/2 & yes\\
\texttt{seed\_invariants\_audit} & Seed invariants (c3, φ0, δ\_top, β) & PASS & 7/7 & yes\\
\texttt{yukawa\_exponent\_index\_audit} & Yukawa exponent-index mapping & WARN & 3/7 & yes\\
\texttt{yukawa\_index\_mapping\_audit} & Yukawa index mapping & WARN & 3/7 & yes\\
\bottomrule
\end{tabular}
\caption{theoryv3 compressed evidence layer (auto-generated from \,\texttt{tfpt-suite/theoryv3/out/}).}
\end{table}

% -----------------------------------------------------------------------------
% AUTO-GENERATED by tfpt-suite/theoryv3/theoryv3_manifest.py — do not edit by hand.
\begin{table}[H]
\centering
\small
\begin{tabular}{@{}lllcc@{}}
\toprule
Module & Signal & Result & Checks & Plots OK\\
\midrule
\texttt{alpha\_backreaction\_sensitivity\_audit} & $\alpha$ backreaction sensitivity (k-grid) & PASS & 2/2 & yes\\
\texttt{axion\_dm\_audit} & Axion DM audit & PASS & 2/2 & yes\\
\texttt{baryon\_consistency\_audit} & Baryon consistency / H0 audit & WARN & 3/4 & yes\\
\texttt{constant\_factory\_audit} & Constant factory audit (derived outputs) & WARN & 2/3 & yes\\
\texttt{dark\_energy\_exponential\_audit} & Dark energy exponential candidate & PASS & 2/2 & yes\\
\texttt{dark\_energy\_norm\_half\_origin\_audit} & Dark energy normalization n=1/2 scan & PASS & 2/2 & yes\\
\texttt{defect\_partition\_g5\_audit} & Two-defect partition ($\delta_2=(5/4)\,\delta_{\rm top}^2$) & PASS & 4/4 & yes\\
\texttt{flavor\_pattern\_audit} & Flavor pattern audit (texture invariants) & PASS & 3/3 & yes\\
\texttt{g5\_crosslink\_audit} & Crosslink consistency (g=5 anchors) & PASS & 3/3 & yes\\
\texttt{g5\_origin\_audit} & Holonomy degeneracy g=5 & PASS & 2/2 & yes\\
\texttt{pmns\_tm1\_audit} & PMNS TM1 audit & PASS & 2/2 & yes\\
\texttt{seed\_invariants\_audit} & Seed invariants (c3, $\varphi_0$, $\delta_{\rm top}$, $\beta$) & PASS & 7/7 & yes\\
\texttt{yukawa\_exponent\_index\_audit} & Yukawa exponent-index mapping & WARN & 3/7 & yes\\
\texttt{yukawa\_index\_mapping\_audit} & Yukawa index mapping & WARN & 3/7 & yes\\
\bottomrule
\end{tabular}
\caption{theoryv3 compressed evidence layer (auto-generated from \,\codepath{tfpt-suite/theoryv3/out/}).}
\end{table}

\begin{tcolorbox}[colback=gray!5, colframe=gray!60,
title={\textbf{Two ``lock-in'' results (negative controls; theoryv3)}}, fonttitle=\bfseries\small]
\textbf{$k=2$ is selected by a sensitivity minimum:} the \texttt{alpha\_backreaction\_sensitivity\_audit} shows the minimum $|\mathrm{ppm}|$ in the $k$ grid occurs at $k=2$, with $|\mathrm{ppm}|\simeq 0.037$.

\textbf{$g=5$ is selected by a defect-partition negative control:} the \texttt{defect\_partition\_g5\_audit} shows $g=5$ minimizes $|z|$ sharply versus $g\in\{4,6\}$ (gap $\sim 45$ in $|z|$).
\end{tcolorbox}

\begin{figure}[H]
\centering
\begin{tikzpicture}
\begin{groupplot}[
    group style={group size=2 by 1, horizontal sep=1.2cm},
    width=0.48\textwidth,
    height=0.42\textwidth,
    grid=both,
    legend style={font=\scriptsize},
]
\nextgroupplot[
    title={$k$-sensitivity (abs ppm)},
    xlabel={$k$},
    ylabel={$|{\rm ppm}|$},
    xmin=-0.1, xmax=3.1,
    ymin=0, ymax=4.2,
    xtick={0,1,1.5,2,2.5,3},
]
\addplot+[mark=*, mark size=2.2pt] coordinates {
    (0.0, 3.665371791638453)
    (1.0, 1.8074246412353618)
    (1.5, 0.883514212976573)
    (2.0, 0.03703710120260405)
    (2.5, 0.9542414840493093)
    (3.0, 1.8681110743677323)
};
\addplot[gray, dashed] coordinates {(2.0,0.0) (2.0,4.2)};

\nextgroupplot[
    title={$g$ negative control (z-score)},
    xlabel={$g$},
    ylabel={$z$},
    xmin=3.7, xmax=6.3,
    ymin=-60, ymax=60,
    xtick={4,5,6},
]
\addplot+[mark=*, mark size=2.2pt] coordinates {
    (4, -46.8455434804275)
    (5, 1.8646873709648437)
    (6, 50.5749164827739)
};
\addplot[gray, dashed] coordinates {(5,-60) (5,60)};
\end{groupplot}
\end{tikzpicture}
\caption{theoryv3 lock-in diagnostics: $k$-sensitivity of $\alpha^{-1}(0)$ (left) and the $g\in\{4,5,6\}$ negative control selecting $g=5$ (right).}
\end{figure}

\begin{tcolorbox}[colback=gray!5, colframe=gray!60,
title={\textbf{Global scorecard (dashboard $\chi^2$; no covariance)}}, fonttitle=\bfseries\small]
The \texttt{global\_consistency\_test} module evaluates a reference-table $\chi^2$ over a small set of headline observables (engineering mode). Current totals:
\[
\chi^2_{\mathrm{core}}=3.083\ (6~\mathrm{dof}),\qquad p\simeq 0.798,
\]
and (excluding $\alpha$ metrology dominance)
\[
\chi^2_{\mathrm{core}\setminus\alpha}=1.354\ (5~\mathrm{dof}),\qquad p\simeq 0.929.
\]
This is a dashboard (independent Gaussians; simplified $r$ bound proxy), not a publication-grade global likelihood. The suite reports per-term contributions and an ``excluding $\alpha$'' total explicitly to avoid misleading optics when a single ultra-precise reference dominates.
\end{tcolorbox}

\begin{figure}[H]
\centering
\begin{tikzpicture}
\begin{groupplot}[
    group style={group size=2 by 1, horizontal sep=1.2cm},
    ybar,
    bar width=10pt,
    width=0.48\textwidth,
    height=0.42\textwidth,
    xmin=0.5, xmax=4.5,
    xtick={1,2,3,4},
    xticklabels={$\beta_{\rm deg}$, $\lambda$, $n_s$, $A_s$},
    x tick label style={rotate=0, font=\scriptsize},
    grid=both,
]
\nextgroupplot[
    title={$\chi^2$ contributions (excluding $\alpha$)},
    ylabel={$\chi^2$},
    ymin=0, ymax=0.8,
]
\addplot coordinates {
    (1,0.5903174784328564)
    (2,0.6542656358088781)
    (3,0.021391549817205795)
    (4,0.08789723472938661)
};

\nextgroupplot[
    title={$z$-scores (excluding $\alpha$)},
    ylabel={$z$},
    ymin=-1.0, ymax=0.15,
]
\addplot coordinates {
    (1,-0.7683212078505034)
    (2,-0.8088668838621581)
    (3,-0.14625850340136054)
    (4,-0.296474678057649)
};
\addplot[black!50, dashed] coordinates {(0.5,0) (4.5,0)};
\end{groupplot}
\end{tikzpicture}
\caption{Engineering-mode dashboard plots from \texttt{global\_consistency\_test}: $\chi^2$ contributions (left) and $z$-scores (right), both excluding $\alpha$.}
\end{figure}

%=============================================================================
\section{\texorpdfstring{E$_8$ Chain Certificate}{E8 Chain Certificate}}
\label{app:E8cert}
%=============================================================================

This appendix provides the complete verification data for the E$_8$ chain uniqueness 
theorem (Theorem~\ref{thm:chainunique}).

\subsection{Orbit Data Structure}

Each nilpotent orbit $\mathcal{O}$ of $\mathfrak{e}_8$ is characterized by:
\begin{itemize}[leftmargin=1.5em, topsep=2pt, itemsep=1pt]
    \item \textbf{Bala--Carter label:} Standard notation (e.g., $A_1$, $2A_1$, $A_2$, \ldots)
    \item \textbf{Dimension:} $\dim(\mathcal{O})$
    \item \textbf{Centralizer dimension:} $D(\mathcal{O}) = 248 - \dim(\mathcal{O})$
    \item \textbf{Height:} $h(\mathcal{O})$ (sum of weighted Dynkin diagram coefficients)
\end{itemize}

\subsection{The Unique Chain (Excerpt)}

The complete chain from $D = 58$ to $D = 8$:

\begin{center}
\small
\begin{tabular}{@{}clccc@{}}
\toprule
$n$ & \textbf{Orbit Label} & $D_n$ & $\dim(\mathcal{O})$ & $h$\\
\midrule
1 & $A_1$ & 58 & 190 & 2\\
2 & $2A_1$ & 56 & 192 & 4\\
3 & $3A_1$ & 54 & 194 & 6\\
$\vdots$ & $\vdots$ & $\vdots$ & $\vdots$ & $\vdots$\\
12 & (EW anchor) & 36 & 212 & 24\\
15 & (Hadronic) & 30 & 218 & 30\\
$\vdots$ & $\vdots$ & $\vdots$ & $\vdots$ & $\vdots$\\
26 & $E_8(a_1)$ & 8 & 240 & 58\\
\bottomrule
\end{tabular}
\end{center}

\textbf{Full data:} Available in machine-readable format at the replication repository 
(see Section~\ref{app:repro}).

\subsection{Verification Algorithm}

The uniqueness is verified by:
\begin{enumerate}[leftmargin=1.5em, topsep=2pt, itemsep=1pt]
    \item Constructing the complete Hasse diagram of $\mathcal{N}(\mathfrak{e}_8)$ 
        from standard tables.
    \item Computing all cover relations with $\Delta D = 2$.
    \item Running dynamic programming to find \emph{all} paths from $D = 58$ to $D = 8$.
    \item Applying the cost function $\mathcal{S} = \sum_n (\Delta^2 \ln D_n)^2$ 
        (second-difference smoothness).
    \item Verifying that the minimum-cost path is unique (cost gap to second-best $= 0$ 
        indicates tie-breaker was not needed for the optimal path).
\end{enumerate}

%=============================================================================
\section{Boundary Trace Derivation}
\label{app:boundary}
%=============================================================================

This appendix derives the block index formula $k_B = \frac{3}{2} I_1(B)$ from 
the APS index theorem.

\subsection{Setup}

Consider a 6D manifold $M^6$ with boundary $\partial M^6 = \Sigma^5$. For a 
Weyl fermion coupled to a $U(1)$ gauge field $A$, the APS index theorem gives:
\begin{equation}
    \text{ind}(\not{\!D}) = \int_{M^6} \text{ch}(F) \wedge \hat{A}(M^6) 
    - \frac{1}{2}\eta(\not{\!D}_\Sigma)
\end{equation}
where $\eta$ is the $\eta$-invariant of the boundary Dirac operator.

\subsection{Flat Connection on Boundary Cycles}

On the orientable double cover $\widetilde{M}$, the boundary decomposes as:
\begin{equation}
    \partial\widetilde{M} = C_1 \cup C_2 \cup C_T
\end{equation}
(two boundary cycles plus the twist seam).

For a \emph{flat} $U(1)$ connection on each $C_i$ with holonomy $\theta_i = \oint_{C_i} A$, 
the $\eta$-invariant contribution becomes:
\begin{equation}
    \eta_i \propto \sum_{\text{fermions}} q^2 \cdot \theta_i
\end{equation}
where $q$ is the $U(1)$ charge.

\subsection{\texorpdfstring{Spectral Trace Coefficient and $\bone$}{Spectral Trace Coefficient and b1}}
The regulated determinant of the hypercharge-coupled Dirac operator contains a universal
UV logarithmic term whose coefficient is the quadratic charge trace:
\begin{equation}
    \ln\det(\not{D}_Y) \supset \frac{1}{2\pi}\left(\frac{5}{3}\sum_f Y_f^2\right)
    \ln\frac{\Lambda}{\mu}\;\int F^2.
\end{equation}
Thus $\bone = \frac{5}{3}\sum_f Y_f^2 = 41/10$ appears as a \emph{spectral index coefficient},
independent of IR thresholds, matching its role in the CFE.

\subsection{\texorpdfstring{Derivation of $k_B$}{Derivation of kB}}

\textbf{Step 1:} Each boundary cycle contributes a factor proportional to $I_1 = \sum q^2$.

\textbf{Step 2:} Three cycles contribute additively: factor 3.

\textbf{Step 3:} The double cover counts each physical state once across two sheets. 
The effective weight is halved: factor $1/2$.

\textbf{Result:}
\begin{equation}
    k_B = 3 \times \frac{1}{2} \times I_1(B) = \frac{3}{2} I_1(B)
\end{equation}

\textbf{Consistency check:} For the electroweak block, 
$I_1^{\text{EW}} = \frac{41}{48}$ (SM hypercharge sum), giving 
$k_{\text{EW}} = \frac{3}{2} \times \frac{41}{48} = \frac{41}{32}$, 
as stated in Table~\ref{tab:blocks}.

%=============================================================================
\section{\texorpdfstring{Seam Gluing and the $\eta$-Invariant Normalization}{Seam Gluing and the eta-Invariant Normalization}}
\label{app:seam}
%=============================================================================

This appendix provides the $\eta$-invariant gluing derivation used in Axiom~\ref{ax:GB}.

\subsection{Cut Space and Operators}
Let $\widetilde{M}$ be the orientable double cover of the M\"obius fiber and cut it along
the seam $\Gamma$ to obtain a manifold-with-boundary $\widetilde{M}_\Gamma$ with
boundary components $(C_1,C_2,\Gamma_+,\Gamma_-)$. Let $\not{D}_\Gamma$ be the Dirac operator
on $\widetilde{M}_\Gamma$ with APS boundary conditions on $C_{1,2}$ and a unitary matching
operator $U_\Gamma$ identifying $\Gamma_+$ with $\Gamma_-$.
On the seam we take the tangential 1D Dirac operator
\begin{equation}
    D_\Gamma = i\,\frac{d}{d\theta}, \qquad \theta\in[0,2\pi).
\end{equation}

\subsection{\texorpdfstring{Gluing Formula for the $\eta$-Invariant}{Gluing Formula for the eta-Invariant}}
The APS gluing theorem expresses the $\eta$-invariant of the glued manifold in terms of
the cut manifold plus a discrete spectral-flow correction:
\begin{equation}
    \eta(\not{D}_{\widetilde{M}}) = \eta(\not{D}_\Gamma) + 2\,\mathrm{SF}(U_\Gamma),
\end{equation}
where $\mathrm{SF}(U_\Gamma)\in\mathbb{Z}$ is the spectral flow of the matching operator.
In TFPT bookkeeping this contributes a quantized seam term
\begin{equation}
    \Delta_\Gamma = 2\pi\,\mathrm{SF}(U_\Gamma)
\end{equation}
to the curvature normalization.

\subsection{M\"obius Matching and the Minimal Gluing Class}
For the M\"obius $\mathbb{Z}_2$ identification, the matching satisfies $U_\Gamma^2=\mathbf{1}$.
The minimal nontrivial class is modeled by the unitary phase
\begin{equation}
    U_\Gamma(\theta)=e^{i\theta}.
\end{equation}

\subsection{Spectral Flow Computation}
Interpolate boundary conditions by a phase $\varphi$,
\begin{equation}
    \psi(\theta+2\pi)=e^{i\varphi}\psi(\theta), \qquad \varphi\in[0,2\pi],
\end{equation}
which gives the eigenvalues
\begin{equation}
    \lambda_n(\varphi)=n+\frac{\varphi}{2\pi}, \qquad n\in\mathbb{Z}.
\end{equation}
As $\varphi$ runs from $0$ to $2\pi$, exactly one eigenvalue crosses zero, so
$\mathrm{SF}(U_\Gamma)=1$ and $\Delta_\Gamma=2\pi$.

\begin{remark}
For unitary matching maps $U_\Gamma:S^1\to U(1)$, the spectral flow equals the winding
number of $\det U_\Gamma$, consistent with the $K^1(S^1)\cong\mathbb{Z}$ classification.
\end{remark}

\subsection{\texorpdfstring{Normalization of $\phitree$}{Normalization of phi_tree}}
The two physical boundary cycles contribute $2\pi+2\pi$, and the seam gluing adds
$\Delta_\Gamma=2\pi$, so the total curvature is $R_{\text{total}}=6\pi$. The stationarity
normalization $\varphi\cdot R_{\text{total}}=1$ therefore yields
\begin{equation}
    \phitree = \frac{1}{6\pi}.
\end{equation}

%=============================================================================
\section{\texorpdfstring{Derivation of the Coefficient 48 in $\deltatop$}{Derivation of the Coefficient 48 in delta_top}}
\label{app:48deriv}
%=============================================================================

This appendix explains the integer coefficient in $\deltatop = 48\cthree^4$.
The key point is that the canonical quantity is
\begin{equation}
    \deltatop = \frac{3}{256\pi^4},
\end{equation}
and the appearance of ``48'' is simply the same number expressed in units of $\cthree^4$.

\subsection{The Physical Setup}

The topological correction $\deltatop$ arises from the leading gauge-invariant correction 
to the tree-level geometric scale $\phitree$. The requirements are:
\begin{enumerate}[leftmargin=1.5em, topsep=2pt, itemsep=1pt]
    \item \textbf{Gauge invariance:} The correction must be built from $\cthree$ only 
        (no explicit $\alpha$ dependence at this order).
    \item \textbf{Dimensionlessness:} Since $\cthree$ is dimensionless, the correction 
        must be a pure power $\cthree^n$ with an integer coefficient.
    \item \textbf{Discreteness:} Topological invariants take discrete values; the 
        coefficient must be a rational number with simple structure.
\end{enumerate}

\subsection{\texorpdfstring{Power Counting: Why $\cthree^4$?}{Power Counting: Why c3^4?}}

The effective action on the M\"obius fiber includes the Chern--Simons term 
$\cthree\, a F\tilde{F}$. The leading topological correction comes from the 
one-loop determinant of the Dirac operator coupled to this background.

\textbf{Dimensional analysis:} The Chern--Simons coupling enters as $\cthree$. The 
Euler characteristic and Pontryagin invariants contribute at order $\cthree^2$ each. 
The product yields:
\begin{equation}
    \deltatop \sim \cthree^2 \times \cthree^2 = \cthree^4
\end{equation}

\textbf{No $\cthree^2$ term:} A correction of order $\cthree^2$ would require a 
pure Euler term, but this is already absorbed into $\phitree$ via Gauss--Bonnet. 
The next allowed correction is $\cthree^4$.

\subsection{Spin-Lifted Deficit Factor}
The non-orientable identification acts as a $\mathbb{Z}_2$ operation on the base,
but in the spin bundle the same element lifts to order $4$ because a $2\pi$ rotation
acts as $-1$ on spinors and only a $4\pi$ rotation is trivial. This produces a
universal deficit factor
\begin{equation}
    1-\frac{1}{4}=\frac{3}{4},
\end{equation}
per minimal sector.

\subsection{Sector Count from Cut Geometry}
Cutting along $\Gamma$ yields boundary components $(C_1,C_2,\Gamma_+,\Gamma_-)$.
The four local corner patches
$(C_1,\Gamma_+)$, $(C_1,\Gamma_-)$, $(C_2,\Gamma_+)$, $(C_2,\Gamma_-)$
constitute the fundamental domain. Summing the spin-lifted deficit factor over these sectors yields
\begin{equation}
    \sum_{p=1}^{F}\left(1-\frac{1}{4}\right)=4\cdot\frac{3}{4}=3.
\end{equation}
In TFPT normalization this yields the canonical form $\deltatop = 3/(256\pi^4)$.

\subsection{\texorpdfstring{Conversion to the $\cthree$ normalization}{Conversion to the c3 normalization}}

Using $\cthree = 1/(8\pi)$:
\begin{equation}
    48\cthree^4 = 48 \times \frac{1}{(8\pi)^4} 
    = \frac{48}{4096\pi^4}
    = \frac{3}{256\pi^4}
\end{equation}
so $\deltatop = 3/(256\pi^4)$ and $\deltatop = 48\cthree^4$ are exactly equivalent.

\subsection{Status note}
The spin-lift argument fixes the $1-1/4$ deficit factor robustly, and the sector count
is fixed by seam symmetry. The seam normalization is provided by the $\eta$-gluing
derivation in Appendix~\ref{app:seam}.

%=============================================================================
\section{Flavor Monodromy and Cross-Ratio}
\label{app:flavor}
%=============================================================================

This appendix derives the M\"obius structure of flavor relations from boundary monodromy.

\subsection{Projective Invariance on Boundaries}

On a conformally flat boundary, the metric is locally:
\begin{equation}
    ds^2 = e^{2\omega(x)} \delta_{ij} dx^i dx^j
\end{equation}
Holomorphic automorphisms (Wilson lines of flat connections) act as M\"obius 
transformations:
\begin{equation}
    z \mapsto \frac{az + b}{cz + d}, \quad ad - bc = 1
\end{equation}

\begin{remark}
\textbf{Context:} $\mathrm{SL}(2,\mathbb{C})$ double-covers the proper orthochronous Lorentz group $\mathrm{SO}^+(3,1)$, 
and its projectivization $\mathrm{PSL}(2,\mathbb{C})$ acts by M\"obius transformations on the Riemann sphere. 
This links the M\"obius structure to standard Lorentz symmetry rather than an ad hoc choice.
\end{remark}

\subsection{Cross-Ratio as Unique Invariant}

Given four marked points $\{z_1, z_2, z_3, z_4\}$ on the boundary, the unique 
projective invariant is the cross-ratio:
\begin{equation}
    [z_1, z_2; z_3, z_4] = \frac{(z_1 - z_3)(z_2 - z_4)}{(z_1 - z_4)(z_2 - z_3)}
\end{equation}

\textbf{Consequence:} Any physical quantity depending only on boundary data must be 
expressible as a function of cross-ratios, hence as a M\"obius map.

\subsection{Cusp Selection}

The cusps $y \in \{1, 1/3, 2/3\}$ arise from:
\begin{enumerate}[leftmargin=1.5em, topsep=2pt, itemsep=1pt]
    \item \textbf{Rationality:} Quantized holonomies $\Rightarrow$ rational cusps.
    \item \textbf{GUT compatibility:} SU(5) hypercharge assignments give 
        $Y = 1/3$ (quarks), $Y = 1$ (leptons), etc.
    \item \textbf{Uniqueness:} These are the only rational values that simultaneously 
        separate the three SM sectors (up, down, lepton) and maintain GUT normalization.
\end{enumerate}

\subsection{\texorpdfstring{Derivation of $\delta_\star$}{Derivation of delta_star}}

The flavor phase is a first-order closure:
\begin{equation}
    \delta_\star = \delta_0 + \delta_1 \cdot \phiz + \mathcal{O}(\phiz^2)
\end{equation}

\textbf{Leading term:} $\delta_0 = 3/5$ from the abelian trace (same as $\bone$ numerator 
divided by 41/10's denominator structure).

\textbf{Correction:} $\delta_1 = 1/6$ from the M\"obius length scale. The factor $1/6$ 
matches the boundary normalization ($6\pi$ total curvature).

\textbf{Result:}
\begin{equation}
    \delta_\star = \frac{3}{5} + \frac{\phiz}{6} = 0.6 + 0.00886 = 0.6089
\end{equation}

%=============================================================================
\section{Reproducibility}
\label{app:repro}
%=============================================================================

\textbf{Replication package:} All numerical results can be reproduced using the 
TFPT verification suite (v5).

\textbf{Contents:}
\begin{itemize}[leftmargin=1.5em, topsep=2pt, itemsep=1pt]
    \item Python modules for each calculation domain
    \item JSON files with all constants and derived quantities
    \item Automated test suite with pass/fail status
    \item Plot generation scripts
    \item Verification suite modules include: \texttt{effective\_action\_r2}, \texttt{bounce\_perturbations} (see \codepath{tfpt-suite/} and the generated \codepath{tfpt-suite/tfpt-test-results.pdf})
\end{itemize}

\textbf{Key output files:}
\begin{itemize}[leftmargin=1.5em, topsep=2pt, itemsep=1pt]
    \item \texttt{tfpt\_complete\_analysis.json}: Master results file
    \item \texttt{e8\_chain\_analysis.json}: Complete E$_8$ chain data
    \item \texttt{gauge\_couplings.csv}: Two-loop RGE output (via PyR@TE)
\end{itemize}

\textbf{Execution:} \texttt{python -m tfpt\_verification5.run\_all}

\textbf{Verification:} The core modules must report ``success'' status; the
$R^2$ effective-action and bounce-perturbation modules are implemented in the shipped suite and included in the generated test report.

%=============================================================================
\section{Prediction Ledger and Falsification}
\label{app:prediction-ledger}
%=============================================================================

\textbf{Purpose:} Provide a single, reviewer-friendly list of predictions and candidates that is tied to the same JSON sources used by the verification suite, to prevent number drift between Abstract, theorems, tables, and appendices.

\begingroup
\footnotesize
\sisetup{round-mode=figures, round-precision=6}
\renewcommand{\arraystretch}{1.15}

% AUTO-GENERATED by tfpt_suite/export_prediction_ledger.py — inlined for standalone-paper build.
\begin{longtable}{@{}p{0.15\textwidth}p{0.27\textwidth}rrrp{0.10\textwidth}p{0.18\textwidth}@{}}
\caption{Prediction ledger (auto-generated from TFPT suite outputs).}\label{tab:prediction-ledger}\\
\toprule
Observable & Formula / dependencies & Pred. & Ref. & $z$ & Status & Suite evidence\\
\midrule
\endfirsthead
\toprule
Observable & Formula / dependencies & Pred. & Ref. & $z$ & Status & Suite evidence\\
\midrule
\endhead
\midrule
\multicolumn{7}{r}{\small Continued on next page.}\\
\midrule
\endfoot
\bottomrule
\endlastfoot
\multicolumn{7}{@{}l}{\textbf{Candidates / open interfaces}}\\
\(\Omega_b\) & \(\Omega_b=(4\pi-1)\beta_{\rm rad}\) under explicit sector-counting assumptions [Candidate] & \num{0.048940662665450112200545653760944651480054952073107413870556556782266783965399009} & \num{0.049301692328524438476424754994611856173233055557122787616860658651215012327847394} $\pm$ \num{0.00085681142747140931509835404112378635706093499964689237552400217734754363189007356} & \num{-0.421364} & C (conditional) & \texttt{omega\_b\_conjecture\_scan}\\ \codepath{out/omega_b_conjecture_scan/results.json}\\
\(\rho_\Lambda\) (normalization audit) & \(\phi_\*\!=n\,e^{-\alpha^{-1}(0)/2}\), \(\rho_\Lambda=(\bar M_{\rm Pl}\phi_\*)^4\); best \(n=1/2\) with log10 mismatch [Candidate] & \num{2.060247122268795528691070560351946768145020625187219242978373038664026793360454e-47} & \num{2.514176465474032e-47} & --- & C (PASS audit) & \texttt{dark\_energy\_exponential\_audit}\\ \codepath{theoryv3/out/dark_energy_exponential_audit/results.json}\\
\(\rho_\Lambda\) (torsion condensate closure) & Gap-equation completion with discrete \(n\) (spectral flow); compare to cosmology target [Candidate] & \num{4.1204942445375910573821411207038935362900412503744384859567460773280535867209079e-47} & \num{2.514176465474032e-47} & \num{2.1403530198484351956856249171986807549266690574189652264117407874776932146992054} & C (FAIL gate) & \texttt{torsion\_condensate}\\ \codepath{out_physics/torsion_condensate/results.json}\\
\multicolumn{7}{@{}l}{\textbf{Comparison layer}}\\
\(\overline{\alpha}^{(5)}(M_Z)^{-1}\) & QED running + matching (declared policy) from \(\alpha(0)\) [Policy] & \num{127.94051870737564248749202130667255420059119636590364698278411016186034311550876} & \num{127.93} $\pm$ \num{0.008} & \num{1.314838421955310936502663334069275073899545737955872848013770232542889438594741} & comparison & \texttt{global\_consistency\_test}\\ \codepath{out/global_consistency_test/results.json}\\
\multicolumn{7}{@{}l}{\textbf{Conditional dynamics}}\\
\(A_s\) & Starobinsky \(R^2\) completion: \(A_s\simeq \frac{N^2}{24\pi^2}(M/\bar M_{\rm Pl})^2\) [Conditional] & \num{0.000000002090191364329459982659058205552757672085364626939873929875219205466615549540939} & \num{0.0000000020989031673} $\pm$ \num{0.00000000002938464434} & \num{-0.29647467805764898160213003439885527348994189187882714815192466383904684207796011} & conditional & \texttt{global\_consistency\_test}\\ \codepath{out/global_consistency_test/results.json}\\
\(\Omega_a h^2\) & Axion relic accounting (scenario-dependent) [Conditional] & \num{0.12275084115028702} & --- & --- & conditional & \texttt{axion\_dm\_audit}\\ \codepath{theoryv3/out/axion_dm_audit/results.json}\\
\(\nu\) [GHz] & \(\nu_{\rm GHz}\approx 0.24179893\, m_a[\mu\mathrm{eV}]\) [Conditional] & \num{15.76402318010866} & --- & --- & conditional & \texttt{axion\_dm\_audit}\\ \codepath{theoryv3/out/axion_dm_audit/results.json}\\
\(f_a\) [GeV] & PQ cascade block (axion claim) [Conditional] & \num{88639886850.34554} & --- & --- & conditional & \texttt{axion\_dm\_audit}\\ \codepath{theoryv3/out/axion_dm_audit/results.json}\\
\(m_a\) [\(\mu\)eV] & PQ cascade block (axion claim) [Conditional] & \num{65.1947598780055} & --- & --- & conditional & \texttt{axion\_dm\_audit}\\ \codepath{theoryv3/out/axion_dm_audit/results.json}\\
\(n_s\) & Starobinsky \(R^2\) completion: \(n_s=1-\frac{2}{N}\) (benchmark \(N=56\)) [Conditional] & \num{0.96428571428571428571428571428571428571428571428571428571428571428571428571428571} & \num{0.9649} $\pm$ \num{0.0042} & \num{-0.14625850340136054421768707482993197278911564625850340136054421768707482993197288} & conditional & \texttt{global\_consistency\_test}\\ \codepath{out/global_consistency_test/results.json}\\
\(r\) & Starobinsky \(R^2\) completion: \(r=\frac{12}{N^2}\) (benchmark \(N=56\)); compared to an upper bound [Conditional] & \num{0.003826530612244897959183673469387755102040816326530612244897959183673469387755102} & < 0.036 & --- & conditional & \texttt{global\_consistency\_test}\\ \codepath{out/global_consistency_test/results.json}\\
\multicolumn{7}{@{}l}{\textbf{Core (parameter-free)}}\\
\(\alpha^{-1}(0)\) & \(\mathrm{CFE} + \varphi_0(\alpha)\) with \(\delta_2=(g/4)\delta_{\mathrm{top}}^2\) and \(g=5\) [Kernel+Grammar] & \num{137.03599921615843479026171751711397398605568500258186146747743646328176776397918} & \num{137.035999177} $\pm$ \num{0.000000021} & \num{1.8646873709648436912911416183836040477419934032132112601562746554275798808614493} & P & \texttt{defect\_partition\_g5\_audit}\\ \codepath{theoryv3/out/defect_partition_g5_audit/results.json}\\
\(\beta_{\mathrm{deg}}\) & \(\beta_{\mathrm{deg}}=\frac{180}{\pi}\frac{\varphi_0}{4\pi}\) [Kernel] & \num{0.24243503090092952849243152047951445246859512239848881592836053961070607871616298} & \num{0.35} $\pm$ \num{0.14} & \num{-0.7683212078505033679112034251463253395100348400107941719402818599235280091702644} & P & \texttt{global\_consistency\_test}\\ \codepath{out/global_consistency_test/results.json}\\
\(\lambda\equiv |V_{us}|\) & \(\lambda=\sqrt{\varphi_0}\left(1-\frac{1}{2}\varphi_0\right)\) [Kernel] & \num{0.22445997051897373245511597747748194821840297007791297390969270711773328122860166} & \num{0.22501} $\pm$ \num{0.00068} & \num{-0.80886688386215815424120959193831144352504400306915601515778365039223348735049879} & P & \texttt{global\_consistency\_test}\\ \codepath{out/global_consistency_test/results.json}\\
\(\sin^2\theta_{12}\) & TM1 sum rule from \(\sin^2\theta_{13}\) [Kernel+Grammar] & \num{0.31792771} & \num{0.307} $\pm$ \num{0.012} & \num{0.910642} & P & \texttt{pmns\_z3\_breaking}\\ \codepath{out/pmns_z3_breaking/results.json}\\
\(\sin^2\theta_{13}\) & Identity + TM1 [Kernel+Grammar] & \num{0.02310844} & \num{0.02224} $\pm$ \num{0.00057} & \num{1.523570} & P & \texttt{pmns\_z3\_breaking}\\ \codepath{out/pmns_z3_breaking/results.json}\\
\(g_{a\gamma\gamma}\) & \(g_{a\gamma\gamma}=-4c_3=-\frac{1}{2\pi}\) [Kernel] & \num{-0.15915494309189535} & --- & --- & P & \texttt{(algebraic)}\newline \codepath{paper kernel (no suite output file)}\\
\end{longtable}
\endgroup

\textbf{Falsification guide (high-level):}
\begin{itemize}[leftmargin=1.5em, topsep=2pt, itemsep=1pt]
    \item \textbf{Core (parameter-free):} Any robust shift of CODATA $\alpha^{-1}(0)$ or a decisive inconsistency in the \texttt{defect\_partition\_g5\_audit} closure would directly falsify the TFPT kernel.
    \item \textbf{Comparison layer:} A stable discrepancy in $\overline{\alpha}^{(5)}(M_Z)^{-1}$ after adopting the declared running/matching policy falsifies the paper-to-observable bridge (not the kernel itself).
    \item \textbf{Conditional dynamics:} Failures here indicate that the adopted completion/policy layer (e.g.\ $R^2$ benchmark choices, relic accounting scenario) is not closed or not correct, and must be treated as conditional rather than unconditional prediction.
    \item \textbf{Candidates / open interfaces:} These rows are explicitly not ToE-closed; a FAIL gate is reported as such and should not be framed as a resolved prediction.
\end{itemize}

%=============================================================================
\section{Identity Catalogue and Status Labels}
\label{app:identities}
%=============================================================================

This appendix labels frequently used equalities by logical status to prevent conflating definitions, algebraic rewrites, and genuine predictions.

\textbf{Legend:}
\begin{itemize}[leftmargin=1.5em, topsep=2pt, itemsep=1pt]
    \item \textbf{D}: Definition (input)
    \item \textbf{K}: Consequence of definitions (algebraic)
    \item \textbf{L}: Lemma from axioms/postulates
    \item \textbf{P}: Prediction (output; not used as input)
\end{itemize}

\begin{center}
\renewcommand{\arraystretch}{1.2}
\begin{tabular}{@{}lp{12cm}@{}}
\toprule
\textbf{Tag} & \textbf{Statement}\\
\midrule
\textbf{D} & $\beta_{\text{rad}} := \phiz/(4\pi)$.\\
\textbf{K} & $\phiz = 4\pi\beta_{\text{rad}}$.\\
\textbf{K} & $\beta_{\text{deg}} := (180/\pi)\beta_{\text{rad}}$.\\
\textbf{K} & $\gagg = -4\cthree = -1/(2\pi)$.\\
\textbf{K} & $\frac{d\beta}{d\eta} = -\frac{1}{2}\gagg\,\frac{d a_{\text{top}}}{d\eta} = 2\cthree\,\frac{d a_{\text{top}}}{d\eta}$ (from $\gagg=-4\cthree$; Theorem~\ref{thm:birefringence}).\\
\textbf{L} & $\gamma(0)=5/6$ (Definition~\ref{def:E8cascade}; discrete closure choice).\\
\textbf{L} & $\phitree = 1/(6\pi)$ (Axiom~\ref{ax:GB}; from $\eta$-gluing seam term).\\
\textbf{L} & $\deltatop = \frac{1}{256\pi^4}\sum_{p=1}^{4}\left(1-\frac{1}{4}\right)=\frac{3}{256\pi^4}$ (Axiom~\ref{ax:TC}).\\
\textbf{P} & CFE closure (two-defect, derived partition): $\alpha^{-1}_{\text{TFPT}}(0)=\num[round-mode=places,round-precision=12]{\TFPTalphaInvGOverFourNum}$ (CODATA $z=\num[round-mode=places,round-precision=3]{\TFPTalphaZGOverFourNum}$; Theorem~\ref{thm:selfconsistent}).\\
\textbf{L} & One-defect truncation (diagnostic baseline): $\alpha^{-1}(0)=\num[round-mode=places,round-precision=10]{\TFPTalphaInvSelfNum}$ (Lemma~\ref{lem:oneDefectTruncation}).\\
\textbf{P} & Cabibbo $\lambda = \sqrt{\phiz}\left(1-\frac{1}{2}\phiz\right)=0.22445997\ldots$ (Corollary~\ref{cor:Cabibbo}).\\
\textbf{P} & Birefringence $\beta_{\text{deg}} \approx 0.2424^\circ$ from $\Delta a_{\text{top}} = n\,\phiz$ (minimal $n=1$; Theorem~\ref{thm:birefringence}; optional realization in Sec.~\ref{sec:deltaa}).\\
\textbf{P} & $R^2$ inflation: $n_s=1-2/N$, $r=12/N^2$, $A_s \simeq \frac{N^2}{24\pi^2}(M/\Mpl)^2$ with $M/\Mpl=\sqrt{8\pi}\cthree^4$ (Theorem~\ref{thm:inflation}; assumptions K1--K3).\\
\textbf{P} (spec.) & $\Omega_b = (4\pi-1)\beta_{\text{rad}}$ (Table~\ref{tab:beta}).\\
\bottomrule
\end{tabular}
\end{center}

%=============================================================================
\section{\texorpdfstring{One-Loop Effective Action in Riemann--Cartan and the $R^2$ Coefficient}{One-Loop Effective Action in Riemann--Cartan and the R2 Coefficient}}
\label{app:r2}
%=============================================================================

\begin{tcolorbox}[colback=gray!5, colframe=gray!60, 
    title={\textbf{Assumptions K1--K4}}, fonttitle=\bfseries\small]
\begin{itemize}[leftmargin=1.5em, topsep=2pt, itemsep=1pt]
    \item \textbf{K1:} Field content and torsionful Lagrangian fixed by the minimal UFE sector.
    \item \textbf{K2:} Quadratic fluctuation operator is of Laplace type after background-field gauge fixing.
    \item \textbf{K3:} Heat-kernel coefficient $a_2$ captures the curvature-squared terms in $\Gamma_{\mathrm{eff}}$.
    \item \textbf{K4:} Matching to $R + R^2/(6M^2)$ after imposing the GR renormalization condition.
\end{itemize}
\end{tcolorbox}

\subsection{K.1 Setup}
We evaluate the one-loop effective action in a Riemann--Cartan background with torsion
decomposed into axial $S_\mu$, trace $T_\mu$, and tensor $q_{\mu\nu\rho}$. The minimal UFE
choice treats torsion at quadratic order, so integrating out torsion is Gaussian and
parameter-free once $\cthree$ fixes the normalization of the coupling.

\subsection{K.2 Operator Form}
After background-field gauge fixing, the quadratic operator for torsion fluctuations can be
written in Laplace type,
\begin{equation}
    \Delta = -\nabla^2 + \mathcal{E}(R,K),
\end{equation}
with $\mathcal{E}$ built from curvature and torsion invariants. The contribution to
$\Gamma_{\mathrm{eff}}$ is $\frac{1}{2}\ln\det\Delta$.

\subsection{K.3 Heat-Kernel Coefficients}
The local part of $\Gamma_{\mathrm{eff}}$ is governed by the heat-kernel expansion
\begin{equation}
    \mathrm{Tr}\,e^{-s\Delta} \sim \sum_{n\ge 0} s^{(n-4)/2} a_n,
\end{equation}
where $a_2$ contains the curvature-squared terms. The torsion-dependent pieces combine into
an effective $R^2$ operator in the low-energy action.

\subsection{K.4 Renormalization Condition}
The GR limit (A8) fixes the renormalization scheme by requiring that the $K\to 0$ limit
reduces to the Einstein--Hilbert action with coefficient $\Mpl^2/2$.

\subsection{K.5 TFPT Closure}
With the TFPT normalization of the torsion sector, the induced $R^2$ coefficient is fixed by
$\cthree$, yielding
\begin{equation}
    \frac{M}{\Mpl}=\sqrt{8\pi}\;\cthree^4,
\end{equation}
which reproduces \eqref{eq:R2scale} without additional parameters.

%=============================================================================
\section{Bounce Perturbations and the Transfer Function}
\label{app:bounce}
%=============================================================================

\begin{tcolorbox}[colback=gray!5, colframe=gray!60, 
    title={\textbf{Status Note}}, fonttitle=\bfseries\small]
This appendix provides a minimal outline; the full numerical transfer-function evaluation
is implemented in the \texttt{bounce\_perturbations} verification module (see \texttt{tfpt-suite/}).
\end{tcolorbox}

\subsection{L.1 Background Solution}
The bounce background is modeled by a smooth scale factor $a(\eta)$ where the torsion-induced
$a^{-6}$ term regularizes $z''/z$ for scalar perturbations. The transition to the $R^2$
inflationary phase is implemented by matching $a(\eta)$ and its derivatives at $\eta=\eta_t$.

\subsection{L.2 Numerical Integration Setup}
Mode functions are initialized in an adiabatic vacuum well before the bounce and integrated
through the bounce and into the $R^2$ era. The numerical scheme records the transfer function
by comparing late-time solutions to the pure $R^2$ baseline.

\subsection{L.3 Transfer Function}
Define the transfer function as
\begin{equation}
    T(k) \equiv \left|\frac{v_k(\eta\gg \eta_t)}{v_k^{(R^2)}(\eta\gg \eta_t)}\right|,
\end{equation}
with $T(k)\to 1$ for modes well inside the horizon during the bounce and deviations confined
to $k\sim k_{\text{bounce}}$.

\TFPTTransferFunctionPlot

\subsection{\texorpdfstring{L.4 The $k\to\ell$ Gate (CMB Observability as Policy Interface)}{L.4 The k->ell Gate (CMB Observability as Policy Interface)}}
\begin{tcolorbox}[colback=yellow!3, colframe=yellow!60!black,
    title={\textbf{What is deterministic vs.\ what is still policy}}, fonttitle=\bfseries\small]
\textbf{Deterministic (given the bounce completion):}
\begin{itemize}[leftmargin=1.5em, topsep=2pt, itemsep=1pt]
    \item The transfer-function shape $T(k)$ (this appendix; implemented in \texttt{bounce\_perturbations}).
    \item Given an explicit expansion-history budget, the mapping $k\to\ell$ and the resulting $C_\ell$ curves are deterministic (implemented in \texttt{k\_calibration} and \texttt{boltzmann\_transfer}).
\end{itemize}

\textbf{Policy / not yet ToE-closed:}
\begin{itemize}[leftmargin=1.5em, topsep=2pt, itemsep=1pt]
    \item The mapping from comoving $k$ to observed CMB multipole $\ell$ depends on the absolute expansion history (reheating/threshold bookkeeping), summarized as an overall scaling $a_0/a_t$ (or more precisely $a_0/a_{\rm transition}$ for the chosen transition surface).
    \item The signature classification (``CMB'' vs ``small-scale'') is therefore a \emph{gate}: it depends on the declared history, and should not be framed as an unconditional claim.
\end{itemize}

\textbf{Notation contract (to prevent $k$-confusion):}
\begin{itemize}[leftmargin=1.5em, topsep=2pt, itemsep=1pt]
    \item $k_{\hat{}}$ is a \emph{dimensionless} wave-number used inside the bounce solver (in $x=M\eta$ units).
    \item The raw bounce scales $k^{(s)}_{\rm bounce}$ and $k^{(t)}_{\rm bounce}$ are reported in these internal units; absolute $k(\mathrm{Mpc}^{-1})$ requires the TFPT scale $M$ and the expansion budget $a_0/a_{\rm transition}$.
    \item The observable proxy is $\ell \approx k(\mathrm{Mpc}^{-1})\,\chi_\*$ with $\chi_\*$ computed in a flat-$\Lambda$CDM snapshot (suite).
\end{itemize}

\textbf{Current suite snapshot (physics mode; threshold-derived budget):}
\begin{itemize}[leftmargin=1.5em, topsep=2pt, itemsep=1pt]
    \item $a_0/a_{\rm transition}\approx 2.4486\times 10^{57}$,
    \item $\chi_\*\approx 1.3867\times 10^4$ Mpc,
    \item $\ell^{(s)}_{\rm bounce}\approx 56.997$ and $\ell^{(t)}_{\rm bounce}\approx 0.212$.
\end{itemize}
Under this explicit gate, the scalar bounce feature is CMB-low-$\ell$, while the tensor feature is super-horizon (below the quadrupole). 

\textbf{Falsifiable signature classes (conditional, gate-dependent):}
\begin{itemize}[leftmargin=1.5em, topsep=2pt, itemsep=1pt]
    \item \textbf{CMB-low-$\ell$ scalar feature}: features near $\ell\sim\mathcal{O}(10^2)$ (current default gate).
    \item \textbf{Small-scale-only}: features shifted to $\ell\gg 10^3$ under alternative histories.
\end{itemize}

\textbf{Suite evidence:} The \texttt{k\_calibration} module reports both (i) the required scaling to place features at target multipoles and (ii) the threshold-derived expansion-budget estimate; \texttt{boltzmann\_transfer} produces explicit CAMB-backed $C_\ell$ curves and the relative deviation $\Delta C_\ell/C_\ell$ between bounce-injected and baseline spectra.
\end{tcolorbox}

\TFPTkCalibrationTable

\begin{figure}[H]
\centering
\begin{tikzpicture}
\begin{groupplot}[
    group style={group size=2 by 1, horizontal sep=1.2cm},
    width=0.48\textwidth,
    height=0.42\textwidth,
    grid=both,
    legend style={font=\scriptsize},
]
\nextgroupplot[
    title={required $a_0/a_{\rm tr}$ vs target $\ell$},
    xmode=log,
    ymode=log,
    xlabel={target $\ell$},
    ylabel={$a_0/a_{\rm tr}$},
    xmin=1.5, xmax=1000,
]
\addplot+[mark=*] coordinates {
    (2,6.978134315108868e58)
    (30,4.652089543405912e57)
    (700,1.9937526614596766e56)
};
\addlegendentry{scalar (required)}
\addplot+[mark=square*] coordinates {
    (2,2.5901704412653403e56)
    (30,1.7267802941768936e55)
    (700,7.400486975043829e53)
};
\addlegendentry{tensor (required)}
\addplot[red, dashed, thick] coordinates {(2,2.4485807685210785e57) (700,2.4485807685210785e57)};
\addlegendentry{current budget ($2.45\times10^{57}$)}

\nextgroupplot[
    title={gate view (CMB window + bounce $\ell$)},
    xmode=log,
    xlabel={$\ell$},
    ymin=0, ymax=1,
    ytick=\empty,
    xmin=0.1, xmax=5000,
    axis y line=none,
]
% CMB window shading: [2,2500]
\addplot[draw=none, fill=blue!10] coordinates {(2,0) (2500,0) (2500,1) (2,1)} \closedcycle;
\addplot[black!40] coordinates {(2,0) (2,1)};
\addplot[black!40] coordinates {(2500,0) (2500,1)};
\node[anchor=south, font=\scriptsize] at (axis cs:20,0.02) {CMB window};

% bounce locations (physics-suite snapshot)
\addplot[green!60!black, thick] coordinates {(56.99737909257206,0) (56.99737909257206,1)};
\node[anchor=south, font=\scriptsize, green!60!black] at (axis cs:56.99737909257206,0.02) {$\ell_b^{(s)}\approx57$};
\addplot[orange!80!black, thick] coordinates {(0.2115650400072186,0) (0.2115650400072186,1)};
\node[anchor=south, font=\scriptsize, orange!80!black] at (axis cs:0.2115650400072186,0.02) {$\ell_b^{(t)}\approx0.21$};
\end{groupplot}
\end{tikzpicture}
\caption{Suite plots from \texttt{k\_calibration} (physics mode): scaling and feasibility diagnostics for the $k\to\ell$ gate under an explicit expansion-history budget.}
\end{figure}

\begin{figure}[H]
\centering
\begin{tikzpicture}
\begin{groupplot}[
    group style={group size=2 by 2, horizontal sep=1.2cm, vertical sep=1.0cm},
    width=0.48\textwidth,
    height=0.33\textwidth,
    xmode=log,
    xmin=2, xmax=3000,
    grid=both,
    legend style={font=\scriptsize},
]
\nextgroupplot[
    title={$D_\ell^{TT}$ (baseline vs bounce)},
    ymode=log,
    ylabel={$D_\ell^{TT}$ [$\mu$K$^2$]},
]
\addplot+[mark=*, mark size=2pt] coordinates {
    (2,1053.602432)
    (3,998.0887611)
    (4,944.9109792)
    (5,905.3057583)
    (10,845.7614033)
    (20,936.246946)
    (30,1090.649646)
    (50,1468.079395)
    (80,2185.897476)
    (100,2782.530282)
    (150,4517.548558)
    (200,5768.991349)
    (300,4189.951412)
    (500,2521.115802)
    (700,1928.629066)
    (1000,1093.569344)
    (1500,706.1670601)
    (2000,237.9250741)
    (2508,78.09810466)
};
\addlegendentry{baseline}
\addplot+[mark=square*, mark size=2pt] coordinates {
    (2,10994.05279)
    (3,10486.33276)
    (4,9385.765294)
    (5,7182.088179)
    (10,466.5297095)
    (20,311.2303573)
    (30,647.8609341)
    (50,1396.409529)
    (80,2178.001377)
    (100,2764.125993)
    (150,4513.504899)
    (200,5742.64759)
    (300,4165.032244)
    (500,2519.57907)
    (700,1945.981149)
    (1000,1115.817699)
    (1500,728.4072936)
    (2000,247.883187)
    (2508,81.98087433)
};
\addlegendentry{bounce injected}

\nextgroupplot[
    title={$D_\ell^{EE}$ (baseline vs bounce)},
    ymode=log,
    ylabel={$D_\ell^{EE}$ [$\mu$K$^2$]},
]
\addplot+[mark=*, mark size=2pt] coordinates {
    (2,0.03176782281)
    (3,0.0407463901)
    (4,0.03535540342)
    (5,0.02363620444)
    (10,0.003157610827)
    (20,0.005394191594)
    (30,0.0223578033)
    (50,0.1235977312)
    (80,0.4863197226)
    (100,0.8033626171)
    (150,1.135920327)
    (200,0.710593948)
    (300,9.377843402)
    (500,8.525653848)
    (700,38.4094504)
    (1000,43.34908241)
    (1500,13.22941615)
    (2000,8.84123834)
    (2508,3.009426377)
};
\addlegendentry{baseline}
\addplot+[mark=square*, mark size=2pt] coordinates {
    (2,0.5094182832)
    (3,0.562182037)
    (4,0.3663058727)
    (5,0.1665671877)
    (10,0.0005850428195)
    (20,0.0006453541167)
    (30,0.004809524067)
    (50,0.09804048815)
    (80,0.489652682)
    (100,0.7951178048)
    (150,1.13708347)
    (200,0.7069214332)
    (300,9.311700761)
    (500,8.492435159)
    (700,38.50920372)
    (1000,43.9978482)
    (1500,13.62199765)
    (2000,9.189186049)
    (2508,3.154751105)
};
\addlegendentry{bounce injected}

\nextgroupplot[
    title={relative deviation (TT)},
    title style={at={(0.5,0.97)}, yshift=-0.6ex},
    xlabel={$\ell$},
    ylabel={$\Delta D_\ell/D_{\ell,0}$},
    ymin=-1.0, ymax=10.5,
]
\addplot+[mark=*, mark size=2pt] coordinates {
    (2,9.43472609)
    (3,9.506413022)
    (4,8.93296247)
    (5,6.933328727)
    (10,-0.4483908727)
    (20,-0.6675766381)
    (30,-0.405986206)
    (50,-0.0488187938)
    (80,-0.003612291319)
    (100,-0.006614227665)
    (150,-0.0008951004397)
    (200,-0.004566441139)
    (300,-0.005947364325)
    (500,-0.0006095444869)
    (700,0.008997107253)
    (1000,0.02034471301)
    (1500,0.03149429467)
    (2000,0.04185398693)
    (2508,0.04971656719)
};
\addplot[black!50, dashed] coordinates {(2,0) (2508,0)};

\nextgroupplot[
    title={relative deviation (EE)},
    title style={at={(0.5,0.97)}, yshift=-0.6ex},
    xlabel={$\ell$},
    ylabel={$\Delta D_\ell/D_{\ell,0}$},
    ymin=-1.0, ymax=16.5,
]
\addplot+[mark=*, mark size=2pt] coordinates {
    (2,15.03566874)
    (3,12.79710045)
    (4,9.360675802)
    (5,6.04712079)
    (10,-0.8147197829)
    (20,-0.8803612913)
    (30,-0.7848838724)
    (50,-0.2067776069)
    (80,0.006853432569)
    (100,-0.01026287768)
    (150,0.00102396489)
    (200,-0.005168232629)
    (300,-0.007053075852)
    (500,-0.003896321651)
    (700,0.002597103541)
    (1000,0.0149660788)
    (1500,0.02967489204)
    (2000,0.0393550876)
    (2508,0.04828984317)
};
\addplot[black!50, dashed] coordinates {(2,0) (2508,0)};
\end{groupplot}
\end{tikzpicture}
\caption{Suite plots from \texttt{boltzmann\_transfer} (physics mode): CAMB-backed $C_\ell$ curves (TT/EE) for baseline vs bounce-injected primordial spectrum, and the relative deviation $\Delta C_\ell/C_\ell$.}
\end{figure}

\subsection{L.5 Falsification Targets}
Large-scale suppression or oscillatory features in $\mathcal{P}_\zeta(k)$ are predicted at
$k\lesssim k_{\text{bounce}}$. Current CMB large-scale bounds constrain the allowable feature
amplitude, providing a direct observational test of the bounce completion.

%=============================================================================
\section*{Reference Ledger (Auto-Generated)}
%=============================================================================
\addcontentsline{toc}{section}{Reference Ledger (Auto-Generated)}
% Inlined (standalone build): formerly % AUTO-GENERATED by tfpt_suite/export_reference_ledger.py — do not edit by hand.
% generated_at_utc: 2026-01-28T07:36:57.179559+00:00
% source: tfpt_suite/data/references.json (+ dereferenced global_reference*.json observables where applicable)

\begin{longtable}{@{}lrrllp{0.36\textwidth}@{}}
\caption{Reference ledger used by the TFPT suite (auto-generated).}\label{tab:reference-ledger}\\
\toprule
Key & Value & $\sigma$ & Units & Version & Source (stable key)\\
\midrule
\endfirsthead
\toprule
Key & Value & $\sigma$ & Units & Version & Source (stable key)\\
\midrule
\endhead
\midrule
\multicolumn{6}{r}{\small Continued on next page.}\\
\midrule
\endfoot
\bottomrule
\endlastfoot
\texttt{A\_s\_planck2018} & \num{2.0989031673e-09} & \num{2.938464434e-11} & dimensionless & Planck 2018 & \cite{planck2018} Planck 2018 results VI (2020), base-ΛCDM TT,TE,EE+lowE+lensing: ln(10\textasciicircum{}\{10\} A\_s)=3.044 ± 0.014 ⇒ A\_s=exp(3.044)/1e10 (https://arxiv.org/abs/1807.06209)\\
\texttt{H0\_planck2018} & \num{67.36} & \num{0.54} & \(\mathrm{km\,s^{-1}\,Mpc^{-1}}\) & Planck 2018 & \cite{planck2018} Planck 2018 results VI (2020), base-ΛCDM TT,TE,EE+lowE+lensing: H0 = 67.36 ± 0.54 (Table 2) (https://arxiv.org/abs/1807.06209)\\
\texttt{H0\_sh0es\_2022} & \num{73.04} & \num{1.04} & \(\mathrm{km\,s^{-1}\,Mpc^{-1}}\) & SH0ES 2022 & \cite{riess2022} Riess et al. 2022 (ApJ 934 L7)\\
\texttt{Mpl\_reduced} & \num{2.435e+18} & \num{0.0} & \(\mathrm{GeV}\) & CODATA & \cite{tfptSuiteData} tfpt\_suite.cosmo\_scale\_map.MPL\_REDUCED\_GEV\\
\texttt{Omega\_dm\_h2\_planck2018} & \num{0.12} & \num{0.001} & dimensionless & Planck 2018 & \cite{planck2018} Planck 2018 base LCDM\\
\texttt{alpha\_bar5\_inv\_MZ\_pdg2024} & \num{127.93} & \num{0.008} & dimensionless & PDG 2024 & \cite{pdg2024} PDG 2024 electroweak review: α̂\textasciicircum{}\{(5)\}(MZ)\textasciicircum{}\{-1\} = 127.930 ± 0.008 (MSbar; see tfpt\_suite/data/alpha\_running\_pdg.json).\\
\texttt{alpha\_inv\_codata\_2022} & \num{137.035999177} & \num{2.1e-08} & dimensionless & CODATA 2022 & \cite{codata2022} NIST CODATA 2022: inverse fine-structure constant α\textasciicircum{}\{-1\}=137.035999177(21) (https://physics.nist.gov/cgi-bin/cuu/Value?alphinv)\\
\texttt{alpha\_s\_mz\_pdg} & \num{0.1179} & \num{0.0011} & dimensionless & PDG (MZ, MSbar) & \cite{pdg2024} sm\_inputs\_mz.json:alpha\_s\\
\texttt{beta\_deg\_minami\_komatsu\_2020} & \num{0.35} & \num{0.14} & \(\mathrm{deg}\) & Minami \& Komatsu 2020 & \cite{minamiKomatsu2020} Minami \& Komatsu (2020), Phys. Rev. Lett. 125, 221301: β = 0.35° ± 0.14° (Planck 2018 polarization, accounting for calibration) (https://doi.org/10.1103/PhysRevLett.125.221301)\\
\texttt{cabibbo\_lambda\_pdg2024} & \num{0.22501} & \num{0.00068} & dimensionless & PDG 2024 & \cite{pdg2024} PDG 2024 CKM review (global fit, Eq. 12.27): |V\_us| = 0.22501 ± 0.00068 (https://pdg.lbl.gov/2024/reviews/rpp2024-rev-ckm-matrix.pdf)\\
\texttt{ln10\_As\_planck2018} & \num{3.044} & \num{0.014} & dimensionless & Planck 2018 & \cite{planck2018} Planck 2018 results VI (2020), base-ΛCDM TT,TE,EE+lowE+lensing: ln(10\textasciicircum{}\{10\} A\_s) = 3.044 ± 0.014 (Table 2) (https://arxiv.org/abs/1807.06209)\\
\texttt{m\_b\_pdg} & \num{4.18} & --- & \(\mathrm{GeV}\) & PDG (scheme dependent) & \cite{pdg2024} sm\_inputs\_mz.json:mb\_GeV\\
\texttt{m\_c\_pdg} & \num{1.27} & --- & \(\mathrm{GeV}\) & PDG (scheme dependent) & \cite{pdg2024} sm\_inputs\_mz.json:mc\_GeV\\
\texttt{m\_e\_pdg} & \num{0.0005109989461} & \num{0.0} & \(\mathrm{GeV}\) & PDG pole mass & \cite{pdg2024} lepton\_masses\_pdg.json:masses.electron\\
\texttt{m\_mu\_pdg} & \num{0.1056583745} & \num{0.0} & \(\mathrm{GeV}\) & PDG pole mass & \cite{pdg2024} lepton\_masses\_pdg.json:masses.muon\\
\texttt{m\_p\_pdg} & \num{0.938272} & --- & \(\mathrm{GeV}\) & PDG pole mass & \cite{pdg2024} mass\_spectrum\_minimal:ledger.placeholders.proton\_mass\_GeV\\
\texttt{m\_t\_pdg} & \num{172.76} & \num{0.3} & \(\mathrm{GeV}\) & PDG pole mass & \cite{pdg2024} sm\_inputs\_mz.json:mt\_GeV\\
\texttt{m\_tau\_pdg} & \num{1.77686} & \num{0.0} & \(\mathrm{GeV}\) & PDG pole mass & \cite{pdg2024} lepton\_masses\_pdg.json:masses.tau\\
\texttt{mass\_ratio\_mb\_over\_ms} & --- & --- & dimensionless & PDG (scheme dependent) & \cite{tfptSuiteData} \\
\texttt{mass\_ratio\_mc\_over\_mu\_quark} & --- & --- & dimensionless & PDG (scheme dependent) & \cite{tfptSuiteData} \\
\texttt{mass\_ratio\_ms\_over\_md} & --- & --- & dimensionless & PDG (scheme dependent) & \cite{tfptSuiteData} \\
\texttt{mass\_ratio\_mt\_over\_mc} & --- & --- & dimensionless & PDG (scheme dependent) & \cite{tfptSuiteData} \\
\texttt{mass\_ratio\_mu\_over\_e\_pdg} & \num{206.76828260879265} & \num{0.0} & dimensionless & PDG pole masses & \cite{pdg2024} lepton\_masses\_pdg.json (mu/e)\\
\texttt{mass\_ratio\_tau\_over\_mu\_pdg} & \num{16.81702949159037} & \num{0.0} & dimensionless & PDG pole masses & \cite{pdg2024} lepton\_masses\_pdg.json (tau/mu)\\
\texttt{n\_s\_planck2018} & \num{0.9649} & \num{0.0042} & dimensionless & Planck 2018 & \cite{planck2018} Planck 2018 results VI (2020), base-ΛCDM TT,TE,EE+lowE+lensing: n\_s = 0.9649 ± 0.0042 (Table 2) (https://arxiv.org/abs/1807.06209)\\
\texttt{omega\_b\_h2\_planck2018} & \num{0.02237} & \num{0.00015} & dimensionless & Planck 2018 & \cite{planck2018} Planck 2018 results VI (2020), base-ΛCDM TT,TE,EE+lowE+lensing: Ω\_b h\textasciicircum{}2 = 0.02237 ± 0.00015 (Table 2) (https://arxiv.org/abs/1807.06209)\\
\texttt{pmns\_delta\_cp\_deg\_nufit53\_no} & \num{232.0} & \num{39.0} & \(\mathrm{deg}\) & NuFIT 5.3 (2024, NO) & \cite{nufit53_2024} pmns\_reference.json:normal\_ordering.delta\_cp\_deg\\
\texttt{pmns\_sin2\_theta12\_nufit53\_no} & \num{0.307} & \num{0.012} & dimensionless & NuFIT 5.3 (2024, NO) & \cite{nufit53_2024} pmns\_reference.json:normal\_ordering.sin2\_theta12\\
\texttt{pmns\_sin2\_theta13\_nufit53\_no} & \num{0.02224} & \num{0.00057} & dimensionless & NuFIT 5.3 (2024, NO) & \cite{nufit53_2024} pmns\_reference.json:normal\_ordering.sin2\_theta13\\
\texttt{pmns\_sin2\_theta23\_nufit53\_no} & \num{0.454} & \num{0.019} & dimensionless & NuFIT 5.3 (2024, NO) & \cite{nufit53_2024} pmns\_reference.json:normal\_ordering.sin2\_theta23\\
\texttt{sin2\_thetaW\_mz\_pdg} & \num{0.23122} & \num{4e-05} & dimensionless & PDG (MZ, MSbar) & \cite{pdg2024} sm\_inputs\_mz.json:sin2\_thetaW\\
\texttt{v\_ew\_246} & \num{246.0} & \num{0.2} & \(\mathrm{GeV}\) & SM Higgs vev & \cite{tfptSuiteData} Standard Model reference\\
\end{longtable}


% -----------------------------------------------------------------------------
% AUTO-GENERATED by tfpt_suite/export_reference_ledger.py — do not edit by hand.
% generated_at_utc: 2026-01-28T07:36:57.179559+00:00
% source: tfpt_suite/data/references.json (+ dereferenced global_reference*.json observables where applicable)

\begingroup
\scriptsize
\sisetup{round-mode=figures, round-precision=6}
\setlength{\tabcolsep}{3pt}
\renewcommand{\_}{\string_\allowbreak}
\Urlmuskip=0mu plus 1mu\relax
\begin{longtable}{@{}>{\raggedright\arraybackslash}p{0.20\textwidth}>{\raggedleft\arraybackslash}p{0.12\textwidth}>{\raggedleft\arraybackslash}p{0.10\textwidth}>{\raggedright\arraybackslash}p{0.10\textwidth}>{\raggedright\arraybackslash}p{0.12\textwidth}>{\raggedright\arraybackslash}p{0.28\textwidth}@{}}
\caption{Reference ledger used by the TFPT suite (auto-generated).}\label{tab:reference-ledger}\\
\toprule
Key & Value & $\sigma$ & Units & Version & Source (stable key)\\
\midrule
\endfirsthead
\toprule
Key & Value & $\sigma$ & Units & Version & Source (stable key)\\
\midrule
\endhead
\midrule
\multicolumn{6}{r}{\small Continued on next page.}\\
\midrule
\endfoot
\bottomrule
\endlastfoot
\texttt{A\_s\_planck2018} & \num{2.0989031673e-09} & \num{2.938464434e-11} & dimensionless & Planck 2018 & \cite{planck2018} Planck 2018 results VI (2020), base-$\Lambda$CDM TT,TE,EE+lowE+lensing: ln(10\textasciicircum{}\{10\} A\_s)=3.044 $\pm$ 0.014 $\Rightarrow$ A\_s=exp(3.044)/1e10 (\url{https://arxiv.org/abs/1807.06209})\\
\texttt{H0\_planck2018} & \num{67.36} & \num{0.54} & \(\mathrm{km\,s^{-1}\,Mpc^{-1}}\) & Planck 2018 & \cite{planck2018} Planck 2018 results VI (2020), base-$\Lambda$CDM TT,TE,EE+lowE+lensing: $H_0 = 67.36 \pm 0.54$ (Table 2) (\url{https://arxiv.org/abs/1807.06209})\\
\texttt{H0\_sh0es\_2022} & \num{73.04} & \num{1.04} & \(\mathrm{km\,s^{-1}\,Mpc^{-1}}\) & SH0ES 2022 & \cite{riess2022} Riess et al. 2022 (ApJ 934 L7)\\
\texttt{Mpl\_reduced} & \num{2.435e+18} & \num{0.0} & \(\mathrm{GeV}\) & CODATA & \cite{tfptSuiteData} tfpt\_suite.cosmo\_scale\_map.MPL\_REDUCED\_GEV\\
\texttt{Omega\_dm\_h2\_planck2018} & \num{0.12} & \num{0.001} & dimensionless & Planck 2018 & \cite{planck2018} Planck 2018 base LCDM\\
\texttt{alpha\_bar5\_inv\_MZ\_pdg2024} & \num{127.93} & \num{0.008} & dimensionless & PDG 2024 & \cite{pdg2024} PDG 2024 electroweak review: $\hat{\alpha}^{(5)}(M_Z)^{-1}$ = 127.930 $\pm$ 0.008 (MSbar; see tfpt\_suite/data/alpha\_running\_pdg.json).\\
\texttt{alpha\_inv\_codata\_2022} & \num{137.035999177} & \num{2.1e-08} & dimensionless & CODATA 2022 & \cite{codata2022} NIST CODATA 2022: inverse fine-structure constant $\alpha^{-1}$=137.035999177(21) (\url{https://physics.nist.gov/cgi-bin/cuu/Value?alphinv})\\
\texttt{alpha\_s\_mz\_pdg} & \num{0.1179} & \num{0.0011} & dimensionless & PDG (MZ, MSbar) & \cite{pdg2024} sm\_inputs\_mz.json:alpha\_s\\
\texttt{beta\_deg\_minami\_komatsu\_2020} & \num{0.35} & \num{0.14} & \(\mathrm{deg}\) & Minami \& Komatsu 2020 & \cite{minamiKomatsu2020} Minami \& Komatsu (2020), Phys. Rev. Lett. 125, 221301: $\beta = 0.35^\circ \pm 0.14^\circ$ (Planck 2018 polarization, accounting for calibration) (\url{https://doi.org/10.1103/PhysRevLett.125.221301})\\
\texttt{cabibbo\_lambda\_pdg2024} & \num{0.22501} & \num{0.00068} & dimensionless & PDG 2024 & \cite{pdg2024} PDG 2024 CKM review (global fit, Eq. 12.27): $|V_{us}| = 0.22501 \pm 0.00068$ (\url{https://pdg.lbl.gov/2024/reviews/rpp2024-rev-ckm-matrix.pdf})\\
\texttt{ln10\_As\_planck2018} & \num{3.044} & \num{0.014} & dimensionless & Planck 2018 & \cite{planck2018} Planck 2018 results VI (2020), base-$\Lambda$CDM TT,TE,EE+lowE+lensing: ln(10\textasciicircum{}\{10\} A\_s) = 3.044 $\pm$ 0.014 (Table 2) (\url{https://arxiv.org/abs/1807.06209})\\
\texttt{m\_b\_pdg} & \num{4.18} & --- & \(\mathrm{GeV}\) & PDG (scheme dependent) & \cite{pdg2024} sm\_inputs\_mz.json:mb\_GeV\\
\texttt{m\_c\_pdg} & \num{1.27} & --- & \(\mathrm{GeV}\) & PDG (scheme dependent) & \cite{pdg2024} sm\_inputs\_mz.json:mc\_GeV\\
\texttt{m\_e\_pdg} & \num{0.0005109989461} & \num{0.0} & \(\mathrm{GeV}\) & PDG pole mass & \cite{pdg2024} lepton\_masses\_pdg.json:masses.electron\\
\texttt{m\_mu\_pdg} & \num{0.1056583745} & \num{0.0} & \(\mathrm{GeV}\) & PDG pole mass & \cite{pdg2024} lepton\_masses\_pdg.json:masses.muon\\
\texttt{m\_p\_pdg} & \num{0.938272} & --- & \(\mathrm{GeV}\) & PDG pole mass & \cite{pdg2024} mass\_spectrum\_minimal:ledger.placeholders.proton\_mass\_GeV\\
\texttt{m\_t\_pdg} & \num{172.76} & \num{0.3} & \(\mathrm{GeV}\) & PDG pole mass & \cite{pdg2024} sm\_inputs\_mz.json:mt\_GeV\\
\texttt{m\_tau\_pdg} & \num{1.77686} & \num{0.0} & \(\mathrm{GeV}\) & PDG pole mass & \cite{pdg2024} lepton\_masses\_pdg.json:masses.tau\\
\texttt{mass\_ratio\_mb\_over\_ms} & --- & --- & dimensionless & PDG (scheme dependent) & \cite{tfptSuiteData} \\
\texttt{mass\_ratio\_mc\_over\_mu\_quark} & --- & --- & dimensionless & PDG (scheme dependent) & \cite{tfptSuiteData} \\
\texttt{mass\_ratio\_ms\_over\_md} & --- & --- & dimensionless & PDG (scheme dependent) & \cite{tfptSuiteData} \\
\texttt{mass\_ratio\_mt\_over\_mc} & --- & --- & dimensionless & PDG (scheme dependent) & \cite{tfptSuiteData} \\
\texttt{mass\_ratio\_mu\_over\_e\_pdg} & \num{206.76828260879265} & \num{0.0} & dimensionless & PDG pole masses & \cite{pdg2024} lepton\_masses\_pdg.json (mu/e)\\
\texttt{mass\_ratio\_tau\_over\_mu\_pdg} & \num{16.81702949159037} & \num{0.0} & dimensionless & PDG pole masses & \cite{pdg2024} lepton\_masses\_pdg.json (tau/mu)\\
\texttt{n\_s\_planck2018} & \num{0.9649} & \num{0.0042} & dimensionless & Planck 2018 & \cite{planck2018} Planck 2018 results VI (2020), base-$\Lambda$CDM TT,TE,EE+lowE+lensing: $n_s = 0.9649 \pm 0.0042$ (Table 2) (\url{https://arxiv.org/abs/1807.06209})\\
\texttt{omega\_b\_h2\_planck2018} & \num{0.02237} & \num{0.00015} & dimensionless & Planck 2018 & \cite{planck2018} Planck 2018 results VI (2020), base-$\Lambda$CDM TT,TE,EE+lowE+lensing: $\Omega_b h^2 = 0.02237 \pm 0.00015$ (Table 2) (\url{https://arxiv.org/abs/1807.06209})\\
\texttt{pmns\_delta\_cp\_deg\_nufit53\_no} & \num{232.0} & \num{39.0} & \(\mathrm{deg}\) & NuFIT 5.3 (2024, NO) & \cite{nufit53_2024} pmns\_reference.json:normal\_ordering.delta\_cp\_deg\\
\texttt{pmns\_sin2\_theta12\_nufit53\_no} & \num{0.307} & \num{0.012} & dimensionless & NuFIT 5.3 (2024, NO) & \cite{nufit53_2024} pmns\_reference.json:normal\_ordering.sin2\_theta12\\
\texttt{pmns\_sin2\_theta13\_nufit53\_no} & \num{0.02224} & \num{0.00057} & dimensionless & NuFIT 5.3 (2024, NO) & \cite{nufit53_2024} pmns\_reference.json:normal\_ordering.sin2\_theta13\\
\texttt{pmns\_sin2\_theta23\_nufit53\_no} & \num{0.454} & \num{0.019} & dimensionless & NuFIT 5.3 (2024, NO) & \cite{nufit53_2024} pmns\_reference.json:normal\_ordering.sin2\_theta23\\
\texttt{sin2\_thetaW\_mz\_pdg} & \num{0.23122} & \num{4e-05} & dimensionless & PDG (MZ, MSbar) & \cite{pdg2024} sm\_inputs\_mz.json:sin2\_thetaW\\
\texttt{v\_ew\_246} & \num{246.0} & \num{0.2} & \(\mathrm{GeV}\) & SM Higgs vev & \cite{tfptSuiteData} Standard Model reference\\
\end{longtable}
\endgroup

%=============================================================================
\section*{References}
%=============================================================================

\begin{thebibliography}{99}
\bibitem{hehl1976}
F.~W.~Hehl, P.~von der Heyde, G.~D.~Kerlick, J.~M.~Nester,
``General Relativity with Spin and Torsion: Foundations and Prospects,''
Rev.\ Mod.\ Phys.\ \textbf{48}, 393 (1976).

\bibitem{fujikawa1979}
K.~Fujikawa,
``Path-Integral Measure for Gauge-Invariant Fermion Theories,''
Phys.\ Rev.\ Lett.\ \textbf{42}, 1195 (1979).

\bibitem{kawasaki1979}
T.~Kawasaki,
``The Riemann--Roch theorem for complex V-manifolds,''
Osaka J.\ Math.\ \textbf{16}, 151--159 (1979).

\bibitem{atiyah1975}
M.~F.~Atiyah, V.~K.~Patodi, I.~M.~Singer,
``Spectral asymmetry and Riemannian geometry I,''
Math.\ Proc.\ Camb.\ Phil.\ Soc.\ \textbf{77}, 43--69 (1975).

\bibitem{bunke1995}
U.~Bunke,
``On the gluing problem for the $\eta$-invariant,''
J.\ Differential Geom.\ \textbf{41}, 397--448 (1995).

\bibitem{diegoPalazuelos2022}
P.~Diego-Palazuelos \textit{et al.},
``Cosmic Birefringence from the Planck Data Release 4,''
Phys.\ Rev.\ Lett.\ \textbf{128}, 091302 (2022).

\bibitem{minamiKomatsu2020}
Y.~Minami, E.~Komatsu,
``New Extraction of the Cosmic Birefringence from the Planck 2018 Polarization Data,''
Phys.\ Rev.\ Lett.\ \textbf{125}, 221301 (2020).

\bibitem{codata2022}
E.~Tiesinga, P.~J.~Mohr, D.~B.~Newell, B.~N.~Taylor,
``CODATA Recommended Values of the Fundamental Physical Constants: 2022,''
J.\ Phys.\ Chem.\ Ref.\ Data \textbf{53}, 043103 (2024).

\bibitem{planck2018}
Planck Collaboration,
``Planck 2018 results. VI. Cosmological parameters,''
\textit{Astron.\ Astrophys.} \textbf{641}, A6 (2020), arXiv:1807.06209.

\bibitem{pdg2024}
Particle Data Group,
``Review of Particle Physics (2024),''
see PDG 2024 online review and summary tables.

\bibitem{nufit53_2024}
NuFIT collaboration,
``NuFIT 5.3 (2024) global analysis of neutrino oscillation data,''
see \texttt{http://www.nu-fit.org/} (snapshot 2024).

\bibitem{riess2022}
A.~G.~Riess \textit{et al.},
``A Comprehensive Measurement of the Local Value of the Hubble Constant with 1 km s$^{-1}$ Mpc$^{-1}$ Uncertainty from the Hubble Space Telescope and the SH0ES Team,''
Astrophys.\ J.\ Lett.\ \textbf{934}, L7 (2022).

\bibitem{kostelecky2008}
V.~A.~Kosteleck\'y, N.~Russell, J.~D.~Tasson,
``Constraints on Torsion from Bounds on Lorentz Violation,''
Phys.\ Rev.\ Lett.\ \textbf{100}, 111102 (2008).

\bibitem{shapiro2002}
I.~L.~Shapiro,
``Physical aspects of the space--time torsion,''
Phys.\ Rept.\ \textbf{357}, 113--213 (2002).

\bibitem{tfptSuiteData}
TFPT Suite (v2.5) data + policy files,
\codepath{tfpt-suite/tfpt_suite/data/} and \codepath{tfpt-suite/out/} artifacts (this repository; 2026-01-27 snapshot).
\end{thebibliography}

\end{document}
